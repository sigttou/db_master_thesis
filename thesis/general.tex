%%%%%% General %%%%%%%%%%%%%%%%%%%%%%%%%%%%%%%%%%%%%%%%%%%%%%%%%%%%%%%%%%%%%%{{{
\chapter{General}\label{sec:general}

This chapter describes terms, techniques, tools and programs used during the
work for this thesis. It provides a general overview to make it easier to
understand the details of the work in the following chapters.

\section{ELF - Executable and Linkable Format}

The ELF format is an object file format. It is used to describe a program in a
way to execute it on a processor without applying changes to the binary itself.
The assembler and linker create the ELF files. The ELF
specification~\cite{elfspec} is followed by most modern Unix like operating
systems, therefore we need to understand the layout of the executable files
before introducing bitflips to it.

\subsection{Structure of an ELF file}

There are two views of an ELF binary, the so called linking view and the the 
execution view. Those views are also called section and segment view. They 
share the same ELF header but serve different purposes for the operating system.

\subsubsection{Sections}

Sections describe the binary for the linking view, it contains instructions,
data, symbol table and relocation information. Sections reserved for the system
start with a dot, there might be additional sections defined by the user.
Sections that are directly loaded into the program's memory image are
initialized data (\texttt{.data, .data1}), read-only data (\texttt{.rodata,
.rodata1}) and executable instructions (\texttt{.text}).

\subsubsection{Segments}

Segments describe the virtual memory layout of a loaded binary. The so called 
process image contains segments with \texttt{text}, \texttt{data}, 
\texttt{stack} and others. When loading the image into memory references inside 
the ELF file need to be resolved and loaded into the memory too. After 
successfully building up the process image and its dependencies the program can 
be executed.

\subsection{Loading ELF files into Memory}

As we focus on GNU/Linux operating systems we will look at how ELF files shall 
be handled by UNIX System V Release 4 based operating systems in order to 
create running programs.
These operating systems use no physical addresses for execution and the 
operating system is free to change position of sections in the virtual address 
space. Therefore the ELF format only contains a base address per section and 
offsets to that address. During the loading step the operating system is free 
to change the base address in the programs virtual address space but keep the 
offsets.

\subsubsection{Dynamic Linking}

In modern operating systems it is widespread to have multiple functions used by
different programs. So-called shared libraries can be used to provide such
functions. When deploying programs, developers usually need to make sure all
libraries exist on the target platform or ship their software with those
included. When depending on libraries by the system, the loader will move the
shared libraries into memory at the desired entry point. It usually reads the
needed library from the ELF file and then checks the library paths provided by
the system if the library is available to be loaded.

\section{Analysis and Testing of Executables}

Testing is a large part of software development in general. There are a lot of
different approaches and styles of testing. Usually, testing was to apply input
to a program and check if the code operates as expected. With increasing amount
of code and an increasing number of bugs found, the style of testing changed
over the years. On the one hand, developers nowadays provide unit tests inside
their code to test their functions and make sure they work as desired. On the
other hand, sometimes the source code is not available to a tester, or it is way
too much effort to write testing code. Therefore there are tools which can be
used to test binaries on their own with just a little configuration.

\subsection{Fuzzing}

Fuzzing is a technology to be used to test programs with just a little knowledge
about the binary to test. Fuzzing is a black box testing approach. Testers do
not need access to the source code. It is a technique to detect erroneous
responses of a program by providing different randomised inputs. Tools applying
this technique are called fuzzers. Fuzzers try to reach corner cases in programs
without knowing exactly which exist and detect undefined behaviour or crashes of
software. Generally, fuzzers give an overall view on the robustness of software,
they are cheap to apply and can find bugs usual white box tests would miss.

Usually fuzzing is quite easy, as just random input is used to apply it. Over
the years the technology improved and different fuzzers evolved. Whereas for
simple programs just testing common input mistakes, like negative numbers,
newlines, end of file characters, format strings or just very long strings might
be enough, advanced software needs better ways of fuzzing. For example, network
protocols, file systems or image formats have a very complex structure which
makes it hard to find possible crashes by random data input. Therefore this kind
of fuzzers generate inputs from given examples and apply small changes to them.
Also, more advanced fuzzers exist which directly interact with the target,
allowing to track execution paths per input and therefore even create even more
test cases automatically.

\todo{link fuzzing technologies}

\subsection{Instrumentation}

Instrumentation usually comes down to adding new code automatically to already
existing one. Developers usually use Instrumentation as a tool for debugging or
performance measures by adding calls to timing functions or information logging
to the code. Information from instrumentation could be something like what
functions are called how often, what execution branches are taken, how long does
a subroutine take, what memory is accessed and many more. It usually also allows
the developer to change the runtime environment of a program, like change return
values or skip instruction calls.

There are two different styles of instrumentation one is source instrumentation,
and one is binary instrumentation. So-called instrumenter-calls are either added
pre-compilation to the source of the program or later directly added to the
binary context of the program. The first style might bring the advantage that it
is easier to apply and the structure of the program is better known, so calls
for performance checks over functions are done by adding timer calls at the
entry and exit points. However, binary instrumentation also has its perks. Users
do not need the source of the program for this. Recompilation of the program is
not needed if the code for the instrumentation changes. With enough debug
information in the binary the instrumentation information might not differ much
from the source code one. With binary instrumentation, it is also possible to
change and read register values before and after CPU instructions are called.

\todo{reference different Instrumentation techniques}


\subsubsection{Intel Pin}

Pin~\cite{pintool} is a proprietary, dynamic, binary instrumentation framework
developed and released by Intel, who supply it for free. The framework provides
a large number of API calls which abstracts most of the work done and gives easy
access to binary analysis. As the framework is dynamic, the source of the to be
analysed program is not needed nor does it need to be recompiled for changes in
the instrumentation tool. Pin allows to log and modify the runtime of a program,
whereas it is no problem to log register values or change those before or after
desired instruction calls or even inject code to be run before given calls.

\section{Transport Layer Security}

Transport Layer Security (TLS) is used to describe a collection of cryptographic
protocols. It is a standard proposed by the Internet Engineering Task Force
(IETF). The IETF regularly updates the collection by adding new protocols and
removing old, now seen as insecure, ones.

TLS provides a secure connection between a server and its clients. The
connection holds the following properties:
\begin{itemize}
  \item Public-key cryptography is used to provide authenticated identities.
  That is usually just needed for clients to make sure they communicate with the
  correct server. To proof the identity of a server, protocols make use of TLS
  certificates, which provide the client with the public key.
  \item Symmetric cryptography is used to provide a private connection between
  the parties. The parties have to agree on how they exchange the key for the
  symmetric encryption.
  \item The so-called message authentication code prevents undetected alteration
  or losses during communication, which makes TLS connections reliable.
\end{itemize}

As TLS is just a collection of protocols, the parties have to agree on which one
they use for each part of the connection, which happens during the TLS
handshake. A common TLS connection protocol is
\texttt{ECDHE-RSA-AES256-GCM-HMAC-SHA1}, which contains all three protocols used
for the connection. Whereas \texttt{ECDHE-RSA} provides the key exchange,
\texttt{AES256-GCM} is the  symmetric cypher and \texttt{HMAC-SHA1} checks data
integrity.

\todo{reference TLS standard}

\subsection{Cryptographic Nonces}

In the early days of encrypted communication replay-attacks were a common
problem.~\todo{reference replay-attack paper} To resolve this issue newer
protocols introduced so-called nonces. Cryptographic nonces are one-time
pseudorandom numbers used for each packet sent. They are combined with the key
so that the encryption differs for each packet. This technique makes
replay-attacks harder. In most cases, nonce reuse is a problem for the protocol
and results in breakage of the encryption of future or past packages. Hence,
most protocol standards define a nonce as a pseudorandom or unpredicted
value~\cite{noncegeneral}. It is also possible to use the nonce can as a counter
which has a random start value and increases for each packet sent.

\subsubsection{AES-GCM Nonce Reuse Attack}

Lorem ipsum dolor sit amet, consetetur sadipscing elitr, sed diam nonumy eirmod
tempor invidunt ut labore et dolore magna aliquyam erat, sed diam voluptua. At
vero eos et accusam et justo duo dolores et ea rebum. Stet clita kasd gubergren,
no sea takimata sanctus est Lorem ipsum dolor sit amet.

\section{Rowhammer}

Technology is steadily changing, with vendors of computer parts being forced to
produce cheaper and hardware every year. For DRAM this resulted in less quality
management and proper checks on possible bitflips which allowed smaller chipsets
into production. These chips provide less space to store loads, which represents
data, and little energy to be used to store the loads. Years after vendors
introduced this increased densities in DRAM to the market researchers like
\todo{references rowhammer} found a critical bug in the chip design. With the
little space and energy stored it was possible to use charges of memory rows to
leak data to nearby rows, which made it possible to even change bits of other
rows with particular crafted memory access patterns like shown by
\todo{references rowhammer}.

This behaviour and bug then caused researchers to dig deeper into the
possibilities, and several attacks were found based on the Rowhammer bug.

\subsection{Types of Rowhammer Attacks}

Lorem ipsum dolor sit amet, consetetur sadipscing elitr, sed diam nonumy eirmod
tempor invidunt ut labore et dolore magna aliquyam erat, sed diam voluptua. At
vero eos et accusam et justo duo dolores et ea rebum. Stet clita kasd gubergren,
no sea takimata sanctus est Lorem ipsum dolor sit amet.
%}}}

%% vim:foldmethod=expr
%% vim:fde=getline(v\:lnum)=~'^%%%%\ .\\+'?'>1'\:'='
%%% Local Variables:
%%% mode: latex
%%% mode: auto-fill
%%% mode: flyspell
%%% eval: (ispell-change-dictionary "en_US")
%%% TeX-master: "main"
%%% End:
