%%%%%% General %%%%%%%%%%%%%%%%%%%%%%%%%%%%%%%%%%%%%%%%%%%%%%%%%%%%%%%%%%%%%%{{{
\chapter{General}\label{sec:general}

This chapter describes term, techniques, tools and programs used for this 
thesis. It provides a general overview to make it easier to understand the 
details of the work in the following chapters.

\section{ELF - Executable and Linkable Format}

The ELF format is an object file format. It is used to describe a program in a 
way to execute it on a processor without applying changes to the binary content. 
The assembler and linker create the ELF files. For our work, we need to look at 
ELF sections containing actual code to run or data which might change the 
execution path of a binary. 

\subsection{Structure of an ELF file}

There are two views of an ELF binary, the so called linking view and the the 
execution view. Those views are also called section and segment view. They 
share the same ELF header but serve different purposes for the operating system.

\subsubsection{Sections}

Sections describe the binary for the linking view, it contains instructions, 
data, symbol table and relocation information. Sections reserved for the system 
start with a dot, there might be other sections in the binary but those are not 
relevant for this work. Sections that contribute to the program's memory image 
are initialized data (\texttt{.data, .data1}), read-only data (\texttt{.rodata, 
.rodata1}) and executable instructions (\texttt{.text}). 

\subsubsection{Segments}

Segments describe the virtual memory layout of a loaded binary. The so called 
process image contains segments with \texttt{text}, \texttt{data}, 
\texttt{stack} and others. When loading the image into memory references inside 
the ELF file need to be resolved and loaded into the memory too. After 
successfully building up the process image and its dependencies the program can 
be executed.

\subsection{Loading ELF files into Memory}

As we focus on GNU/Linux operating systems we will look at how ELF files shall 
be handled by UNIX System V Release 4 based operating systems in order to 
create running programs.
These operating systems use no physical addresses for execution and the 
operating system is free to change position of sections in the virtual address 
space. Therefore the ELF format only contains a base address per section and 
offsets to that address. During the loading step the operating system is free 
to change the base address in the programs virtual address space but keep the 
offsets. For our work we therefore need to be able to track back the position 
of sections from the virtual address space to the ELF file and vice versa.

\subsubsection{Dynamic Linking}

\section{Analysis and Testing of ELFs}

Lorem ipsum dolor sit amet, consetetur sadipscing elitr, sed diam nonumy eirmod
tempor invidunt ut labore et dolore magna aliquyam erat, sed diam voluptua. At
vero eos et accusam et justo duo dolores et ea rebum. Stet clita kasd gubergren,
no sea takimata sanctus est Lorem ipsum dolor sit amet.

\subsection{Instrumentation}

Lorem ipsum dolor sit amet, consetetur sadipscing elitr, sed diam nonumy eirmod
tempor invidunt ut labore et dolore magna aliquyam erat, sed diam voluptua. At
vero eos et accusam et justo duo dolores et ea rebum. Stet clita kasd gubergren,
no sea takimata sanctus est Lorem ipsum dolor sit amet.

\subsection{Fuzzing}

Lorem ipsum dolor sit amet, consetetur sadipscing elitr, sed diam nonumy eirmod
tempor invidunt ut labore et dolore magna aliquyam erat, sed diam voluptua. At
vero eos et accusam et justo duo dolores et ea rebum. Stet clita kasd gubergren,
no sea takimata sanctus est Lorem ipsum dolor sit amet.

\subsubsection{ELF Code Fuzzing}

Lorem ipsum dolor sit amet, consetetur sadipscing elitr, sed diam nonumy eirmod
tempor invidunt ut labore et dolore magna aliquyam erat, sed diam voluptua. At
vero eos et accusam et justo duo dolores et ea rebum. Stet clita kasd gubergren,
no sea takimata sanctus est Lorem ipsum dolor sit amet.

\section{Transport Layer Security}

Lorem ipsum dolor sit amet, consetetur sadipscing elitr, sed diam nonumy eirmod
tempor invidunt ut labore et dolore magna aliquyam erat, sed diam voluptua. At
vero eos et accusam et justo duo dolores et ea rebum. Stet clita kasd gubergren,
no sea takimata sanctus est Lorem ipsum dolor sit amet.

\subsection{Ciphers with Counting Nonces}

Lorem ipsum dolor sit amet, consetetur sadipscing elitr, sed diam nonumy eirmod
tempor invidunt ut labore et dolore magna aliquyam erat, sed diam voluptua. At
vero eos et accusam et justo duo dolores et ea rebum. Stet clita kasd gubergren,
no sea takimata sanctus est Lorem ipsum dolor sit amet.

\subsubsection{Attacks based on Nonce Misuse}

Lorem ipsum dolor sit amet, consetetur sadipscing elitr, sed diam nonumy eirmod
tempor invidunt ut labore et dolore magna aliquyam erat, sed diam voluptua. At
vero eos et accusam et justo duo dolores et ea rebum. Stet clita kasd gubergren,
no sea takimata sanctus est Lorem ipsum dolor sit amet.

\subsubsection{AES-GCM Nonce Reuse Attack}

Lorem ipsum dolor sit amet, consetetur sadipscing elitr, sed diam nonumy eirmod
tempor invidunt ut labore et dolore magna aliquyam erat, sed diam voluptua. At
vero eos et accusam et justo duo dolores et ea rebum. Stet clita kasd gubergren,
no sea takimata sanctus est Lorem ipsum dolor sit amet.

\section{Rowhammer}

Lorem ipsum dolor sit amet, consetetur sadipscing elitr, sed diam nonumy eirmod
tempor invidunt ut labore et dolore magna aliquyam erat, sed diam voluptua. At
vero eos et accusam et justo duo dolores et ea rebum. Stet clita kasd gubergren,
no sea takimata sanctus est Lorem ipsum dolor sit amet.

\subsection{Types of Rowhammer Attacks}

Lorem ipsum dolor sit amet, consetetur sadipscing elitr, sed diam nonumy eirmod
tempor invidunt ut labore et dolore magna aliquyam erat, sed diam voluptua. At
vero eos et accusam et justo duo dolores et ea rebum. Stet clita kasd gubergren,
no sea takimata sanctus est Lorem ipsum dolor sit amet.
%}}}

%% vim:foldmethod=expr
%% vim:fde=getline(v\:lnum)=~'^%%%%\ .\\+'?'>1'\:'='
%%% Local Variables:
%%% mode: latex
%%% mode: auto-fill
%%% mode: flyspell
%%% eval: (ispell-change-dictionary "en_US")
%%% TeX-master: "main"
%%% End:
