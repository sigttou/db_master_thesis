%%%%%% General %%%%%%%%%%%%%%%%%%%%%%%%%%%%%%%%%%%%%%%%%%%%%%%%%%%%%%%%%%%%%%{{{
\chapter{General}\label{sec:general}

This chapter describes terms, techniques, tools and programs used during the
work for this thesis. It provides a general overview to make it easier to
understand the details of the work in the following chapters.

\section{ELF - Executable and Linkable Format}

The ELF format is an object file format. It is used to describe a program in a
way to execute it on a processor without applying changes to the binary itself.
The assembler and linker create the ELF files. The ELF
specification~\cite{elfspec} is followed by most modern Unix like operating
systems, therefore we need to understand the layout of the executable files
before introducing bitflips to it.

\subsection{Structure of an ELF file}

There are two views of an ELF binary, the so called linking view and the the 
execution view. Those views are also called section and segment view. They 
share the same ELF header but serve different purposes for the operating system.

\subsubsection{Sections}

Sections describe the binary for the linking view, it contains instructions,
data, symbol table and relocation information. Sections reserved for the system
start with a dot, there might be additional sections defined by the user.
Sections that are directly loaded into the program's memory image are
initialized data (\texttt{.data, .data1}), read-only data (\texttt{.rodata,
.rodata1}) and executable instructions (\texttt{.text}).

\subsubsection{Segments}

Segments describe the virtual memory layout of a loaded binary. The so called 
process image contains segments with \texttt{text}, \texttt{data}, 
\texttt{stack} and others. When loading the image into memory references inside 
the ELF file need to be resolved and loaded into the memory too. After 
successfully building up the process image and its dependencies the program can 
be executed.

\subsection{Loading ELF files into Memory}

As we focus on GNU/Linux operating systems we will look at how ELF files shall 
be handled by UNIX System V Release 4 based operating systems in order to 
create running programs.
These operating systems use no physical addresses for execution and the 
operating system is free to change position of sections in the virtual address 
space. Therefore the ELF format only contains a base address per section and 
offsets to that address. During the loading step the operating system is free 
to change the base address in the programs virtual address space but keep the 
offsets.

\subsubsection{Dynamic Linking}

In modern operating systems it is widespread to have multiple functions used
by different programs. So-called shared libraries can be used to provide such 
functions. When deploying programs, developers usually need to make sure all 
needed libraries are already provided or ship their software with those 
included. When depending on libraries by the system, the loader will move 
the shared libraries into memory at the desired entry point. It 
usually reads the needed library from the ELF file and then checks the library 
paths provided by the system if the library is available to be loaded.

\section{Analysis and Testing of Executables}

Testing is a large part of software development in general. There are a lot of
different approaches and styles of testing. Usually, testing was to apply input
to a program and check if the code operates as expected. With increasing amount
of code and an increasing number of bugs found, the style of testing changed
over the years. On the one hand, developers nowadays provide unit tests inside
their code to test their functions and make sure they work as desired. On the
other hand, sometimes the source code is not available to a tester, or it is way
too much effort to write testing code. Therefore there are tools which can be
used to test binaries on their own with just a little configuration.

\subsection{Fuzzing}

Lorem ipsum dolor sit amet, consetetur sadipscing elitr, sed diam nonumy eirmod
tempor invidunt ut labore et dolore magna aliquyam erat, sed diam voluptua. At
vero eos et accusam et justo duo dolores et ea rebum. Stet clita kasd gubergren,
no sea takimata sanctus est Lorem ipsum dolor sit amet.

\subsection{Instrumentation}

Lorem ipsum dolor sit amet, consetetur sadipscing elitr, sed diam nonumy eirmod
tempor invidunt ut labore et dolore magna aliquyam erat, sed diam voluptua. At
vero eos et accusam et justo duo dolores et ea rebum. Stet clita kasd gubergren,
no sea takimata sanctus est Lorem ipsum dolor sit amet.

\subsubsection{ELF Code Fuzzing}

Lorem ipsum dolor sit amet, consetetur sadipscing elitr, sed diam nonumy eirmod
tempor invidunt ut labore et dolore magna aliquyam erat, sed diam voluptua. At
vero eos et accusam et justo duo dolores et ea rebum. Stet clita kasd gubergren,
no sea takimata sanctus est Lorem ipsum dolor sit amet.

\section{Transport Layer Security}

Lorem ipsum dolor sit amet, consetetur sadipscing elitr, sed diam nonumy eirmod
tempor invidunt ut labore et dolore magna aliquyam erat, sed diam voluptua. At
vero eos et accusam et justo duo dolores et ea rebum. Stet clita kasd gubergren,
no sea takimata sanctus est Lorem ipsum dolor sit amet.

\subsection{Nonce Misuse Attacks}

Lorem ipsum dolor sit amet, consetetur sadipscing elitr, sed diam nonumy eirmod
tempor invidunt ut labore et dolore magna aliquyam erat, sed diam voluptua. At
vero eos et accusam et justo duo dolores et ea rebum. Stet clita kasd gubergren,
no sea takimata sanctus est Lorem ipsum dolor sit amet.

\subsubsection{AES-GCM Nonce Reuse Attack}

Lorem ipsum dolor sit amet, consetetur sadipscing elitr, sed diam nonumy eirmod
tempor invidunt ut labore et dolore magna aliquyam erat, sed diam voluptua. At
vero eos et accusam et justo duo dolores et ea rebum. Stet clita kasd gubergren,
no sea takimata sanctus est Lorem ipsum dolor sit amet.

\section{Rowhammer}

Technology is steadily changing, with vendors of computer parts being forced to
produce cheaper and hardware every year. For DRAM this resulted in less quality
management and proper checks on possible bitflips which allowed smaller chipsets
into production. These chips provide less space to store loads, which represents
data, and little energy to be used to store the loads. Years after vendors
introduced this increased densities in DRAM to the market researchers like
<TODO: REFERENCE ROWHAMMER> found a critical bug in the chip design. With the
little space and energy stored it was possible to use charges of memory rows to
leak data to nearby rows, which made it possible to even change bits of other
rows with particular crafted memory access patterns like shown by
<TODO: REFERENCE ROWHAMMER>.

This behaviour and bug then caused researchers to dig deeper into the
possibilities, and several attacks were found based on the Rowhammer bug.

\subsection{Types of Rowhammer Attacks}

Lorem ipsum dolor sit amet, consetetur sadipscing elitr, sed diam nonumy eirmod
tempor invidunt ut labore et dolore magna aliquyam erat, sed diam voluptua. At
vero eos et accusam et justo duo dolores et ea rebum. Stet clita kasd gubergren,
no sea takimata sanctus est Lorem ipsum dolor sit amet.
%}}}

%% vim:foldmethod=expr
%% vim:fde=getline(v\:lnum)=~'^%%%%\ .\\+'?'>1'\:'='
%%% Local Variables:
%%% mode: latex
%%% mode: auto-fill
%%% mode: flyspell
%%% eval: (ispell-change-dictionary "en_US")
%%% TeX-master: "main"
%%% End:
