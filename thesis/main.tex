%%%% Time-stamp: <2017-02-05 16:13:04 vk>
%% ========================================================================
%%%% Disclaimer
%% ========================================================================
%%
%% created by
%%
%%      Karl Voit
%%

%% ========================================================================
%%%% Basic settings
%% ========================================================================
%% (idea of using newcommands for basic documentclass settings from: Thomas Schlager)

\newcommand{\mypapersize}{A4}
%% e.g., "A4", "letter", "legal", "executive", ...
%% The size of the paper of the resulting PDF file.

%\newcommand{\mylaterality}{twoside}
\newcommand{\mylaterality}{oneside}
%% "oneside" or "twoside"
%% Either you are creating a document which is printed on both, left pages
%% and right pages (twoside) or you create a document which is printed
%% on right pages only (oneside).

\newcommand{\mydraft}{false}
%% "true" or "false"
%% Use draft mode? If true, included graphics are replaced by empty
%% rectangles (of same size) and overfull boxes (in margin space) are
%% marked with black box (-> easy to spot!)

\newcommand{\myparskip}{half}
%% e.g., "no", "full", "half", ...
%% How to separate paragraphs: indention ("no") or spacing ("half",
%% "full", ...).

\newcommand{\myBCOR}{0mm}
%% Inner binding correction. This value depends on the method which is
%% being used to bind your printed result. Some techniques do not
%% require a binding correction at all ("0mm"), other require for
%% example "5mm". Refer to KOMA script documentation for a detailed
%% explanation what a binding correction is and how to measure it.

\newcommand{\myfontsize}{11pt}
%% e.g., 10pt, 11pt, 12pt
%% The font size of the main text in pt (points).

\newcommand{\mylinespread}{1.0}
%% e.g., 1.0, 1.5, 2.0
%% Line spacing in %/100. For example 1.5 means 150% of the usual line
%% spacing. Please use with caution: 100% ("1.0") is fine because the
%% font was designed for it.

\newcommand{\mylanguage}{ngerman,american}
%% "english,ngerman", "ngerman,english", ...
%% NOTE: The *last* language is the active one!
%% See babel documentation for further details.

%% BibLaTeX-settings: (see biblatex reference for further description)
\newcommand{\mybiblatexstyle}{numeric}
%% e.g., "alphabetic", "authoryear", ...
%% The biblatex style which is being used for referencing. See
%% biblatex documentation for further details and more values.
%%
%% CAUTION: if you change the style, please check for (in)compatible
%%          "biblatex" package options in the file
%%          "template/preamble.tex"! For example: "alphabetic" does
%%          not have an option "dashed=..." and causes an error if it
%%          does not get removed from the list of options.

\newcommand{\mybiblatexdashed}{false}  %% "true" or "false"
%% If true: replace recurring reference authors with a dash.

\newcommand{\mybiblatexbackref}{true}  %% "true" or "false"
%% If true: create backward links from reference to citations.

% removed for vim-latex to find references
%\newcommand{\mybiblatexfile}{references-biblatex.bib}
%% Name of the biblatex file that holds the references.

%\newcommand{\mydispositioncolor}{30,103,182}
%% e.g., "30,103,182" (blue/turquois), "0,0,0" (black), ...
%% Color of the headings and so forth in RGB (red,green,blue) values.
%% NOTE: if you are using "0,0,0" for black, printers might still
%%       recognize pages as color pages. In case this is a problem
%%       (paying for color print-outs vs. paying for b/w-printouts)
%%       please edit file "template/preamble.tex" and change
%%       "\definecolor{DispositionColor}{RGB}{\mydispositioncolor}"
%%       to "\definecolor{DispositionColor}{gray}{0}" and thus
%%       overwriting the value of \mydispositioncolor above.

\newcommand{\mycolorlinks}{true}  %% "true" or "false"
%% Enables or disables colored links (hyperref package).

\newcommand{\mytitlepage}{template/title_Thesis_TU_Graz_-_kazemakase}
%% Your own or one of following pre-defined title pages:
%% "template/title_plain_maketitle": simple maketitle page
%% "template/title_Diplomarbeit_KF_Uni_Graz.tex": fancy (german) title page for KF Uni Graz
%% "template/title_Thesis_TU_Graz":
%%             titlepage for Graz University of Technology (correct
%%             (old?) Corporate Design) by Karl Voit (2012)
%% "template/title_Thesis_TU_Graz_-_kazemakase":
%%             titlepage for Graz University of Technology
%%             (correct new Corporate Design) by kazemakase (2013):
%%             see https://github.com/novoid/LaTeX-KOMA-template/issues/5
%% "template/title_VWA": titlepage for Vorwissenschaftliche Arbeit

\newcommand{\mytodonotesoptions}{}
%% e.g., "" (empty), "disable", ...
%% Options for the todonotes-package. If "disable", all todonotes will
%% be hidden (including listoftodos).

%% Load main settings for document preamble:
%% Time-stamp: <2015-04-30 17:23:24 vk>
%%%% === Disclaimer: =======================================================
%% created by
%%
%%      Karl Voit
%%
%% using GNU/Linux, GNU Emacs & LaTeX 2e
%%

%doc% %% overriding preamble/preamble.tex %%
%doc% \newcommand{\mylinespread}{1.0}  \newcommand{\mycolorlinks}{true}
%doc% \documentclass[12pt,paper=a4,parskip=half,DIV=calc,oneside,%%
%doc% headinclude,footinclude=false,open=right,bibliography=totoc]{scrartcl}
%doc% \usepackage[utf8]{inputenc}\usepackage[ngerman,american]{babel}\usepackage{scrpage2}
%doc% \usepackage{ifthen}\usepackage{eurosym}\usepackage{xspace}\usepackage[usenames,dvipsnames]{xcolor}
%doc% \usepackage[protrusion=true,factor=900]{microtype}
%doc% \usepackage{enumitem}
%doc% \usepackage[pdftex]{graphicx}
%doc% \usepackage{todonotes}
%doc% \usepackage{dingbat,bbding} %% special characters
%doc% \definecolor{DispositionColor}{RGB}{30,103,182}
%doc%
%doc% \usepackage[backend=biber,style=authoryear,dashed=false,natbib=true,hyperref=true%%
%doc% ]{biblatex}
%doc%
%doc% \addbibresource{references-biblatex.bib} %% remove, if using BibTeX instead of biblatex
%doc%
%doc% %% overriding userdata %%
%doc% \newcommand{\myauthor}{Karl Voit}\newcommand{\mytitle}{LaTeX Template Documentation}
%doc% \newcommand{\mysubject}{A Comprehensive Guide to Use the
%doc% Template from https://github.com/novoid/LaTeX-KOMA-template}
%doc% \newcommand{\mykeywords}{LaTeX, pdflatex, template, documentation, biber, biblatex}
%doc%
%doc% \newcommand{\myLaT}{\LaTeX{}@TUG\xspace}
%doc%
%doc% %% for future use?
%doc% % \usepackage{filecontents}
%doc% % \begin{filecontents}{filename.example}
%doc% %
%doc% % \end{filecontents}
%doc%
%doc%
%doc% %% using existing TeX files %%
%doc% %% Time-stamp: <2015-04-30 17:19:58 vk>
%%%% === Disclaimer: =======================================================
%% created by
%%
%%      Karl Voit
%%
%% using GNU/Linux, GNU Emacs & LaTeX 2e
%%

%doc%
%doc% \section{\texttt{mycommands.tex} --- various definitions}\myinteresting
%doc% \label{sec:mycommands}
%doc%
%doc% In file \verb#template/mycommands.tex# many useful commands are being
%doc% defined. 
%doc% 
%doc% \paragraph{What should I do with this file?} Please take a look at its 
%doc% content to get the most out of your document.
%doc% 

%doc% 
%doc% One of the best advantages of \LaTeX{} compared to \myacro{WYSIWYG} software products is
%doc% the possibility to define and use macros within text. This empowers the user to
%doc% a great extend.  Many things can be defined using \verb#\newcommand{}# and
%doc% automates repeating tasks. It is recommended to use macros not only for
%doc% repetitive tasks but also for separating form from content such as \myacro{CSS}
%doc% does for \myacro{XHTML}. Think of including graphics in your document: after
%doc% writing your book, you might want to change all captions to the upper side of
%doc% each figure. In this case you either have to modify all
%doc% \texttt{includegraphics} commands or you were clever enough to define something
%doc% like \verb#\myfig#\footnote{See below for a detailed description}. Using a
%doc% macro for including graphics enables you to modify the position caption on only
%doc% \emph{one} place: at the definition of the macro.
%doc% 
%doc% The following section describes some macros that came with this document template
%doc% from \myLaT and you are welcome to modify or extend them or to create
%doc% your own macros!
%doc% 

%doc% 
%doc% \subsection{\texttt{myfig} --- including graphics made easy}
%doc% 
%doc% The classic: you can easily add graphics to your document with \verb#\myfig#:
%doc% \begin{verbatim}
%doc%  \myfig{flower}%% filename w/o extension in the folder figures
%doc%        {width=0.7\textwidth}%% maximum width/height, aspect ratio will be kept
%doc%        {This flower was photographed at my home town in 2010}%% caption
%doc%        {Home town flower}%% optional (short) caption for list of figures
%doc%        {fig:flower}%% label
%doc% \end{verbatim}
%doc% 
%doc% There are many advantages of this command (compared to manual
%doc% \texttt{figure} environments and \texttt{includegraphics} commands:
%doc% \begin{itemize}
%doc% \item consistent style throughout the whole document
%doc% \item easy to change; for example move caption on top
%doc% \item much less characters to type (faster, error prone)
%doc% \item less visual clutter in the \TeX{}-files
%doc% \end{itemize}
%doc% 
%doc% 
\newcommand{\myfig}[5]{
%% example:
% \myfig{}%% filename in figures folder
%       {width=0.5\textwidth,height=0.5\textheight}%% maximum width/height, aspect ratio will be kept
%       {}%% caption
%       {}%% optional (short) caption for list of figures
%       {}%% label
\begin{figure}%[htp]
  \centering
  \includegraphics[keepaspectratio,#2]{figures/#1}
  \caption[#4]{#3}
  \label{#5} %% NOTE: always label *after* caption!
\end{figure}
}


%doc% 
%doc% \subsection{\texttt{myclone} --- repeat things!}
%doc% 
%doc% Using \verb#\myclone[42]{foobar}# results the text \enquote{foobar} printed 42 times.
%doc% But you can not only repeat text output with \texttt{myclone}. 
%doc%
%doc% Default argument
%doc% for the optional parameter \enquote{number of times} (like \enquote{42} in the example above) 
%doc% is set to two.
%doc% 
%% \myclone[x]{text}
\newcounter{myclonecnt}
\newcommand{\myclone}[2][2]{%
  \setcounter{myclonecnt}{#1}%
  \whiledo{\value{myclonecnt}>0}{#2\addtocounter{myclonecnt}{-1}}%
}

%old% %d oc% 
%old% %d oc% \subsection{\texttt{fixxme} --- sidemark something as unfinished}
%old% %d oc% 
%old% %d oc% You know it: something has to be fixed and you can not do it right
%old% %d oc% now. In order to \texttt{not} forget about it, you might want to add a
%old% %d oc% note like \verb+\fixxme{check again}+ which inserts a note on the page
%old% %d oc% margin such as this\fixxme{check again} example.
%old% %d oc%
%old% \newcommand{\fixxme}[1]{%%
%old% \textcolor{red}{FIXXME}\marginpar{\textcolor{red}{#1}}%%
%old% }


%%%% End 
%%% Local Variables:
%%% mode: latex
%%% mode: auto-fill
%%% mode: flyspell
%%% eval: (ispell-change-dictionary "en_US")
%%% TeX-master: "../main"
%%% End:
%% vim:foldmethod=expr
%% vim:fde=getline(v\:lnum)=~'^%%%%'?0\:getline(v\:lnum)=~'^%doc.*\ .\\%(sub\\)\\?section{.\\+'?'>1'\:'1':

%doc% %%%% Time-stamp: <2015-08-22 17:20:32 vk>
%%%% === Disclaimer: =======================================================
%% created by
%%
%%      Karl Voit
%%
%% using GNU/Linux, GNU Emacs & LaTeX 2e
%%
%doc%
%doc% \section{\texttt{typographic\_settings.tex} --- Typographic finetuning}
%doc%
%doc% The settings of file \verb#template/typographic_settings.tex# contain
%doc% typographic finetuning related to things mentioned in literature.  The
%doc% settings in this file relates to personal taste and most of all: 
%doc% \emph{typographic experience}. 
%doc% 
%doc% \paragraph{What should I do with this file?} You might as well skip the whole
%doc% file by excluding the \verb#%%%% Time-stamp: <2015-08-22 17:20:32 vk>
%%%% === Disclaimer: =======================================================
%% created by
%%
%%      Karl Voit
%%
%% using GNU/Linux, GNU Emacs & LaTeX 2e
%%
%doc%
%doc% \section{\texttt{typographic\_settings.tex} --- Typographic finetuning}
%doc%
%doc% The settings of file \verb#template/typographic_settings.tex# contain
%doc% typographic finetuning related to things mentioned in literature.  The
%doc% settings in this file relates to personal taste and most of all: 
%doc% \emph{typographic experience}. 
%doc% 
%doc% \paragraph{What should I do with this file?} You might as well skip the whole
%doc% file by excluding the \verb#\input{template/typographic_settings.tex}# command
%doc% in \texttt{main.tex}.  For standard usage it is recommended to stay with the
%doc% default settings.
%doc% 
%doc% 
%% ========================================================================

%doc%
%doc% Some basic microtypographic settings are provided by the
%doc% \texttt{microtype}
%doc% package\footnote{\url{http://ctan.org/pkg/microtype}}. This template
%doc% uses the rather conservative package parameters: \texttt{protrusion=true,factor=900}.
\usepackage[protrusion=true,factor=900]{microtype}

%doc%
%doc% \subsection{French spacing}
%doc%
%doc% \paragraph{Why?} see~\textcite[p.\,28, p.\,30]{Bringhurst1993}: `2.1.4 Use a single word space between sentences.'
%doc%
%doc% \paragraph{How?} see~\textcite[p.\,185]{Eijkhout2008}:\\
%doc% \verb#\frenchspacing  %% Macro to switch off extra space after punctuation.# \\
\frenchspacing  %% Macro to switch off extra space after punctuation.
%doc%
%doc% Note: This setting might be default for \myacro{KOMA} script.
%doc%


%doc%
%doc% \subsection{Font}
%doc% 
%doc% This template is using the Palatino font (package \texttt{mathpazo}) which results
%doc% in a legible document and matching mathematical fonts for printout.
%doc% 
%doc% It is highly recommended that you either stick to the Palatino font or use the
%doc% \LaTeX{} default fonts (by removing the package \texttt{mathpazo}).
%doc% 
%doc% Chosing different fonts is not
%doc% an easy task. Please leave this to people with good knowledge on this subject.
%doc% 
%doc% One valid reason to change the default fonts is when your document is mainly
%doc% read on a computer screen. In this case it is recommended to switch to a font
%doc% \textsf{which is sans-serif like this}. This template contains several alternative
%doc% font packages which can be activated in this file.
%doc% 

% for changing the default font, please go to the next subsection!

%doc%
%doc% \subsection{Text figures}
%doc% 
%doc% \ldots also called old style numbers such as 0123456789. 
%doc% (German: \enquote{Mediäval\-ziffern\footnote{\url{https://secure.wikimedia.org/wikibooks/de/wiki/LaTeX-W\%C3\%B6rterbuch:\_Medi\%C3\%A4valziffern}}})
%doc% \paragraph{Why?} see~\textcite[p.\,44f]{Bringhurst1993}: 
%doc% \begin{quote}
%doc% `3.2.1 If the font includes both text figures and titling figures, use
%doc%  titling figures only with full caps, and text figures in all other
%doc%  circumstances.'
%doc% \end{quote}
%doc% 
%doc% \paragraph{How?} 
%doc% Quoted from Wikibooks\footnote{\url{https://secure.wikimedia.org/wikibooks/en/wiki/LaTeX/Formatting\#Text\_figures\_.28.22old\_style.22\_numerals.29}}:
%doc% \begin{quote}
%doc% Some fonts do not have text figures built in; the textcomp package attempts to
%doc% remedy this by effectively generating text figures from the currently-selected
%doc% font. Put \verb#\usepackage{textcomp}# in your preamble. textcomp also allows you to
%doc% use decimal points, properly formatted dollar signs, etc. within
%doc% \verb#\oldstylenums{}#.
%doc% \end{quote}
%doc% \ldots but proposed \LaTeX{} method does not work out well. Instead use:\\
%doc% \verb#\usepackage{hfoldsty}#  (enables text figures using additional font) or \\
%doc% \verb#\usepackage[sc,osf]{mathpazo}# (switches to Palatino font with small caps and old style figures enabled).
%doc%
%\usepackage{hfoldsty}  %% enables text figures using additional font
%% ... OR use ...
%\usepackage[sc,osf]{mathpazo} %% switches to Palatino with small caps and old style figures

%% Font selection from:
%%     http://www.matthiaspospiech.de/latex/vorlagen/allgemein/preambel/fonts/
%% use following lines *instead* of the mathpazo package above:
%% ===== Serif =========================================================
%% for Computer Modern (LaTeX default font), simply remove the mathpazo above
%\usepackage{charter}\linespread{1.05} %% Charter
%\usepackage{bookman}                  %% Bookman (laedt Avant Garde !!)
%\usepackage{newcent}                  %% New Century Schoolbook (laedt Avant Garde !!)
%% ===== Sans Serif ====================================================
%\renewcommand{\familydefault}{\sfdefault}  %% this one in *combination* with the default mathpazo package
%\usepackage{cmbright}                  %% CM-Bright (eigntlich eine Familie)
%\usepackage{tpslifonts}                %% tpslifonts % Font for Slides


%doc% 
%doc% \subsection{\texttt{myacro} --- Abbrevations using \textsc{small caps}}\myinteresting
%doc% \label{sec:myacro}
%doc% 
%doc% \paragraph{Why?} see~\textcite[p.\,45f]{Bringhurst1993}: `3.2.2 For abbrevations and
%doc% acronyms in the midst of normal text, use spaced small caps.'
%doc% 
%doc% \paragraph{How?} Using the predefined macro \verb#\myacro{}# for things like
%doc% \myacro{UNO} or \myacro{UNESCO} using \verb#\myacro{UNO}# or \verb#\myacro{UNESCO}#.
%doc% 
\DeclareRobustCommand{\myacro}[1]{\textsc{\lowercase{#1}}} %%  abbrevations using small caps


%doc% 
%doc% \subsection{Colorized headings and links}
%doc% 
%doc% This document template is able to generate an output that uses colorized
%doc% headings, captions, page numbers, and links. The color named `DispositionColor'
%doc% used in this document is defined near the definition of package \texttt{color}
%doc% in the preamble (see section~\ref{subsec:miscpackages}). The changes required
%doc% for headings, page numbers, and captions are defined here.
%doc% 
%doc% Settings for colored links are handled by the definitions of the
%doc% \texttt{hyperref} package (see section~\ref{sec:pdf}).
%doc% 
\setheadsepline{.4pt}[\color{DispositionColor}]
\renewcommand{\headfont}{\normalfont\sffamily\color{DispositionColor}}
\renewcommand{\pnumfont}{\normalfont\sffamily\color{DispositionColor}}
\addtokomafont{disposition}{\color{DispositionColor}}
\addtokomafont{caption}{\color{DispositionColor}\footnotesize}
\addtokomafont{captionlabel}{\color{DispositionColor}}

%doc% 
%doc% \subsection{No figures or tables below footnotes}
%doc% 
%doc% \LaTeX{} places floating environments below footnotes if \texttt{b}
%doc% (bottom) is used as (default) placement algorithm. This is certainly
%doc% not appealing for most people and is deactivated in this template by
%doc% using the package \texttt{footmisc} with its option \texttt{bottom}.
%doc% 
%% see also: http://www.komascript.de/node/858 (German description)
\usepackage[bottom]{footmisc}

%doc% 
%doc% \subsection{Spacings of list environments}
%doc% 
%doc% By default, \LaTeX{} is using vertical spaces between items of enumerate, 
%doc% itemize and description environments. This is fine for multi-line items.
%doc% Many times, the user does just write single-line items where the larger
%doc% vertical space is inappropriate. The \href{http://ctan.org/pkg/enumitem}{enumitem}
%doc% package provides replacements for the pre-defined list environments and
%doc% offers many options to modify their appearances.
%doc% This template is using the package option for \texttt{noitemsep} which
%doc% mimimizes the vertical space between list items.
%doc% 
\usepackage{enumitem}
\setlist{noitemsep}   %% kills the space between items

%doc% 
%doc% \subsection{\texttt{csquotes} --- Correct quotation marks}\myinteresting
%doc% \label{sub:csquotes}
%doc% 
%doc% \emph{Never} use quotation marks found on your keyboard.
%doc% They end up in strange characters or false looking quotation marks.
%doc% 
%doc% In \LaTeX{} you are able to use typographically correct quotation marks. The package 
%doc% \href{http://www.ctan.org/pkg/csquotes}{\texttt{csquotes}} offers you with 
%doc% \verb#\enquote{foobar}# a command to get correct quotation marks around \enquote{foobar}.
%doc% Please do check the package options in order to modify
%doc% its settings according to the language used\footnote{most of the time in 
%doc% combination with the language set in the options of the \texttt{babel} package}.
%doc% 
%doc% \href{http://www.ctan.org/pkg/csquotes}{\texttt{csquotes}} is also recommended 
%doc% by \texttt{biblatex} (see Section~\ref{sec:references}). 
\usepackage[babel=true,strict=true,english=american,german=guillemets]{csquotes}

%doc% 
%doc% \subsection{Line spread}
%doc% 
%doc% If you have to enlarge the distance between two lines of text, you can
%doc% increase it using the \texttt{\mylinespread} command in \texttt{main.tex}. By default, it is
%doc% deactivated (set to 100~percent). Modify only with caution since it influences the
%doc% page layout and could lead to ugly looking documents.
\linespread{\mylinespread}

%doc% 
%doc% \subsection{Optional: Lines above and below the chapter head}
%doc% 
%doc% This is not quite something typographic but rather a matter of taste.
%doc% \myacro{KOMA} Script offers \href{http://www.komascript.de/node/24}{a method to
%doc% add lines above and below chapter head} which is disabled by
%doc% default. If you want to enable this feature, remove corresponding
%doc% comment characters from the settings.
%doc% 
%% Source: http://www.komascript.de/node/24
%disabled% %% 1st get a new command
%disabled% \newcommand*{\ORIGchapterheadstartvskip}{}%
%disabled% %% 2nd save the original definition to the new command
%disabled% \let\ORIGchapterheadstartvskip=\chapterheadstartvskip
%disabled% %% 3rd redefine the command using the saved original command
%disabled% \renewcommand*{\chapterheadstartvskip}{%
%disabled%   \ORIGchapterheadstartvskip
%disabled%   {%
%disabled%     \setlength{\parskip}{0pt}%
%disabled%     \noindent\color{DispositionColor}\rule[.3\baselineskip]{\linewidth}{1pt}\par
%disabled%   }%
%disabled% }
%disabled% %% see above
%disabled% \newcommand*{\ORIGchapterheadendvskip}{}%
%disabled% \let\ORIGchapterheadendvskip=\chapterheadendvskip
%disabled% \renewcommand*{\chapterheadendvskip}{%
%disabled%   {%
%disabled%     \setlength{\parskip}{0pt}%
%disabled%     \noindent\color{DispositionColor}\rule[.3\baselineskip]{\linewidth}{1pt}\par
%disabled%   }%
%disabled%   \ORIGchapterheadendvskip
%disabled% }

%doc% 
%doc% \subsection{Optional: Chapter thumbs}
%doc% 
%doc% This is not quite something typographic but rather a matter of taste.
%doc% \myacro{KOMA} Script offers \href{http://www.komascript.de/chapterthumbs-example}{a method to
%doc% add chapter thumbs} (in combination with the package \texttt{scrpage2}) which is disabled by
%doc% default. If you want to enable this feature, remove corresponding
%doc% comment characters from the settings.
%doc% 
%disabled% \makeatletter
%disabled% % Safty first
%disabled% \@ifundefined{chapter}{\let\chapter\undefined
%disabled%   \chapter must be defined to use chapter thumbs!}{%
%disabled%  
%disabled% % Two new commands for the width and height of the boxes with the
%disabled% % chapter number at the thumbs (use of commands instead of lengths
%disabled% % for sparing registers)
%disabled% \newcommand*{\chapterthumbwidth}{2em}
%disabled% \newcommand*{\chapterthumbheight}{1em}
%disabled%  
%disabled% % Two new commands for the colors of the box background and the
%disabled% % chapter numbers of the thumbs
%disabled% \newcommand*{\chapterthumbboxcolor}{black}
%disabled% \newcommand*{\chapterthumbtextcolor}{white}
%disabled%  
%disabled% % New command to set a chapter thumb. I'm using a group at this
%disabled% % command, because I'm changing the temporary dimension \@tempdima
%disabled% \newcommand*{\putchapterthumb}{%
%disabled%   \begingroup
%disabled%     \Large
%disabled%     % calculate the horizontal possition of the right paper border
%disabled%     % (I ignore \hoffset, because I interprete \hoffset moves the page
%disabled%     % at the paper e.g. if you are using cropmarks)
%disabled%     \setlength{\@tempdima}{\@oddheadshift}% (internal from scrpage2)
%disabled%     \setlength{\@tempdima}{-\@tempdima}%
%disabled%     \addtolength{\@tempdima}{\paperwidth}%
%disabled%     \addtolength{\@tempdima}{-\oddsidemargin}%
%disabled%     \addtolength{\@tempdima}{-1in}%
%disabled%     % putting the thumbs should not change the horizontal
%disabled%     % possition
%disabled%     \rlap{%
%disabled%       % move to the calculated horizontal possition
%disabled%       \hspace*{\@tempdima}%
%disabled%       % putting the thumbs should not change the vertical
%disabled%       % possition
%disabled%       \vbox to 0pt{%
%disabled%         % calculate the vertical possition of the thumbs (I ignore
%disabled%         % \voffset for the same reasons told above)
%disabled%         \setlength{\@tempdima}{\chapterthumbwidth}%
%disabled%         \multiply\@tempdima by\value{chapter}%
%disabled%         \addtolength{\@tempdima}{-\chapterthumbwidth}%
%disabled%         \addtolength{\@tempdima}{-\baselineskip}%
%disabled%         % move to the calculated vertical possition
%disabled%         \vspace*{\@tempdima}%
%disabled%         % put the thumbs left so the current horizontal possition
%disabled%         \llap{%
%disabled%           % and rotate them
%disabled%           \rotatebox{90}{\colorbox{\chapterthumbboxcolor}{%
%disabled%               \parbox[c][\chapterthumbheight][c]{\chapterthumbwidth}{%
%disabled%                 \centering
%disabled%                 \textcolor{\chapterthumbtextcolor}{%
%disabled%                   \strut\thechapter}\\
%disabled%               }%
%disabled%             }%
%disabled%           }%
%disabled%         }%
%disabled%         % avoid overfull \vbox messages
%disabled%         \vss
%disabled%       }%
%disabled%     }%
%disabled%   \endgroup
%disabled% }
%disabled%  
%disabled% % New command, which works like \lohead but also puts the thumbs (you
%disabled% % cannot use \ihead with this definition but you may change this, if
%disabled% % you use more internal scrpage2 commands)
%disabled% \newcommand*{\loheadwithchapterthumbs}[2][]{%
%disabled%   \lohead[\putchapterthumb#1]{\putchapterthumb#2}%
%disabled% }
%disabled%  
%disabled% % initial use
%disabled% \loheadwithchapterthumbs{}
%disabled% \pagestyle{scrheadings}
%disabled%  
%disabled% }
%disabled% \makeatother

%%%% END
%%% Local Variables:
%%% mode: latex
%%% mode: auto-fill
%%% mode: flyspell
%%% eval: (ispell-change-dictionary "en_US")
%%% TeX-master: "../main"
%%% End:
%% vim:foldmethod=expr
%% vim:fde=getline(v\:lnum)=~'^%%%%'?0\:getline(v\:lnum)=~'^%doc.*\ .\\%(sub\\)\\?section{.\\+'?'>1'\:'1':
# command
%doc% in \texttt{main.tex}.  For standard usage it is recommended to stay with the
%doc% default settings.
%doc% 
%doc% 
%% ========================================================================

%doc%
%doc% Some basic microtypographic settings are provided by the
%doc% \texttt{microtype}
%doc% package\footnote{\url{http://ctan.org/pkg/microtype}}. This template
%doc% uses the rather conservative package parameters: \texttt{protrusion=true,factor=900}.
\usepackage[protrusion=true,factor=900]{microtype}

%doc%
%doc% \subsection{French spacing}
%doc%
%doc% \paragraph{Why?} see~\textcite[p.\,28, p.\,30]{Bringhurst1993}: `2.1.4 Use a single word space between sentences.'
%doc%
%doc% \paragraph{How?} see~\textcite[p.\,185]{Eijkhout2008}:\\
%doc% \verb#\frenchspacing  %% Macro to switch off extra space after punctuation.# \\
\frenchspacing  %% Macro to switch off extra space after punctuation.
%doc%
%doc% Note: This setting might be default for \myacro{KOMA} script.
%doc%


%doc%
%doc% \subsection{Font}
%doc% 
%doc% This template is using the Palatino font (package \texttt{mathpazo}) which results
%doc% in a legible document and matching mathematical fonts for printout.
%doc% 
%doc% It is highly recommended that you either stick to the Palatino font or use the
%doc% \LaTeX{} default fonts (by removing the package \texttt{mathpazo}).
%doc% 
%doc% Chosing different fonts is not
%doc% an easy task. Please leave this to people with good knowledge on this subject.
%doc% 
%doc% One valid reason to change the default fonts is when your document is mainly
%doc% read on a computer screen. In this case it is recommended to switch to a font
%doc% \textsf{which is sans-serif like this}. This template contains several alternative
%doc% font packages which can be activated in this file.
%doc% 

% for changing the default font, please go to the next subsection!

%doc%
%doc% \subsection{Text figures}
%doc% 
%doc% \ldots also called old style numbers such as 0123456789. 
%doc% (German: \enquote{Mediäval\-ziffern\footnote{\url{https://secure.wikimedia.org/wikibooks/de/wiki/LaTeX-W\%C3\%B6rterbuch:\_Medi\%C3\%A4valziffern}}})
%doc% \paragraph{Why?} see~\textcite[p.\,44f]{Bringhurst1993}: 
%doc% \begin{quote}
%doc% `3.2.1 If the font includes both text figures and titling figures, use
%doc%  titling figures only with full caps, and text figures in all other
%doc%  circumstances.'
%doc% \end{quote}
%doc% 
%doc% \paragraph{How?} 
%doc% Quoted from Wikibooks\footnote{\url{https://secure.wikimedia.org/wikibooks/en/wiki/LaTeX/Formatting\#Text\_figures\_.28.22old\_style.22\_numerals.29}}:
%doc% \begin{quote}
%doc% Some fonts do not have text figures built in; the textcomp package attempts to
%doc% remedy this by effectively generating text figures from the currently-selected
%doc% font. Put \verb#\usepackage{textcomp}# in your preamble. textcomp also allows you to
%doc% use decimal points, properly formatted dollar signs, etc. within
%doc% \verb#\oldstylenums{}#.
%doc% \end{quote}
%doc% \ldots but proposed \LaTeX{} method does not work out well. Instead use:\\
%doc% \verb#\usepackage{hfoldsty}#  (enables text figures using additional font) or \\
%doc% \verb#\usepackage[sc,osf]{mathpazo}# (switches to Palatino font with small caps and old style figures enabled).
%doc%
%\usepackage{hfoldsty}  %% enables text figures using additional font
%% ... OR use ...
%\usepackage[sc,osf]{mathpazo} %% switches to Palatino with small caps and old style figures

%% Font selection from:
%%     http://www.matthiaspospiech.de/latex/vorlagen/allgemein/preambel/fonts/
%% use following lines *instead* of the mathpazo package above:
%% ===== Serif =========================================================
%% for Computer Modern (LaTeX default font), simply remove the mathpazo above
%\usepackage{charter}\linespread{1.05} %% Charter
%\usepackage{bookman}                  %% Bookman (laedt Avant Garde !!)
%\usepackage{newcent}                  %% New Century Schoolbook (laedt Avant Garde !!)
%% ===== Sans Serif ====================================================
%\renewcommand{\familydefault}{\sfdefault}  %% this one in *combination* with the default mathpazo package
%\usepackage{cmbright}                  %% CM-Bright (eigntlich eine Familie)
%\usepackage{tpslifonts}                %% tpslifonts % Font for Slides


%doc% 
%doc% \subsection{\texttt{myacro} --- Abbrevations using \textsc{small caps}}\myinteresting
%doc% \label{sec:myacro}
%doc% 
%doc% \paragraph{Why?} see~\textcite[p.\,45f]{Bringhurst1993}: `3.2.2 For abbrevations and
%doc% acronyms in the midst of normal text, use spaced small caps.'
%doc% 
%doc% \paragraph{How?} Using the predefined macro \verb#\myacro{}# for things like
%doc% \myacro{UNO} or \myacro{UNESCO} using \verb#\myacro{UNO}# or \verb#\myacro{UNESCO}#.
%doc% 
\DeclareRobustCommand{\myacro}[1]{\textsc{\lowercase{#1}}} %%  abbrevations using small caps


%doc% 
%doc% \subsection{Colorized headings and links}
%doc% 
%doc% This document template is able to generate an output that uses colorized
%doc% headings, captions, page numbers, and links. The color named `DispositionColor'
%doc% used in this document is defined near the definition of package \texttt{color}
%doc% in the preamble (see section~\ref{subsec:miscpackages}). The changes required
%doc% for headings, page numbers, and captions are defined here.
%doc% 
%doc% Settings for colored links are handled by the definitions of the
%doc% \texttt{hyperref} package (see section~\ref{sec:pdf}).
%doc% 
\setheadsepline{.4pt}[\color{DispositionColor}]
\renewcommand{\headfont}{\normalfont\sffamily\color{DispositionColor}}
\renewcommand{\pnumfont}{\normalfont\sffamily\color{DispositionColor}}
\addtokomafont{disposition}{\color{DispositionColor}}
\addtokomafont{caption}{\color{DispositionColor}\footnotesize}
\addtokomafont{captionlabel}{\color{DispositionColor}}

%doc% 
%doc% \subsection{No figures or tables below footnotes}
%doc% 
%doc% \LaTeX{} places floating environments below footnotes if \texttt{b}
%doc% (bottom) is used as (default) placement algorithm. This is certainly
%doc% not appealing for most people and is deactivated in this template by
%doc% using the package \texttt{footmisc} with its option \texttt{bottom}.
%doc% 
%% see also: http://www.komascript.de/node/858 (German description)
\usepackage[bottom]{footmisc}

%doc% 
%doc% \subsection{Spacings of list environments}
%doc% 
%doc% By default, \LaTeX{} is using vertical spaces between items of enumerate, 
%doc% itemize and description environments. This is fine for multi-line items.
%doc% Many times, the user does just write single-line items where the larger
%doc% vertical space is inappropriate. The \href{http://ctan.org/pkg/enumitem}{enumitem}
%doc% package provides replacements for the pre-defined list environments and
%doc% offers many options to modify their appearances.
%doc% This template is using the package option for \texttt{noitemsep} which
%doc% mimimizes the vertical space between list items.
%doc% 
\usepackage{enumitem}
\setlist{noitemsep}   %% kills the space between items

%doc% 
%doc% \subsection{\texttt{csquotes} --- Correct quotation marks}\myinteresting
%doc% \label{sub:csquotes}
%doc% 
%doc% \emph{Never} use quotation marks found on your keyboard.
%doc% They end up in strange characters or false looking quotation marks.
%doc% 
%doc% In \LaTeX{} you are able to use typographically correct quotation marks. The package 
%doc% \href{http://www.ctan.org/pkg/csquotes}{\texttt{csquotes}} offers you with 
%doc% \verb#\enquote{foobar}# a command to get correct quotation marks around \enquote{foobar}.
%doc% Please do check the package options in order to modify
%doc% its settings according to the language used\footnote{most of the time in 
%doc% combination with the language set in the options of the \texttt{babel} package}.
%doc% 
%doc% \href{http://www.ctan.org/pkg/csquotes}{\texttt{csquotes}} is also recommended 
%doc% by \texttt{biblatex} (see Section~\ref{sec:references}). 
\usepackage[babel=true,strict=true,english=american,german=guillemets]{csquotes}

%doc% 
%doc% \subsection{Line spread}
%doc% 
%doc% If you have to enlarge the distance between two lines of text, you can
%doc% increase it using the \texttt{\mylinespread} command in \texttt{main.tex}. By default, it is
%doc% deactivated (set to 100~percent). Modify only with caution since it influences the
%doc% page layout and could lead to ugly looking documents.
\linespread{\mylinespread}

%doc% 
%doc% \subsection{Optional: Lines above and below the chapter head}
%doc% 
%doc% This is not quite something typographic but rather a matter of taste.
%doc% \myacro{KOMA} Script offers \href{http://www.komascript.de/node/24}{a method to
%doc% add lines above and below chapter head} which is disabled by
%doc% default. If you want to enable this feature, remove corresponding
%doc% comment characters from the settings.
%doc% 
%% Source: http://www.komascript.de/node/24
%disabled% %% 1st get a new command
%disabled% \newcommand*{\ORIGchapterheadstartvskip}{}%
%disabled% %% 2nd save the original definition to the new command
%disabled% \let\ORIGchapterheadstartvskip=\chapterheadstartvskip
%disabled% %% 3rd redefine the command using the saved original command
%disabled% \renewcommand*{\chapterheadstartvskip}{%
%disabled%   \ORIGchapterheadstartvskip
%disabled%   {%
%disabled%     \setlength{\parskip}{0pt}%
%disabled%     \noindent\color{DispositionColor}\rule[.3\baselineskip]{\linewidth}{1pt}\par
%disabled%   }%
%disabled% }
%disabled% %% see above
%disabled% \newcommand*{\ORIGchapterheadendvskip}{}%
%disabled% \let\ORIGchapterheadendvskip=\chapterheadendvskip
%disabled% \renewcommand*{\chapterheadendvskip}{%
%disabled%   {%
%disabled%     \setlength{\parskip}{0pt}%
%disabled%     \noindent\color{DispositionColor}\rule[.3\baselineskip]{\linewidth}{1pt}\par
%disabled%   }%
%disabled%   \ORIGchapterheadendvskip
%disabled% }

%doc% 
%doc% \subsection{Optional: Chapter thumbs}
%doc% 
%doc% This is not quite something typographic but rather a matter of taste.
%doc% \myacro{KOMA} Script offers \href{http://www.komascript.de/chapterthumbs-example}{a method to
%doc% add chapter thumbs} (in combination with the package \texttt{scrpage2}) which is disabled by
%doc% default. If you want to enable this feature, remove corresponding
%doc% comment characters from the settings.
%doc% 
%disabled% \makeatletter
%disabled% % Safty first
%disabled% \@ifundefined{chapter}{\let\chapter\undefined
%disabled%   \chapter must be defined to use chapter thumbs!}{%
%disabled%  
%disabled% % Two new commands for the width and height of the boxes with the
%disabled% % chapter number at the thumbs (use of commands instead of lengths
%disabled% % for sparing registers)
%disabled% \newcommand*{\chapterthumbwidth}{2em}
%disabled% \newcommand*{\chapterthumbheight}{1em}
%disabled%  
%disabled% % Two new commands for the colors of the box background and the
%disabled% % chapter numbers of the thumbs
%disabled% \newcommand*{\chapterthumbboxcolor}{black}
%disabled% \newcommand*{\chapterthumbtextcolor}{white}
%disabled%  
%disabled% % New command to set a chapter thumb. I'm using a group at this
%disabled% % command, because I'm changing the temporary dimension \@tempdima
%disabled% \newcommand*{\putchapterthumb}{%
%disabled%   \begingroup
%disabled%     \Large
%disabled%     % calculate the horizontal possition of the right paper border
%disabled%     % (I ignore \hoffset, because I interprete \hoffset moves the page
%disabled%     % at the paper e.g. if you are using cropmarks)
%disabled%     \setlength{\@tempdima}{\@oddheadshift}% (internal from scrpage2)
%disabled%     \setlength{\@tempdima}{-\@tempdima}%
%disabled%     \addtolength{\@tempdima}{\paperwidth}%
%disabled%     \addtolength{\@tempdima}{-\oddsidemargin}%
%disabled%     \addtolength{\@tempdima}{-1in}%
%disabled%     % putting the thumbs should not change the horizontal
%disabled%     % possition
%disabled%     \rlap{%
%disabled%       % move to the calculated horizontal possition
%disabled%       \hspace*{\@tempdima}%
%disabled%       % putting the thumbs should not change the vertical
%disabled%       % possition
%disabled%       \vbox to 0pt{%
%disabled%         % calculate the vertical possition of the thumbs (I ignore
%disabled%         % \voffset for the same reasons told above)
%disabled%         \setlength{\@tempdima}{\chapterthumbwidth}%
%disabled%         \multiply\@tempdima by\value{chapter}%
%disabled%         \addtolength{\@tempdima}{-\chapterthumbwidth}%
%disabled%         \addtolength{\@tempdima}{-\baselineskip}%
%disabled%         % move to the calculated vertical possition
%disabled%         \vspace*{\@tempdima}%
%disabled%         % put the thumbs left so the current horizontal possition
%disabled%         \llap{%
%disabled%           % and rotate them
%disabled%           \rotatebox{90}{\colorbox{\chapterthumbboxcolor}{%
%disabled%               \parbox[c][\chapterthumbheight][c]{\chapterthumbwidth}{%
%disabled%                 \centering
%disabled%                 \textcolor{\chapterthumbtextcolor}{%
%disabled%                   \strut\thechapter}\\
%disabled%               }%
%disabled%             }%
%disabled%           }%
%disabled%         }%
%disabled%         % avoid overfull \vbox messages
%disabled%         \vss
%disabled%       }%
%disabled%     }%
%disabled%   \endgroup
%disabled% }
%disabled%  
%disabled% % New command, which works like \lohead but also puts the thumbs (you
%disabled% % cannot use \ihead with this definition but you may change this, if
%disabled% % you use more internal scrpage2 commands)
%disabled% \newcommand*{\loheadwithchapterthumbs}[2][]{%
%disabled%   \lohead[\putchapterthumb#1]{\putchapterthumb#2}%
%disabled% }
%disabled%  
%disabled% % initial use
%disabled% \loheadwithchapterthumbs{}
%disabled% \pagestyle{scrheadings}
%disabled%  
%disabled% }
%disabled% \makeatother

%%%% END
%%% Local Variables:
%%% mode: latex
%%% mode: auto-fill
%%% mode: flyspell
%%% eval: (ispell-change-dictionary "en_US")
%%% TeX-master: "../main"
%%% End:
%% vim:foldmethod=expr
%% vim:fde=getline(v\:lnum)=~'^%%%%'?0\:getline(v\:lnum)=~'^%doc.*\ .\\%(sub\\)\\?section{.\\+'?'>1'\:'1':

%doc% %%%% Time-stamp: <2014-03-23 13:40:59 vk>
%%%% === Disclaimer: =======================================================
%% created by
%%
%%      Karl Voit
%%
%% using GNU/Linux, GNU Emacs & LaTeX 2e
%%

%doc%
%doc% \section{\texttt{pdf\_settings.tex} --- Settings related to PDF output}
%doc% \label{sec:pdf}
%doc% 
%doc% The file \verb#template/pdf_settings.tex# basically contains the definitions for
%doc% the \href{http://tug.org/applications/hyperref/}{\texttt{hyperref} package}
%doc% including the
%doc% \href{http://www.ctan.org/tex-archive/macros/latex/required/graphics/}{\texttt{graphicx}
%doc% package}. Since these settings should be the last things of any \LaTeX{}
%doc% preamble, they got their own \TeX{} file which is included in \texttt{main.tex}.
%doc% 
%doc% \paragraph{What should I do with this file?} The settings in this file are
%doc% important for \myacro{PDF} output and including graphics. Do not exclude the
%doc% related \texttt{input} command in \texttt{main.tex}. But you might want to
%doc% modify some settings after you read the
%doc% \href{http://tug.org/applications/hyperref/}{documentation of the \texttt{hyperref} package}.
%doc% 


%% Fix positioning of images in PDF viewers. (disabled by
%% default; see https://github.com/novoid/LaTeX-KOMA-template/issues/4
%% for more information) 
%% I do not have time to read about possible side-effect of this
%% package for now.
% \usepackage[hypcap]{caption}

%% Declarations of hyperref should be the last definitions of the preamble:
%% FIXXME: black-and-white-version for printing!

\pdfcompresslevel=9

\usepackage[%
unicode=true, % loads with unicode support
%a4paper=true, %
pdftex, %
backref, %
pagebackref=false, % creates backward references too
bookmarks=false, %
bookmarksopen=false, % when starting with AcrobatReader, the Bookmarkcolumn is opened
pdfpagemode=UseNone,% UseNone, UseOutlines, UseThumbs, FullScreen
plainpages=false, % correct, if pdflatex complains: ``destination with same identifier already exists''
%% colors: https://secure.wikimedia.org/wikibooks/en/wiki/LaTeX/Colors
urlcolor=DispositionColor, %%
linkcolor=DispositionColor, %%
%pagecolor=DispositionColor, %%
citecolor=DispositionColor, %%
anchorcolor=DispositionColor, %%
colorlinks=\mycolorlinks, % turn on/off colored links (on: better for
                          % on-screen reading; off: better for printout versions)
]{hyperref}

%% all strings need to be loaded after hyperref was loaded with unicode support
%% if not the field is garbled in the output for characters like ČŽĆŠĐ
\hypersetup{
pdftitle={\mytitle}, %
pdfauthor={\myauthor}, %
pdfsubject={\mysubject}, %
pdfcreator={Accomplished with: pdfLaTeX, biber, and hyperref-package. No animals, MS-EULA or BSA-rules were harmed.},
pdfproducer={\myauthor},
pdfkeywords={\mykeywords}
}

%\DeclareGraphicsExtensions{.pdf}

%%%% END
%%% Local Variables:
%%% TeX-master: "../main"
%%% mode: latex
%%% mode: auto-fill
%%% mode: flyspell
%%% eval: (ispell-change-dictionary "en_US")
%%% End:
%% vim:foldmethod=expr
%% vim:fde=getline(v\:lnum)=~'^%%%%'?0\:getline(v\:lnum)=~'^%doc.*\ .\\%(sub\\)\\?section{.\\+'?'>1'\:'1':

%doc%
%doc% \begin{document}
%doc% %% title page %%
%doc% \title{\mytitle}\subtitle{\mysubject}
%doc% \author{\myauthor}
%doc% \date{\today}
%doc%
%doc% \maketitle\newpage
%doc%
%doc% \tableofcontents\newpage
%doc% %%---------------------------------------%%

%doc%
%doc% \section{How to use this \LaTeX{} document template}
%doc%
%doc% This \LaTeX{} document template from
%doc% \myLaT\footnote{\url{http://LaTeX.TUGraz.at}} is based on \myacro{KOMA}
%doc% script\footnote{\url{http://komascript.de/}}. You don't need any
%doc% special \myacro{KOMA} knowledge (but it woun't hurt either). It provides an easy to use and
%doc% easy to modify template. All settings are documented and many references to
%doc% additional information sources are given.
%doc%

%doc% In general, there should not be any reason to modify a file in
%doc% the \texttt{template} folder. \emph{All important settings are
%doc% accessible in the main folder, mostly in the \texttt{main.tex}
%doc% file.} This way, it is easy to get what you need and you can update
%doc% the template independent of the content of the document.
%doc%
%doc% \newcommand{\myimportant}{%% mark important chapters
%doc%   \marginpar{\vspace{-1em}\rightpointleft}
%doc% }
%doc% \newcommand{\myinteresting}{\marginpar{\vspace{-2em}\PencilLeftDown}}

%doc%
%doc% The \emph{absolute minimum you should read} is listed below and
%doc% marked with the hand symbol:\myimportant
%doc% \begin{itemize}
%doc% \item Section~\ref{sec:modifytemplate}: basic configuration of this template.
%doc% \item Section~\ref{sec:howtocompile}: how to generate the \myacro{PDF} file
%doc% \item Section~\ref{sec:references}: using biblatex (instead of bibtex)
%doc% \end{itemize}
%doc%
%doc% In order to get a perfect resulting document and to get an
%doc% exciting experience with this template, you should definitely consider reading
%doc% following sections which are also marked with the pencil symbol:\myinteresting
%doc% \begin{itemize}
%doc% \item Section~\ref{sec:extending-template}: extend the template with
%doc%   your own usepackages, newcommands, and so forth
%doc% \item Section~\ref{sec:mycommands}: pre-defined commands to make your life easier (e.g., including graphics)
%doc% \item Section~\ref{sec:myacro}: how to do acronyms (like \myacro{ACME}) beautifully
%doc% \item Section~\ref{sub:csquotes}: how to \enquote{quote} text and use parentheses correctly
%doc% \end{itemize}
%doc%
%doc% The other sections describe all other settings for the sake of completeness. This is
%doc% interesting for learning more about \LaTeX{} and modifying this template to a higher level of detail.

%doc%
%doc% \newpage
%doc% \subsection{Six Steps to Customize Your Document}\myimportant
%doc% \label{sec:modifytemplate}
%doc%
%doc% This template is optimized to get to the first draft of your thesis
%doc% very quickly. Follow these instructions and you get most of your
%doc% customizing done in a few minutes:
%doc%
%doc% \newcommand{\myfile}[1]{\texttt{\href{file:#1}{#1}}}
%doc%
%doc% \begin{enumerate}
%doc% \item Modify settings in \texttt{main.tex} to meet your requirements:
%doc%   \begin{itemize}
%doc%   \item Basic settings
%doc%     \begin{itemize}
%doc%     \item Paper size, languages, font size, citation style,
%doc%           title page, and so forth
%doc%     \end{itemize}
%doc%   \item Document metadata
%doc%     \begin{itemize}
%doc%     \item Preferences like \verb+myauthor+, \verb+mytitle+, and so forth
%doc%     \end{itemize}
%doc%   \end{itemize}
%doc% \item Replace \myfile{figures/institution.pdf} with the logo of
%doc% your institution in either \myacro{PDF} or \myacro{PNG}
%doc% format.\footnote{Avoid \myacro{JPEG} format for
%doc% computer-generated (pixcel-oriented) graphics like logos or
%doc% screenshots in general. The \myacro{JEPG} format is for
%doc% photographs \emph{only}.}
%doc% \item Further down in \myfile{main.tex}:
%doc%   \begin{itemize}
%doc%   \item Create your desired structure for the chapters
%doc%         (\verb+%%%%%% Introduction %%%%%%%%%%%%%%%%%%%%%%%%%%%%%%%%%%%%%%%%%%%%%%%%%%%%%%%%%{{{
\chapter{Introduction}\label{sec:intro}

With the latest releases of Microarchitectural attacks like
Meltdown~\cite{meltdown} and Spectre~\cite{spectre}, the topic of flaws in
hardware implementations became known to the general public. Many media outlets
reported on these issues of modern CPUs. Where it was mostly vendors of
x86-architecture CPUs like Intel or AMD, also ARM, and with it, most mobile
devices are affected by such flaws.

Issues like these show that these vendors have set performance above security by
neglecting quality management and testing. The demand for better releases of
hardware is rising all the time, and not only vendors of CPUs are affected by
this demand. Another field of silicon chip design ran into a similar problem in
the past, namely DRAM chip vendors.  

In 2015, Kim~\etal~\cite{rowhammergeneral} released their paper ``RowHammer'',
showing how specially crafted memory access routines can cause bits in DRAM
chips to flip, without accessing them directly. This work showed how the demand
for higher memory density caused faults where interfering voltages and leaking
currents to influence other memory storage cells. While at first, this was just
seen as a stability issue, Google\textquotesingle s Project Zero showed how
Rowhammer can be used for privilege escalation and sandbox
escapes~\cite{projectzerorow}. With reports like this, the interest in
researching the field of Rowhammer increased. Gruss~\etal~\cite{rowhammerjs}
showed that it is not only possible to target systems by executing native code,
but also that Rowhammer can be triggered by using JavaScript. Van der
Veen~\etal\cite{drammer} published their work named ``Drammer'', where they show
how not only desktop computers are affected by the Rowhammer bug, but also
mobile devices. Earlier this year, Gruss~\etal~\cite{nethammer}, released a way
to trigger bitflips by only sending specially-crafted network requests.
Publications like these show that Rowhammer is an active research topic, where
still new findings come up.

Our work builds on work released by Gruss~\etal~\cite{flipinthewall}, where they
showed that application code can directly be attacked with Rowhammer. They show
that a bitflip applied to \texttt{sudo} can result in a bypass of the password
check. They show some bitflips causing such a bypass. They look at the
disassembly of the authentication check code and find opcodes which when changed
would result in a different outcome. With our work, we want to automate, and
therefore simplify, this process. We want to find a higher number of bitflips in
a shorter time. In addition to that, we want to provide a toolset, which allows
us to apply similar searches to other applications. We want to run lots of tests
in parallel and verify the outcomes. Therefore we want to make use of modern
testing techniques.

Testing and debugging were always a significant part of software technology, and
with rising sizes of projects and an increasing number of old code bases, it is
more vital than ever. Not only developers are putting much work into these
topics but also researchers releasing new ways of testing regularly. With modern
approaches for testing like fuzzying, bug searching in unknown code got more
successful. Also, the field of proving software\textquotesingle s correctness
got much attention. With symbolic execution techniques, the possibility to prove
each software state on its own got more practical. The release of the open
source symbolic execution framework \texttt{angr}~\cite{angrpaper} made it
possible for a wide range of users to apply symbolic execution to programs. This
tool mostly gets used in testing, in combination with fuzzing, but also security
researchers use \texttt{angr} to find exploitable code segments and execution
paths.

Understanding what programs do, and how they are executed by the CPU, gets
harder with every improvement and change in hardware design. Instrumentation is
a technique to inject code to programs providing the possibility to collect
runtime information. With tools like Intel Pin~\cite{pintool} it is possible to
check changes to processor registers, log accessed memory and performance
measuring at machine code level.

\section{Goals and Motivation for the Thesis}

As we know from previous work done by Gruss~\etal~\cite{flipinthewall}, there
are bitflips in the ELF files loaded by the \texttt{sudo} program which allow
privilege escalation by providing a password check bypass. They only looked at
the binary section providing the permission check. However, they could not claim
to find all flips, and their approach is very time-consuming. We want to
simplify the search by automatic testing of flips. Also, we also want to make it
easier for future applications to be tested for possible bitflip outcomes by
providing a test framework.

Common Unix-based operating systems use package management to roll-out
applications to users. Every instance of the operating system then uses the same
binary to execute. With this in mind, a bitflip found in the \texttt{sudo}
application distributed with Debian, can be used to attack all instances of that
installation. An attacker, therefore, can use the test framework to find
bitflips in widely distributed binaries.

With our work, we want to provide an easy-to-apply framework to search for
bitflips providing a pre-defined outcome. To show how this framework works we
apply it to real-world applications and compare our results to the ones reported
by Gruss~\etal~\cite{flipinthewall}. We want to show how likely such bitflips
are in applications.

\section{Contributions of this Work}

Our contribution to the field of microarchitectural attacks and Rowhammer is
providing a practical analysis of real-world applications and how bitflips can
affect them. We present a framework which can be used to find bitflips changing
a program\textquotesingle s behaviour to a pre-defined outcome. The structure of
the framework is designed to be extensible and adaptable for multiple purposes.

We apply our tool to real-world applications to show the impact bitflips could
have on users of personal computers. On one hand, we show how privilege
escalation bitflips can be found in the \texttt{sudo} program. We show bits,
which when flipped, allow us to skip the password check. On the other hand, we
also analyse the popular \texttt{nginx} web server. For this application, we
show bitflips which permit an attacker to bypass HTTP authentication measures.
We present results for these two applications and if there exist bits, which
when flipped, allow us to achieve our set outcome. Besides analysing the bits
inside the program\textquotesingle s executable, we also examine any dynamically
loaded library it uses. By that, we also cover possibilities where external
functions could change the application\textquotesingle s outcome.

In addition to that, we look at possible cryptographic vulnerabilities
introduced by bitflips. As a basis, we took the work by
Böck~\etal~\cite{gcmnonceattack}, who showed how web servers were misusing
nonces when using AES-GCM. We build on their approach to bypass the fixes
applied by server software to re-introduce this nonce misusage via bitflips.
Here we look at the current implementation of AES-GCM in the TLS library
OpenSSL, used by most web servers. We show that nonce misusage can be
reintroduced by bitflips and give a probability for them.

\section{Outline of this Work}

This thesis is structured as follows: In section~\ref{sec:general}, we describe
general terms and technologies our work is built on or makes use of. We discuss
other microarchitectural attacks and give an overview of the functionality of
programs which our work targets. In section~\ref{sec:elfattack}, we discuss our
work regarding the automatic bitflip search. We show the tested programs, what
additions had to be made for testing and present the found bitflips. In
section~\ref{sec:dynattack}, we discuss our work regarding Rowhammer attacks
targeting dynamic memory. We thereby show how the OpenSSL implementation of
AES-GCM can be attacked by flipping bits. In section~\ref{sec:countermeasure},
we discuss countermeasures which could be applied to improve system security. We
discuss countermeasures against microarchitectural attacks in general, and
discuss on what could be done to reduce the impact of our testing. In
section~\ref{sec:futurework}, we show possible future works, and an overview of
possible directions the research in the field of microarchitectural attacks
could take. In section~\ref{sec:conclusion}, we close our thesis with a summary
and give a conclusion of our work.

\section{Merge to 1.0}

We present this work separated into different chapters. We start with a general
overview of topics in this field. Beginning with describing how programs get
executed on modern computers and how operating systems handle applications. For
this, we describe the general design of executables on Unix-like systems, as we
discuss the executable and linkable format (ELF) in more detail.

As ELF files hold machine code executed by CPUs, and our work relies heavily on
how machine code is built, we discuss the design of instructions in modern
CPU architectures.  We then go on and look at different testing techniques in
software development, where we compare and describe several options and check
their advantages and disadvantages. In particular, we work out details about
fuzzing, symbolic execution and instrumentation. For our work, we try to change
the behaviour of programs by modifying their execution path, most of the time
this behaviour is changed to gain improved privileges. Therefore, we discuss the
permission model used by common Unix-based operating systems.

We also look at details of how permission switches work on these systems,
especially of how the \texttt{setuid} property of executables works. For
permission separation and testing purposes we also take a look at the
\texttt{chroot} environment provided by most Unix-like operating systems. We
do not only target behaviour changes to gain a higher privilege but also target
changes to bypass permission checks.

Moreover, besides local attacks, we also look at remote possibilities. We
describe the networking and security principles used by most computers and
servers. We look at common web servers, TLS libraries and how they provide
security for users. As our work targets cryptographic implementations, we take
a detailed look at the advanced encryption standard (AES) and a variant of it
using Galois/counter mode (AES-GCM).

Our work relies heavily on Rowhammer, which is a software-based
microarchitectural attack. We take a look at these attacks in general and give
an overview of state of the art attacks using similar techniques. As timing
plays a vital role in exploiting side-channels in this area, we describe precise
timing measurement methods. We look at recent cache attacks and the impact
those have on modern systems. We close our background overview with a detailed
description of the Rowhammer bug.

We continue with describing our distribution. This is split into two parts, one
being our testing of bitflips in ELF files and the other being attacks against
cryptographic functions during runtime. We start with the ELF analysis and
describe what impact a single bitflip can have to the execution paths of a
program. We go on with outlining possibilities to find bitflips which would
change the behaviour in a manner so that it benefits an attacker. We describe
the design of our automated bitflip-search framework and how we applied it to
real-world applications. We resume by showing the results of our tests for
the applications of \texttt{sudo} and \texttt{nginx}. We also mention how we
would apply these bitflips to a system by using the Rowhammer bug.

In the second part of our contribution we describe the influence of bitflips on
cryptographic implementations, we discuss the problems of nonce misuse and how
this problem could be introduced to AES-GCM implemented in OpenSSL with
bitflips. We give numbers for a likelihood of such a nonce-misuse introduced by
a single bit flip. We show how we tested this issue in a practical setup with a
simple web server using the OpenSSL library.

We go on with looking at countermeasures for issues in the field of
microarchitectural attacks. We look at possible ways to prevent cache attacks
and Rowhammer. We also line out ways of reducing the impact of our tests on
real-world setups. We close this section by describing how general system
security could be improved.

As microarchitectural attacks are an on-going and growing research field, we
also look at possible future works. We describe a possible future of these
attacks in general, how open source architectures could help to prevent some of
them. We also talk about a possible attribution of machine learning to help to
find new attack vectors. We also find possibilities to improve the impact of
Rowhammer, by looking at new attacks coming by applying it to further
implementations of cryptographic algorithms. In the end, we talk about how our
framework could be helpful for future work and how it can be used in various
testing and research environments with just small code changes.

We close our thesis by giving a conclusion describing our results, provide a
summary of the outcome of our work and point out again how vital secure hardware
is as a basis for secure systems. Therefore we recommend hardware vendors to
improve their quality checks and by this improve general security for users.
%}}}

%% vim:foldmethod=expr
%% vim:fde=getline(v\:lnum)=~'^%%%%\ .\\+'?'>1'\:'='
%%% Local Variables:
%%% mode: latex
%%% mode: auto-fill
%%% mode: flyspell
%%% eval: (ispell-change-dictionary "en_US")
%%% TeX-master: "main"
%%% End:
+, \verb+\include{evaluation}+, \ldots)
%doc%   \end{itemize}
%doc% \item Create the \TeX{} files and fill your content into these files you defined in the previous step.
%doc% \item Optionally: Modify \myfile{colophon.tex} to meet your situation.
%doc%   \begin{itemize}
%doc%   \item Please spend a couple of minutes and think about putting your work
%doc%         under an open license\footnote{\url{https://creativecommons.org/licenses/}}
%doc%         in order to follow the spirit of Open Science\footnote{\url{https://en.wikipedia.org/wiki/Open_science}}.
%doc%   \end{itemize}
%doc% \item In case you are using \myacro{GNU} make\footnote{If you
%doc%       don't know, what \myacro{GNU} make is, you are not using it (yet).}:
%doc%       Put your desired \myacro{PDF} file name in the second line of file
%doc%    \myfile{Makefile}
%doc%    \begin{itemize}
%doc%    \item replace \enquote{Projectname} with your filename
%doc%    \item do not use any file extension like \texttt{.tex} or \texttt{.pdf}
%doc%    \end{itemize}
%doc% \end{enumerate}
%doc%
%doc%

%doc%
%doc% \subsection{License}\myimportant
%doc% \label{sec:license}
%doc%
%doc% This template is licensed under a Creative Commons Attribution-ShareAlike 3.0 Unported (CC BY-SA 3.0)
%doc%         license\footnote{\url{https://creativecommons.org/licenses/by-sa/3.0/}}:
%doc%     \begin{itemize}
%doc%     \item You can share (to copy, distribute and transmit) this template.
%doc%     \item You can remix (adapt) this template.
%doc%     \item You can make commercial use of the template.
%doc%     \item In case you modify this template and share the derived
%doc%           template: You must attribute the template such that you do not
%doc%           remove (co-)authorship of Karl Voit and you must not remove
%doc%           the URL to the original repository on
%doc%           github\footnote{\url{https://github.com/novoid/LaTeX-KOMA-template}}.
%doc%     \item If you alter, transform, or build a new template upon
%doc%           this template, you may distribute the resulting
%doc%           template only under the same or similar license to this one.
%doc%     \item There are \emph{no restrictions} of any kind, however, related to the
%doc%           resulting (PDF) document!
%doc%     \item You may remove the colophon (but it's not recommended).
%doc%     \end{itemize}


%doc%
%doc%
%doc% \subsection{How to compile this document}\myimportant
%doc% \label{sec:howtocompile}
%doc%
%doc% I assume that compiling \LaTeX{} documents within your software
%doc% environment is something you have already learned. This template is
%doc% almost like any other \LaTeX{} document except it uses
%doc% state-of-the-art tools for generating things like the list of
%doc% references using biblatex/biber (see
%doc% Section~\ref{sec:references} for details). Unfortunately, some \LaTeX{} editors
%doc% do not support this much better way of working with bibliography
%doc% references yet. This section describes how to compile this template.
%doc%
%doc% \subsubsection{Compiling Using a \LaTeX{} Editor}
%doc%
%doc% Please do select \myfile{main.tex} as the \enquote{main project file} or make
%doc% sure to compile/run only \myfile{main.tex} (and not \myfile{introduction.tex}
%doc% or other \TeX{} files of this template).
%doc%
%doc% Choose \texttt{biber} for generating the references. Modern LaTeX{}
%doc% environments offer this option. Older tools might not be that up to
%doc% date yet.
%doc%

%doc% \subsubsection{Activating \texttt{biber} in the \LaTeX{} editor TeXworks}
%doc% \label{sec:biberTeXworks}
%doc%
%doc% The \href{https://www.tug.org/texworks/}{TeXworks} editor is a very
%doc% basic (but fine) \LaTeX{} editor to start with. It is included in
%doc% \href{http://miktex.org/}{MiKTeX} and
%doc% \href{http://miktex.org/portable}{MiKTeX portable} and supports
%doc% \href{https://en.wikipedia.org/wiki/Syntax_highlighting}{syntax
%doc%   highlighting} and
%doc% \href{http://itexmac.sourceforge.net/SyncTeX.html}{SyncTeX} to
%doc% synchronize \myacro{PDF} output and \LaTeX{} source code.
%doc%
%doc% Unfortunately, TeXworks shipped with MiKTeX does not support compiling
%doc% using \texttt{biber} (biblatex) out of the box. Here is a solution to
%doc% this issue. Go to TeXworks: \texttt{Edit} $\rightarrow$
%doc% \texttt{Preferences~\ldots} $\rightarrow$ \texttt{Typesetting} $\rightarrow$
%doc% \texttt{Processing tools} and add a new entry (using the plus icon):
%doc%
%doc% \begin{tabbing}
%doc%   Arguments: \= foobar  \kill
%doc%   Name:      \> \verb#pdflatex+biber# \\
%doc%   Program:   \> \emph{find the \texttt{template/pdflatex+biber.bat} file from your disk} \\
%doc%   Arguments: \> \verb+$fullname+ \\
%doc%              \> \verb+$basename+
%doc% \end{tabbing}
%doc%
%doc% Activate the \enquote{View PDF after running} option.
%doc%
%doc% Close the preferences dialog and you will now have an additional
%doc% choice in the drop down list for compiling your document. Choose the
%doc% new entry called \verb#pdflatex+biber# and start a happier life with
%doc% \texttt{biber}.
%doc%
%doc% In case your TeXworks has a German user interface, here the key
%doc% aspects in German as well:
%doc%
%doc% \begin{otherlanguage}{ngerman}
%doc%
%doc%   \texttt{Bearbeiten} $\rightarrow$ \texttt{Einstellungen~\ldots} $\rightarrow$
%doc%   \texttt{Textsatz} $\rightarrow$ \texttt{Verarbeitungsprogramme} $\rightarrow$
%doc%   + \emph{(neues Verarbeitungsprogramm)}:
%doc%
%doc% \begin{tabbing}
%doc%   Befehl/Datei: \= foobar  \kill
%doc%     Name: \> pdflatex+biber \\
%doc%     Befehl/Datei: \> \emph{die \texttt{template/pdflatex+biber.bat} im Laufwerk suchen} \\
%doc%     Argumente: \> \verb+$fullname+ \\
%doc%                \> \verb+$basename+
%doc% \end{tabbing}
%doc%
%doc% \enquote{PDF nach Beendigung anzeigen} aktivieren.
%doc%
%doc% \end{otherlanguage}
%doc%

%doc% \subsubsection{Compiling Using \myacro{GNU} make}
%doc%
%doc% With \myacro{GNU}
%doc% make\footnote{\url{https://secure.wikimedia.org/wikipedia/en/wiki/Make\_\%28software\%29}}
%doc% it is just simple as that: \texttt{make pdf}
%doc%
%doc% Several other targets are available. You can check them out by
%doc% executing: \texttt{make help}
%doc%
%doc% In case you are using TeXLive (instead of MiKTeX as I do), you might
%doc% want to modify the line \texttt{PDFLATEX\_CMD = pdflatex} within
%doc% the file \texttt{Makefile} to: \texttt{PDFLATEX\_CMD = pdflatex -synctex=1 -undump=pdflatex}
%doc%
%doc%

%doc% \subsubsection{Compiling in a Text-Shell}
%doc%
%doc% To generate a document using \texttt{Biber}, you can stick to
%doc% following example:
%doc% \begin{verbatim}
%doc% pdflatex main.tex
%doc% biber main
%doc% pdflatex main.tex
%doc% pdflatex main.tex
%doc% \end{verbatim}
%doc% 
%doc% Users of TeXLive with Microsoft Windows might want to try the
%doc% following script\footnote{Thanks to Florian Brucker for provinding
%doc%   this script.} which could be stored as, e.g., \texttt{compile.bat}:
%doc% \begin{verbatim}
%doc% REM call pdflatex using parameters suitable for TeXLive:
%doc% pdflatex.exe  "main.tex"
%doc% REM generate the references metadata for biblatex (using biber):
%doc% biber.exe "main"
%doc% REM call pdflatex twice to compile the references and finalize PDF:
%doc% pdflatex.exe  "main.tex"
%doc% pdflatex.exe -synctex=-1 -interaction=nonstopmode "main.tex"
%doc% \end{verbatim}
%doc% 


%doc%
%doc% \subsection{How to get rid of the template documentation}
%doc%
%doc% Simply remove the files \verb#Template_Documentation.pdf# and
%doc% \verb#Template_Documentation.tex# (if it exists) in the main folder
%doc% of this template.
%doc%
%doc% \subsection{What about modifying or extending the template?}\myinteresting
%doc% \label{sec:extending-template}
%doc%
%doc% This template provides an easy to start \LaTeX{} document template with sound
%doc% default settings. You can modify each setting any time. It is recommended that
%doc% you are familiar with the documentation of the command whose settings you want
%doc% to modify.
%doc%
%doc% It is recommended that for \emph{adding} things to the preambel (newcommands,
%doc% setting variables, defining headers, \dots) you should use the file
%doc% \texttt{main.tex}.
%doc% There are comment lines which help you find the right spot.
%doc% This way you still have the chance to update your \texttt{template}
%doc% folder from the template repository without losing your own added things.
%doc%
%doc% The following sections describe the settings and commands of this template and
%doc% give a short overview of its features.

%doc% \subsection{How to change the title page}
%doc%
%doc% This template comes with a variety of title pages. They are located in
%doc% the folder \texttt{template}. You can switch to a specific title
%doc% page by including the corresponding title page file in the file
%doc% \texttt{main.tex}.
%doc%
%doc% Please note that you may not need to modify any title page document by
%doc% yourself since all relevant information is defined in the file
%doc% \texttt{main.tex}.

%doc%
%doc% \section{\texttt{preamble.tex} --- Main preamble file}
%doc%
%doc% In the file \verb#preamble/preamble.tex# you will find the basic
%doc% definitions related to your document. This template uses the \myacro{KOMA} script
%doc% extension package of \LaTeX{}.
%doc%
%doc% There are comments added to the \verb#\documentclass{}# definitions. Please
%doc% refer to the great documentation of \myacro{KOMA}\footnote{\texttt{scrguide.pdf} for
%doc% German users} for further details.
%doc%
%doc% \paragraph{What should I do with this file?} For standard purposes you might
%doc% use the default values it provides. You must not remove its \texttt{include} command
%doc% in \texttt{main.tex} since it contains important definitions. This file contains
%doc% settings which are documented well and can be modified according to your needs.
%doc% It is recommended that you fully understand each setting you modify in order to
%doc% get a good document result. However, you can set basic values in the
%doc% \texttt{main.tex} file: font size, paper size,
%doc% paragraph separation mode, draft mode, binding correction, and whether
%doc% your document will be a one sided document or you are planning to
%doc% create a document which is printed on both, left side and right side.
%doc%

\documentclass[%
fontsize=\myfontsize,%% size of the main text
paper=\mypapersize,  %% paper format
parskip=\myparskip,  %% vertical space between paragraphs (instead of indenting first par-line)
DIV=calc,            %% calculates a good DIV value for type area; 66 characters/line is great
headinclude=true,    %% is header part of margin space or part of page content?
footinclude=false,   %% is footer part of margin space or part of page content?
open=right,          %% "right" or "left": start new chapter on right or left page
appendixprefix=true, %% adds appendix prefix; only for book-classes with \backmatter
bibliography=totoc,  %% adds the bibliography to table of contents (without number)
draft=\mydraft,      %% if true: included graphics are omitted and black boxes
                     %%          mark overfull boxes in margin space
BCOR=\myBCOR,        %% binding correction (depends on how you bind
                     %% the resulting printout.
\mylaterality        %% oneside: document is not printed on left and right sides, only right side
                     %% twoside: document is printed on left and right sides
]{scrbook}  %% article class of KOMA: "scrartcl", "scrreprt", or "scrbook".
            %% CAUTION: If documentclass will be changed, *many* other things
            %%          change as well like heading structure, ...



% FIXXME: adopting class usage:
% from scrbook -> scrartcl OR scrreport:
% - remove appendixprefix from class options
% - remove \frontmatter \mainmatter \backmatter \appendix from main.tex

% FIXXME: adopting language:
% add or modify language parameter of package »babel« and use language switches described in babel-documentation

%doc%
%doc% \subsection{\texttt{inputenc}: UTF8 as input charset}
%doc%
%doc% You are able and should use \myacro{UTF8} character settings for writing these \TeX{}-files.
%doc%
%\usepackage{ucs}             %% UTF8 as input characters; UCS incompatible to biblatex
\usepackage[utf8]{inputenc} %% UTF8 as input characters
%% Source: http://latex.tugraz.at/latex/tutorial#laden_von_paketen


%doc%
%doc% \subsection{\texttt{babel}: Language settings}
%doc%
%doc% The default setting of the language is American. Please change settings for
%doc% additional or alternative languages used in \texttt{main.tex}.
%doc%
%doc% Please note that the default language of the document is the \emph{last} language
%doc% which is added to the package options.
%doc%
%doc% To set only parts of your document in a different language as the rest, use for example\newline
%doc% \verb+\foreignlanguage{ngerman}{Beispieltext in deutscher Sprache}+\newline
%doc% For using foreign language quotes, please refer to the \verb+\foreignquote+,
%doc% \verb+\foreigntextquote+, or \verb+\foreignblockquote+ provided by
%doc% \texttt{csquotes} (see Section~\ref{sub:csquotes}).
%doc%
\usepackage[\mylanguage]{babel}  %% used languages; default language is *last* language of options

%doc%
%doc% \subsection{\texttt{scrpage2}: Headers and footers}
%doc%
%doc% Since this template is based on \myacro{KOMA} script it uses its great \texttt{scrpage2}
%doc% package for defining header and footer information. Please refer to the \myacro{KOMA}
%doc% script documentation how to use this package.
%doc%
\usepackage{scrpage2} %%  advanced page style using KOMA


%doc%
%doc% \subsection{References}\myimportant
%doc% \label{sec:references}
%doc%
%doc% This template is using
%doc% \href{http://www.tex.ac.uk/tex-archive/info/translations/biblatex/de/}{\texttt{biblatex}}
%doc% and \href{http://en.wikipedia.org/wiki/Biber_(LaTeX)}{\texttt{Biber}}
%doc% instead of
%doc% \href{http://en.wikipedia.org/wiki/BibTeX}{\textsc{Bib}\TeX{}}. This has the following
%doc% advantages:
%doc% \begin{itemize}
%doc% \item better documentation
%doc% \item Unicode-support like German umlauts (ö, ä, ü, ß) for references
%doc% \item flexible definition of citation styles
%doc% \item multiple bibliographies e.\,g. for printed and online resources
%doc% \item cleaner reference definition e.\,g. inheriting information from
%doc%   \texttt{Proceedings} to all related \texttt{InProceedings}
%doc% \item modern implementation
%doc% \end{itemize}
%doc%
%doc% In short, \texttt{biblatex} is able to handle your \texttt{bib}-files
%doc% and offers additional features. To get the most out of
%doc% \texttt{biblatex}, you should read the very good package
%doc% documentation. Be warned: you'll probably never want to change back
%doc% to \textsc{Bib}\TeX{} again.
%doc%
%doc% Take a look at the files \texttt{references-bibtex.bib} and
%doc% \texttt{references-biblatex.bib}: they contain the three
%doc% references \texttt{tagstore}, \texttt{Voit2009}, and
%doc% \texttt{Voit2011}.
%doc% The second file is optimized for \texttt{biblatex} and
%doc% takes advantage of some features that are not possible with
%doc% \textsc{Bib}\TeX{}.
%doc%
%doc% This template is ready to use \texttt{biblatex} with \texttt{Biber} as
%doc% reference compiler. You should make sure that you have installed an up
%doc% to date binary of \texttt{Biber} from its
%doc% homepage\footnote{\url{http://biblatex-biber.sourceforge.net/}}.
%doc%
%doc%
%doc% In \texttt{main.tex} you can define several general \texttt{biblatex}
%doc% options: citation style, whether or not multiple occurrences of
%doc% authors are replaced with dashes, or if backward references (from
%doc% references to citations) should be added.
%doc%
%doc%
%doc% If you are using the LaTeX{} editor TeXworks, please make sure that
%doc% you have read Section~\ref{sec:biberTeXworks} in order to use
%doc% \texttt{biber}.
%doc%

%doc% \subsubsection{Example citation commands}
%doc%
%doc% This section demonstrates some example citations using the style \texttt{authoryear}.
%doc% You can change the citation style in \texttt{main.tex} (\texttt{mybiblatexstyle}).
%doc%
%doc% \begin{itemize}
%doc% \item cite \cite{Eijkhout2008} and cite \cite{Bringhurst1993, Eijkhout2008}.
%doc% \item citet \citet{Eijkhout2008} and citet \citet{Bringhurst1993, Eijkhout2008}.
%doc% \item autocite \autocite{Eijkhout2008} and autocite \autocite{Bringhurst1993, Eijkhout2008}.
%doc% \item autocites \autocites{Eijkhout2008} and autocites \autocites{Bringhurst1993, Eijkhout2008}.
%doc% \item citeauthor \citeauthor{Eijkhout2008} and citeauthor \citeauthor{Bringhurst1993, Eijkhout2008}.
%doc% \item citetitle \citetitle{Eijkhout2008} and citetitle \citetitle{Bringhurst1993, Eijkhout2008}.
%doc% \item citeyear \citeyear{Eijkhout2008} and citeyear \citeyear{Bringhurst1993, Eijkhout2008}.
%doc% \item textcite \textcite{Eijkhout2008} and textcite \textcite{Bringhurst1993, Eijkhout2008}.
%doc% \item smartcite \smartcite{Eijkhout2008} and smartcite \smartcite{Bringhurst1993, Eijkhout2008}.
%doc% \item footcite \footcite{Eijkhout2008} and footcite \footcite{Bringhurst1993, Eijkhout2008}.
%doc% \item footcite with page \footcite[p.42]{Eijkhout2008} and footcite with page \footcite[compare][p.\,42]{Eijkhout2008}.
%doc% \item fullcite \fullcite{Eijkhout2008} and fullcite \fullcite{Bringhurst1993, Eijkhout2008}.
%doc% \end{itemize}
%doc%
%doc% Please note that the citation style as well as the bibliography style
%doc% can be changed very easily. Refer to the settings in
%doc% \texttt{main.tex} as well as the very good documentation of \texttt{biblatex}.
%doc%

%doc% \subsubsection{Using this template with \myacro{APA} style}
%doc%
%doc% First, you have to have the \myacro{APA} biblatex style
%doc% installed. Modern \LaTeX{} distributions do come with
%doc% \texttt{biblatex} and \myacro{APA} style. If so, you will find the
%doc% files \texttt{biblatex-apa.pdf} (style documentation) and
%doc% \texttt{biblatex-apa-test.pdf} (file with citation examples) on your
%doc% hard disk.
%doc%
%doc% \begin{enumerate}
%doc% \item Change the style according to \verb#\newcommand{\mybiblatexstyle}{apa}#
%doc% \item Add \verb#\DeclareLanguageMapping{american}{american-apa}# or \\
%doc%   \verb#\DeclareLanguageMapping{german}{german-apa}# to your
%doc%   preamble\footnote{You might want to use section \enquote{MISC
%doc%       self-defined commands and settings} for this.}
%doc% \end{enumerate}
%doc%
%doc% These steps change the biblatex style to \myacro{APA} style

%doc%
%doc% \subsubsection{Using this template with \textsc{Bib}\TeX{}}
%doc%
%doc% If you do not want to use \texttt{Biber} and \texttt{biblatex}, you
%doc% have to change several things:
%doc% \begin{itemize}
%doc% \item in \verb#preamble/preamble.tex#
%doc%   \begin{itemize}
%doc%   \item remove the usepackage command of \texttt{biblatex}
%doc%   \item remove the \verb#\addbibresource{...}# command
%doc%   \end{itemize}
%doc% \item in \verb#main.tex#
%doc%   \begin{itemize}
%doc%   \item replace \verb=\printbibliography= with the usual
%doc%     \verb=\bibliographystyle{yourstyle}= and \verb=\bibliography{yourbibfile}=
%doc%   \end{itemize}
%doc% \item if you are using \myacro{GNU} \texttt{make}: modify \verb=Makefile=
%doc%   \begin{itemize}
%doc%   \item replace \verb#BIBTEX_CMD = biber# with \verb#BIBTEX_CMD = bibtex#
%doc%   \end{itemize}
%doc% \item Use the reference file \texttt{references-bibtex.bib}
%doc%   instead of \texttt{references-biblatex.bib}
%doc% \end{itemize}
%doc%
%doc%
\usepackage[backend=biber, %% using "biber" to compile references (instead of "biblatex")
style=\mybiblatexstyle, %% see biblatex documentation
%style=alphabetic, %% see biblatex documentation
%dashed=\mybiblatexdashed, %% do *not* replace recurring reference authors with a dash
isbn=true,
doi=true,
url=true,
giveninits=true,
maxnames=16,
minnames=16,
maxcitenames=2,
mincitenames=1,
backref=\mybiblatexbackref, %% create backlings from references to citations
natbib=true, %% offering natbib-compatible commands
hyperref=true, %% using hyperref-package references
]{biblatex}  %% remove, if using BibTeX instead of biblatex

% moved to main.tex for vim-latex
%\addbibresource{\mybiblatexfile} %% remove, if using BibTeX instead of biblatex



%doc%
%doc% \subsection{Miscellaneous packages} \label{subsec:miscpackages}
%doc%
%doc% There are several packages included by default. You might want to activate or
%doc% deactivate them according to your requirements:
%doc%
%doc% \begin{enumerate}

%doc% \item[\texttt{\href{http://www.ctan.org/pkg/graphicx}{%%
%doc% graphicx%%
%doc% }}]
%doc% The widely used package to use graphical images within a \LaTeX{} document.
\usepackage[pdftex]{graphicx}

%doc% \item[\texttt{\href{https://secure.wikimedia.org/wikibooks/en/wiki/LaTeX/Formatting\#Other\_symbols}{%%
%doc% pifont%%
%doc% }}]
%doc% For additional special characters available by \verb#\ding{}#
\usepackage{pifont}


%doc% \item[\texttt{\href{http://ctan.org/pkg/ifthen}{%%
%doc% ifthen%%
%doc% }}]
%doc% For using if/then/else statements for example in macros
\usepackage{ifthen}

%% pre-define ifthen-boolean variables:
\newboolean{myaddcolophon}
\newboolean{myaddlistoftodos}
\newboolean{english_affidavit}


%doc% \item[\texttt{\href{http://www.ctan.org/tex-archive/fonts/eurosym}{%%
%doc% eurosym%%
%doc% }}]
%doc% Using the character for Euro with \verb#\officialeuro{}#
%\usepackage{eurosym}

%doc% \item[\texttt{\href{http://www.ctan.org/tex-archive/help/Catalogue/entries/xspace.html}{%%
%doc% xspace%%
%doc% }}]
%doc% This package is required for intelligent spacing after commands
\usepackage{xspace}

%doc% \item[\texttt{\href{https://secure.wikimedia.org/wikibooks/en/wiki/LaTeX/Colors}{%%
%doc% xcolor%%
%doc% }}]
%doc% This package defines basic colors. If you want to get rid of colored links and headings
%doc% please change corresponding value in \texttt{main.tex} to \{0,0,0\}.
\usepackage[usenames,dvipsnames]{xcolor}
%\definecolor{DispositionColor}{RGB}{\mydispositioncolor} %% used for links and so forth in screen-version
\definecolor{DispositionColor}{gray}{0} %% used for links and so forth in screen-version

%doc% \item[\texttt{\href{http://www.ctan.org/pkg/ulem}{%%
%doc% ulem%%
%doc% }}]
%doc% This package offers strikethrough command \verb+\sout{foobar}+.
\usepackage[normalem]{ulem}

%doc% \item[\texttt{\href{http://www.ctan.org/pkg/framed}{%%
%doc% framed%%
%doc% }}]
%doc% Create framed, shaded, or differently highlighted regions that can
%doc% break across pages.  The environments defined are
%doc% \begin{itemize}
%doc%   \item framed: ordinary frame box (\verb+\fbox+) with edge at margin
%doc%   \item shaded: shaded background (\verb+\colorbox+) bleeding into margin
%doc%   \item snugshade: similar
%doc%   \item leftbar: thick vertical line in left margin
%doc% \end{itemize}
\usepackage{framed}

%doc% \item[\texttt{\href{http://www.ctan.org/pkg/eso-pic}{%%
%doc% eso-pic%%
%doc% }}]
%doc% For example on title pages you might want to have a logo on the upper right corner of
%doc% the first page (only). The package \texttt{eso-pic} is able to place things on absolute
%doc% and relative positions on the whole page.
\usepackage{eso-pic}

%doc% \item[\texttt{\href{http://ctan.org/pkg/enumitem}{%%
%doc% enumitem%%
%doc% }}]
%doc% This package replaces the built-in definitions for enumerate, itemize and description.
%doc% With \texttt{enumitem} the user has more control over the layout of those environments.
\usepackage{enumitem}

%doc% \item[\texttt{\href{http://www.ctan.org/tex-archive/macros/latex/contrib/todonotes/}{%%
%doc% todonotes%%
%doc% }}]
%doc% This packages is \emph{very} handy to add notes\footnote{\texttt{todonotes} replaced
%doc% the \texttt{fixxme}-command which previously was defined in the
%doc% \texttt{preamble\_mycommands.tex} file.}. Using for example \verb#\todo{check}#
%doc% results in something like this \todo{check} in the document. Do read the
%doc% great package documentation for usage of other very helpful commands such as
%doc% \verb#\missingfigure{}# and \verb#\listoftodos#. The latter one creates an index of all
%doc% open todos which is very useful for getting an overview of open issues.
%doc% The package \texttt{todonotes} require the packages \texttt{ifthen}, \texttt{xkeyval}, \texttt{xcolor},
%doc% \texttt{tikz}, \texttt{calc}, and \texttt{graphicx}. Activate
%doc% and configure \verb#\listoftodos# in \texttt{main.tex}.
%\usepackage{todonotes}
\usepackage[\mytodonotesoptions]{todonotes}  %% option "disable" removes all todonotes output from resulting document

%disabled% \item[\texttt{\href{http://www.ctan.org/tex-archive/macros/latex/contrib/blindtext}{%%
%disabled% blindtext%%
%disabled% }}]
%disabled% This package is used to generate blind text for demonstration purposes.
%disabled% %% This is undocumented due to problems using american english; author informed
%disabled% \usepackage{blindtext}  %% provides commands for blind text:
%disabled% %% \blindtext creates some text,
%disabled% %% \Blindtext creates more text.
%disabled% %% \blinddocument creates a small document with sections, lists...
%disabled% %% \Blinddocument creates a large document with sections, lists...
%% 2012-03-10: vk: author published a corrected version which is able to handle "american english" as well. Did not have time to check new package version for this template here.

%doc% \item[\texttt{\href{http://ctan.org/tex-archive/macros/latex/contrib/units}{%%
%doc% units%%
%doc% }}]
%doc% For setting correctly typesetted units and nice fractions with \verb+\unit[42]{m}+ and \verb+\unitfrac[100]{km}{h}+.
\usepackage{units}


%doc% \end{enumerate}




%%%% End
%%% Local Variables:
%%% TeX-master: "../main"
%%% mode: latex
%%% mode: auto-fill
%%% mode: flyspell
%%% eval: (ispell-change-dictionary "en_US")
%%% End:
%% vim:foldmethod=expr
%% vim:fde=getline(v\:lnum)=~'^%%%%'?0\:getline(v\:lnum)=~'^%doc.*\ .\\%(sub\\)\\?section{.\\+'?'>1'\:'1':
%% DO NOT REMOVE THIS LINE!

% bibresource moved rere for vim-latex
\addbibresource{references-biblatex.bib} %% remove, if using BibTeX instead of biblatex

\setboolean{myaddcolophon}{false}  %% "true" or "false"
%% If set to "true": a colophon (with notes about this document
%% template, LaTeX, ...) is added after the title page.
%% Please do not set to "false" without a good reason. The colophon
%% helps your readers to get in touch with LaTeX and to find this template.

\setboolean{myaddlistoftodos}{true}  %% "true" or "false"
%% If set to "true": the current list of open todos is added after the
%% table of contents. If \mytodonotesoptions is set to "disable", no
%% list of todos is added, independent of this setting here.

\setboolean{english_affidavit}{true}  %% "true" or "false"
%% If set to "true": the language of the statutory declaration text is set to
%% English, otherwise it is in German.


%% ========================================================================
%%%% Document metadata
%% ========================================================================

%% general metadata:
\newcommand{\myauthor}{David Bidner}  %% also used for PDF metadata (hyperref)
\newcommand{\myauthorwithexistingtitles}{\myauthor{}, BSc}  %% including
                                %% university degree already held
                                %% (BSc, MSc, ...)
\newcommand{\mytitle}{Title TBD}  %% also used for PDF metadata (hyperref)
\newcommand{\mysubtitle}{Subtitle TBD}
\newcommand{\mysubject}{Subject TBD}  %% also used for PDF metadata (hyperref)
\newcommand{\mykeywords}{Keywords, TBD}  %% also used for PDF metadata (hyperref)

%% this information is used only for generating the title page:
\newcommand{\myworktitle}{Master's Thesis}  %% official type of work like ``Master theses''
\newcommand{\mygrade}{Diplom-Ingenieur} %% title you are getting with this work like ``Master of ...''
\newcommand{\mystudy}{Information and Computer Engineering} %% your study like ``Arts''
\newcommand{\mydegreeprogramme}{Master's degree programme: \mystudy} %% Master's or PhD degree programme
\newcommand{\myuniversity}{Graz University of Technology} %% your university/school
\newcommand{\myinstitute}{Institute of Applied Information Processing and Communications} %% affiliation
\newcommand{\myfaculty}{Faculty of Computer Science and Biomedical Engineering} % affiliation
%\newcommand{\myinstitutehead}{Univ.-Prof.\,Dipl-Ing.\,Dr.techn.~Some One} %% head of institute
\newcommand{\mysupervisor}{Dipl.-Ing. Dr.techn. Daniel Gruß} %% your supervisor
%\newcommand{\mycosupervisor}{Dr.~Some Body} %% your supervisor
\newcommand{\myevaluator}{Univ.-Prof. Dipl.-Ing. Dr.techn. Stefan Mangard} %% your evaluator
\newcommand{\myhomestreet}{} %% your home street (with house number)
\newcommand{\myhometown}{Graz} %% your home town
\newcommand{\myhomepostalnumber}{8010} %% your postal number of home town
\newcommand{\mysubmissionmonth}{TBD} %% month you are handing in
\newcommand{\mysubmissionyear}{2018} %% year you are handing in
\newcommand{\mysubmissiontown}{\myhometown} %% town of handing in (or \myhometown)

%% additional information for generic_documentation title page
\newcommand{\myid}{} %% Matrikelnummer
\newcommand{\mylecture}{LECTURE} %%


%% ========================================================================
%%%% MISC command definitions
%% ========================================================================
%% Time-stamp: <2015-04-30 17:19:58 vk>
%%%% === Disclaimer: =======================================================
%% created by
%%
%%      Karl Voit
%%
%% using GNU/Linux, GNU Emacs & LaTeX 2e
%%

%doc%
%doc% \section{\texttt{mycommands.tex} --- various definitions}\myinteresting
%doc% \label{sec:mycommands}
%doc%
%doc% In file \verb#template/mycommands.tex# many useful commands are being
%doc% defined. 
%doc% 
%doc% \paragraph{What should I do with this file?} Please take a look at its 
%doc% content to get the most out of your document.
%doc% 

%doc% 
%doc% One of the best advantages of \LaTeX{} compared to \myacro{WYSIWYG} software products is
%doc% the possibility to define and use macros within text. This empowers the user to
%doc% a great extend.  Many things can be defined using \verb#\newcommand{}# and
%doc% automates repeating tasks. It is recommended to use macros not only for
%doc% repetitive tasks but also for separating form from content such as \myacro{CSS}
%doc% does for \myacro{XHTML}. Think of including graphics in your document: after
%doc% writing your book, you might want to change all captions to the upper side of
%doc% each figure. In this case you either have to modify all
%doc% \texttt{includegraphics} commands or you were clever enough to define something
%doc% like \verb#\myfig#\footnote{See below for a detailed description}. Using a
%doc% macro for including graphics enables you to modify the position caption on only
%doc% \emph{one} place: at the definition of the macro.
%doc% 
%doc% The following section describes some macros that came with this document template
%doc% from \myLaT and you are welcome to modify or extend them or to create
%doc% your own macros!
%doc% 

%doc% 
%doc% \subsection{\texttt{myfig} --- including graphics made easy}
%doc% 
%doc% The classic: you can easily add graphics to your document with \verb#\myfig#:
%doc% \begin{verbatim}
%doc%  \myfig{flower}%% filename w/o extension in the folder figures
%doc%        {width=0.7\textwidth}%% maximum width/height, aspect ratio will be kept
%doc%        {This flower was photographed at my home town in 2010}%% caption
%doc%        {Home town flower}%% optional (short) caption for list of figures
%doc%        {fig:flower}%% label
%doc% \end{verbatim}
%doc% 
%doc% There are many advantages of this command (compared to manual
%doc% \texttt{figure} environments and \texttt{includegraphics} commands:
%doc% \begin{itemize}
%doc% \item consistent style throughout the whole document
%doc% \item easy to change; for example move caption on top
%doc% \item much less characters to type (faster, error prone)
%doc% \item less visual clutter in the \TeX{}-files
%doc% \end{itemize}
%doc% 
%doc% 
\newcommand{\myfig}[5]{
%% example:
% \myfig{}%% filename in figures folder
%       {width=0.5\textwidth,height=0.5\textheight}%% maximum width/height, aspect ratio will be kept
%       {}%% caption
%       {}%% optional (short) caption for list of figures
%       {}%% label
\begin{figure}%[htp]
  \centering
  \includegraphics[keepaspectratio,#2]{figures/#1}
  \caption[#4]{#3}
  \label{#5} %% NOTE: always label *after* caption!
\end{figure}
}


%doc% 
%doc% \subsection{\texttt{myclone} --- repeat things!}
%doc% 
%doc% Using \verb#\myclone[42]{foobar}# results the text \enquote{foobar} printed 42 times.
%doc% But you can not only repeat text output with \texttt{myclone}. 
%doc%
%doc% Default argument
%doc% for the optional parameter \enquote{number of times} (like \enquote{42} in the example above) 
%doc% is set to two.
%doc% 
%% \myclone[x]{text}
\newcounter{myclonecnt}
\newcommand{\myclone}[2][2]{%
  \setcounter{myclonecnt}{#1}%
  \whiledo{\value{myclonecnt}>0}{#2\addtocounter{myclonecnt}{-1}}%
}

%old% %d oc% 
%old% %d oc% \subsection{\texttt{fixxme} --- sidemark something as unfinished}
%old% %d oc% 
%old% %d oc% You know it: something has to be fixed and you can not do it right
%old% %d oc% now. In order to \texttt{not} forget about it, you might want to add a
%old% %d oc% note like \verb+\fixxme{check again}+ which inserts a note on the page
%old% %d oc% margin such as this\fixxme{check again} example.
%old% %d oc%
%old% \newcommand{\fixxme}[1]{%%
%old% \textcolor{red}{FIXXME}\marginpar{\textcolor{red}{#1}}%%
%old% }


%%%% End 
%%% Local Variables:
%%% mode: latex
%%% mode: auto-fill
%%% mode: flyspell
%%% eval: (ispell-change-dictionary "en_US")
%%% TeX-master: "../main"
%%% End:
%% vim:foldmethod=expr
%% vim:fde=getline(v\:lnum)=~'^%%%%'?0\:getline(v\:lnum)=~'^%doc.*\ .\\%(sub\\)\\?section{.\\+'?'>1'\:'1':


%% ========================================================================
%%%% Typographic settings
%% ========================================================================
%%%% Time-stamp: <2015-08-22 17:20:32 vk>
%%%% === Disclaimer: =======================================================
%% created by
%%
%%      Karl Voit
%%
%% using GNU/Linux, GNU Emacs & LaTeX 2e
%%
%doc%
%doc% \section{\texttt{typographic\_settings.tex} --- Typographic finetuning}
%doc%
%doc% The settings of file \verb#template/typographic_settings.tex# contain
%doc% typographic finetuning related to things mentioned in literature.  The
%doc% settings in this file relates to personal taste and most of all: 
%doc% \emph{typographic experience}. 
%doc% 
%doc% \paragraph{What should I do with this file?} You might as well skip the whole
%doc% file by excluding the \verb#%%%% Time-stamp: <2015-08-22 17:20:32 vk>
%%%% === Disclaimer: =======================================================
%% created by
%%
%%      Karl Voit
%%
%% using GNU/Linux, GNU Emacs & LaTeX 2e
%%
%doc%
%doc% \section{\texttt{typographic\_settings.tex} --- Typographic finetuning}
%doc%
%doc% The settings of file \verb#template/typographic_settings.tex# contain
%doc% typographic finetuning related to things mentioned in literature.  The
%doc% settings in this file relates to personal taste and most of all: 
%doc% \emph{typographic experience}. 
%doc% 
%doc% \paragraph{What should I do with this file?} You might as well skip the whole
%doc% file by excluding the \verb#%%%% Time-stamp: <2015-08-22 17:20:32 vk>
%%%% === Disclaimer: =======================================================
%% created by
%%
%%      Karl Voit
%%
%% using GNU/Linux, GNU Emacs & LaTeX 2e
%%
%doc%
%doc% \section{\texttt{typographic\_settings.tex} --- Typographic finetuning}
%doc%
%doc% The settings of file \verb#template/typographic_settings.tex# contain
%doc% typographic finetuning related to things mentioned in literature.  The
%doc% settings in this file relates to personal taste and most of all: 
%doc% \emph{typographic experience}. 
%doc% 
%doc% \paragraph{What should I do with this file?} You might as well skip the whole
%doc% file by excluding the \verb#\input{template/typographic_settings.tex}# command
%doc% in \texttt{main.tex}.  For standard usage it is recommended to stay with the
%doc% default settings.
%doc% 
%doc% 
%% ========================================================================

%doc%
%doc% Some basic microtypographic settings are provided by the
%doc% \texttt{microtype}
%doc% package\footnote{\url{http://ctan.org/pkg/microtype}}. This template
%doc% uses the rather conservative package parameters: \texttt{protrusion=true,factor=900}.
\usepackage[protrusion=true,factor=900]{microtype}

%doc%
%doc% \subsection{French spacing}
%doc%
%doc% \paragraph{Why?} see~\textcite[p.\,28, p.\,30]{Bringhurst1993}: `2.1.4 Use a single word space between sentences.'
%doc%
%doc% \paragraph{How?} see~\textcite[p.\,185]{Eijkhout2008}:\\
%doc% \verb#\frenchspacing  %% Macro to switch off extra space after punctuation.# \\
\frenchspacing  %% Macro to switch off extra space after punctuation.
%doc%
%doc% Note: This setting might be default for \myacro{KOMA} script.
%doc%


%doc%
%doc% \subsection{Font}
%doc% 
%doc% This template is using the Palatino font (package \texttt{mathpazo}) which results
%doc% in a legible document and matching mathematical fonts for printout.
%doc% 
%doc% It is highly recommended that you either stick to the Palatino font or use the
%doc% \LaTeX{} default fonts (by removing the package \texttt{mathpazo}).
%doc% 
%doc% Chosing different fonts is not
%doc% an easy task. Please leave this to people with good knowledge on this subject.
%doc% 
%doc% One valid reason to change the default fonts is when your document is mainly
%doc% read on a computer screen. In this case it is recommended to switch to a font
%doc% \textsf{which is sans-serif like this}. This template contains several alternative
%doc% font packages which can be activated in this file.
%doc% 

% for changing the default font, please go to the next subsection!

%doc%
%doc% \subsection{Text figures}
%doc% 
%doc% \ldots also called old style numbers such as 0123456789. 
%doc% (German: \enquote{Mediäval\-ziffern\footnote{\url{https://secure.wikimedia.org/wikibooks/de/wiki/LaTeX-W\%C3\%B6rterbuch:\_Medi\%C3\%A4valziffern}}})
%doc% \paragraph{Why?} see~\textcite[p.\,44f]{Bringhurst1993}: 
%doc% \begin{quote}
%doc% `3.2.1 If the font includes both text figures and titling figures, use
%doc%  titling figures only with full caps, and text figures in all other
%doc%  circumstances.'
%doc% \end{quote}
%doc% 
%doc% \paragraph{How?} 
%doc% Quoted from Wikibooks\footnote{\url{https://secure.wikimedia.org/wikibooks/en/wiki/LaTeX/Formatting\#Text\_figures\_.28.22old\_style.22\_numerals.29}}:
%doc% \begin{quote}
%doc% Some fonts do not have text figures built in; the textcomp package attempts to
%doc% remedy this by effectively generating text figures from the currently-selected
%doc% font. Put \verb#\usepackage{textcomp}# in your preamble. textcomp also allows you to
%doc% use decimal points, properly formatted dollar signs, etc. within
%doc% \verb#\oldstylenums{}#.
%doc% \end{quote}
%doc% \ldots but proposed \LaTeX{} method does not work out well. Instead use:\\
%doc% \verb#\usepackage{hfoldsty}#  (enables text figures using additional font) or \\
%doc% \verb#\usepackage[sc,osf]{mathpazo}# (switches to Palatino font with small caps and old style figures enabled).
%doc%
%\usepackage{hfoldsty}  %% enables text figures using additional font
%% ... OR use ...
%\usepackage[sc,osf]{mathpazo} %% switches to Palatino with small caps and old style figures

%% Font selection from:
%%     http://www.matthiaspospiech.de/latex/vorlagen/allgemein/preambel/fonts/
%% use following lines *instead* of the mathpazo package above:
%% ===== Serif =========================================================
%% for Computer Modern (LaTeX default font), simply remove the mathpazo above
%\usepackage{charter}\linespread{1.05} %% Charter
%\usepackage{bookman}                  %% Bookman (laedt Avant Garde !!)
%\usepackage{newcent}                  %% New Century Schoolbook (laedt Avant Garde !!)
%% ===== Sans Serif ====================================================
%\renewcommand{\familydefault}{\sfdefault}  %% this one in *combination* with the default mathpazo package
%\usepackage{cmbright}                  %% CM-Bright (eigntlich eine Familie)
%\usepackage{tpslifonts}                %% tpslifonts % Font for Slides


%doc% 
%doc% \subsection{\texttt{myacro} --- Abbrevations using \textsc{small caps}}\myinteresting
%doc% \label{sec:myacro}
%doc% 
%doc% \paragraph{Why?} see~\textcite[p.\,45f]{Bringhurst1993}: `3.2.2 For abbrevations and
%doc% acronyms in the midst of normal text, use spaced small caps.'
%doc% 
%doc% \paragraph{How?} Using the predefined macro \verb#\myacro{}# for things like
%doc% \myacro{UNO} or \myacro{UNESCO} using \verb#\myacro{UNO}# or \verb#\myacro{UNESCO}#.
%doc% 
\DeclareRobustCommand{\myacro}[1]{\textsc{\lowercase{#1}}} %%  abbrevations using small caps


%doc% 
%doc% \subsection{Colorized headings and links}
%doc% 
%doc% This document template is able to generate an output that uses colorized
%doc% headings, captions, page numbers, and links. The color named `DispositionColor'
%doc% used in this document is defined near the definition of package \texttt{color}
%doc% in the preamble (see section~\ref{subsec:miscpackages}). The changes required
%doc% for headings, page numbers, and captions are defined here.
%doc% 
%doc% Settings for colored links are handled by the definitions of the
%doc% \texttt{hyperref} package (see section~\ref{sec:pdf}).
%doc% 
\setheadsepline{.4pt}[\color{DispositionColor}]
\renewcommand{\headfont}{\normalfont\sffamily\color{DispositionColor}}
\renewcommand{\pnumfont}{\normalfont\sffamily\color{DispositionColor}}
\addtokomafont{disposition}{\color{DispositionColor}}
\addtokomafont{caption}{\color{DispositionColor}\footnotesize}
\addtokomafont{captionlabel}{\color{DispositionColor}}

%doc% 
%doc% \subsection{No figures or tables below footnotes}
%doc% 
%doc% \LaTeX{} places floating environments below footnotes if \texttt{b}
%doc% (bottom) is used as (default) placement algorithm. This is certainly
%doc% not appealing for most people and is deactivated in this template by
%doc% using the package \texttt{footmisc} with its option \texttt{bottom}.
%doc% 
%% see also: http://www.komascript.de/node/858 (German description)
\usepackage[bottom]{footmisc}

%doc% 
%doc% \subsection{Spacings of list environments}
%doc% 
%doc% By default, \LaTeX{} is using vertical spaces between items of enumerate, 
%doc% itemize and description environments. This is fine for multi-line items.
%doc% Many times, the user does just write single-line items where the larger
%doc% vertical space is inappropriate. The \href{http://ctan.org/pkg/enumitem}{enumitem}
%doc% package provides replacements for the pre-defined list environments and
%doc% offers many options to modify their appearances.
%doc% This template is using the package option for \texttt{noitemsep} which
%doc% mimimizes the vertical space between list items.
%doc% 
\usepackage{enumitem}
\setlist{noitemsep}   %% kills the space between items

%doc% 
%doc% \subsection{\texttt{csquotes} --- Correct quotation marks}\myinteresting
%doc% \label{sub:csquotes}
%doc% 
%doc% \emph{Never} use quotation marks found on your keyboard.
%doc% They end up in strange characters or false looking quotation marks.
%doc% 
%doc% In \LaTeX{} you are able to use typographically correct quotation marks. The package 
%doc% \href{http://www.ctan.org/pkg/csquotes}{\texttt{csquotes}} offers you with 
%doc% \verb#\enquote{foobar}# a command to get correct quotation marks around \enquote{foobar}.
%doc% Please do check the package options in order to modify
%doc% its settings according to the language used\footnote{most of the time in 
%doc% combination with the language set in the options of the \texttt{babel} package}.
%doc% 
%doc% \href{http://www.ctan.org/pkg/csquotes}{\texttt{csquotes}} is also recommended 
%doc% by \texttt{biblatex} (see Section~\ref{sec:references}). 
\usepackage[babel=true,strict=true,english=american,german=guillemets]{csquotes}

%doc% 
%doc% \subsection{Line spread}
%doc% 
%doc% If you have to enlarge the distance between two lines of text, you can
%doc% increase it using the \texttt{\mylinespread} command in \texttt{main.tex}. By default, it is
%doc% deactivated (set to 100~percent). Modify only with caution since it influences the
%doc% page layout and could lead to ugly looking documents.
\linespread{\mylinespread}

%doc% 
%doc% \subsection{Optional: Lines above and below the chapter head}
%doc% 
%doc% This is not quite something typographic but rather a matter of taste.
%doc% \myacro{KOMA} Script offers \href{http://www.komascript.de/node/24}{a method to
%doc% add lines above and below chapter head} which is disabled by
%doc% default. If you want to enable this feature, remove corresponding
%doc% comment characters from the settings.
%doc% 
%% Source: http://www.komascript.de/node/24
%disabled% %% 1st get a new command
%disabled% \newcommand*{\ORIGchapterheadstartvskip}{}%
%disabled% %% 2nd save the original definition to the new command
%disabled% \let\ORIGchapterheadstartvskip=\chapterheadstartvskip
%disabled% %% 3rd redefine the command using the saved original command
%disabled% \renewcommand*{\chapterheadstartvskip}{%
%disabled%   \ORIGchapterheadstartvskip
%disabled%   {%
%disabled%     \setlength{\parskip}{0pt}%
%disabled%     \noindent\color{DispositionColor}\rule[.3\baselineskip]{\linewidth}{1pt}\par
%disabled%   }%
%disabled% }
%disabled% %% see above
%disabled% \newcommand*{\ORIGchapterheadendvskip}{}%
%disabled% \let\ORIGchapterheadendvskip=\chapterheadendvskip
%disabled% \renewcommand*{\chapterheadendvskip}{%
%disabled%   {%
%disabled%     \setlength{\parskip}{0pt}%
%disabled%     \noindent\color{DispositionColor}\rule[.3\baselineskip]{\linewidth}{1pt}\par
%disabled%   }%
%disabled%   \ORIGchapterheadendvskip
%disabled% }

%doc% 
%doc% \subsection{Optional: Chapter thumbs}
%doc% 
%doc% This is not quite something typographic but rather a matter of taste.
%doc% \myacro{KOMA} Script offers \href{http://www.komascript.de/chapterthumbs-example}{a method to
%doc% add chapter thumbs} (in combination with the package \texttt{scrpage2}) which is disabled by
%doc% default. If you want to enable this feature, remove corresponding
%doc% comment characters from the settings.
%doc% 
%disabled% \makeatletter
%disabled% % Safty first
%disabled% \@ifundefined{chapter}{\let\chapter\undefined
%disabled%   \chapter must be defined to use chapter thumbs!}{%
%disabled%  
%disabled% % Two new commands for the width and height of the boxes with the
%disabled% % chapter number at the thumbs (use of commands instead of lengths
%disabled% % for sparing registers)
%disabled% \newcommand*{\chapterthumbwidth}{2em}
%disabled% \newcommand*{\chapterthumbheight}{1em}
%disabled%  
%disabled% % Two new commands for the colors of the box background and the
%disabled% % chapter numbers of the thumbs
%disabled% \newcommand*{\chapterthumbboxcolor}{black}
%disabled% \newcommand*{\chapterthumbtextcolor}{white}
%disabled%  
%disabled% % New command to set a chapter thumb. I'm using a group at this
%disabled% % command, because I'm changing the temporary dimension \@tempdima
%disabled% \newcommand*{\putchapterthumb}{%
%disabled%   \begingroup
%disabled%     \Large
%disabled%     % calculate the horizontal possition of the right paper border
%disabled%     % (I ignore \hoffset, because I interprete \hoffset moves the page
%disabled%     % at the paper e.g. if you are using cropmarks)
%disabled%     \setlength{\@tempdima}{\@oddheadshift}% (internal from scrpage2)
%disabled%     \setlength{\@tempdima}{-\@tempdima}%
%disabled%     \addtolength{\@tempdima}{\paperwidth}%
%disabled%     \addtolength{\@tempdima}{-\oddsidemargin}%
%disabled%     \addtolength{\@tempdima}{-1in}%
%disabled%     % putting the thumbs should not change the horizontal
%disabled%     % possition
%disabled%     \rlap{%
%disabled%       % move to the calculated horizontal possition
%disabled%       \hspace*{\@tempdima}%
%disabled%       % putting the thumbs should not change the vertical
%disabled%       % possition
%disabled%       \vbox to 0pt{%
%disabled%         % calculate the vertical possition of the thumbs (I ignore
%disabled%         % \voffset for the same reasons told above)
%disabled%         \setlength{\@tempdima}{\chapterthumbwidth}%
%disabled%         \multiply\@tempdima by\value{chapter}%
%disabled%         \addtolength{\@tempdima}{-\chapterthumbwidth}%
%disabled%         \addtolength{\@tempdima}{-\baselineskip}%
%disabled%         % move to the calculated vertical possition
%disabled%         \vspace*{\@tempdima}%
%disabled%         % put the thumbs left so the current horizontal possition
%disabled%         \llap{%
%disabled%           % and rotate them
%disabled%           \rotatebox{90}{\colorbox{\chapterthumbboxcolor}{%
%disabled%               \parbox[c][\chapterthumbheight][c]{\chapterthumbwidth}{%
%disabled%                 \centering
%disabled%                 \textcolor{\chapterthumbtextcolor}{%
%disabled%                   \strut\thechapter}\\
%disabled%               }%
%disabled%             }%
%disabled%           }%
%disabled%         }%
%disabled%         % avoid overfull \vbox messages
%disabled%         \vss
%disabled%       }%
%disabled%     }%
%disabled%   \endgroup
%disabled% }
%disabled%  
%disabled% % New command, which works like \lohead but also puts the thumbs (you
%disabled% % cannot use \ihead with this definition but you may change this, if
%disabled% % you use more internal scrpage2 commands)
%disabled% \newcommand*{\loheadwithchapterthumbs}[2][]{%
%disabled%   \lohead[\putchapterthumb#1]{\putchapterthumb#2}%
%disabled% }
%disabled%  
%disabled% % initial use
%disabled% \loheadwithchapterthumbs{}
%disabled% \pagestyle{scrheadings}
%disabled%  
%disabled% }
%disabled% \makeatother

%%%% END
%%% Local Variables:
%%% mode: latex
%%% mode: auto-fill
%%% mode: flyspell
%%% eval: (ispell-change-dictionary "en_US")
%%% TeX-master: "../main"
%%% End:
%% vim:foldmethod=expr
%% vim:fde=getline(v\:lnum)=~'^%%%%'?0\:getline(v\:lnum)=~'^%doc.*\ .\\%(sub\\)\\?section{.\\+'?'>1'\:'1':
# command
%doc% in \texttt{main.tex}.  For standard usage it is recommended to stay with the
%doc% default settings.
%doc% 
%doc% 
%% ========================================================================

%doc%
%doc% Some basic microtypographic settings are provided by the
%doc% \texttt{microtype}
%doc% package\footnote{\url{http://ctan.org/pkg/microtype}}. This template
%doc% uses the rather conservative package parameters: \texttt{protrusion=true,factor=900}.
\usepackage[protrusion=true,factor=900]{microtype}

%doc%
%doc% \subsection{French spacing}
%doc%
%doc% \paragraph{Why?} see~\textcite[p.\,28, p.\,30]{Bringhurst1993}: `2.1.4 Use a single word space between sentences.'
%doc%
%doc% \paragraph{How?} see~\textcite[p.\,185]{Eijkhout2008}:\\
%doc% \verb#\frenchspacing  %% Macro to switch off extra space after punctuation.# \\
\frenchspacing  %% Macro to switch off extra space after punctuation.
%doc%
%doc% Note: This setting might be default for \myacro{KOMA} script.
%doc%


%doc%
%doc% \subsection{Font}
%doc% 
%doc% This template is using the Palatino font (package \texttt{mathpazo}) which results
%doc% in a legible document and matching mathematical fonts for printout.
%doc% 
%doc% It is highly recommended that you either stick to the Palatino font or use the
%doc% \LaTeX{} default fonts (by removing the package \texttt{mathpazo}).
%doc% 
%doc% Chosing different fonts is not
%doc% an easy task. Please leave this to people with good knowledge on this subject.
%doc% 
%doc% One valid reason to change the default fonts is when your document is mainly
%doc% read on a computer screen. In this case it is recommended to switch to a font
%doc% \textsf{which is sans-serif like this}. This template contains several alternative
%doc% font packages which can be activated in this file.
%doc% 

% for changing the default font, please go to the next subsection!

%doc%
%doc% \subsection{Text figures}
%doc% 
%doc% \ldots also called old style numbers such as 0123456789. 
%doc% (German: \enquote{Mediäval\-ziffern\footnote{\url{https://secure.wikimedia.org/wikibooks/de/wiki/LaTeX-W\%C3\%B6rterbuch:\_Medi\%C3\%A4valziffern}}})
%doc% \paragraph{Why?} see~\textcite[p.\,44f]{Bringhurst1993}: 
%doc% \begin{quote}
%doc% `3.2.1 If the font includes both text figures and titling figures, use
%doc%  titling figures only with full caps, and text figures in all other
%doc%  circumstances.'
%doc% \end{quote}
%doc% 
%doc% \paragraph{How?} 
%doc% Quoted from Wikibooks\footnote{\url{https://secure.wikimedia.org/wikibooks/en/wiki/LaTeX/Formatting\#Text\_figures\_.28.22old\_style.22\_numerals.29}}:
%doc% \begin{quote}
%doc% Some fonts do not have text figures built in; the textcomp package attempts to
%doc% remedy this by effectively generating text figures from the currently-selected
%doc% font. Put \verb#\usepackage{textcomp}# in your preamble. textcomp also allows you to
%doc% use decimal points, properly formatted dollar signs, etc. within
%doc% \verb#\oldstylenums{}#.
%doc% \end{quote}
%doc% \ldots but proposed \LaTeX{} method does not work out well. Instead use:\\
%doc% \verb#\usepackage{hfoldsty}#  (enables text figures using additional font) or \\
%doc% \verb#\usepackage[sc,osf]{mathpazo}# (switches to Palatino font with small caps and old style figures enabled).
%doc%
%\usepackage{hfoldsty}  %% enables text figures using additional font
%% ... OR use ...
%\usepackage[sc,osf]{mathpazo} %% switches to Palatino with small caps and old style figures

%% Font selection from:
%%     http://www.matthiaspospiech.de/latex/vorlagen/allgemein/preambel/fonts/
%% use following lines *instead* of the mathpazo package above:
%% ===== Serif =========================================================
%% for Computer Modern (LaTeX default font), simply remove the mathpazo above
%\usepackage{charter}\linespread{1.05} %% Charter
%\usepackage{bookman}                  %% Bookman (laedt Avant Garde !!)
%\usepackage{newcent}                  %% New Century Schoolbook (laedt Avant Garde !!)
%% ===== Sans Serif ====================================================
%\renewcommand{\familydefault}{\sfdefault}  %% this one in *combination* with the default mathpazo package
%\usepackage{cmbright}                  %% CM-Bright (eigntlich eine Familie)
%\usepackage{tpslifonts}                %% tpslifonts % Font for Slides


%doc% 
%doc% \subsection{\texttt{myacro} --- Abbrevations using \textsc{small caps}}\myinteresting
%doc% \label{sec:myacro}
%doc% 
%doc% \paragraph{Why?} see~\textcite[p.\,45f]{Bringhurst1993}: `3.2.2 For abbrevations and
%doc% acronyms in the midst of normal text, use spaced small caps.'
%doc% 
%doc% \paragraph{How?} Using the predefined macro \verb#\myacro{}# for things like
%doc% \myacro{UNO} or \myacro{UNESCO} using \verb#\myacro{UNO}# or \verb#\myacro{UNESCO}#.
%doc% 
\DeclareRobustCommand{\myacro}[1]{\textsc{\lowercase{#1}}} %%  abbrevations using small caps


%doc% 
%doc% \subsection{Colorized headings and links}
%doc% 
%doc% This document template is able to generate an output that uses colorized
%doc% headings, captions, page numbers, and links. The color named `DispositionColor'
%doc% used in this document is defined near the definition of package \texttt{color}
%doc% in the preamble (see section~\ref{subsec:miscpackages}). The changes required
%doc% for headings, page numbers, and captions are defined here.
%doc% 
%doc% Settings for colored links are handled by the definitions of the
%doc% \texttt{hyperref} package (see section~\ref{sec:pdf}).
%doc% 
\setheadsepline{.4pt}[\color{DispositionColor}]
\renewcommand{\headfont}{\normalfont\sffamily\color{DispositionColor}}
\renewcommand{\pnumfont}{\normalfont\sffamily\color{DispositionColor}}
\addtokomafont{disposition}{\color{DispositionColor}}
\addtokomafont{caption}{\color{DispositionColor}\footnotesize}
\addtokomafont{captionlabel}{\color{DispositionColor}}

%doc% 
%doc% \subsection{No figures or tables below footnotes}
%doc% 
%doc% \LaTeX{} places floating environments below footnotes if \texttt{b}
%doc% (bottom) is used as (default) placement algorithm. This is certainly
%doc% not appealing for most people and is deactivated in this template by
%doc% using the package \texttt{footmisc} with its option \texttt{bottom}.
%doc% 
%% see also: http://www.komascript.de/node/858 (German description)
\usepackage[bottom]{footmisc}

%doc% 
%doc% \subsection{Spacings of list environments}
%doc% 
%doc% By default, \LaTeX{} is using vertical spaces between items of enumerate, 
%doc% itemize and description environments. This is fine for multi-line items.
%doc% Many times, the user does just write single-line items where the larger
%doc% vertical space is inappropriate. The \href{http://ctan.org/pkg/enumitem}{enumitem}
%doc% package provides replacements for the pre-defined list environments and
%doc% offers many options to modify their appearances.
%doc% This template is using the package option for \texttt{noitemsep} which
%doc% mimimizes the vertical space between list items.
%doc% 
\usepackage{enumitem}
\setlist{noitemsep}   %% kills the space between items

%doc% 
%doc% \subsection{\texttt{csquotes} --- Correct quotation marks}\myinteresting
%doc% \label{sub:csquotes}
%doc% 
%doc% \emph{Never} use quotation marks found on your keyboard.
%doc% They end up in strange characters or false looking quotation marks.
%doc% 
%doc% In \LaTeX{} you are able to use typographically correct quotation marks. The package 
%doc% \href{http://www.ctan.org/pkg/csquotes}{\texttt{csquotes}} offers you with 
%doc% \verb#\enquote{foobar}# a command to get correct quotation marks around \enquote{foobar}.
%doc% Please do check the package options in order to modify
%doc% its settings according to the language used\footnote{most of the time in 
%doc% combination with the language set in the options of the \texttt{babel} package}.
%doc% 
%doc% \href{http://www.ctan.org/pkg/csquotes}{\texttt{csquotes}} is also recommended 
%doc% by \texttt{biblatex} (see Section~\ref{sec:references}). 
\usepackage[babel=true,strict=true,english=american,german=guillemets]{csquotes}

%doc% 
%doc% \subsection{Line spread}
%doc% 
%doc% If you have to enlarge the distance between two lines of text, you can
%doc% increase it using the \texttt{\mylinespread} command in \texttt{main.tex}. By default, it is
%doc% deactivated (set to 100~percent). Modify only with caution since it influences the
%doc% page layout and could lead to ugly looking documents.
\linespread{\mylinespread}

%doc% 
%doc% \subsection{Optional: Lines above and below the chapter head}
%doc% 
%doc% This is not quite something typographic but rather a matter of taste.
%doc% \myacro{KOMA} Script offers \href{http://www.komascript.de/node/24}{a method to
%doc% add lines above and below chapter head} which is disabled by
%doc% default. If you want to enable this feature, remove corresponding
%doc% comment characters from the settings.
%doc% 
%% Source: http://www.komascript.de/node/24
%disabled% %% 1st get a new command
%disabled% \newcommand*{\ORIGchapterheadstartvskip}{}%
%disabled% %% 2nd save the original definition to the new command
%disabled% \let\ORIGchapterheadstartvskip=\chapterheadstartvskip
%disabled% %% 3rd redefine the command using the saved original command
%disabled% \renewcommand*{\chapterheadstartvskip}{%
%disabled%   \ORIGchapterheadstartvskip
%disabled%   {%
%disabled%     \setlength{\parskip}{0pt}%
%disabled%     \noindent\color{DispositionColor}\rule[.3\baselineskip]{\linewidth}{1pt}\par
%disabled%   }%
%disabled% }
%disabled% %% see above
%disabled% \newcommand*{\ORIGchapterheadendvskip}{}%
%disabled% \let\ORIGchapterheadendvskip=\chapterheadendvskip
%disabled% \renewcommand*{\chapterheadendvskip}{%
%disabled%   {%
%disabled%     \setlength{\parskip}{0pt}%
%disabled%     \noindent\color{DispositionColor}\rule[.3\baselineskip]{\linewidth}{1pt}\par
%disabled%   }%
%disabled%   \ORIGchapterheadendvskip
%disabled% }

%doc% 
%doc% \subsection{Optional: Chapter thumbs}
%doc% 
%doc% This is not quite something typographic but rather a matter of taste.
%doc% \myacro{KOMA} Script offers \href{http://www.komascript.de/chapterthumbs-example}{a method to
%doc% add chapter thumbs} (in combination with the package \texttt{scrpage2}) which is disabled by
%doc% default. If you want to enable this feature, remove corresponding
%doc% comment characters from the settings.
%doc% 
%disabled% \makeatletter
%disabled% % Safty first
%disabled% \@ifundefined{chapter}{\let\chapter\undefined
%disabled%   \chapter must be defined to use chapter thumbs!}{%
%disabled%  
%disabled% % Two new commands for the width and height of the boxes with the
%disabled% % chapter number at the thumbs (use of commands instead of lengths
%disabled% % for sparing registers)
%disabled% \newcommand*{\chapterthumbwidth}{2em}
%disabled% \newcommand*{\chapterthumbheight}{1em}
%disabled%  
%disabled% % Two new commands for the colors of the box background and the
%disabled% % chapter numbers of the thumbs
%disabled% \newcommand*{\chapterthumbboxcolor}{black}
%disabled% \newcommand*{\chapterthumbtextcolor}{white}
%disabled%  
%disabled% % New command to set a chapter thumb. I'm using a group at this
%disabled% % command, because I'm changing the temporary dimension \@tempdima
%disabled% \newcommand*{\putchapterthumb}{%
%disabled%   \begingroup
%disabled%     \Large
%disabled%     % calculate the horizontal possition of the right paper border
%disabled%     % (I ignore \hoffset, because I interprete \hoffset moves the page
%disabled%     % at the paper e.g. if you are using cropmarks)
%disabled%     \setlength{\@tempdima}{\@oddheadshift}% (internal from scrpage2)
%disabled%     \setlength{\@tempdima}{-\@tempdima}%
%disabled%     \addtolength{\@tempdima}{\paperwidth}%
%disabled%     \addtolength{\@tempdima}{-\oddsidemargin}%
%disabled%     \addtolength{\@tempdima}{-1in}%
%disabled%     % putting the thumbs should not change the horizontal
%disabled%     % possition
%disabled%     \rlap{%
%disabled%       % move to the calculated horizontal possition
%disabled%       \hspace*{\@tempdima}%
%disabled%       % putting the thumbs should not change the vertical
%disabled%       % possition
%disabled%       \vbox to 0pt{%
%disabled%         % calculate the vertical possition of the thumbs (I ignore
%disabled%         % \voffset for the same reasons told above)
%disabled%         \setlength{\@tempdima}{\chapterthumbwidth}%
%disabled%         \multiply\@tempdima by\value{chapter}%
%disabled%         \addtolength{\@tempdima}{-\chapterthumbwidth}%
%disabled%         \addtolength{\@tempdima}{-\baselineskip}%
%disabled%         % move to the calculated vertical possition
%disabled%         \vspace*{\@tempdima}%
%disabled%         % put the thumbs left so the current horizontal possition
%disabled%         \llap{%
%disabled%           % and rotate them
%disabled%           \rotatebox{90}{\colorbox{\chapterthumbboxcolor}{%
%disabled%               \parbox[c][\chapterthumbheight][c]{\chapterthumbwidth}{%
%disabled%                 \centering
%disabled%                 \textcolor{\chapterthumbtextcolor}{%
%disabled%                   \strut\thechapter}\\
%disabled%               }%
%disabled%             }%
%disabled%           }%
%disabled%         }%
%disabled%         % avoid overfull \vbox messages
%disabled%         \vss
%disabled%       }%
%disabled%     }%
%disabled%   \endgroup
%disabled% }
%disabled%  
%disabled% % New command, which works like \lohead but also puts the thumbs (you
%disabled% % cannot use \ihead with this definition but you may change this, if
%disabled% % you use more internal scrpage2 commands)
%disabled% \newcommand*{\loheadwithchapterthumbs}[2][]{%
%disabled%   \lohead[\putchapterthumb#1]{\putchapterthumb#2}%
%disabled% }
%disabled%  
%disabled% % initial use
%disabled% \loheadwithchapterthumbs{}
%disabled% \pagestyle{scrheadings}
%disabled%  
%disabled% }
%disabled% \makeatother

%%%% END
%%% Local Variables:
%%% mode: latex
%%% mode: auto-fill
%%% mode: flyspell
%%% eval: (ispell-change-dictionary "en_US")
%%% TeX-master: "../main"
%%% End:
%% vim:foldmethod=expr
%% vim:fde=getline(v\:lnum)=~'^%%%%'?0\:getline(v\:lnum)=~'^%doc.*\ .\\%(sub\\)\\?section{.\\+'?'>1'\:'1':
# command
%doc% in \texttt{main.tex}.  For standard usage it is recommended to stay with the
%doc% default settings.
%doc% 
%doc% 
%% ========================================================================

%doc%
%doc% Some basic microtypographic settings are provided by the
%doc% \texttt{microtype}
%doc% package\footnote{\url{http://ctan.org/pkg/microtype}}. This template
%doc% uses the rather conservative package parameters: \texttt{protrusion=true,factor=900}.
\usepackage[protrusion=true,factor=900]{microtype}

%doc%
%doc% \subsection{French spacing}
%doc%
%doc% \paragraph{Why?} see~\textcite[p.\,28, p.\,30]{Bringhurst1993}: `2.1.4 Use a single word space between sentences.'
%doc%
%doc% \paragraph{How?} see~\textcite[p.\,185]{Eijkhout2008}:\\
%doc% \verb#\frenchspacing  %% Macro to switch off extra space after punctuation.# \\
\frenchspacing  %% Macro to switch off extra space after punctuation.
%doc%
%doc% Note: This setting might be default for \myacro{KOMA} script.
%doc%


%doc%
%doc% \subsection{Font}
%doc% 
%doc% This template is using the Palatino font (package \texttt{mathpazo}) which results
%doc% in a legible document and matching mathematical fonts for printout.
%doc% 
%doc% It is highly recommended that you either stick to the Palatino font or use the
%doc% \LaTeX{} default fonts (by removing the package \texttt{mathpazo}).
%doc% 
%doc% Chosing different fonts is not
%doc% an easy task. Please leave this to people with good knowledge on this subject.
%doc% 
%doc% One valid reason to change the default fonts is when your document is mainly
%doc% read on a computer screen. In this case it is recommended to switch to a font
%doc% \textsf{which is sans-serif like this}. This template contains several alternative
%doc% font packages which can be activated in this file.
%doc% 

% for changing the default font, please go to the next subsection!

%doc%
%doc% \subsection{Text figures}
%doc% 
%doc% \ldots also called old style numbers such as 0123456789. 
%doc% (German: \enquote{Mediäval\-ziffern\footnote{\url{https://secure.wikimedia.org/wikibooks/de/wiki/LaTeX-W\%C3\%B6rterbuch:\_Medi\%C3\%A4valziffern}}})
%doc% \paragraph{Why?} see~\textcite[p.\,44f]{Bringhurst1993}: 
%doc% \begin{quote}
%doc% `3.2.1 If the font includes both text figures and titling figures, use
%doc%  titling figures only with full caps, and text figures in all other
%doc%  circumstances.'
%doc% \end{quote}
%doc% 
%doc% \paragraph{How?} 
%doc% Quoted from Wikibooks\footnote{\url{https://secure.wikimedia.org/wikibooks/en/wiki/LaTeX/Formatting\#Text\_figures\_.28.22old\_style.22\_numerals.29}}:
%doc% \begin{quote}
%doc% Some fonts do not have text figures built in; the textcomp package attempts to
%doc% remedy this by effectively generating text figures from the currently-selected
%doc% font. Put \verb#\usepackage{textcomp}# in your preamble. textcomp also allows you to
%doc% use decimal points, properly formatted dollar signs, etc. within
%doc% \verb#\oldstylenums{}#.
%doc% \end{quote}
%doc% \ldots but proposed \LaTeX{} method does not work out well. Instead use:\\
%doc% \verb#\usepackage{hfoldsty}#  (enables text figures using additional font) or \\
%doc% \verb#\usepackage[sc,osf]{mathpazo}# (switches to Palatino font with small caps and old style figures enabled).
%doc%
%\usepackage{hfoldsty}  %% enables text figures using additional font
%% ... OR use ...
%\usepackage[sc,osf]{mathpazo} %% switches to Palatino with small caps and old style figures

%% Font selection from:
%%     http://www.matthiaspospiech.de/latex/vorlagen/allgemein/preambel/fonts/
%% use following lines *instead* of the mathpazo package above:
%% ===== Serif =========================================================
%% for Computer Modern (LaTeX default font), simply remove the mathpazo above
%\usepackage{charter}\linespread{1.05} %% Charter
%\usepackage{bookman}                  %% Bookman (laedt Avant Garde !!)
%\usepackage{newcent}                  %% New Century Schoolbook (laedt Avant Garde !!)
%% ===== Sans Serif ====================================================
%\renewcommand{\familydefault}{\sfdefault}  %% this one in *combination* with the default mathpazo package
%\usepackage{cmbright}                  %% CM-Bright (eigntlich eine Familie)
%\usepackage{tpslifonts}                %% tpslifonts % Font for Slides


%doc% 
%doc% \subsection{\texttt{myacro} --- Abbrevations using \textsc{small caps}}\myinteresting
%doc% \label{sec:myacro}
%doc% 
%doc% \paragraph{Why?} see~\textcite[p.\,45f]{Bringhurst1993}: `3.2.2 For abbrevations and
%doc% acronyms in the midst of normal text, use spaced small caps.'
%doc% 
%doc% \paragraph{How?} Using the predefined macro \verb#\myacro{}# for things like
%doc% \myacro{UNO} or \myacro{UNESCO} using \verb#\myacro{UNO}# or \verb#\myacro{UNESCO}#.
%doc% 
\DeclareRobustCommand{\myacro}[1]{\textsc{\lowercase{#1}}} %%  abbrevations using small caps


%doc% 
%doc% \subsection{Colorized headings and links}
%doc% 
%doc% This document template is able to generate an output that uses colorized
%doc% headings, captions, page numbers, and links. The color named `DispositionColor'
%doc% used in this document is defined near the definition of package \texttt{color}
%doc% in the preamble (see section~\ref{subsec:miscpackages}). The changes required
%doc% for headings, page numbers, and captions are defined here.
%doc% 
%doc% Settings for colored links are handled by the definitions of the
%doc% \texttt{hyperref} package (see section~\ref{sec:pdf}).
%doc% 
\setheadsepline{.4pt}[\color{DispositionColor}]
\renewcommand{\headfont}{\normalfont\sffamily\color{DispositionColor}}
\renewcommand{\pnumfont}{\normalfont\sffamily\color{DispositionColor}}
\addtokomafont{disposition}{\color{DispositionColor}}
\addtokomafont{caption}{\color{DispositionColor}\footnotesize}
\addtokomafont{captionlabel}{\color{DispositionColor}}

%doc% 
%doc% \subsection{No figures or tables below footnotes}
%doc% 
%doc% \LaTeX{} places floating environments below footnotes if \texttt{b}
%doc% (bottom) is used as (default) placement algorithm. This is certainly
%doc% not appealing for most people and is deactivated in this template by
%doc% using the package \texttt{footmisc} with its option \texttt{bottom}.
%doc% 
%% see also: http://www.komascript.de/node/858 (German description)
\usepackage[bottom]{footmisc}

%doc% 
%doc% \subsection{Spacings of list environments}
%doc% 
%doc% By default, \LaTeX{} is using vertical spaces between items of enumerate, 
%doc% itemize and description environments. This is fine for multi-line items.
%doc% Many times, the user does just write single-line items where the larger
%doc% vertical space is inappropriate. The \href{http://ctan.org/pkg/enumitem}{enumitem}
%doc% package provides replacements for the pre-defined list environments and
%doc% offers many options to modify their appearances.
%doc% This template is using the package option for \texttt{noitemsep} which
%doc% mimimizes the vertical space between list items.
%doc% 
\usepackage{enumitem}
\setlist{noitemsep}   %% kills the space between items

%doc% 
%doc% \subsection{\texttt{csquotes} --- Correct quotation marks}\myinteresting
%doc% \label{sub:csquotes}
%doc% 
%doc% \emph{Never} use quotation marks found on your keyboard.
%doc% They end up in strange characters or false looking quotation marks.
%doc% 
%doc% In \LaTeX{} you are able to use typographically correct quotation marks. The package 
%doc% \href{http://www.ctan.org/pkg/csquotes}{\texttt{csquotes}} offers you with 
%doc% \verb#\enquote{foobar}# a command to get correct quotation marks around \enquote{foobar}.
%doc% Please do check the package options in order to modify
%doc% its settings according to the language used\footnote{most of the time in 
%doc% combination with the language set in the options of the \texttt{babel} package}.
%doc% 
%doc% \href{http://www.ctan.org/pkg/csquotes}{\texttt{csquotes}} is also recommended 
%doc% by \texttt{biblatex} (see Section~\ref{sec:references}). 
\usepackage[babel=true,strict=true,english=american,german=guillemets]{csquotes}

%doc% 
%doc% \subsection{Line spread}
%doc% 
%doc% If you have to enlarge the distance between two lines of text, you can
%doc% increase it using the \texttt{\mylinespread} command in \texttt{main.tex}. By default, it is
%doc% deactivated (set to 100~percent). Modify only with caution since it influences the
%doc% page layout and could lead to ugly looking documents.
\linespread{\mylinespread}

%doc% 
%doc% \subsection{Optional: Lines above and below the chapter head}
%doc% 
%doc% This is not quite something typographic but rather a matter of taste.
%doc% \myacro{KOMA} Script offers \href{http://www.komascript.de/node/24}{a method to
%doc% add lines above and below chapter head} which is disabled by
%doc% default. If you want to enable this feature, remove corresponding
%doc% comment characters from the settings.
%doc% 
%% Source: http://www.komascript.de/node/24
%disabled% %% 1st get a new command
%disabled% \newcommand*{\ORIGchapterheadstartvskip}{}%
%disabled% %% 2nd save the original definition to the new command
%disabled% \let\ORIGchapterheadstartvskip=\chapterheadstartvskip
%disabled% %% 3rd redefine the command using the saved original command
%disabled% \renewcommand*{\chapterheadstartvskip}{%
%disabled%   \ORIGchapterheadstartvskip
%disabled%   {%
%disabled%     \setlength{\parskip}{0pt}%
%disabled%     \noindent\color{DispositionColor}\rule[.3\baselineskip]{\linewidth}{1pt}\par
%disabled%   }%
%disabled% }
%disabled% %% see above
%disabled% \newcommand*{\ORIGchapterheadendvskip}{}%
%disabled% \let\ORIGchapterheadendvskip=\chapterheadendvskip
%disabled% \renewcommand*{\chapterheadendvskip}{%
%disabled%   {%
%disabled%     \setlength{\parskip}{0pt}%
%disabled%     \noindent\color{DispositionColor}\rule[.3\baselineskip]{\linewidth}{1pt}\par
%disabled%   }%
%disabled%   \ORIGchapterheadendvskip
%disabled% }

%doc% 
%doc% \subsection{Optional: Chapter thumbs}
%doc% 
%doc% This is not quite something typographic but rather a matter of taste.
%doc% \myacro{KOMA} Script offers \href{http://www.komascript.de/chapterthumbs-example}{a method to
%doc% add chapter thumbs} (in combination with the package \texttt{scrpage2}) which is disabled by
%doc% default. If you want to enable this feature, remove corresponding
%doc% comment characters from the settings.
%doc% 
%disabled% \makeatletter
%disabled% % Safty first
%disabled% \@ifundefined{chapter}{\let\chapter\undefined
%disabled%   \chapter must be defined to use chapter thumbs!}{%
%disabled%  
%disabled% % Two new commands for the width and height of the boxes with the
%disabled% % chapter number at the thumbs (use of commands instead of lengths
%disabled% % for sparing registers)
%disabled% \newcommand*{\chapterthumbwidth}{2em}
%disabled% \newcommand*{\chapterthumbheight}{1em}
%disabled%  
%disabled% % Two new commands for the colors of the box background and the
%disabled% % chapter numbers of the thumbs
%disabled% \newcommand*{\chapterthumbboxcolor}{black}
%disabled% \newcommand*{\chapterthumbtextcolor}{white}
%disabled%  
%disabled% % New command to set a chapter thumb. I'm using a group at this
%disabled% % command, because I'm changing the temporary dimension \@tempdima
%disabled% \newcommand*{\putchapterthumb}{%
%disabled%   \begingroup
%disabled%     \Large
%disabled%     % calculate the horizontal possition of the right paper border
%disabled%     % (I ignore \hoffset, because I interprete \hoffset moves the page
%disabled%     % at the paper e.g. if you are using cropmarks)
%disabled%     \setlength{\@tempdima}{\@oddheadshift}% (internal from scrpage2)
%disabled%     \setlength{\@tempdima}{-\@tempdima}%
%disabled%     \addtolength{\@tempdima}{\paperwidth}%
%disabled%     \addtolength{\@tempdima}{-\oddsidemargin}%
%disabled%     \addtolength{\@tempdima}{-1in}%
%disabled%     % putting the thumbs should not change the horizontal
%disabled%     % possition
%disabled%     \rlap{%
%disabled%       % move to the calculated horizontal possition
%disabled%       \hspace*{\@tempdima}%
%disabled%       % putting the thumbs should not change the vertical
%disabled%       % possition
%disabled%       \vbox to 0pt{%
%disabled%         % calculate the vertical possition of the thumbs (I ignore
%disabled%         % \voffset for the same reasons told above)
%disabled%         \setlength{\@tempdima}{\chapterthumbwidth}%
%disabled%         \multiply\@tempdima by\value{chapter}%
%disabled%         \addtolength{\@tempdima}{-\chapterthumbwidth}%
%disabled%         \addtolength{\@tempdima}{-\baselineskip}%
%disabled%         % move to the calculated vertical possition
%disabled%         \vspace*{\@tempdima}%
%disabled%         % put the thumbs left so the current horizontal possition
%disabled%         \llap{%
%disabled%           % and rotate them
%disabled%           \rotatebox{90}{\colorbox{\chapterthumbboxcolor}{%
%disabled%               \parbox[c][\chapterthumbheight][c]{\chapterthumbwidth}{%
%disabled%                 \centering
%disabled%                 \textcolor{\chapterthumbtextcolor}{%
%disabled%                   \strut\thechapter}\\
%disabled%               }%
%disabled%             }%
%disabled%           }%
%disabled%         }%
%disabled%         % avoid overfull \vbox messages
%disabled%         \vss
%disabled%       }%
%disabled%     }%
%disabled%   \endgroup
%disabled% }
%disabled%  
%disabled% % New command, which works like \lohead but also puts the thumbs (you
%disabled% % cannot use \ihead with this definition but you may change this, if
%disabled% % you use more internal scrpage2 commands)
%disabled% \newcommand*{\loheadwithchapterthumbs}[2][]{%
%disabled%   \lohead[\putchapterthumb#1]{\putchapterthumb#2}%
%disabled% }
%disabled%  
%disabled% % initial use
%disabled% \loheadwithchapterthumbs{}
%disabled% \pagestyle{scrheadings}
%disabled%  
%disabled% }
%disabled% \makeatother

%%%% END
%%% Local Variables:
%%% mode: latex
%%% mode: auto-fill
%%% mode: flyspell
%%% eval: (ispell-change-dictionary "en_US")
%%% TeX-master: "../main"
%%% End:
%% vim:foldmethod=expr
%% vim:fde=getline(v\:lnum)=~'^%%%%'?0\:getline(v\:lnum)=~'^%doc.*\ .\\%(sub\\)\\?section{.\\+'?'>1'\:'1':



%% ========================================================================
%%%% MISC usepackages
%% ========================================================================

%% ... it's OK to put here your own usepackage commands ...
\usepackage{subcaption}
\usepackage{amsmath,amssymb,amsfonts,xfrac}
\usepackage{tikz}
\usetikzlibrary{shapes.geometric, arrows}
\usepackage{multirow}
\usepackage{booktabs}


%% ========================================================================
%%%% TikZ packages
%% ========================================================================

% for external filenames
\usetikzlibrary{external}


%% ========================================================================
%%%% Custom colors
%% ========================================================================
% Solarized color scheme: http://ethanschoonover.com/solarized
\definecolor{myyellow}{HTML}{B58900}
\definecolor{myorange}{HTML}{CB4B16}
\definecolor{myred}{HTML}{DC322F}
\definecolor{mymagenta}{HTML}{D33682}
\definecolor{myviolet}{HTML}{6C71C4}
\definecolor{myblue}{HTML}{268BD2}
\definecolor{mycyan}{HTML}{2AA198}
\definecolor{mygreen}{HTML}{859900}
\definecolor{mybase03}{HTML}{002B36}
\definecolor{mybase02}{HTML}{073642}
\definecolor{mybase01}{HTML}{586E75}
\definecolor{mybase00}{HTML}{657B83}
\definecolor{mybase0}{HTML}{839496}
\definecolor{mybase1}{HTML}{93A1A1}
\definecolor{mybase2}{HTML}{EEE8D5}
\definecolor{mybase3}{HTML}{FDF6E3}

%%% TikZ colors
\colorlet{raster}{black}
\colorlet{col0}{mybase03}
\colorlet{col1}{mybase02}
\colorlet{col2}{mybase01}
\colorlet{col3}{mybase0}
\colorlet{col4}{mybase1!75}

\colorlet{fa}{myyellow}
\colorlet{a}{myred}
\colorlet{fda5}{mygreen}
\colorlet{Sfda5}{myviolet}
\colorlet{SSfda5}{mymagenta}
\colorlet{SaSfda5}{myorange}
\colorlet{Sf750}{mybase02}
\colorlet{dunno}{mybase02}
\colorlet{trunc}{mybase02}


%% ========================================================================
%%%% MISC self-defined commands and settings
%% ========================================================================


%%% Text %%%%%%%%%%%%%%%%%%%%%%%%%%%%%%%%%%%%%%%%%%%%%%%%%%%%%%%%%%%%%%%%%%%%%%%
\newcommand{\todow}[1]{\tikz[baseline]{\node[anchor=base, draw=black, fill=mygreen, rounded corners=3pt, inner sep=2pt] {#1};}}
\newcommand{\todoi}[1]{\todo[inline]{#1}}
\newcommand{\fixme}[1]{\textcolor{mygreen}{\uwave{#1}} }

%%%%%%%%%%%%%%%%%%%%%%%%%%%%%%%%%%%%%%%%%%%%%%%%%%%%%%%%%%%%%%%%%%%%%%%%%%%%%%%
\newcommand{\myLaT}{\LaTeX{}@TUG\xspace} %% LaTeX@TUG text "logo"

\newcommand{\etal}{\textit{et~al}. }
\newcommand{\ie}{\textit{i}.\textit{e}., }
\newcommand{\eg}{\textit{e}.\textit{g}. }

\hyphenation{ex-am-ple hy-phen-ate}  %% in order to use German umlauts
%% here (Ver-\"of-fent-li-chung), you have to check for
%% activated \usepackage[T1]{fontenc} in the preamble

%% override default language of babel: (be sure to know, what you're
%% doing here)
%\selectlanguage{american}
%\selectlanguage{ngerman}

%% ========================================================================
%%%% Templates
%% ========================================================================

%% template for inserting figures:
% \myfig{}%% filename
%       {}%% width/height
%       {}%% caption
%       {}%% optional (short) caption for list of figures
%       {fig:}%% label

%% acronyms in small caps: \myacro{UNESCO}


%%%% Time-stamp: <2014-03-23 13:40:59 vk>
%%%% === Disclaimer: =======================================================
%% created by
%%
%%      Karl Voit
%%
%% using GNU/Linux, GNU Emacs & LaTeX 2e
%%

%doc%
%doc% \section{\texttt{pdf\_settings.tex} --- Settings related to PDF output}
%doc% \label{sec:pdf}
%doc% 
%doc% The file \verb#template/pdf_settings.tex# basically contains the definitions for
%doc% the \href{http://tug.org/applications/hyperref/}{\texttt{hyperref} package}
%doc% including the
%doc% \href{http://www.ctan.org/tex-archive/macros/latex/required/graphics/}{\texttt{graphicx}
%doc% package}. Since these settings should be the last things of any \LaTeX{}
%doc% preamble, they got their own \TeX{} file which is included in \texttt{main.tex}.
%doc% 
%doc% \paragraph{What should I do with this file?} The settings in this file are
%doc% important for \myacro{PDF} output and including graphics. Do not exclude the
%doc% related \texttt{input} command in \texttt{main.tex}. But you might want to
%doc% modify some settings after you read the
%doc% \href{http://tug.org/applications/hyperref/}{documentation of the \texttt{hyperref} package}.
%doc% 


%% Fix positioning of images in PDF viewers. (disabled by
%% default; see https://github.com/novoid/LaTeX-KOMA-template/issues/4
%% for more information) 
%% I do not have time to read about possible side-effect of this
%% package for now.
% \usepackage[hypcap]{caption}

%% Declarations of hyperref should be the last definitions of the preamble:
%% FIXXME: black-and-white-version for printing!

\pdfcompresslevel=9

\usepackage[%
unicode=true, % loads with unicode support
%a4paper=true, %
pdftex, %
backref, %
pagebackref=false, % creates backward references too
bookmarks=false, %
bookmarksopen=false, % when starting with AcrobatReader, the Bookmarkcolumn is opened
pdfpagemode=UseNone,% UseNone, UseOutlines, UseThumbs, FullScreen
plainpages=false, % correct, if pdflatex complains: ``destination with same identifier already exists''
%% colors: https://secure.wikimedia.org/wikibooks/en/wiki/LaTeX/Colors
urlcolor=DispositionColor, %%
linkcolor=DispositionColor, %%
%pagecolor=DispositionColor, %%
citecolor=DispositionColor, %%
anchorcolor=DispositionColor, %%
colorlinks=\mycolorlinks, % turn on/off colored links (on: better for
                          % on-screen reading; off: better for printout versions)
]{hyperref}

%% all strings need to be loaded after hyperref was loaded with unicode support
%% if not the field is garbled in the output for characters like ČŽĆŠĐ
\hypersetup{
pdftitle={\mytitle}, %
pdfauthor={\myauthor}, %
pdfsubject={\mysubject}, %
pdfcreator={Accomplished with: pdfLaTeX, biber, and hyperref-package. No animals, MS-EULA or BSA-rules were harmed.},
pdfproducer={\myauthor},
pdfkeywords={\mykeywords}
}

%\DeclareGraphicsExtensions{.pdf}

%%%% END
%%% Local Variables:
%%% TeX-master: "../main"
%%% mode: latex
%%% mode: auto-fill
%%% mode: flyspell
%%% eval: (ispell-change-dictionary "en_US")
%%% End:
%% vim:foldmethod=expr
%% vim:fde=getline(v\:lnum)=~'^%%%%'?0\:getline(v\:lnum)=~'^%doc.*\ .\\%(sub\\)\\?section{.\\+'?'>1'\:'1':
  %% should be *last* definitions in preamble!
%%% Autoref Fixes %%%%%%%%%%%%%%%%%%%%%%%%%%%%%%%%%%%%%%%%%%%%%%%%%%%%%%%%%%%%%%
\def\algorithmautorefname{Algorithm}
%\def\itemautorefname{Step}
\def\chapterautorefname{Chapter}
\def\sectionautorefname{Section}
\def\subsectionautorefname{Section}
\def\subsubsectionautorefname{Section}

%% ========================================================================
%%%% begin{document}
%% ========================================================================
\begin{document}

\frontmatter                    %% KOMA: roman page numbers and such; only available in scrbook

%%%% Time-stamp: <2013-03-18 14:35:00 vk>
%% ========================================================================
%%%% Disclaimer
%% ========================================================================
%%
%% created by
%%
%%      Karl Voit


\newcommand{\mycolophon}{%%
  This document 
  %% was written with \myacro{GNU}~Emacs, 
  is set in Palatino, compiled with
  \href{http://LaTeX.TUGraz.at}{pdf\LaTeX2e} and
  \href{http://en.wikipedia.org/wiki/Biber_(LaTeX)}{\texttt{Biber}}.

  The \LaTeX{} template from Karl Voit is based on
  \href{http://www.komascript.de/}{KOMA script} and can be found 
  online: \href{https://github.com/novoid/LaTeX-KOMA-template}{https://github.com/novoid/LaTeX-KOMA-template}
}


%%% Local Variables: 
%%% mode: latex
%%% mode: auto-fill
%%% mode: flyspell
%%% eval: (ispell-change-dictionary "en_US")
%%% TeX-master: "../main"
%%% End: 
                %% defines information about editor, LaTeX, font, ...

%% Choose your desired title page:
\input{\mytitlepage}            %% include title page


\cleardoublepage
%%%% Time-stamp: <2017-02-14 16:01:12 vk>
%% ========================================================================
%%%% Disclaimer
%% ========================================================================
%%
%% created by
%%
%%      Karl Voit, and Matthias Wölbitsch
%%
%%
%% code for the date and signature fields adapted from
%% http://tex.stackexchange.com/a/20562


\newcommand{\textfield}[2]{
  \vbox{
    \hsize=#1\kern3cm\hrule\kern1ex
    \hbox to \hsize{\strut\hfil\footnotesize#2\hfil}
  }
}


\ifthenelse{\boolean{english_affidavit}}{
  \section*{Affidavit}
  I declare that I have authored this thesis independently, that I have
  not used other than the declared sources/resources, and that I have
  explicitly indicated all material which has been quoted either
  literally or by content from the sources used. The text document
  uploaded to \myacro{TUGRAZ}online is identical to the present master's
  thesis.

  \hbox to \hsize{\textfield{4cm}{Date}\hfil\hfil\textfield{7cm}{Signature}}
}{
  \section*{Eidesstattliche Erklärung}
  \foreignlanguage{ngerman}{%
    Ich erkläre an Eides statt, dass ich die vorliegende Arbeit
    selbstständig verfasst, andere als die angegebenen
    Quellen/Hilfsmittel nicht benutzt, und die den benutzten Quellen
    wörtlich und inhaltlich entnommenen Stellen als solche kenntlich
    gemacht habe. Das in \myacro{TUGRAZ}online hochgeladene Textdokument
    ist mit der vorliegenden Dissertation identisch.}

    \hbox to \hsize{\textfield{4cm}{Datum}\hfil\hfil\textfield{7cm}{Unterschrift}}
}

\newpage


%%% Local Variables:
%%% mode: latex
%%% mode: auto-fill
%%% mode: flyspell
%%% TeX-master: "../main"
%%% End:
  %% Statutory Declaration
% \input{thanks}                %% this is a suggestion: you have to create this file on demand
% \input{foreword}              %% this is a suggestion: you have to create this file on demand


%% include the abstract without chapter number but include it on table of contents (is done in abstract.tex)
\cleardoublepage
%%%% Time-stamp: <2013-02-25 10:31:01 vk>

\phantomsection
\addcontentsline{toc}{chapter}{Abstract}
\chapter*{Abstract}
\label{cha:abstract}

Microarchitectural attacks exploit design choices made by hardware vendors.
Cache attacks, or attacks like Meltdown and Spectre, target the CPU.
Microarchitectural attacks either target design choices or exploit bugs
introduced by mistakes in the hardware or the microcode. Rowhammer attacks
exploit a design flaw in DRAM chips. Hardware vendors steadily decreased sizes
and lowered refresh rates of the cells. With specially crafted memory access
routines it is possible to access DRAM cells fast enough to create interference
between cells, which will cause them to change their load, hence, changing their
logical state.

Rowhammer attacks use this phenomenon to target memory areas where bitflips
would benefit the attacker. Researchers have shown that bitflips in page tables
can cause privilege escalation and bypass security mechanisms implemented in
modern operating systems. It has also been shown that similar attacks work for
memory chips used by solid-state disks. Besides flipping bits in page tables,
researchers showed that when flipping bits in executables, the program's
behaviour can change. With such changes, it is possible for an attacker to abuse
programs for privilege escalation, or to bypass authentication-checks.

In this thesis, we present a way of testing binaries for possible execution-path
changes introduced by bitflips. In our work, we pre-define an outcome for a
program and then search for all single bitflips which cause the program to
behave in the desired way. We scan the entire address space of the program,
including all dynamically loaded libraries. We show results for searched
bitflips in programs used for authentication on Linux-based systems. We bypass
user privilege checks, which lead to privilege escalation, or make login
possible without knowing the user\textquotesingle s password. We also show a
bypass of HTTP basic authentication, allowing an attacker to download files
which unauthenticated users are not allowed to access. In addition to searching
for bitflips in executable files, we also look at possible other attack vectors
for Rowhammer. We show that bitflips applied to the runtime of cryptographic
calculations can break assumptions made by the communicating parties and can
even allow key leakage. We apply bitflips to the implementation of AES-GCM in
OpenSSL and show how Rowhammer can be used to cause reusing of nonces.

With our work, we want to increase the awareness of Rowhammer and show how
software security is affected by bitflips. We call on to all vendors of hardware
to not forget to keep their systems secure and do not put lower prices and
higher performance ahead of security, which would harm their users.
\cleardoublepage
\phantomsection
\addcontentsline{toc}{chapter}{Kurzfassung}
\begin{otherlanguage}{ngerman}
\chapter*{Kurzfassung}
\label{cha:kurzfassung}

Angriffe auf die Mikroarchitektur zielen meistens auf Fehler von
Hardwareherstellern ab. Attacken auf Caches nutzen hierbei ein gewolltes
Verhalten des Systems aus. CPU Lücken wie Meltdown und Spectre machen
sich Fehler im Design der Hardware zu Nutzen, welche vom Hersteller
entweder durch Mikrocode Updates oder einem Tausch der CPU berichtigt werden
müssen. Rowhammer macht sich einen Designfehler in DRAM Chips zu Nutze.
Hersteller dieser Speicherchips produzieren immer kleinere Chips mit
geringeren Refresh-Zyklen. Spezielle Reihenfolgen von Speicherzugriffen
ermöglichen es Interferenzen zu erzeugen welche Ladungsänderungen in
benachbarten Speicherzellen verursachen, diese können dadurch ihren logischen
Zustand wechseln.

Rowhammer Angriffe machen sich dieses Verhalten zu Nutze und zielen damit auf
Speicherbereiche ab, welche durch eine Änderung dem Angreifer einen Vorteil
verschaffen. Forscher haben gezeigt, dass es möglich ist mit Bitflips in Page
Tables Privilegien einen Superusers zu bekommen. Ebenso wurde gezeigt, dass
ähnliche Angriffe auch auf Speicherchips in Solid-State-Disks möglich sind.
Neben Flips in Page Tables wurde auch gezeigt, dass Änderungen in ausführbaren
Files Folgen auf das Verhalten des Programms haben, sie ändern dieses durch das
Wechseln eines einzigen Bits. Dies kann zum Beispiel dazu führen dass
Berechtigungsüberprüfungen umgangen werden.

Wir zeigen eine Möglichkeit Programme auf solche Verhaltensänderungen durch
Bitflips zu testen. Wir geben hier ein gewünschtes Verhalten vor und suchen
danach nach allen möglichen Bitflips welche das Verhalten in den gewünschten
Zustand ändern. Im Gegensatz zu früheren Arbeiten haben wir diesen Vorgang
automatisiert um eine größere Anzahl von Programmen abzudecken, zusätzlich
betrachten wir den gesamten Speicher in ausführbaren Files und dynamisch
geladenen Software-Bibliotheken. Wir zeigen die gefundenen Bitflips welche
Sicherheitsüberprüfungen umgehen, wie zum Beispiel Passwortabfragen. Auch
zeigen wir Bitflips, welche es erlauben HTTP-Authentication Checks in einem
Webserver zu umgehen. Zusätzlich zur Untersuchung von statischen Files
betrachten wir auch die Auswirkung von Bitflips auf kryptographische
Algorithmen. Hier untersuchen wir, wie Rowhammer dazu genutzt werden kann um in
der AES-GCM Implementation von OpenSSL eine falsche Verwendung von Noncen zu
verursachen.

Unsere Arbeit soll auf die Sicherheitsrisiken die durch Fehler wie Rowhammer
entstehen hinweisen und als Aufruf an die Hersteller von Hardware dienen, damit
diese nicht billigere Hardware und bessere Performance über die Sicherheit
ihrer Anwender stellen.

\end{otherlanguage}

%\glsresetall %% all glossary entries should be used in long form (again)
%% vim:foldmethod=expr
%% vim:fde=getline(v\:lnum)=~'^%%%%\ .\\+'?'>1'\:'='
%%% Local Variables:
%%% mode: latex
%%% mode: auto-fill
%%% mode: flyspell
%%% eval: (ispell-change-dictionary "en_US")
%%% TeX-master: "main"
%%% End:
              %% Abstract


\tableofcontents                %% this produces the table of contents - you might have guessed :-)

% \listoffigures
% \listoftables

%% if myaddlistoftodos is set to "true", the current list of open todos is added:
\ifthenelse{\boolean{myaddlistoftodos}}{
  \newpage\listoftodos          %% handy if you are using todonotes with \todo{}
}{}                             %% with todonotes-package option "disable" you can get rid of any todo in the output

\mainmatter                     %% KOMA: marks main part using arabic page numbers and such; only available in scrbook


% include tex file chapters:
%%%%%% Introduction %%%%%%%%%%%%%%%%%%%%%%%%%%%%%%%%%%%%%%%%%%%%%%%%%%%%%%%%%{{{
\chapter{Introduction}\label{sec:intro}

With the latest releases of Microarchitectural attacks like
Meltdown~\cite{meltdown} and Spectre~\cite{spectre}, the topic of flaws in
hardware implementations became known to the general public. Many media outlets
reported on these issues of modern CPUs. Where it was mostly vendors of
x86-architecture CPUs like Intel or AMD, also ARM, and with it, most mobile
devices are affected by such flaws.

Issues like these show that these vendors have set performance above security by
neglecting quality management and testing. The demand for better releases of
hardware is rising all the time, and not only vendors of CPUs are affected by
this demand. Another field of silicon chip design ran into a similar problem in
the past, namely DRAM chip vendors.  

In 2015, Kim~\etal~\cite{rowhammergeneral} released their paper ``RowHammer'',
showing how specially crafted memory access routines can cause bits in DRAM
chips to flip, without accessing them directly. This work showed how the demand
for higher memory density caused faults where interfering voltages and leaking
currents to influence other memory storage cells. While at first, this was just
seen as a stability issue, Google\textquotesingle s Project Zero showed how
Rowhammer can be used for privilege escalation and sandbox
escapes~\cite{projectzerorow}. With reports like this, the interest in
researching the field of Rowhammer increased. Gruss~\etal~\cite{rowhammerjs}
showed that it is not only possible to target systems by executing native code,
but also that Rowhammer can be triggered by using JavaScript. Van der
Veen~\etal\cite{drammer} published their work named ``Drammer'', where they show
how not only desktop computers are affected by the Rowhammer bug, but also
mobile devices. Earlier this year, Gruss~\etal~\cite{nethammer}, released a way
to trigger bitflips by only sending specially-crafted network requests.
Publications like these show that Rowhammer is an active research topic, where
still new findings come up.

Our work builds on work released by Gruss~\etal~\cite{flipinthewall}, where they
showed that application code can directly be attacked with Rowhammer. They show
that a bitflip applied to \texttt{sudo} can result in a bypass of the password
check. They show some bitflips causing such a bypass. They look at the
disassembly of the authentication check code and find opcodes which when changed
would result in a different outcome. With our work, we want to automate, and
therefore simplify, this process. We want to find a higher number of bitflips in
a shorter time. In addition to that, we want to provide a toolset, which allows
us to apply similar searches to other applications. We want to run lots of tests
in parallel and verify the outcomes. Therefore we want to make use of modern
testing techniques.

Testing and debugging were always a significant part of software technology, and
with rising sizes of projects and an increasing number of old code bases, it is
more vital than ever. Not only developers are putting much work into these
topics but also researchers releasing new ways of testing regularly. With modern
approaches for testing like fuzzying, bug searching in unknown code got more
successful. Also, the field of proving software\textquotesingle s correctness
got much attention. With symbolic execution techniques, the possibility to prove
each software state on its own got more practical. The release of the open
source symbolic execution framework \texttt{angr}~\cite{angrpaper} made it
possible for a wide range of users to apply symbolic execution to programs. This
tool mostly gets used in testing, in combination with fuzzing, but also security
researchers use \texttt{angr} to find exploitable code segments and execution
paths.

Understanding what programs do, and how they are executed by the CPU, gets
harder with every improvement and change in hardware design. Instrumentation is
a technique to inject code to programs providing the possibility to collect
runtime information. With tools like Intel Pin~\cite{pintool} it is possible to
check changes to processor registers, log accessed memory and performance
measuring at machine code level.

\section{Goals and Motivation for the Thesis}

As we know from previous work done by Gruss~\etal~\cite{flipinthewall}, there
are bitflips in the ELF files loaded by the \texttt{sudo} program which allow
privilege escalation by providing a password check bypass. They only looked at
the binary section providing the permission check. However, they could not claim
to find all flips, and their approach is very time-consuming. We want to
simplify the search by automatic testing of flips. Also, we also want to make it
easier for future applications to be tested for possible bitflip outcomes by
providing a test framework.

Common Unix-based operating systems use package management to roll-out
applications to users. Every instance of the operating system then uses the same
binary to execute. With this in mind, a bitflip found in the \texttt{sudo}
application distributed with Debian, can be used to attack all instances of that
installation. An attacker, therefore, can use the test framework to find
bitflips in widely distributed binaries.

With our work, we want to provide an easy-to-apply framework to search for
bitflips providing a pre-defined outcome. To show how this framework works we
apply it to real-world applications and compare our results to the ones reported
by Gruss~\etal~\cite{flipinthewall}. We want to show how likely such bitflips
are in applications.

\section{Contributions of this Work}

Our contribution to the field of microarchitectural attacks and Rowhammer is
providing a practical analysis of real-world applications and how bitflips can
affect them. We present a framework which can be used to find bitflips changing
a program\textquotesingle s behaviour to a pre-defined outcome. The structure of
the framework is designed to be extensible and adaptable for multiple purposes.

We apply our tool to real-world applications to show the impact bitflips could
have on users of personal computers. On one hand, we show how privilege
escalation bitflips can be found in the \texttt{sudo} program. We show bits,
which when flipped, allow us to skip the password check. On the other hand, we
also analyse the popular \texttt{nginx} web server. For this application, we
show bitflips which permit an attacker to bypass HTTP authentication measures.
We present results for these two applications and if there exist bits, which
when flipped, allow us to achieve our set outcome. Besides analysing the bits
inside the program\textquotesingle s executable, we also examine any dynamically
loaded library it uses. By that, we also cover possibilities where external
functions could change the application\textquotesingle s outcome.

In addition to that, we look at possible cryptographic vulnerabilities
introduced by bitflips. As a basis, we took the work by
Böck~\etal~\cite{gcmnonceattack}, who showed how web servers were misusing
nonces when using AES-GCM. We build on their approach to bypass the fixes
applied by server software to re-introduce this nonce misusage via bitflips.
Here we look at the current implementation of AES-GCM in the TLS library
OpenSSL, used by most web servers. We show that nonce misusage can be
reintroduced by bitflips and give a probability for them.

\section{Outline of this Work}

This thesis is structured as follows: In section~\ref{sec:general}, we describe
general terms and technologies our work is built on or makes use of. We discuss
other microarchitectural attacks and give an overview of the functionality of
programs which our work targets. In section~\ref{sec:elfattack}, we discuss our
work regarding the automatic bitflip search. We show the tested programs, what
additions had to be made for testing and present the found bitflips. In
section~\ref{sec:dynattack}, we discuss our work regarding Rowhammer attacks
targeting dynamic memory. We thereby show how the OpenSSL implementation of
AES-GCM can be attacked by flipping bits. In section~\ref{sec:countermeasure},
we discuss countermeasures which could be applied to improve system security. We
discuss countermeasures against microarchitectural attacks in general, and
discuss on what could be done to reduce the impact of our testing. In
section~\ref{sec:futurework}, we show possible future works, and an overview of
possible directions the research in the field of microarchitectural attacks
could take. In section~\ref{sec:conclusion}, we close our thesis with a summary
and give a conclusion of our work.

\section{Merge to 1.0}

We present this work separated into different chapters. We start with a general
overview of topics in this field. Beginning with describing how programs get
executed on modern computers and how operating systems handle applications. For
this, we describe the general design of executables on Unix-like systems, as we
discuss the executable and linkable format (ELF) in more detail.

As ELF files hold machine code executed by CPUs, and our work relies heavily on
how machine code is built, we discuss the design of instructions in modern
CPU architectures.  We then go on and look at different testing techniques in
software development, where we compare and describe several options and check
their advantages and disadvantages. In particular, we work out details about
fuzzing, symbolic execution and instrumentation. For our work, we try to change
the behaviour of programs by modifying their execution path, most of the time
this behaviour is changed to gain improved privileges. Therefore, we discuss the
permission model used by common Unix-based operating systems.

We also look at details of how permission switches work on these systems,
especially of how the \texttt{setuid} property of executables works. For
permission separation and testing purposes we also take a look at the
\texttt{chroot} environment provided by most Unix-like operating systems. We
do not only target behaviour changes to gain a higher privilege but also target
changes to bypass permission checks.

Moreover, besides local attacks, we also look at remote possibilities. We
describe the networking and security principles used by most computers and
servers. We look at common web servers, TLS libraries and how they provide
security for users. As our work targets cryptographic implementations, we take
a detailed look at the advanced encryption standard (AES) and a variant of it
using Galois/counter mode (AES-GCM).

Our work relies heavily on Rowhammer, which is a software-based
microarchitectural attack. We take a look at these attacks in general and give
an overview of state of the art attacks using similar techniques. As timing
plays a vital role in exploiting side-channels in this area, we describe precise
timing measurement methods. We look at recent cache attacks and the impact
those have on modern systems. We close our background overview with a detailed
description of the Rowhammer bug.

We continue with describing our distribution. This is split into two parts, one
being our testing of bitflips in ELF files and the other being attacks against
cryptographic functions during runtime. We start with the ELF analysis and
describe what impact a single bitflip can have to the execution paths of a
program. We go on with outlining possibilities to find bitflips which would
change the behaviour in a manner so that it benefits an attacker. We describe
the design of our automated bitflip-search framework and how we applied it to
real-world applications. We resume by showing the results of our tests for
the applications of \texttt{sudo} and \texttt{nginx}. We also mention how we
would apply these bitflips to a system by using the Rowhammer bug.

In the second part of our contribution we describe the influence of bitflips on
cryptographic implementations, we discuss the problems of nonce misuse and how
this problem could be introduced to AES-GCM implemented in OpenSSL with
bitflips. We give numbers for a likelihood of such a nonce-misuse introduced by
a single bit flip. We show how we tested this issue in a practical setup with a
simple web server using the OpenSSL library.

We go on with looking at countermeasures for issues in the field of
microarchitectural attacks. We look at possible ways to prevent cache attacks
and Rowhammer. We also line out ways of reducing the impact of our tests on
real-world setups. We close this section by describing how general system
security could be improved.

As microarchitectural attacks are an on-going and growing research field, we
also look at possible future works. We describe a possible future of these
attacks in general, how open source architectures could help to prevent some of
them. We also talk about a possible attribution of machine learning to help to
find new attack vectors. We also find possibilities to improve the impact of
Rowhammer, by looking at new attacks coming by applying it to further
implementations of cryptographic algorithms. In the end, we talk about how our
framework could be helpful for future work and how it can be used in various
testing and research environments with just small code changes.

We close our thesis by giving a conclusion describing our results, provide a
summary of the outcome of our work and point out again how vital secure hardware
is as a basis for secure systems. Therefore we recommend hardware vendors to
improve their quality checks and by this improve general security for users.
%}}}

%% vim:foldmethod=expr
%% vim:fde=getline(v\:lnum)=~'^%%%%\ .\\+'?'>1'\:'='
%%% Local Variables:
%%% mode: latex
%%% mode: auto-fill
%%% mode: flyspell
%%% eval: (ispell-change-dictionary "en_US")
%%% TeX-master: "main"
%%% End:

%%%%%% General %%%%%%%%%%%%%%%%%%%%%%%%%%%%%%%%%%%%%%%%%%%%%%%%%%%%%%%%%%%%%%{{{
\chapter{Background}\label{sec:general}

This chapter describes the terms, techniques, tools and programs used during the
work for this thesis. It provides a general overview to make it easier to
understand the details of the work in the following chapters. We start by
showing how modern computers execute programs and code, by looking at file
formats needed to load programs and how common computing architectures structure
their machine code. We then look at how programs and code is tested and what new
technologies evolved in that matter. As we deal with Unix based systems, we take
a look at the permission system used by those.
Additionally, we face topics in secure communication. Therefore we also describe
the basic principles of network connections, web servers and their security. The
thesis provides automation for searching for bits to flip with rowhammer,
because of this we take a look at this kind of attack and how it works and also
look at microarchitectural attacks in general.

\section{Executing Programs}

Processing units (PU) execute machine code. Any device running programs contains
at least one. Usually, the central processing unit (CPU) runs programs and
triggers other PUs to run other tasks if needed. Often referred to as the main
processor, the CPU carries out the instruction given to it by the machine code
representing a program. A PU can only step over single instructions and execute
them one by one. This is done by moving the instruction pointer to different
memory locations. Simple processors may only execute one single program. On
ordinary desktop computers, the operating system is the main program which can
load other programs as processes and manage them. CPUs can only execute machine
code, but programs usually are written in programming languages which are then
translated to machine language by either a compiler or an interpreter. We refer
to files created by a compiler as executables or binaries. These files can
directly be loaded into memory by the OS. Operating Systems use defined
structures for executables to make this loading possible, for example,
Unix/Linux systems use ELF and Windows uses PE.

\subsection{CPU Instructions}

Machine code, in general, is a list of instructions, encoded in some binary
format. Usually, an instruction consists of an operation code (opcode), telling
the processor what to do, and parameters for the opcode, which the processor
uses for the operation. Instructions can operate on registers or locations in
memory directly. Available instructions and their design are depending on the
architecture used. The most common instruction set architectures (ISA) are Intel
$x86$, mostly referred to as \texttt{x86\_64} or \texttt{AMD64}, and ARM ISA
which is used by most mobile processors.

\begin{table}[]
\centering
\begin{tabular}{ccccccc}
\cline{2-7}
\multicolumn{1}{c|}{} & \multicolumn{1}{c|}{Prefix} &
\multicolumn{1}{c|}{Opcode} & \multicolumn{1}{c|}{ModR/M} &
\multicolumn{1}{c|}{SIB} & \multicolumn{1}{c|}{Displacement} &
\multicolumn{1}{c|}{Immediate} \\ \cline{2-7}
bytes         & 1 per prefix                & 1, 2 or 3
 & 0 or 1                      & 0 or 1                   & 0, 1, 2 or 4
             & 0, 1, 2 or 4
\end{tabular}
\caption{Instruction Format for Intel 64 and IA-32 architectures. It shows how
many different instruction lengths are possible in modern architectures and
also which parts they may contain.}
\label{tab:instrfor}
\end{table}

Table~\ref{tab:instrfor} shows the instruction format for Intel 64 and IA-32
Architectures. The format splits an instruction into the following sections:

\begin{itemize}
  \item Prefix - Used to give the CPU further instructions on how to handle the
opcode. There are prefixes for memory locks in a multi-core environment or
repeat instructions to apply an operation on each byte in a string or I/O data.
There are also prefixes overriding the operand sizes or introducing SSE
instruction calls, or even others give branching hints to the CPU.
  \item Opcode - Tells the CPU what to do and how to handle the following
parameters.
  \item ModR/M - This byte is used behind the opcode to tell the CPU what
addressing mode to use for memory accesses.
  \item SIB - For some ModR/M addressing-modes, this byte also needs to be
parsed and evaluated.
  \item Displacement - Some addressing-modes referred in the ModR/M or SIB byte
might include a displacement following them.
  \item Immediate - Immediate data might follow if referred in previous
parameters.
\end{itemize}

We refer to the Intel Manual~\cite{intelsys} section 2.1 for further details on
the instruction format used in \texttt{x86\_64}.

\subsection{Executable and Linkable Format (ELF)}

ELF is an object file format. It is used to describe a program in a way to load
a program into memory to make it execution ready for the processor without
applying changes to the binary itself. A linker creates ELF files after an
assembler turned the program's source code into machine code. Most modern
Unix-like operating systems follow the ELF specification released by the Tool
Interface Standard Committee~\cite{elfspec}. Therefore, we need to understand
the layout of such executable files before introducing bitflips to them.

\subsubsection{Structure of an ELF file}

\begin{figure}
  \centering
  \begin{tikzpicture}
  \node (eh) [block] {ELF Header};
  \node (ph) [block, right of=eh, xshift=3cm] {Program Header Table};
  \node (sh) [block, below of=eh, yshift=-2cm] {Section Header Table};
  \node (dot0) [dot, below of=ph] {};
  \node (dot1) [dot, below of=dot0, yshift=0.5cm] {};
  \node (text) [block, below of=dot1] {.text};
  \node (rodata) [block, below of=text] {.rodata};
  \node (data) [block, below of=rodata] {.data};
  \node (bss) [block, below of=data] {.bss};
  \node (dot2) [dot, below of=bss] {};
  \node (dot3) [dot, below of=dot2, yshift=0.5cm] {};
  \node (seg02) [segblock, right of=text, xshift=1cm, yshift=-0.5cm] {02};
  \node (dot4) [dot, above of=seg02] {};
  \node (dot5) [dot, above of=dot4, yshift=-0.5cm] {};
  \node (seg03) [segblock, right of=data, xshift=1cm, yshift=-0.5cm] {03};
  \node (dot6) [dot, below of=seg03] {};
  \node (dot7) [dot, below of=dot6, yshift=0.5cm] {};
  % connections
  \path [thick,->,>=stealth] (eh) edge[out=0, in=180] (ph);
  \path [thick,->,>=stealth] (eh) edge[out=180, in=180] (sh);
  \path [thick,->,>=stealth] (sh) edge[out=0, in=180] (text);
  \path [thick,->,>=stealth] (sh) edge[out=0, in=180] (rodata);
  \path [thick,->,>=stealth] (sh) edge[out=0, in=180] (data);
  \path [thick,->,>=stealth] (sh) edge[out=0, in=180] (bss);
  \path [thick,->,>=stealth] (ph) edge[out=0, in=0] (seg02);
  \path [thick,->,>=stealth] (ph) edge[out=0, in=0] (seg03);
  \path [thick,->,>=stealth] (seg02) edge[out=180, in=0] (text);
  \path [thick,->,>=stealth] (seg02) edge[out=180, in=0] (rodata);
  \path [thick,->,>=stealth] (seg03) edge[out=180, in=0] (data);
  \path [thick,->,>=stealth] (seg03) edge[out=180, in=0] (bss);
  \end{tikzpicture}
  \caption{Structure of an ELF file showing the four main sections inside an
executable. The Figure shows the connection of between sections and segments via
the program header table and the section header table. Segments usually are just
numbers and can refer to multiple sections inside the section header table.}
\label{fig:elfstruct}
\end{figure}

There are two views of an ELF file, the linking view and the execution view.
Those are also called section and segment view respectively. They share the same
ELF header but serve different purposes for the operating system. Figure
~\ref{fig:elfstruct} show the connection between sections and segments via the
two different headers. Sections and segments can be seen as follows:

\paragraph{Sections} describe the binary for the linking view. A section
contains instructions, data, symbol table and relocation information. Sections
reserved for the system start with a dot, there might be additional sections
defined by the user. Sections are created and managed by the linker. Sections
that are directly loaded into the program's memory image are initialized data
(\texttt{.data, .data1}), read-only data (\texttt{.rodata, .rodata1}) and
executable instructions (\texttt{.text}).

\paragraph{Segments} describe the virtual memory layout of a loaded binary.
Figure~\ref{fig:elfstruct} shows how segments are applied in an ELF file.
Usually, the linker splits segments into different behaviours the loader needs
to apply. For example, all read-only data sections are in the same segment. When
loading the image into memory references inside the ELF file need to be resolved
and loaded into the memory too. After successfully loading the image and its
dependencies the program can be executed.

\subsubsection{Loading ELF files into Memory}

As we focus on GNU/Linux operating systems, we will look at how UNIX System V
Release 4 based operating systems handle ELF files in order to create running
programs. Operating systems use no physical addresses for execution, and the OS
is free to change the position of sections in the virtual address space.
Therefore, in an ELF, a section only contains a base address and offsets to that
address. During the loading step, the operating system is free to change the
base address in the programs virtual address space. The loader then places
additional memory according to these offsets.

\paragraph{Dynamic Linking} is needed because it is widespread to have multiple
functions used by different programs. Shared libraries can be used to provide
such general functions. When deploying programs, developers usually make sure
all needed shared libraries exist on the target platform or ship their software
with those included. When depending on libraries by the system, the loader will
move the shared libraries into memory at the desired entry point. The loader
fetches the information about the needed library from the ELF file and then
checks the library paths provided by the operating system if the library is
available to be loaded.

\section{Analysis and Testing of Executables}

Testing is a significant part of software development. There are a lot of
different approaches and styles for testing. One approach to test a program is
to apply input to it and check if the code operates as expected by validating
the output. With increasing amount of code and an increasing number of bugs
found, the style of testing changed over the years. On the one hand, developers
nowadays provide unit tests inside their code to test single functions and make
sure they work as desired. On the other hand, sometimes the source code is not
available to a tester, or it is way too much work to read the whole source code
to such tests. Therefore, developers came up with tools which can be used to
test binaries on their own without knowing the source code.

\subsection{Fuzzing}

Fuzzing is an approach for testing programs with just a little knowledge about
the program to test. Such an approach is called black box testing. Fuzzing is a
technique to detect erroneous responses of a program by providing different,
automatically created, mostly random, inputs. Tools applying this technique are
called fuzzers. Fuzzers try to reach corner cases in programs without knowing
which exist and detect undefined behaviour or crashes. Generally, fuzzers give
an overall view on the robustness of a program, they are cheap to apply and even
can find bugs usual white box tests would miss.

If a fuzzer only applies random inputs, testing would take a very long time, as
a lot of unneeded tests would run. Because of this, the technology improved over
the time and different fuzzers evolved. For simple programs just testing common
input mistakes, like negative numbers, newlines, end of file characters, format
strings or just very long strings might be enough. Advanced software needs
better ways of fuzzing, for example, network protocols, file systems or image
formats have a very complex structure which makes it hard to find possible
crashes by random data input. Fuzzers for such kind of software generate inputs
from given examples and apply small changes to them. Like,
Xmlfuzzer~\cite{xmlfuzzer} is a tool crafted specially to test XML schemes in
programs. Besides special format fuzzers, more advanced ones exist which
directly interact with the target, allowing to track execution paths per input
and therefore even create more test cases automatically. The most popular fuzzer
is "american fuzzy lop" (afl)~\cite{aflweb}. It contains described algorithms to
get the best possible code coverage. Another more advanced fuzzer is Steelix by
Yuekang~\etal~\cite{steelix}, which provides program-state based binary fuzzing.
Steelix uses static-analysis and binary instrumentation to provide more
information about the fuzzing process and therefore gains better test cases and
execution paths for the fuzzed program. Shoshitaishvili~\etal showed how fuzzing
could find authentication bypasses in binary firmware with their tool
Firmalice~\cite{firmalice}. In their work Wang~\etal~\cite{inmemfuzzing} showed
how in-memory fuzzing is used to detect similarities in binary code. This is
done by fuzzing every function available and compare traces of program
behaviours with machine learning models. They show that they can find
similarities in binaries even when different compilers, optimisations or
obfuscation is applied.

\subsection{Symbolic Execution}

Fuzzing provides a pseudorandom input to a program to test it. Therefore it
might not catch all possible execution paths of a program. Scientists developed
techniques to test all possible paths by simplifying the program,  one of this
techniques is symbolic execution. It makes use of the possible simplification of
inputs where they use a logic symbol with less possible states instead. For
example, a number can be represented by the possible states: maximal value,
minimum value, positive, negative, zero. The possible symbolic-state inputs then
run through the program in combination with a constraint solver to report
possible violations of the program's specification. The solver can also be used
to find possible paths to reach a pre-defined goal inside the binary.

One of the most famous symbolic analysis frameworks is angr~\cite{angrpaper}.
Besides the symbolic execution of binaries, it also allows various other
analysis methods such as control-flow graph recovery, automatic exploit
generation or automatic binary hardening.

Stephens~\etal~\cite{driller} showed with Driller how symbolic execution could
be beneficial to fuzzing approaches. This allows keeping the number of execution
paths to test lower but also try to cover as much code as possible. Driller uses
concolic execution for the path exploring. Concolic execution is the combination
of symbolic and concrete execution of a program.

\subsection{Instrumentation}

Instrumentation comes down to adding new code to already existing one in an
automated manner. Developers use instrumentation as a tool for debugging or
performance measures by adding calls to timing functions or information logging
to the code. Information from instrumentation could be something like what
functions are called and how often, what execution branches are taken, how long
does a subroutine take, what memory is accessed and many more. It usually also
allows the developer to change the runtime environment of a program, like change
return values or skip instruction calls.

Two different styles of instrumentation exist, one is source instrumentation,
and one is binary instrumentation, where so-called instrumenter-calls are either
added pre-compilation to the source of the program or later added to the binary
context of the program. The first style might bring the advantage that it is
easier to apply and the structure of the program is better known to the
instrumenter, so calls for performance checks over functions are done by adding
timer calls at the entry and exit points. However, binary instrumentation also
has its perks. For example, users do not need the source of the program, which
might not be available in proprietary programs. Also, recompilation of the
program is not needed if the code for the instrumentation changes. Given enough
debug information in the binary the instrumentation information might not differ
much from the source based one. With binary instrumentation it is also possible
to change and read register values before and after CPU instructions are called,
which might not be possible at source code level, as used registers and memory
areas are unknown to the instrumenter at this point.

\subsubsection{Intel Pin}

Pin~\cite{pintool} is a proprietary, dynamic, binary instrumentation framework
developed and released by Intel, who supply it for free. The framework provides
a large number of API calls which abstracts most of the work done by Pin and
gives easy access to binary analysis. As Pin is a binary instrumentation
framework, the source of the to be analysed program is not needed, and the
program does it need to be recompiled if the instrumentation tool changes. The
framework allows to log and modify the runtime of a program, whereas it is no
problem to log register values or change those before or after desired
instruction calls or even inject code to be run before given calls. Programs
instrumenting a binary are called pintools, the framework provides their API
calls for C++, but also bindings for Python can be found in
\texttt{python-pin}~\cite{pythonpin}.

In the paper about Pin, Luk~\etal~\cite{pintool} describe the system of Pin and
how the written pintool is combined with the application in more detail. When
instrumenting a program, three binaries are being used. One is the application
to instrument, one is the compiled pintool, and one is Pin itself. Pin can be
seen as a userland virtual machine. When the instrumentation starts Pin is
loaded into the address space of the application, then the \texttt{ptrace}
system-call is used to capture the program's behaviour. When Pin has initialised
itself the tool overtakes the entry point of the program, or the current program
counter if it was attached to a running application, and uses its just-in-time
(JIT) compiler to continue the execution. They state that their approach of
loading Pin has various advantages over other instrumentation tools, as those
mostly use \texttt{LD\_PRELOAD}, which would not work on statically linked
binaries and also using an additional library, the address space would be
changed from a normal execution.

The JIT compiler inside Pin is used to inject the instrumentation API into the
application and then the pintool to use it. The code is loaded into a preserved
code-cache and then executed. Pin always tries to execute as many instructions
as possible without interfering it for debugging. That is why they state,
execution of the code happens one trace at a time. Whereas a trace is either
till a control change happens, like branching, a call or return, or a
pre-defined API call is needed for the instruction. With this approach, they
gain better performance than other instrumentation techniques. In their paper,
they also state that the JIT is not the bottleneck but the instrumenting code
inside the application. That is the reason why Pin tries to optimise the
injected code as good as possible. As their first routine, they inline all
analysis functions. This is faster as no calls happen, which would cause
registers to be used for the arguments. Most other debugging tools use the so
\texttt{eflags} registers, which is not ideal, as storing and restoring these
flags is time-consuming, and an extra stack has to be used, as the original
stack is not allowed to be modified. Pin detects in their JIT if the
\texttt{eflag} content is needed later on, and only if it is, saving routines
are called. Another optimisation for their framework is the API call
\texttt{IPOINT\_ANYWHERE}, which specifies that the analysis call can happen
anywhere in the instrumentation context, which allows better optimisation of the
instruction order.

\section{Permission Model in Unix-based Systems}

The POSIX standard~\cite{posix} defines many properties for operating systems,
also permissions for users and their groups. For practical reasons, a user can
be part of multiple groups. A unique integer ID per user and group provides
identification to the system. The POSIX-defined permission structure applies to
files, which can have properties like readable, writable, executable. On
creation, the user can set these properties separately for the owner, the group
of the owner and others. POSIX also allows users with special privileges to
suppress such permission checks. These users are called super-users.

Besides permission for files most Unix based systems also provide permission
checks for processes and their owner. For example, only processes owned by a
super-user might be allowed special syscalls. In most systems, only one
super-user exists, namely \texttt{root}. Usually, users do not want to login
another time if they need super-permissions for a task. Therefore, POSIX
compliant operating systems ship tools such as \texttt{su} or the more advanced
\texttt{sudo}. \texttt{su} stands for switch user and allows a user to change
the current context to another user or execute a command with the other user's
permissions, providing the correct password of the other user. \texttt{sudo}
includes the same functionality as \texttt{su}. Additionally, it has a better
management system in the background and given the correct permissions the user
does not need to know the other user's password to execute commands as the
target user.

\subsection{\texttt{setuid} Binary Property}

\texttt{setuid} stands for "set user identity"~\cite{ogroupsetuid}. It allows an
executable to change its user ID on execution. An additional bit in the file
permissions represents the \texttt{setuid} property, namely \texttt{S\_ISUID}.
This property allows a user to run a program with the permissions of the file's
owner. Tools like \texttt{su} and \texttt{sudo} need this as they require the
permission to change the current context. It is also practical to allow normal
users the usage of tools like \texttt{ping} or \texttt{ip}, without allowing
them to use \texttt{sudo}. We refer to the GNU libc documentation for more
details~\cite{libcpermission}.

Whereas the \texttt{setuid} property might seem practical, it also brings
security risks. A bug allowing code execution in such a program allows an
attacker to execute that code with the permission of another user. Therefore,
the number of \texttt{setuid} binaries should be low and programs having that
property should be well audited.

\subsection{\texttt{chroot}}

\texttt{chroot}~\cite{ogroupchroot} stands for "change root directory". One can
either use it via the system call or the wrapper program. A \texttt{chroot} call
changes the root directory for the running process. Meaning, the environment is
changed, and the program runs inside a completely different system. It is not
possible to access files from outside the new environment nor to interact with
processes running in the old environment. The tool is either used for testing
purposes, to create packages in clean environments, to check compatibility, as
the chrooted system could be a different system or for sandboxing and privilege
separation.

\section{Connecting Computers}

People are exchanging data between devices for several decades now. The Internet
started when scientists found a way to have universities share research results
between computers over the back then used Advanced Research Projects Agency
Network (ARPANET).  The ARPANET already used the TCP/IP protocol and showed
requirements for future interconnection networks. With easier access to
computers, more universities, companies and private households wanted to be
connected and exchange data. Many protocols were introduced with the creation of
the Internet, where Hypertext protocols are shared in the World Wide Web,
electronic mail is used to transmit messages, and file sharing became common.
With this growing amount of users, security in the network and the connected
devices became more relevant every day. Therefore, protocols to ensure security
for the participants became standard and regularly improved.

\subsection{Webserver}

Web servers provide users with content on the Internet, mostly referred to as
the World Wide Web (WWW). There are many different software implementations for
such servers available, most of them being open source and free, such as
Apache\cite{apacheweb} and \texttt{nginx}\cite{nginxweb}. Web servers use the
Hypertext Transfer Protocol (HTTP) to handle requests and deliver the content to
the client. Web servers usually deliver websites encoded in Hypertext Markup
Language (HTML) but can also deliver all kind of files such as JavaScript (JS)
and Content Style Sheet (CSS).

A client requests content from the server by sending a Uniform Resource Locator
(URL). The server then looks up the given path in the URL and answers the
request by sending the desired file or an error if one occurs during the
handling of the request. As not all requests should publicly visible web servers
adapted TLS for security and privacy reasons to provide content over HTTPS. The
client, most of the time a web browser, can thereby verify the server's identity
and apply encryption to the connection if needed.

\subsubsection{HTTP - Basic Access Authentication (BA)}

To allow only authorised users to access specified paths on a web server HTTP
added access-control to the protocol described in RFC 7617~\cite{rfc7617}. This
technique makes use of a header field in HTTP, namely \texttt{WWW-Authenticate}.
RFC 3986~\cite{rfc3986} deprecates the functionality to transmit credentials
inside the URL. BA is a straightforward access-control technique and does not
protect the transmitted credentials. Therefore, it is used together with HTTPS
at most times. The client has to transmit the credentials with every request,
and there is no logout function as in other techniques using cookies or session
identifiers.

\paragraph{Access control in Apache and \texttt{nginx}} is done by providing
Basic Access Authentication in the default configuration. Apache makes use of
\texttt{ht} files, which are an extension to the usual server configuration for
users with limited rights, see their documentation~\cite{apacheauthdoc} for
further details. Apache introduced \texttt{htpasswd}-files to make use of BA,
which stores the user credentials. \texttt{nginx} later adapted these files for
their web server implementation. The \texttt{htpasswd}-files keep the data
either in plaintext or hashed. To create these files Apache provides the tool
\texttt{htpasswd}~\cite{htpasswddoc}. This tool uses MD5 hashing per default (as
of version 2.2.18), but it is possible to change the hashing algorithm to
something more secure. On a request including BA, the web server will look up
the \texttt{htpasswd}-file to verify given credentials and then return the
requested content or an error according to RFC 7231\cite{rfc7231}.

\subsection{Transport Layer Security}

Transport Layer Security (TLS) describes a collection of cryptographic
protocols. It is a standard proposed by the Internet Engineering Task Force
(IETF), RFC 5246\cite{rfc5246} describes the standard in detail. The IETF
regularly updates the TLS standard by adding new protocols and removing old,
seen as insecure, ones.

TLS provides a secure connection between various devices, usually clients and a
server. The connection holds the following properties:
\begin{itemize}
  \item Authenticated identities are provided by public-key cryptography. This
is needed for clients to make sure they communicate with the correct server. TLS
certificates provide the ability to prove the identity of a server. Protocols
use the certificates as they contain the public key of the server.
  \item Private connections between the parties are provided by symmetric
cryptography. The participants in a connection have to agree on how they
exchange the key for the symmetric encryption.
  \item Reliability is provided by the so-called message authentication code
(MAC), which prevents undetected alteration or losses during TLS communication.
\end{itemize}

As TLS is just a collection of protocols, the parties have to agree on which one
they use for each part of the connection, which happens during the TLS
Handshake. A common name for a TLS connection protocol would be
\texttt{ECDHE-RSA-AES256-GCM-HMAC-SHA1}, which contains all three protocols used
for the connection. Whereas \texttt{ECDHE-RSA} provides the key exchange,
\texttt{AES256-GCM} is the  symmetric cypher and \texttt{HMAC-SHA1} checks data
integrity.

TLS is commonly used in web servers to guarantee user's safety, but also
applications like Virtual Private Networks (VPNs) and Secure Shell (SSH) make
use of protocols and cyphers described in the TLS standard. The most commonly
used TLS implementation is OpenSSL~\cite{opensslweb}.

\subsection{Cryptographic Nonces}

In the early days of encrypted communication, replay-attacks were a common
problem, as Syverson showed in his taxonomy of replay attacks in
1994~\cite{replaytax}. Newer protocols introduced nonces to resolve this issue.
Cryptographic nonces are one-time, often pseudorandom, numbers used for each
packet sent. They are combined with the key so that the encryption differs for
each packet. This technique makes replay-attacks harder.

In most cases, nonce reuse is a problem for the protocol and results in breakage
of the encryption of future or past packages. Hence, most protocol standards
define a nonce as a pseudorandom, non-repeating or unpredicted
value~\cite{noncegeneral}. It is also possible to use the nonce as a counter
which has a random start value and increases for each packet sent. However, also
current protocols and technologies still face problems with nonces, like
Vanhoef~\etal showed in their work in 2017, that it is possible to force nonce
reuse in WPA2\cite{wpanoncereuse}.

\subsection{Advanced Encryption Standard (AES)}

The Advanced Encryption Standard is a block cypher proposed to the US National
Institute of Standards and Technology  (NIST) by Vincent Rijmen and Joan Daemen
in 1999~\cite{aesproposal}. In 2001 NIST released AES as a specification for
encryption of electronic data. It is the successor of the Data Encryption
Standard (DES). AES is also part of ISO/IEC 18033-3~\cite{iso18033}, the ISO
standard related to Security techniques and encryption algorithms.

\subsubsection{Design of AES}

AES is a block cypher, it is applied to data blocks, as the plaintext and a
given key are permutated together to create the ciphertext.

\paragraph{Substitution Boxes} so-called S-boxes are a significant part of the
AES algorithm. In a simplified view, they can be seen as a lookup table which
shows which value has to be replaced which what other value. These lookup tables
are designed in a way that there is no mathematical analysis possible, based on
linear or differential dependencies between values. In AES the Rijndael S-box is
used, the lookup-table describing this S-box can be seen in
table~\ref{tab:sbox}. A lookup in the S-box for a byte works as the least
significant nibble (four bytes) determines the column to be used, and the most
significant nibble determines the row. For example, the entry byte of  0x3C
would output 0xEB, using the given S-box.

\begin{table}[]
\centering
\begin{tabular}{c|cccccccccccccccc}
   & 00 & 01 & 02 & 03 & 04 & 05 & 06 & 07 & 08 & 09 & 0a & 0b & 0c & 0d & 0e &
0f \\ \hline
00 & 63 & 7c & 77 & 7b & f2 & 6b & 6f & c5 & 30 & 01 & 67 & 2b & fe & d7 & ab &
76 \\
10 & ca & 82 & c9 & 7d & fa & 59 & 47 & f0 & ad & d4 & a2 & af & 9c & a4 & 72 &
c0 \\
20 & b7 & fd & 93 & 26 & 36 & 3f & f7 & cc & 34 & a5 & e5 & f1 & 71 & d8 & 31 &
15 \\
30 & 04 & c7 & 23 & c3 & 18 & 96 & 05 & 9a & 07 & 12 & 80 & e2 & eb & 27 & b2 &
75 \\
40 & 09 & 83 & 2c & 1a & 1b & 6e & 5a & a0 & 52 & 3b & d6 & b3 & 29 & e3 & 2f &
84 \\
50 & 53 & d1 & 00 & ed & 20 & fc & b1 & 5b & 6a & cb & be & 39 & 4a & 4c & 58 &
cf \\
60 & d0 & ef & aa & fb & 43 & 4d & 33 & 85 & 45 & f9 & 02 & 7f & 50 & 3c & 9f &
a8 \\
70 & 51 & a3 & 40 & 8f & 92 & 9d & 38 & f5 & bc & b6 & da & 21 & 10 & ff & f3 &
d2 \\
80 & cd & 0c & 13 & ec & 5f & 97 & 44 & 17 & c4 & a7 & 7e & 3d & 64 & 5d & 19 &
73 \\
90 & 60 & 81 & 4f & dc & 22 & 2a & 90 & 88 & 46 & ee & b8 & 14 & de & 5e & 0b &
db \\
a0 & e0 & 32 & 3a & 0a & 49 & 06 & 24 & 5c & c2 & d3 & ac & 62 & 91 & 95 & e4 &
79 \\
b0 & e7 & c8 & 37 & 6d & 8d & d5 & 4e & a9 & 6c & 56 & f4 & ea & 65 & 7a & ae &
08 \\
c0 & ba & 78 & 25 & 2e & 1c & a6 & b4 & c6 & e8 & dd & 74 & 1f & 4b & bd & 8b &
8a \\
d0 & 70 & 3e & b5 & 66 & 48 & 03 & f6 & 0e & 61 & 35 & 57 & b9 & 86 & c1 & 1d &
9e \\
e0 & e1 & f8 & 98 & 11 & 69 & d9 & 8e & 94 & 9b & 1e & 87 & e9 & ce & 55 & 28 &
df \\
f0 & 8c & a1 & 89 & 0d & bf & e6 & 42 & 68 & 41 & 99 & 2d & 0f & b0 & 54 & bb &
16
\end{tabular}
\caption{The S-box for AES as a lookup-table for single bytes. The least
significant nibble determines the column the most significant one the row.}
\label{tab:sbox}
\end{table}

The permutation of plaintext is done as the algorithm applies some steps to the
data. These steps are repeated as so-called rounds. Depending on the key length
the used the number of rounds differs. Usually, in AES these steps are:

\begin{itemize}
\item \texttt{AddRoundKey} - In this step, the round-key is added to the block
via the XOR operation.
\item \texttt{SubBytes} - In the substitute bytes step, every byte inside the
block is substituted with the equivalent S-box entry. This is an alphabetic
replace encryption.
\item \texttt{ShiftRows} - Blocks can be seen as two-dimensional arrays. In this
phase, the rows of the arrays are shifted to the left. The number of shifts
depends on the block-size and the number of rows. The overflowing rows are
re-added on the right side.
\item \texttt{MixColumns} - In this phase, the data in the columns get changed.
This happens with a matrix multiplication where each column-vector is multiplied
with a constant matrix to form a new vector. The multiplication happens in the
modulo of a Galois-Field polynomial, making this operation irreversible.
\end{itemize}

The algorithm applies these steps in the following order:

\begin{enumerate}
\item Key expansion - In this phase the round keys are generated to be later
applied in the repeating steps. Those keys are derived from the main key. The
round keys have the same size as the blocks used.
\item Pre-round - In the first round only the key is added, no S-box lookup or
shifting is applied.
\begin{itemize}
\item \texttt{AddRoundKey}
\end{itemize}
\item Encryption rounds - These are the default round, which can be applied
multiple times. The number of repetitions depends on the keysize.
\begin{itemize}
\item \texttt{SubBytes}
\item \texttt{ShiftRows}
\item \texttt{MixColumns}
\item \texttt{AddRoundKey}
\end{itemize}
\item Last-round - The last round  does not include the call to
\texttt{MixColumns}, it was removed because of implementation reasons and with
the ommiting the security was still provided, this matter is discussed in the
Rijnadael design document~\cite{aesproposal}.
\begin{itemize}
\item \texttt{SubBytes}
\item \texttt{ShiftRows}
\item \texttt{AddRoundKey}
\end{itemize}
\end{enumerate}

Decryption happens in the same manner as encryption with just the order of
function calls reversed. The only thing is that a different S-box is used, but
that one can be derived from the encryption one. All other operations usually
have an inverse implementation or use the same, as for example, XOR calls do not
differ in that regard.

AES is used in most encryption schemes and protocols.  Like WPA2 adapts it to
secure wireless networks. SSH uses it to have secure communication between
computers. Intel CPUs even include AES instructions in their architectures,
which speeds up encryption and decryption for libraries as OpenSSL.

\subsubsection{Attacks on AES}

Since AES was released, researchers and attackers targeted the cypher and tried
to find weak spots in it. Most of these attacks targeted a smaller number of
rounds than the algorithm describes. In 2011, Bogdanov~\etal~\cite{bicliqueaes}
showed an attack against AES which targets the whole number of rounds. Their
attack weakens the security of AES slightly, where for 128-bit key recovery the
computational complexity drops to $2^{126.1}$ and for 256-bit keys to
$2^{254.4}$. As these are still very high complexities, the attack does not
affect real-world usages of AES.

Way more common than such cryptoanalysis attacks on AES are attacks targeting
software and hardware implementations. For example,
O'Flynn~\etal~\cite{aespowerboot} showed an attack on AES in a bootloader.
Whereas most attacks target 128 bits, they showed one against 256. They use a
Correlation Power Analysis (CPA) on a bootloader which loads an AES-256-CBC
encrypted firmware. Besides recovering the full key, they also achieve recovery
of the initialisation vector. Another hardware side-channel attack was made by
Lo~\etal~\cite{sboxpoweranal}. They target the implementation of the S-box used
in AES-128 besides using CPA, they also use differential power analysis (DPA).
In their work they compare both methods as they try to recover cypher keys from
the \texttt{AddRoundKey} and \texttt{SubBytes} function calls. Their work was
done with widespread hardware, the Arduino Uno, making it possible for more
people to learn from such attacks, as they can replicate such attacks with cheap
hardware. With an increasing number of side-channel attacks on software and
hardware, vendors increased the key-size of AES in their implementations. In
their work, Neve~\etal~\cite{sidecomplex}, show that increasing the key-size
does not increase the complexity of side-channel attacks in the same manner.
They show for cache-based side-channel attacks an upgrade from 128 bits to 256
only increases the complexity of an attack by a factor of 6 to 7.

\subsection{AES-GCM}

Galois Counter Mode (GCM) in combination with Advanced Encryption Standard (AES)
provides encryption and simultaneously performs message
authentication~\cite{gcm, gcmnist}. Figure~\ref{fig:aesgcm} shows the schematic
of AES-GCM. Where we use close to the same notation as Böck~\etal in their
paper~\cite{gcmnonceattack}:

\begin{itemize}
  \item[$CNT_i$] The $i$-th counter block, computed using the concatenation of
the $IV$ and the counter value $cnt$, $cnt = (i+1) \mod{2^{32}}$, to achieve 128
bits. $CNT_0 = IV \parallel 0 ^{31} \parallel 1$. With $IV$ being the
initialisation vector with a size of 96 bits.
  \item[$E_k$] AES encryption with symmetric key $k$.
  \item[$a \oplus b$] XOR operation between $a$ and $b$.
  \item[$P_i$] The $i$-th plaintext block.
  \item[$C_i$] The $i$-th ciphertext block.
  \item[$mult_H$] Multiplication $H \cdot X$ in Galois Field GF($2^{128}$),
with the irreducible polynomial $g = g(x) = x^{128} + x^{7} + x^{2} + x + 1$.
  \item[$A$] Authenticated data.
  \item[$len(X)$] Bit-length of $X$.
  \item[$a \parallel b$] Concatenation of $a$ and $b$.
  \item[$TAG$] Authentication tag.
\end{itemize}

\begin{figure}
  \centering
  \begin{tikzpicture}
    \node (cnt0) [block]  {$CNT_0$};
    \node (cnt1) [block, right of=cnt0, xshift=3cm]  {$CNT_1$};
    \node (cnt2) [block, right of=cnt1, xshift=3cm]  {$CNT_2$};
    \node (enc0) [function, below of=cnt0, yshift=-0.5cm] {$E_k$};
    \node (enc1) [function, below of=cnt1, yshift=-0.5cm] {$E_k$};
    \node (enc2) [function, below of=cnt2, yshift=-0.5cm] {$E_k$};
    \node (xor11) [XOR, below of=enc1, scale=2, yshift=-0.5cm] {};
    \node (xor12) [XOR, below of=enc2, scale=2, yshift=-0.5cm] {};
    \node (pt1) [block, left of=xor11, xshift=-1cm] {$P_1$};
    \node (pt2) [block, left of=xor12, xshift=-1cm] {$P_2$};
    \node (ct1) [block, below of=xor11, yshift=-0.5cm] {$C_1$};
    \node (ct2) [block, below of=xor12, yshift=-0.5cm] {$C_2$};
    \node (xor21) [XOR, below of=ct1, scale=2, yshift=-0.25cm] {};
    \node (xor22) [XOR, below of=ct2, scale=2, yshift=-0.25cm] {};
    \node (mult1) [function, left of=xor21, xshift=-1.5cm] {$mult_H$};
    \node (mult2) [function, right of=xor21, xshift=1cm] {$mult_H$};
    \node (mult3) [function, below of=xor22, yshift=-0.5cm] {$mult_H$};
    \node (auth) [block, below of=mult1, yshift=-0.5cm] {$A$};
    \node (xor31) [XOR, below of=mult3, scale=2, yshift=-0.25cm] {};
    \node (len) [block, left of=xor31, xshift=-1cm] {$len(A) || len(C)$};
    \node (mult4) [function, below of=xor31, yshift=-0.5cm] {$mult_H$};
    \node (xor41) [XOR, below of=mult4, scale=2, yshift=-0.25cm] {};
    \node (tag) [block, right of=xor41, xshift=1cm] {$TAG$};

    \draw [arrow] (cnt0) -- (cnt1);
    \draw [arrow] (cnt1) -- (cnt2);
    \draw [arrow] (cnt0) -- (enc0);
    \draw [arrow] (cnt1) -- (enc1);
    \draw [arrow] (cnt2) -- (enc2);
    \draw [arrow] (enc0) |- (xor41);
    \draw [arrow] (enc1) -- (xor11);
    \draw [arrow] (enc2) -- (xor12);
    \draw [arrow] (pt1) -- (xor11);
    \draw [arrow] (pt2) -- (xor12);
    \draw [arrow] (xor11) -- (ct1);
    \draw [arrow] (xor12) -- (ct2);
    \draw [arrow] (ct1) -- (xor21);
    \draw [arrow] (ct2) -- (xor22);
    \draw [arrow] (mult1) -- (xor21);
    \draw [arrow] (auth) -- (mult1);
    \draw [arrow] (xor21) -- (mult2);
    \draw [arrow] (mult2) -- (xor22);
    \draw [arrow] (xor22) -- (mult3);
    \draw [arrow] (mult3) -- (xor31);
    \draw [arrow] (len) -- (xor31);
    \draw [arrow] (xor31) -- (mult4);
    \draw [arrow] (mult4) -- (xor41);
    \draw [arrow] (xor41) -- (tag);
  \end{tikzpicture}
  \caption{Schematic of AES-GCM. Whereas the $E_k$ blocks apply
AES-encryption with the key $k$. The $mult_H$ blocks apply a multiplication in
the Galois-field. $A$ represents a block of authenticated data, which is added
to the MAC $TAG$ but not encrypted.} \label{fig:aesgcm}
\end{figure}

Following steps describe how AES-GCM is used to encrypt data:

\begin{enumerate}
  \item The initialisation vector $IV$ of 96 bits is generated.
  \item The counters $CNT_i$ of 128 bits are generated by $CNT_i = IV \parallel
cnt$, where $cnt = (i + 1) \bmod{2^{32}}$, for $i \in\{0 .. n\}$ and $n$
represents the number of plaintext blocks.
  \item The ciphertexts $C_i$ are generated by encrypting the counter values
$CNT_i$ and XORing it to the plaintext $P_i$, as $C_i = P_i \oplus E_k(CNT_i)$.
\end{enumerate}

To get the authentication tag $TAG$, the $GHASH$ function,
$GHASH(H, A, C) = X_{m+n+1}$, is applied. Where, $H$ is the hash key, computed
by encrypting 128 zero bits with the AES block cypher, $C$ is the ciphertext,
and $A$ is non-encrypted but authenticated data. Figure~\ref{fig:aesgcm}
shows how the $GHASH$ is computed for one block of authenticated data and two
blocks of plaintext. Following steps describes the computation of the $TAG$:

\begin{enumerate}
  \item The Galois field multiplication is applied to the block of
authenticated data. $R_0 = mult_H(A_1)$.
  \item The result of this is XORed to the first ciphertext. The output
of this XOR is again multiplied in the Galois field. $R_1 = mult_H(R_0 \oplus
C_1)$.
  \item This output is then XORed to the second ciphertext and again
multiplied. $R_2 = mult_H(R_1 \oplus C_2)$
  \item Then the length of the authenticated data and the ciphertext is
added. $R_3 = mult_H(R_2 \oplus (len(A) \parallel len(C)))$
  \item The hash key is added in the end to form the tag $TAG$. $TAG = H \oplus
mult_H(R_3)$.
\end{enumerate}

The authenticated tag $TAG$ can therefore be found by solving the polynomial
$g(X) = A_{1}X^{m+n+1} + \cdots + A_{m}X^{n+2} + C_{1}X^{n+1} + \cdots +
C_{n}X^2 + LX + S$, as $g(H) = TAG$. Where $L$ is the combined length of $A$
and $C$, and $S$ is the nonce plus the counter. We refer to the NIST
standard about AES-GCM~\cite{gcmnist} for more details.

\paragraph{The OpenSSL source code} is structured in a way that each part of a
cypher is implemented on its own. For our thesis and analysation we want to
refer to version \texttt{1.1.0g} of OpenSSL\cite{opensslsource}. The AES-GCM
code part is located at \texttt{crypto/evp/e\_aes.c}. The structures used by the
cypher are defined in \texttt{crypto/evp/evp\_locl.h},
\texttt{crypto/include/internal/evp\_int.h} and \texttt{ssl/ssl\_locl.h}. The
file \texttt{crypto/evp/evp\_lib.c} shows the implementation for the creation of
the cypher data.

\subsubsection{AES-GCM Nonce Reuse Attack}

Joux\cite{NISTGCMcomment} calls his attack against GCM "a forbidden attack"
because in cryptography the uniqueness of nonces is seen as given.
Böck~\etal~\cite{gcmnonceattack} describe how relevant nonce reuse still is in
real-world application and why clear definitions for nonces in protocols are
needed.

We look at the simplified attack of Joux\textquotesingle s
comment~\cite{NISTGCMcomment}, described by Böck~\etal~\cite{gcmnonceattack}.
Assuming that the same nonce is used for two messages, consisting of just a
single ciphertext block and no authenticated data. Also,  we note that all
values besides $S$ are known to the attacker:

\[g_1(X) = C{_{1}X^2 \oplus L_1X \oplus S}\]
\[g_2(X) = C{_{2}X^2 \oplus L_2X \oplus S}\]

We add the tag $TAG$ to each polynomial to get $g_i'(X)$ as follows:

\[g_1'(X) = C{_{1}X^2 \oplus L_1X \oplus S \oplus T_1}\]
\[g_2'(X) = C{_{2}X^2 \oplus L_2X \oplus S \oplus T_2}\]

Knowing that $g_x'(H) = 0$, because $g_x(H) = T_x$, we combine those two
formulas to following polynomial:

\[g_{1+2}'(X) = (C_{1} + C_{2})X^2 + (L_1 + L_2)X + T_1 + T_2\]

This polynomial results in $0$ at the point $h$, $g_{1+2}'(H) = 0$.
Therefore, an attacker can calculate a list of possible solutions for $H$.
With an increasing number of nonce reuses the number of possible values for $H$
decreases.

\section{Software-based Microarchitectural Attacks}

Attacks against desktop computers and mobile devices have been widespread over
the last couple of years. As programmers and vendors became more aware of
classic issues in such systems, attackers moved to other layers to find bugs and
security holes. One of these other layers is side-channel information leakage.
Like Kelsey~\etal\cite{kelsey1998side} showed how little information was needed
to break cryptographic cypher implementations. They used time, processor flags,
and power consumption as their sources to break various implementations. But not
only cryptography was affected by side-channel attacks. As
Weinberg~\etal\cite{weinberg2011still} showed, it is also possible to gather
information about a users browser history by using side-channels.

\subsection{Low Level Performance Measuring}

Many side-channel attacks are based on timings. Like, predicting what a machine
currently does because of the time needed to compute something. To measure
timings in a high resolution, researchers need as precise time stamp as
possible.

\subsubsection{Native Timing Measurement}

In native code, researchers use the highly precise timestamp counter (TSC)
provided on x86 architectures since the first Pentium CPU~\cite{intelsys}. The
CPU stores this time stamp in a 64-bit register. The TSC counts the number of
execution cycles since the last resetting of the CPU. As CPUs tend to change
their clock frequency during runtime, the TSC might deliver different results
depending on the current frequency. Therefore, vendors changed the behaviour of
the TSC so that it is increased in the same way as it would on maximum
frequency. The instruction \texttt{RDTSC} provides the TSC's value in the
\texttt{EDX:EAX} CPU registers. With out-of-order execution, problems occurred
with measuring timings, as the CPU read the TSC at the wrong time. This issue
can be resolved by calling serialising instructions such as \texttt{CPUID} or
use the serialising \texttt{RDTSCP} instruction, which reads the timestamp
counter to \texttt{EDX:EAX} and the processor ID into \texttt{EXC}.

\subsubsection{JavaScript Timing Measurement}

Calls to the \texttt{RDTSC} instruction are generally only possible in native
code, but as Gruss~\etal~\cite{memdedupjs} and Lipp~\etal~\cite{keytimejs}
showed, it is also possible to receive highly precise timestamps in JavaScript.
In their work, we see another countermeasure by browser vendors defeated by
researchers. In JavaScript, a time stamp function exists, namely
\texttt{performance.now()}. As researchers started to use the JavaScript time
stamp counter in their side-channel attacks, browser vendors were forced to
reduce the resolution of the reported value, which was nanosecond precise in the
past and is only milliseconds precise now. Browser vendors thought with this
countermeasures JavaScript is more secure, but researchers like
Gruss~\etal~\cite{memdedupjs} and Vila~\etal~\cite{loophole} showed that there
are ways to bring back highly-precise timestamps in the area of single-digit
microseconds.

\subsection{Caches}

Vendors of processors introduced the principle of caching to memory management
to speed up access times for memory. Hardware components closer to the computing
core holding a part of memory for future requests reply faster. In modern CPUs,
there are three or four levels of caching applied. As shown in
figure~\ref{fig:intelcache}, the design in Intel Kaby Lake assigns caches for
level one and two (L1, L2) to each core and the cores share the level three
cache (L3) between each other. L1 caches can be split up in a data and an
instruction cache. There might also be an L0 cache for micro-operations, as in
Intel Haswell.

Caches are updated on each memory access. Inclusive caching is a form of
managing the caches in a way where data stored on one level is also at lower
levels. The architecture defines if the CPU uses inclusive caches and which ones
are inclusive to others. The memory management part of the processor needs to
validate memory all the time to make sure memory accesses do not get wrong data
from other cache levels or the DRAM. Computation at rates of about four
gigahertz would not be possible without such caching mechanisms.

We refer to the Intel manual~\cite{intelsys} chapter 11.1 for more details about
cache management in modern CPUs.

\begin{figure}
  \centering
  \begin{tikzpicture}
    % Level 1 caching blocks
    \node (C1L1) [block] {$C_1-L_1$};
    \node (C2L1) [block, right of=C1L1, xshift=1.5cm] {$C_2-L_1$};
    \node (C3L1) [block, right of=C2L1, xshift=1.5cm] {$C_3-L_1$};
    \node (C4L1) [block, right of=C3L1, xshift=1.5cm] {$C_4-L_1$};
    % Level 2 caching blocks
    \node (C1L2) [block, below of=C1L1, yshift=-1cm] {$C_1-L_2$};
    \node (C2L2) [block, right of=C1L2, xshift=1.5cm] {$C_2-L_2$};
    \node (C3L2) [block, right of=C2L2, xshift=1.5cm] {$C_3-L_2$};
    \node (C4L2) [block, right of=C3L2, xshift=1.5cm] {$C_4-L_2$};
    % Level 3 block
    \node (L3) [level3block, below of=C2L2, yshift=-1cm, xshift=1.25cm] {$L_3$};
    % Connections
    \draw [doublearrow] (C1L1) -- (C1L2);
    \draw [doublearrow] (C2L1) -- (C2L2);
    \draw [doublearrow] (C3L1) -- (C3L2);
    \draw [doublearrow] (C4L1) -- (C4L2);
    \draw [doublearrow] (C1L2) -- (0,-3.6cm);
    \draw [doublearrow] (C2L2) -- (2.5cm,-3.6cm);
    \draw [doublearrow] (C3L2) -- (5cm,-3.6cm);
    \draw [doublearrow] (C4L2) -- (7.5cm,-3.6cm);
    % Memory Controler and DDR3
    \node (MC) [mcblock, below of=L3, yshift=-1cm] {Memory Controller};
    \node (DDR31) [block, below of=MC, yshift=0.2cm, xshift=-1cm] {DRAM};
    \node (DDR32) [block, below of=MC, yshift=0.2cm, xshift=1cm] {DRAM};
    \draw [doublearrow] (L3) -- (MC);
  \end{tikzpicture}
  \caption{Schematic of Intel Kaby Lake caching architecture. Whereas $C_x$
represents each processing core and $L_x$ the level of the cache. Requests are
sent from the processing core to each memory storage, the one finding the
desired memory first will answer before the others. Caches get updated with
each request.} \label{fig:intelcache}
\end{figure}

\subsubsection{Cache Attacks}

Cache Attacks are part of microarchitectural attacks, where scientists have
shown that caching also introduces security risks to systems.
Gruss~\etal\cite{gruss2015cache} showed how caches provide information by
measuring memory access-timings. They also showed how to automate attacks on
last-level caches. Knowledge gained from this kind of attack has also played
a role in the exploitation of the rowhammer bug or other hardware bugs, such as
Meltdown~\cite{meltdown} and Spectre~\cite{spectre}.

\paragraph{FLUSH+RELOAD} is a low noise, L3 cache side-channel attack, released
by Yarom~\etal~\cite{flushreload}. The attack provides a possibility of leaking
information about memory accesses on shared pages between two processes. As the
attack targets the last-level cache, it does not require the two processes to
run on the same core. Pages are either shared when actively mapped with such
property or mapped as shared with copy-on-write when a process is forked. One
process is referred to as spy and the other as victim. The attack consists of
three phases.

\begin{enumerate}
\item The spy uses the instruction \texttt{clflush} to remove the given address
from the CPU caches.
\item The spy waits a pre-defined time.
\item The spy accesses the flushed memory with reading access. If it takes less
than a defined time, the victim has accessed the memory in this timespan.
\end{enumerate}

With this attack, certain information can be leaked to the spy processes. For
example, in cryptographic algorithms, accessed code lines could leak the private
key to the attacker.

\paragraph{FLUSH+FLUSH} is a cache attack designed by
Gruss~\etal~\cite{flushflush} to circumvent detection of cache attacks such as
FLUSH+RELOAD. This attack only uses only the \texttt{clflush} instruction and no
memory accesses. With this idea, the attack bypasses cache attack detection by
performance counters. Instead of the three phases in FLUSH+RELOAD, this attack
only has one phase. The spy process gains a memory address to target, will call
\texttt{clflush} on it and measure the time needed to flush the cache. There are
two possible outcomes for this instruction call:

\begin{itemize}
\item The CPU has not loaded the memory into the cache and the \texttt{clflush}
results in a cache-miss. Therefore the instruction can return immediately and
not flush any other cache levels.
\item The victim process has caused the CPU to load the memory into the cache.
Therefore the flush needs to be applied to all cache levels, which results in
longer execution time for the \texttt{clflush} instruction.
\end{itemize}

With measuring the execution time of \texttt{clflush}, the spy process can again
tell when a victim has accessed targeted memory, as the timing will be higher in
such cases. Besides the leakages similar to the FLUSH+RELOAD results,
Gruss~\etal also built a side-channel with this principle to transmit
information between two processes, according to their paper~\cite{flushflush},
they reach a transmission speed of $496$ kilobytes per second.

\subsection{Rowhammer}

Technology is steadily changing and improving, with vendors of computer parts
being forced to produce cheaper and better hardware every year. For DRAM this
resulted in less quality management and a declining number of proper checks for
hardware faults. However, the industry produced smaller and faster memory chips.
These new chips need less space and energy to store electric load, which
represents data. Years after vendors introduced increased densities in DRAM to
the market, researchers like Kim~\etal\cite{rowhammergeneral} found a critical
bug in the chip design. With the little space and energy stored it was possible
to use changes of charges in memory rows to interfere with data of nearby cells.
This interference resulted in a change of bits in other memory rows.

This behaviour and bug then caused researchers to dig deeper into the
possibilities, and several attacks were found based on the rowhammer bug. For
example, Google\textquotesingle s "Project Zero"~\cite{projectzerorow} showed,
it is possible to use rowhammer for privilege escalation and sandbox escapes.
Van der Veen~\etal~\cite{drammer} state that this attack not only affects
desktop computers but also memory in mobile devices. In their work,
Gruss~\etal\cite{flipinthewall} show that flipping bits in memory has further
risks than previously thought and that Rowhammer defences need a general
overhaul. They state that the only solution is to fix the hardware.
Gruss~\etal~\cite{rowhammerjs} also showed that it is possible to trigger
bitflips from JavaScript.

\subsubsection{Design of Dynamic Random-Access Memory (DRAM)}

Random-access memory is designed to store information as bits. A simplified view
of RAM is a circuit containing amplifiers, one-bit storage cell matrix, an
address decoder and buffers for input and output of the memory.
Figure~\ref{fig:DRAMscheme} shows a simple schematic of a DRAM module. The
storage matrix is a two-dimensional array of one-bit storage cells. Accesses to
DRAM do not happen per bit but per row. The size of a row depends on the DRAM
design, but it is usually a multiple of the word size. In modern systems, rows
are larger than 64 kilobytes. The CPU sends requests for memory to the
controller which then translates it to a sequence, where first the column
entries to read are set via the bit lines and then the desired row via the word
lines. DRAM chips contain many memory arrays and controllers which a data bus
connects with the processor.

A one-transistor (1-T) memory cell, as seen in figure~\ref{fig:1Tstorage}, is
used to store a single bit. To access memory, first, the desired bit lines are
charged with $\frac{V_{CC}}{2}$, then the word line is also charged to switch on
the transistors. An amplifier is then used to detect changes of charge on the
bit lines. A discharged capacitor would cause the voltage on the bit line to
drop, whereas a charged capacitor would raise the voltage to its load. If a
charged or discharged capacitor represents true or false for the bit state,
depends on the design of the DRAM chip. Writing the bits works in a similar way,
where the bit lines of a row are either set to $V_{CC}$ or $0$, and then the
word line is used to switch on the capacitors on the row, making the capacitors
change their charge to the level of the connected bit line.

The capacitor used for the 1-T cell will either lose load over time or on
reading access. Therefore, these capacitors need to be periodically refreshed.
The JEDEC standard~\cite{jedec} sets the refresh cycle to a 64-millisecond
interval. For a usual chip with $8192$ rows, this means a refresh happens all
$7,8$ microseconds. The easiest way to do this is by reading a bit and writing
the response back to it. In the DRAM design, it is possible to refresh each row
at a time with a RAS without the following CAS. With this refresh cycle, the
chip needs to hold the information which was the last refreshed row. Some DRAM
designs, therefore, implement a so-called row refresh pointer holding this
information.

\begin{figure}
  \centering
  \begin{circuitikz}
  \ctikzset{label/align = smart}
  \draw
  (0,0) to[short, l_=$bit\ line$] (0,-4)
  (-1,-1) to[short, l^=$word\ line$] (4,-1)
  (1,-2) node[nmos, rotate=-90] (mos) {}
  (mos.gate) node[anchor=north] {}
  (mos.drain) node[anchor=east] {}
  (mos.source) node[anchor=west] {}
  ;
  \draw
  (mos.gate) to[short] (1,-1)
  (mos.source) to[short] (0,-2)
  ;
  \draw
  (mos.drain) to[short] (2,-2)
  (2,-3) node[rground] (A) {}
  node[anchor=west, yshift=-0.4cm, xshift=0.2cm] {} to[pC, l_=$C$] (2,-2)
  ;
  \end{circuitikz}
  \caption{Schematic of a 1-T memory storage element. The capacitor $C$ is used
to store information. The transistor makes it possible to access the storage by
applying charge on the connected lines. Depending on the voltage used it is a
read or a write access.}
  \label{fig:1Tstorage}
\end{figure}

\begin{figure}
  \centering
  \begin{tikzpicture}
  %blocks
  \node (arr) [memarray]  {Memory Array};
  \node (rowdec) [memfunc, left of=arr, rotate=90, yshift=1.5cm] {Row decoder};
  \node (amp) [memfunc, below of=arr, yshift=-1.5cm] {Amplifier};
  \node (coldec) [memfunc, below of=amp, yshift=-0.5cm] {Column decoder};
  \node (memcon) [memcon, left of=amp, align=center, xshift=-2cm,
                  yshift=-0.5cm] {Memory \\ Controller};
  \node (buf) [memfunc, right of=arr, rotate=90, yshift=-1.5cm] {Data Buffer};
  %connections
  \draw [arrow] (amp) -| (buf);
  % memcon -> rowdec
  \draw [arrow] (-2.5,-2) -- (-2.5,-1.5);
  % memcon -> coldec
  \draw [arrow] (-2,-3.75) -- (-1.5,-3.75);
  % coldec -> amp
  \draw [arrow] (-1,-3.5) -- (-1,-3);
  \draw [arrow] (1,-3.5) -- (1,-3);
  % amp -> arr
  \draw [arrow] (-1,-2) -- (-1,-1.5);
  \draw [arrow] (1,-2) -- (1,-1.5);
  \node at (0,-1.75) {bit lines};
  % rowdec -> arr
  \draw [arrow] (-2,-1) -- (-1.5,-1);
  \draw [arrow] (-2,1) -- (-1.5,1);
  \node at (-1.75,0) [rotate=90] {word lines};
  % arr <-> buf
  \draw [doublearrow] (1.5,-1) -- (2,-1);
  \draw [doublearrow] (1.5,1) -- (2,1);
  % memcon <-> buf
  \draw [arrow] (-3,-5) -- (memcon.south);
  \draw [line] (-3,-5) -- (3.5,-5);
  \draw [line] (3.5,-5) -- (3.5,0);
  \draw [arrow] (3.5,0) -- (3,0);
  \end{tikzpicture}
  \caption{Schematic of a single DRAM module. The memory controller is managing
the data buffer to fill or read the buffer via a bus connection to the CPU. The
decoders are used to make sure the correct bits are accessed. The amplifier is
used because the bit voltage differences are usually too small to change the
memory in the buffer directly.}
  \label{fig:DRAMscheme}
\end{figure}

\subsubsection{Introducing bitflips to DRAM}

Kim~\etal\cite{rowhammergeneral} show how it is possible to cause bits to flip
inside memory by correct access patterns. They use two rows as so-called
aggressors, and one as target row. As the code in
listing~\ref{lst:rowhammercode} shows, the two aggressor rows are repeatedly
read, causing memory accesses in the DRAM. The changes in electrical charge
caused by the read accesses will by chance interfere with the loads in the
target row and change content there. The \texttt{clflush} instructions are used
to empty the memory in the used cache lines, causing the System to access the
DRAM for each read instead of getting the value out of the cache. These accesses
happen fast enough so that the refresh cycle of the DRAM cannot ensure that the
capacitors hold enough voltage.

\begin{minipage}{\linewidth}
\begin{lstlisting}[
  label={lst:rowhammercode},
  style=nasm,
  caption={Code to trigger the rowhammer bug, as of
Kim~\etal\cite{rowhammergeneral}. The preating accesses to the aggressor
locations cause bitflips in a third victim row. The \texttt{CFLUSH} is
neededed to make sure the DRAM rows are accessed directly. According to
Project Zero\cite{projectzerorow} the \texttt{mfence} is not needed and even
lowered the number of flips.},]
code1a:
  mov (X), %eax
  mov (Y), %ebx
  clflush (X)
  clflush (Y)
  mfence
  jmp code1a
\end{lstlisting}
\end{minipage}

The memory addresses to be accessed to gain the rows next to each other are hard
to find. Linux provides a file, \texttt{/proc/<PID>/pagemap}, where physical
positioning is stored. Some systems allow huge pages, where two megabytes of
memory are used contiguously. These two megabytes will use more than one row,
other than normal pages with the size of four kilobytes. In the paper,
Kim~\etal~\cite{rowhammergeneral} use addresses calculated based on their
knowledge of physical address mapping used by common CPU vendors. Project
Zero~\cite{projectzerorow} show how it is possible to use rowhammer to
gain privilege escalation on a standard \texttt{x86\_64} CPU and an escape from
a sandbox. After vendors of operating systems started to introduce
countermeasures, scientists like Gruss~\etal\cite{rowhammerjs, flipinthewall},
came up with further, improved attacks making use of the rowhammer bug. They
show more possible targets for flips and also show that rowhammer is possible in
JavaScript.
%}}}

%% vim:foldmethod=expr
%% vim:fde=getline(v\:lnum)=~'^%%%%\ .\\+'?'>1'\:'='
%%% Local Variables:
%%% mode: latex
%%% mode: auto-fill
%%% mode: flyspell
%%% eval: (ispell-change-dictionary "en_US")
%%% TeX-master: "main"
%%% End:

%%%%%% Static Data Flips %%%%%%%%%%%%%%%%%%%%%%%%%%%%%%%%%%%%%%%%%%%%%%%%%%%%{{{
\chapter{Bitflip Attacks on ELF Files}\label{sec:bitflip}

In this chapter, we will discuss our approach to automate the finding bitflips
in ELF files to gain privilege escalation. This is an approach which
Gruss~\etal~\cite{flipinthewall} already have shown by disassembling binaries
and manually looking for them. We start by describing the impact of a single
bitflip. Then, we look at our work and the framework we created to find bitflips
in binaries to change their behaviour to a pre-defined state. We take a look at
the framework design in general. Then we look at the details on how we find bits
to check, how the definition for searching works, how we test the bitflips and
how results are captured and verified.

\section{Changing the Execution Path with a Single Bitflip}

The task of our thesis was to find a bit in an ELF file which we can flip to
change the behaviour in a way that it benefits us. See the code in
listing~\ref{lst:csimbranch} for example. We want to change the binary in a way
it will print \texttt{success.} instead of \texttt{fail.}, and all this by just
toggling a single bit. For this, we look at the disassembly of the code in
listing~\ref{lst:disasmsimplebranch}. From this, we first mention that we can
change the \texttt{1} can to a \texttt{0} at the \texttt{mov} instruction on
address \texttt{0x0642}. Besides that, also the opcode of \texttt{jnz}, namely
\texttt{0x75 (1110101)}, can be changed into a \texttt{jz}, as this is
\texttt{0x74 (1110100)}, at address \texttt{0x064d}. For compiling we
used~\texttt{gcc (Ubuntu 7.3.0-16ubuntu3) 7.3.0}~\cite{gccubuntu}, for
disassembling~\texttt{radare2 2.3.0-git 17214}~\cite{radare2web}.

\begin{minipage}{\linewidth}
\begin{lstlisting}[style=CStyle,
                   caption={Simple branching code to show an example for a
single bitflip to change the execution path.},
                   label={lst:csimbranch}]
#include <stdio.h>
int main(void)
{
  int x = 0;
  if(x == 1)
    printf("success.\n");
  else
    printf("fail.\n");
  return 0;
}
\end{lstlisting}
\end{minipage}

\begin{minipage}{\linewidth}
\begin{lstlisting}[style=nasm,
                   caption={Disassemby of the main function created by the
code in listing~\ref{lst:csimbranch}. Shows machine code at given address
inside the ELF file, starting at \texttt{0x063a}.},
                   label={lst:disasmsimplebranch}]
0x063a  push rbp
0x063b  mov rbp, rsp
0x063e  sub rsp, 0x10
0x0642  mov dword [local_4h], 0
0x0649  cmp dword [local_4h], 1  ; [0x1:4]=0x2464c45
0x064d  jnz 0x65d
0x064f  lea rdi, str.success.    ; 0x6f4 ; "success."
0x0656  call sym.imp.puts        ; int puts(const char *s)
0x065b  jmp 0x669
0x065d  lea rdi, str.fail.       ; 0x6fd ; "fail."
0x0664  call sym.imp.puts        ; int puts(const char *s)
0x0669  mov eax, 0
0x066e  leave
0x066f  ret
\end{lstlisting}
\end{minipage}

To line out what other possible changes could be done when changing the
\texttt{jnz}, we flip each bit in the \texttt{0x75}-byte and look at the output
of the resulting binary. The flips and results are pointed out in
table~\ref{tab:jnzflips}. We see that in three cases the program would print
\texttt{success.}, in one case the output would not change and in four cases the
program would crash for different reasons.

\begin{table}[]
\begin{tabular}{|l|l|l|l|}
\hline
Opcode & Hex Value & Describtion                                           &
Output              \\ \hline
JZ     & 0x74      & Jump short if zero/equal                              &
success.            \\ \hline
JNBE   & 0x77      & Jump short if not below or equal/above                &
success.            \\ \hline
JNO    & 0x71      & Jump short if not overflow                            &
fail.               \\ \hline
JNL    & 0x7D      & Jump short if less or equal/not greater               &
success.            \\ \hline
GS     & 0x65      & GS segment override prefix                            &
Illegal instruction \\ \hline
PUSH   & 0x55      & ($50+r$) Push onto the Stack &
Illegal instruction \\ \hline
XOR    & 0x35      & Logical Exclusive OR                                  &
Segmentation fault  \\ \hline
CMC    & 0xF5      & Complement Carry Flag                                 &
Illegal instruction \\ \hline
\end{tabular}
\caption{Possible opcodes resulting from changing a single bit in
\texttt{JNZ}~\texttt{0x75} and the output when applied in the assembly showed in
listing~\ref{lst:disasmsimplebranch}.}
\label{tab:jnzflips}
\end{table}

From this, we gain the knowledge that at least three bitflips to the opcode give
us our desired behaviour. Additionally, we can add the bitflip to the $1$, which
makes it four, which we can find manually quite fast. Now, the question arises
how do we automate this searching process?

\section{Automating the Finding of Feasible Bitflips}

Testing all possible bitflips inside the binary and report each output would be
a way to gain all of those which would result in the printing of
\texttt{success.}. The binary resulting from the compilation of the small
branching code from listing~\ref{lst:csimbranch} has a size of $8288$ bytes,
which means there are $66304$ bitflips to test. With an execution time of
$0,003$ seconds per run, this means that it would take close to $200$ seconds to
test all bits inside the binary. For our automation, we want a drastically lower
number.

\subsection{Design of the Testing Framework}

We implemented a framework to search bits to flip in binaries.
Figure~\ref{fig:frameworkdesign} shows a diagram describing the framework. The
framework is split into multiple parts:

\begin{enumerate}
\item We log the accessed memory areas from all the ELF files the binary uses.
\item We apply pre-defined filters to the results.
\item We generate ELF files for each of the bit flips
\item We execute each ELF file and check if the output fits a pre-defined
success state
\end{enumerate}

\begin{figure}
  \centering
  \begin{tikzpicture}
  % Nodes of the framework
  \node (start) [root] {Start};
  \node (conf) [function, right of=start, xshift=2cm] {Load config};
  \node (instr) [function, right of=conf, align=center, xshift=2cm]
                {Instrument \\ binary};
  \node (filter) [function, right of=instr, align=center, xshift=2cm]
                 {Filter \\ memory areas};
  \node (gen) [function, below of=filter, align=center, yshift=-1cm]
              {Generate \\ flipped binaries};
  \node (test) [function, below of=gen, align=center, yshift=-1cm]
               {Test \\ generated files};
  \node (report) [function, below of=test, align=center, yshift=-1cm]
                 {Log results};
  \node (clean) [function, right of=report, align=center, xshift=2cm]
                {Delete \\ generated files};
  \node (dec) [decision, below of=report, align=center, yshift=-1cm]
              {Finished?};
  \node (propclean) [function, left of=dec, align=center, xshift=-2cm]
                    {Cleanup \\ Testing \\ Environment};
  \node (printres) [function, left of=propclean, align=center, xshift=-2cm]
                   {Print Results};
  \node (finish) [finish, left of=printres, xshift=-2cm] {Done};
  % Connections
  \draw [arrow] (start) -- (conf);
  \draw [arrow] (conf) -- (instr);
  \draw [arrow] (instr) -- (filter);
  \draw [arrow] (filter) -- (gen);
  \draw [arrow] (gen) -- (test);
  \draw [arrow] (test) -- (report);
  \draw [arrow] (report) -- (dec);
  \draw [arrow] (dec) -- (propclean) node[near start, above] {Yes};
  \draw [arrow] (propclean) -- (printres);
  \draw [arrow] (printres) -- (finish);
  \draw [arrow] (dec) -| (clean) node[near start, above] {No};
  \draw [arrow] (clean) |- (gen);
  \end{tikzpicture}
  \caption{Flowchart showing each part of the framework and their connections.}
  \label{fig:frameworkdesign}
\end{figure}

Each part of the framework is designed to work on its own and has a clearly
defined input and output. The parts can quickly be adopted or used for other
testing purposes. The configuration file for the framework defines how each step
is applied to the program as instrumentation and verification differ for each
application. The file in listing~\ref{lst:expconfig} shows an example config
file for the framework. Figure~\ref{fig:framfilesys} shows the layout of the
framework in the file system. The configuration file tells the framework where
to find each of the needed files. The template chroot describes the system used
for testing.

\begin{minipage}{\linewidth}
\begin{lstlisting}[style=nasm,
                   caption={JSON style config file for the framework, showing
all parameters used to tweak each part of the framework. Entries
starting with \texttt{CR\_} are used inside the testing \texttt{chroot}.},
                   label={lst:expconfig}]
{
  "instrumenter_call": "./instr_tool.sh",
  "instrumenter_outfile": "tool.out",
  "chroot_template": "chroot_tmpl/bionic",
  "tmp_chroot_folder": "/media/ramdisk/chroot/",
  "folder_with_flips": "/media/ramdisk/flips/",
  "num_of_parallel_checks": 1,
  "CR_exec_file": "./run_tool_chroot.sh",
  "CR_flip_folder": "/media/flips",
  "CR_success_folder": "/media/success/",
  "CR_log_file": "/tmp/succ.file"
}
\end{lstlisting}
\end{minipage}

\begin{figure}
\centering
\begin{tikzpicture}[%
  grow via three points={one child at (0.5,-0.7) and
  two children at (0.5,-0.7) and (0.5,-1.4)},
  edge from parent path={(\tikzparentnode.south) |- (\tikzchildnode.west)}]
  \node [fsnode] {proj.root}
    child {node[fsnode] {tool\_config.json}}
    child {node [fsnode] {instr\_tool.sh}}
    child {node [fsnode] {run\_tool\_chroot.sh}}
    child {node [fsnode, optional] {tool.out}}
    child {node [fsnode] {chroot\_tmpl/}
      child {node [fsnode] {bionic/}}
      };
  \node [fsnode, xshift=5cm] {chroot.root}
    child {node [fsnode] {media/}
      child {node [fsnode] {success/}}
      child {node [fsnode] {flips/}}
    }
    child [missing] {}
    child [missing] {}
    child {node [fsnode] {tmp/}
      child {node [fsnode] {succ.file}}
    }
    child [missing] {}
    child {node [fsnode, optional] {run\_tool\_chroot.sh}};
\end{tikzpicture}
\caption{File system structure of framework and the chroot loaded for the test
runs, whereas the chroot is created from a templated loaded from the
\texttt{proj.root} and the \texttt{run\_tool\_chroot.sh} is copied over to the
created chroot.}
\label{fig:framfilesys}
\end{figure}

\subsubsection{Instrumenting the Program}

We use Intel Pin~\cite{pintool} for instrumentation. The instrumentation is used
to log every memory access during execution of the program, whereas we want to
log all accesses which happen inside of ELF files, which means the program's
binary and all libraries used by it during runtime. The used pintool is the same
for all test runs. It adds instrumenter calls to all memory accesses. The tools
stores all accesses per file and what base offsets are used. This information is
needed to determine if the accessed area is mapped to a file later on.

We instrument the program with a run defining a failed state, after the run, we
filter the accessed memory areas for each file and check if areas were written
before reading. If so, we can also skip testing those bits, as the program would
overwrite the flips again. Listing~\ref{lst:pinlogcode} shows an example code
snippet of a pintool which logs all memory accesses as it instruments each
instruction and logs the opcode position and its parameters if they address
memory. The listing~\ref{lst:pinlogcode} also shows some API-calls for the PIN
framework. In Pin, the Image~\texttt{img} refers to files. These are most of the
time dynamically loaded libraries. The \texttt{INS\_InsertPredicatedCall}
function is used to insert a function call on the machine code layer. The
function can have an arbitrary number of parameters, as the user have to define
the function with each parameter and the type. Pin then inserts a call to the
given function pointer and manages the handling of the parameters and return
values in a way that the instrumented binary's execution is not affected.

Instrumentation is done via the configured \texttt{instrumenter\_call}. That
entry is executed and expected to generate a file named in the
\texttt{instrumenter\_outfile} configuration parameter. This file needs to
follow an output format for the accesses as in \texttt{<hex:position\_in\_file>
- str:filename}. The framework then applies a filter to that before it tests the
actual bitflips.

\begin{minipage}{\linewidth}
\begin{lstlisting}[style=CStyle,
                   caption={Example C++ code for a pintool logging memory
accesses. At first, the tool stores the instruction bytes, and after the
operands are stored separately, depending on if they are writing or reading
memory.},
                   label={lst:pinlogcode}]
ADDRINT ins_addr = INS_Address(ins);
IMG img = IMG_FindByAddress(ins_addr);
ADDRINT base = IMG_LowAddress(img);
img_offsets[base] = IMG_LoadOffset(img);

for(size_t i = 0; i < INS_Size(ins); i++)
  accesses.insert(std::make_pair(ins_addr + i, base));

UINT32 memOperands = INS_MemoryOperandCount(ins);
for (UINT32 memOp = 0; memOp < memOperands; memOp++)
{
  if(INS_MemoryOperandIsRead(ins, memOp))
  {
    INS_InsertPredicatedCall(
      ins, IPOINT_BEFORE, (AFUNPTR)RecordMemRead,
      IARG_INST_PTR,
      IARG_MEMORYREAD_EA,
      IARG_ADDRINT, base,
      IARG_MEMORYREAD_SIZE,
      IARG_END);
  }
  if(INS_MemoryOperandIsWritten(ins, memOp))
  {
    INS_InsertPredicatedCall(
      ins, IPOINT_BEFORE, (AFUNPTR)RecordMemWrite,
      IARG_INST_PTR,
      IARG_MEMORYWRITE_EA,
      IARG_ADDRINT, base,
      IARG_MEMORYWRITE_SIZE,
      IARG_END);
  }
}
\end{lstlisting}
\end{minipage}

\subsubsection{Filtering Memory Accesses reported by the Instrumentation}

We apply a filter to the memory accesses at two different points. At first, we
already filter inside the written pintool. As we log reading and writing
accesses to memory separately, we can check on reading accesses if the program
has written to the position before, if so, there is no need to flip this
location in the binary.

At second, after the program run, we translate all reported accessed addresses
to positions in ELF files. At this point, every access which does not refer to
the file content is thrown away. We can find this as we know the base address
from the loaded library from the pintool report.

Additionally, we add addresses of ELF structure headers, as remapping sections
or change other property bits inside the ELF structure can affect the way the
binary runs. After applying the filters, we gain a list of bytes to flip.

\subsubsection{Generating and Testing the Flips}

Next, we create a new ELF file for each bit for each of the reported bytes. We
generate those files into a \texttt{tmpfs} file-system placing its memory in the
DRAM to have lower access times than to the hard drive. Depending on the
configured number of parallel checks, chroots are created in the \texttt{tmpfs}
too. The framework makes sure the number of created ELF files does not need more
than the memory available. If there are more files to test than the available
memory allows, the framework takes multiple runs, as seen in
figure~\ref{fig:frameworkdesign}. A test run starts configured
\texttt{CR\_exec\_file} which will copy a flipped file from
\texttt{CR\_flip\_folder} to the original position, then run it and check if the
output matches a defined success state. If the flipped file reaches the success
state, the framework writes the flipped address to \texttt{CR\_log\_file} and
copies the flipped ELF file to \texttt{CR\_success\_folder}. The framework
passes the file to replace to the \texttt{CR\_exec\_file} when it calls it. To
keep track of which bitflip is tested the filename for generated ELF files is
structured as \texttt{<original\_filename>\_<address\_in\_file>\_<bit\_number>}.

\subsection{Applying the Framework to real-world Applications}

In our thesis, we want to apply our tool to more extensive programs, programs
used in the real world and with bitflips in mind which an attacker also would
use. As Gruss~\etal\cite{flipinthewall} already showed, there are bits in the
\texttt{sudoers} library which when toggled, allow using ~\texttt{sudo} without
or with a wrong password to still gain super-user privilege. We want to show,
that our framework can find a more significant number of possible bitflips to
achieve this. Also, we also want to show how we can search for bitflips in the
\texttt{nginx} program to bypass HTTP basic authentication.

\subsubsection{\texttt{sudo} - Privilege Escalation}

For \texttt{sudo}, the task was to find all possible bitflips which allow
switching to \texttt{root}, without knowing the correct user's password.
\texttt{sudo} is a \texttt{setuid}-binary, which changes it's process owner to
\texttt{root} when being executed. We ran into problems with our framework
because of this property of the binary.

\paragraph{Instrumentation} for \texttt{sudo} became more advanced than for
simple programs, because of the way Pin works. Usually, when starting a program
with a pintool, Pin would modify the program's memory to add its internal
instrumenter calls and then execute the program. As \texttt{sudo} is owned by
\texttt{root}, Pin is not allowed to modify the memory when running as regular
user. Running the \texttt{sudo} with Pin attached as \texttt{root} is possible,
but starting \texttt{sudo} as super-user would not trigger the password dialogue
nor trigger any of the authentication check execution paths.

\begin{minipage}{\linewidth}
\begin{lstlisting}[style=CStyle,
                   caption={Code of the pre-loaded library to keep the process
waiting for some milliseconds, which gives enough time for Pin to attach to the
process.},
label=lst:presleeplib]
extern const char *__progname;

__attribute__((constructor)) void init(void)
{
  if(!strcmp(__progname, "sudo_instr"))
  {
    size_t i = 0;
    while(i++ < 600000000);
  }
}
\end{lstlisting}
\end{minipage}

To bypass this, we use the attach-to-process functionality of Pin, which
wouldn't change the context of the program and still run it as a regular user.
The problem for this is, finding the program ID and attach to it takes time,
which allows us only to attach at the point where the password dialogue is
already popped-up. Pre-loading a wait library which keeps the process sleeping
resolves this issue, and we would instrument enough of the code.

Listing~\ref{lst:presleeplib} shows the code used to generate such a pre-loaded
library. We use the counting while-loop instead of a call to the sleep system
call because attaching Pinto a process in sleeping state results in a state,
where the process would never wake up again. The number of increments was chosen
to have enough time to search the process ID and attach to it. The compiler
attribute \texttt{constructor} makes sure the program executes this code on
loading the library before any other code. We use the
\texttt{/etc/ld.so.preload} file to make our library load for any program, to
not have all of them waiting on start-up, we check the name of the binary.

\paragraph{Testing} the \texttt{sudo} program is easy, we call it and give it a
wrong password. There are multiple ways to verify a successful privilege change.
For once, it would be possible to use the \texttt{whoami} program to check if
the system changed the current user to \texttt{root}. On the other side, it is
also possible to verify by reading a file owned by the other user.

\subsubsection{\texttt{nginx} - Basic Authentication}

For \texttt{nginx} we want to achieve an authentication bypass by flipping a
single bit. This means we need to set up a page which is only accessible given a
correct user and password credential pair.

\paragraph{Instrumentation} for \texttt{nginx} is straightforward. We had to
modify the default configuration a little to make sure the server runs as a
single process. This was done by not allowing \texttt{nginx} to run as a system
daemon or use their master process. We also only want to instrument the code run
when an actual request to the protected site happens. Therefore, we again use
the functionality to attach Pin to an already running process. For
instrumentation, we start the web server, attach Pin, sent a request to the
protected site with wrong credentials, wait for the denying answer from the
server and then end instrumentation. This should give us only the bytes used
during a basic authentication request.

\paragraph{Testing} the bitflips in \texttt{nginx} works the same as in the
instrumentation. We copy the flipped ELF, start the server, verify it is running
by requesting a standard page, then requesting the protected site with wrong
credentials, a successful state is if the web server returns the protected site
to a request with wrong credentials. For requesting the protected site, we would
use \texttt{netcat} instead of \texttt{curl} or \texttt{wget}, as we only want
to make sure the content is delivered to us and not if the answer got malformed
in any way by the bitflip.

\subsection{Results given by the Framework}

Looking back at the introduced flips for the code in
listing~\ref{lst:disasmsimplebranch}, we can see that permission checks can be
bypassed in the same way for both programs we looked at. As at some point jump
instructions will point to code executed only in a successful state. We can also
see flips causing a success which change bits in other sections than the code
one. We want to show which ELF files have bits where flips would cause
successes and what areas in ELF files are most promising for flips to be found.

\subsubsection{\texttt{sudo} - Privilege Escalation}

Table~\ref{tab:sudores} shows the number of found flips per file. We see that
the possible flips are spread over multiple files.  The \texttt{sudoers} library
contains most of them, whereas we also can find flips affecting a privilege
change in the linker (\texttt{ld-linux}) and the threading library
(\texttt{libpthread}). As expected, most flips are in the \texttt{.text}
section. Additionally, other main sections such as \texttt{.rodata} contain
flips. The \texttt{.plt} section refers to the Procedure Linkage Table, which is
used to resolve mappings between position-independent functions and their
absolute addresses. Close to the same is the \texttt{.got.plt} section, which
represents the Global Offset Table, which resolves mappings for position
independent code parts.

\begin{table}[]
\begin{tabular}{c|cccc|c}
ELF File & \texttt{.text}  & \texttt{.rodata} & \texttt{.got.plt} &
\texttt{.plt} & Sum of flips:                             \\ \hline
\texttt{ld-linux-x86-64.so.2} & 1   & 0  & 0  & 0  & 1    \\
\texttt{libpthread.so.0}      & 2   & 0  & 0  & 0  & 2    \\
\texttt{sudo}                 & 6   & 0  & 0  & 0  & 6    \\
\texttt{libnns\_compat.so.2}  & 6   & 0  & 1  & 0  & 7    \\
\texttt{libpam.so.0}          & 64  & 0  & 0  & 0  & 64   \\
\texttt{pam\_unix.so}         & 85  & 0  & 0  & 1  & 86   \\
\texttt{sudoers.so}           & 219 & 35 & 5  & 0  & 259  \\ \hline
Sum of flips:                 & 383 & 35 & 6  & 1  & 425
\end{tabular}
\caption{List of ELF files used by the \texttt{sudo(1.8.19p1)} program and the
number of flips causing privilege escalation listed per section.}
\label{tab:sudores}
\end{table}

\subsubsection{\texttt{nginx} - Basic Authentication}

For the basic authentication bypass in \texttt{nginx} we can report that only
flips inside the program's binary itself were useable. Table~\ref{tab:nginxres}
shows the number of flips found and that all flips are inside the \texttt{.text}
section of the binary.

\begin{table}[]
\centering
\begin{tabular}{c|c}
ELF File               & \texttt{.text} \\ \hline
\texttt{nginx}         & $298$
\end{tabular}
\caption{List of flips inside the \texttt{nginx(nginx/1.10.3)} program causing a
basic authentication bypass, whereas all of them were in the \texttt{.text}
section.}
\label{tab:nginxres}
\end{table}
%}}}

%% vim:foldmethod=expr
%% vim:fde=getline(v\:lnum)=~'^%%%%\ .\\+'?'>1'\:'='
%%% Local Variables:
%%% mode: latex
%%% mode: auto-fill
%%% mode: flyspell
%%% eval: (ispell-change-dictionary "en_US")
%%% TeX-master: "main"
%%% End:

%%%%%% Dynamic Data Attack %%%%%%%%%%%%%%%%%%%%%%%%%%%%%%%%%%%%%%%%%%%%%%%%%%{{{
\chapter{Bitflip Attacks on Dynamic Data}\label{sec:automate}

In this chapter, we want to take a look at data created at the runtime of the
program and how flips to that could benefit an attack. We will go into details
about OpenSSL and how bitflips can make attacks like a nonce-misuse possible. We
want to apply a similar attack as Böck~\etal describe in their work about
"practical nonce misusage attacks"~\cite{gcmnonceattack}. By this, we want to
show how rowhammer can reintroduce different previous attacks.

\section{Analysis of OpenSSL for possible Nonce Misuse Flips}

We look at the implementation of AES-GCM inside OpenSSL. We see the general
AES-GCM context struct in listing~\ref{lst:aesstruct}. Additional to that we
take a look at the cypher context struct in listing~\ref{lst:ciphctx}. The
interesting connection between those two is the initialisation vector
\texttt{iv} which is updated on each call to the GCM function. The
\texttt{GCM128\_CONTEXT} provides the hashing and multiplying function in the
Galois field.

\begin{minipage}{\linewidth}
\begin{lstlisting}[style=CStyle,
                   caption={Struct used by OpenSSL to describe the AES-GCM
context. The IV used is stored in the memory pointed to by \texttt{iv}. Source
is taken from OpenSSL version $1.1.0g$},
                   label={lst:aesstruct}]
typedef struct {
  union {
    double align;
    AES_KEY ks;
  } ks;                       /* AES key schedule to use */
  int key_set;                /* Set if key initialised */
  int iv_set;                 /* Set if an iv is set */
  GCM128_CONTEXT gcm;
  unsigned char *iv;          /* Temporary IV store */
  int ivlen;                  /* IV length */
  int taglen;
  int iv_gen;                 /* It is OK to generate IVs */
  int tls_aad_len;            /* TLS AAD length */
  ctr128_f ctr;
} EVP_AES_GCM_CTX;
\end{lstlisting}
\end{minipage}

\begin{minipage}{\linewidth}
\begin{lstlisting}[style=CStyle,
                   caption={Context struct describing the Cipher used in TLS.
This struct is used as the SSL context inside OpenSSL. Source is taken from
OpenSSL version $1.1.0g$},
                   label={lst:ciphctx}]
struct evp_cipher_ctx_st {
  const EVP_CIPHER *cipher;
  ENGINE *engine;     /* functional reference if
                       * 'cipher' is ENGINE-provided */
  int encrypt;        /* encrypt or decrypt */
  int buf_len;        /* number we have left */
  unsigned char oiv[EVP_MAX_IV_LENGTH]; /* original iv */
  unsigned char iv[EVP_MAX_IV_LENGTH]; /* working iv */
  unsigned char buf[EVP_MAX_BLOCK_LENGTH]; /* saved partial
                                            * block */
  int num;           /* used by cfb/ofb/ctr mode */
  /* FIXME: Should this even exist? It appears unused */
  void *app_data;    /* application stuff */
  int key_len;       /* May change for variable length cipher */
  unsigned long flags; /* Various flags */
  void *cipher_data; /* per EVP data */
  int final_used;
  int block_mask;
  unsigned char final[EVP_MAX_BLOCK_LENGTH]; /* possible final
                                              * block */
} /* EVP_CIPHER_CTX */ ;
\end{lstlisting}
\end{minipage}

For the attack to work, we would need the IV to be reused, just as described by
Böck~\etal~\cite{gcmnonceattack}. As OpenSSL uses a counter IV for AES-GCM
instead of a random value, flipping back a lower bit would probably cause a
decrement of the counter which makes reuse very likely. As the described attack
states, this would cause a break of the cryptography for future messages and the
attacker could even craft valid ciphertexts and overtake sessions. For the
attack with rowhammer, we would target the \texttt{iv} array in the
\texttt{EVP\_CIPHER\_CTX} struct, as seen in the listing~\ref{lst:ciphctx}.

\subsection{Likelihood of a Nonce-Misuse introduced by Rowhammer}

In their work, nethammer, Lipp~\etal~\cite{nethammer}, stated how it is possible
to send network requests which cause memory accesses in a manner that they
introduce rowhammer faults. With this knowledge and the possibility of
nonce-misuse attacks in OpenSSL by bitflips, an attacker could overtake TLS
sessions with only remote access to the network.

Looking at the OpenSSL source code, we can see that for each TLS connection, at
least those two context structs are generated. There is one general SSL struct
needed, one AES-GCM context struct and two cypher contexts, as one is used for
sending and one for receiving. The working IVs make 32 bytes of this memory. The
sum of the structs for one TLS connection is 1960 bytes. If we could fill the
DRAM with just these structs, a rowhammer flip hitting an IV happens with a
probability of about 1.5 per cent. The attack is therefore already quite
unlikely, as also, only lower bits are allowed to be hit, to make a nonce
counter reuse likely. Also, we cannot just hold these structs in memory, and the
operating system usually limits the number of parallel TLS connections.

\subsection{Practical Analysis of Nonce-Reuse caused by Rowhammer}

We set up two versions for the practical analysis of the nonce misusage, where
one uses multiple processes, and one uses multithreading. In both cases, we
implemented a simple endless sending loop which will keep the TLS connection to
any client open and sent some string every second. Listing~\ref{lst:ssltestcode}
shows the code parts used in both cases. The code makes sure the connection will
send the reply every second, as long as the client accepts the write.

\begin{minipage}{\linewidth}
\begin{lstlisting}[style=CStyle,
                   caption={},
                   label={lst:ssltestcode}]
if (SSL_accept(ssl) <= 0) {
    ERR_print_errors_fp(stderr);
}
else {
  int run = 1;
  while(run)
  {
    if(SSL_write(ssl, reply, strlen(reply)) < 0)
      run = 0;
    sleep(1);
  }
}
SSL_free(ssl);
close(client);
\end{lstlisting}
\end{minipage}

We force the server to use AES-GCM by using the \texttt{netcat} tool with the
parameter \texttt{--ssl-ciphers ECDHE-RSA-AES256-GCM-SHA384} as client.

For our experiments, we had a memory increase of about 5440 bytes for each
forked process. With threading, the increase was slightly less. We never
achieved filling a major part of the memory with TLS structs so that we came
near the 1.5 per cent, the probability in practical implementations is therefore
even lower. If it is possible to get a thousand parallel connections, with 5440
bytes each, with 32 bytes in initialisation vectors, it will result in 5.19
megabytes of memory representing TLS connections, with 0.59% of this being IVs.
In a setup with four gigabyte DRAM, the IVs would only make $74.5\cdot10^{-6}$
per cent. Even if an attacker could gain knowledge about the four-kilobyte pages
used to store TLS structs, it would be tough to hit IVs with rowhammer. Given
this knowledge, we can conclude that a remote attack with nethammer is very
unlikely to succeed in nonce misusage.
%}}}

%% vim:foldmethod=expr
%% vim:fde=getline(v\:lnum)=~'^%%%%\ .\\+'?'>1'\:'='
%%% Local Variables:
%%% mode: latex
%%% mode: auto-fill
%%% mode: flyspell
%%% eval: (ispell-change-dictionary "en_US")
%%% TeX-master: "main"
%%% End:

%%%%%% Countermeasures %%%%%%%%%%%%%%%%%%%%%%%%%%%%%%%%%%%%%%%%%%%%%%%%%%%%%%{{{
\chapter{Countermeasures}\label{sec:countermeasure}

In this chapter, we want to look at countermeasures. At first, we look at
measures against Rowhammer and microarchitectural attacks in general. Then we
want to look at measures which would reduce the impact of our work. In the end,
we want to close with measures which could be applied to computer systems in
general to make them more secure.

\section{Microarchitectural Attacks}

For this kind of bugs, it is vital to check who is to blame for it, to know who
needs to define the countermeasure. For attacks like
Foreshadow~\cite{foreshadow}, Meltdown~\cite{meltdown} and
Spectre~\cite{spectre} it is clear that that bug is caused by the CPUs
manufactured by Intel or AMD. Even if there are countermeasures like
Kaiser~\cite{kaiserpaper}, which would prevent leakage of kernel memory by
Meltdown, the bug remains in the microcode, and other processes\textquotesingle
memory can still be leaked. Vendors need to address such problems directly, by
either deprecating CPUs or patch the bugs with microcode updates.
Countermeasures applied to other layers might not be as effective and bring a
larger performance drop with them.

\subsection{Cache Attacks}

Cache attacks are mostly timing-attacks, where based on the content of the
cache, accesses to memory take less time. CPU vendors cannot, and should not,
fix this, as the attack is based on the principle of how caches work.
Cache-timing attacks often try to gain information on other processes, by
loading the same library and check what parts are accessed.
Gruss~\etal~\cite{gruss2015cache} showed how such an attack could be used to
build a keylogger, as memory accesses differ for different keys. They also show
how to generate cache-attacks automatically without an in-depth knowledge of
the targeted program and system. They also state that a possible removing of
the \texttt{clflush} instruction is not a sufficient countermeasure, as there
are ways to build cache-attacks with Evict+Reload. They propose disabling
shared-memory as a countermeasure, but operating-systems use this technology a
lot to have lower memory-footprints. They propose a solution where security
relevant functions can be marked as not shared, to remain the low
memory-footprint but protect functions which could leak sensitive information.
It would be down to developers of libraries and programs to mark these
functions. Overall there might be no perfect countermeasure to cache-attacks,
but there are ways to lower their impact on security.

\subsection{Rowhammer}

Rowhammer is an issue which occurs because of missing high-quality checks by
vendors, the leakage of bit-charges to change bits in other memory areas is a
hardware fault and needs fixing on this layer. Also, users of current DRAM
chips need to be protected, and their risk of using their hardware should be
minimalised. In their work about Rowhammer, Gruss~\etal~\cite{flipinthewall}
analyse various countermeasures. Tools like the Micro-Architectural Side
Channel Attack Trapper (MASCAT) by Irazoqui~\etal~\cite{mascat} could be used by
anti-virus software to prevent users from executing software making use of
cache-attacks. With a physical memory separation of kernel memory and user
memory, privilege escalation by applying bitflips to page-tables can be
prevented.

Brasser~\etal~\cite{canttouch} showed how such a separation could be built
inside an OS by changing the code of the physical memory allocator. Another
way to prevent Rowhammer would be to detect an ongoing attack during the runtime
of a program. With their work about ANVIL, Aweke~\etal~\cite{anvil} showed such
a detection technique. They track DRAM accesses using hardware performance
measures and check for frequently accessed memory rows to detect the so-called
aggressors. As prevention, they refresh nearby rows more often to make them
secure against possible flipping. Their benchmarking shows that they have less
than one per cent false-positives and an average performance drop of about one
per cent.

Operating system vendors could implement another countermeasure for the file
cache. Our work relies on the file cache to be used when we execute the ELF
file. The systems could either recheck file contents before execution, as
in check some checksum, or force loading binaries from the disk instead of
using the cache. In such cases, the performance drop would be rather small, but
it would not prevent attacks using Rowhammer on SSDs.

\section{Reducing the Impact of Bit-Flip Testing}

From looking at our binary analysis and the impact we get from knowing that
operating system ship the same ELF files to all their users, we can conclude
that having different ELF files per instance would drastically lower our impact.
Doing so would mean, an attacker would need to analyse bitflips per target.
Whereas this would be a valid countermeasure, it is not practical. Also, having
different ELF files per system breaks other security improvements. Compiler
developers and developers of wide-spread software currently push the technique
of reproducible builds. Whereas the Debian-close organisation
reproducible-builds~\cite{reprobuilds} is one of the most significant
contributors in this field. The advantage of reproducible builds is that all
binaries compiled with the same compiler, same configuration and same
source code will not differ, this makes detection of changes very easy. The
technology allows a bit-by-bit compared verification of the full build chain
used for the program. By this, also changes or backdoors introduced by a
malformed compiler or linker could be detected.

Reproducible builds could also help to define other countermeasures to bit flips
in the binary memory. Whereas the operating system can verify ELF files by using
stored checksums, it would also be possible to use similar techniques to do this
with in-memory code or data at runtime. As Suh~\etal~\cite{memintegrity} have
shown in their work about memory integrity verification and encryption for
secure processors. Other work that could be looked at as countermeasure is
DRIVE~\cite{drive} by Rein~\etal or SPEE~\cite{spee} by Gelbart~\etal. Those apply
code verification to binaries before execution and during runtime.

The encryption of binaries could in general seen as a countermeasure to
bitflips. If decryption happens on each access to the file stored in memory. If
the entire file is stored decrypted it would not improve security. If a bit is
flipped inside an encrypted ELF file, decryption is likely to fail, meaning the
binary could not continue execution and the program would crash in case of an
applied bitflip.

\section{Improving General System Security}

There is no unbreakable, unhackable, totally secure system. Users and vendors
need to be aware of this. If vulnerabilities are found vendors should react to
provide possibilities for their users to be secure. Like provide patches for
either their microcode, operating system kernels or programs itself. Users
should be informed about security updates and apply them to their environment
as soon as possible. For updates to be in place on time, it is vital for
vendors to work closely with operating system developers, to push patches very
soon after or even before public disclosure.

As systems should generally be seen as unsafe and programs running cannot be
trusted, a defensive approach for operating system development might be a right
approach. Defensive coding is usually based on a multiple wall approach. This
allows having defensive structures in place even if some defences are broken.
Whereas developers should find motivation in writing secure code, mistakes
happen, and old software is sometimes run. Bugs like buffer-overflow often lead
to code execution. Operating systems introduced techniques like address space
layout randomisation (ASLR), to make attacks abusing buffer-overflow harder as
an attacker would not know positions of injected code anymore. From this other
attack were designed, using return-oriented programming (ROP) and using known
positions inside shared libraries to craft code from code-pieces inside those.
In their work Ruan~\etal~\cite{ropsur}, show a survey of ROP defence mechanisms.
They point out the effectiveness of randomisation in memory, which would also
increase the defence for Rowhammer. Additional there are approaches described
using instrumentation to detect ROP attacks, which could again also be used to
detect the usage of Rowhammer.
%}}}

%% vim:foldmethod=expr
%% vim:fde=getline(v\:lnum)=~'^%%%%\ .\\+'?'>1'\:'='
%%% Local Variables:
%%% mode: latex
%%% mode: auto-fill
%%% mode: flyspell
%%% eval: (ispell-change-dictionary "en_US")
%%% TeX-master: "main"
%%% End:

%%%%%% Conclusion %%%%%%%%%%%%%%%%%%%%%%%%%%%%%%%%%%%%%%%%%%%%%%%%%%%%%%%%%%%{{{
\chapter{Conclusion}\label{sec:conclusion}

We have shown that the Rowhammer bug is still relevant to system
engineers and that users and vendors need to be aware of this issue.

We present a framework to test any binary to find bitflips which change the
execution path to a given, pre-defined outcome. We apply this framework to
programs available for GNU/Linux operating systems. We gain privilege
escalation by exploiting \texttt{sudo}, bypass the credential check for local
and remote login measures, and bypass HTTP basic authentication implemented by
the \texttt{nginx} web server.

For \texttt{sudo}, Gruss~\etal~\cite{flipinthewall} show \num{29} bitflips which
yield privilege escalation, which they find by manual analysis of the ELF files
used by the program. We show that our framework finds over \num{10} times more
bitflips gaining the same outcome in a few hours of testing time.

Besides searching for exploitable bitflips in ELF files, we also look at the
consequences of bitflips applied to memory generated and used at the runtime of
programs. As an example, we show how bitflips can introduce nonce reuse in the
AES-GCM implementation of OpenSSL.

We present a summary of countermeasures already applied to systems against
Rowhammer and how developers need to extend them. Also, we discuss measures
which make systems more secure by checking the ELF files before
execution, or check the memory of the process during the runtime.

We close with a recommendation for vendors of hardware, such as CPUs or DRAM
chips, that besides increasing performance and space reduction, also the
security of products should be improved. Hardware vendors and researchers should
work together more often to find new microarchitectural flaws in hardware, and
thereby improve the security for users.
%}}}

%% vim:foldmethod=expr
%% vim:fde=getline(v\:lnum)=~'^%%%%\ .\\+'?'>1'\:'='
%%% Local Variables:
%%% mode: latex
%%% mode: auto-fill
%%% mode: flyspell
%%% eval: (ispell-change-dictionary "en_US")
%%% TeX-master: "main"
%%% End:

%%%%%% Future Work %%%%%%%%%%%%%%%%%%%%%%%%%%%%%%%%%%%%%%%%%%%%%%%%%%%%%%%%%%{{{
\chapter{Future Work}\label{sec:futurework}

\section{Future Work regarding Rowhammer}

We see Rowhammer as a still open field of research, while vendors could fix the
exploitation in DRAM chips, by increasing refresh rates or built
non-load-leaking chips, there is other hardware which already faces similar
issues, as Kurmus~\etal~\cite{rowssdhammer} showed for SSDs. We can see more
devices and techniques targeted with such attacks.

Also, an in-depth analysis of the Rowhammer is still an open issue in the
research of microarchitectural attacks. Analysing this issue might bring better
knowledge of the hardware and why it has its faults. With this kind of
knowledge is might be possible to target bits better in the DRAM, allowing the
Rowhammer attack to gain more attention and increase their attack vector even
more.

\subsection{Possible Attacks against Cryptography using Rowhammer}

We showed how nonce-misusage could be introduced by single bitflips to target
AES-GCM implementations. Attackers could use this idea to target other
cryptographic implementations too. Especially those using a counter nonce. Also,
from our work and research inside the OpenSSL source code, we can see other
memory areas to be attacked, such as session reuse by flipping bits, or changing
used cyphers if identifiers are enumerated. Such attacks could not only break
encryption or allow key-recovery but also allow to break authenticity and
confidentiality when an attacker flips bits in tags inside message
authentication codes.

Dobraunig~\etal~\cite{noncestat} showed statistical attacks against nonce-based
authenticated encryption schemes. They show fault attacks on various modes of
AES, such as GCM, CCM, EAX or OCB. They show how faults would affect the
algorithm when they are introduced before the MixColumn step of AES. Work on
faults in memory, like Rowhammer, could be used as a fault source here and work
like this could be used to show future attacks using single bitflips. We can see
more upcoming work in that regard in the future of fault attack research.

Another possible field of research would be to check if it is possible to bring
linear or differential properties to S-box calls via bitflips, or if targeting
the functionality of shifting and replacing could break cryptographic
assumptions made by AES.

We already saw a combination between microarchitectural attacks and
cryptographic attacks. Such as Yarom~\etal~\cite{noncerec} showed, that it is
possible to use FLUSH+RELOAD cache attacks to recover ECDSA nonces in OpenSSL.
They show an attack which allows a partial key-recovery in they Elliptic Curve
Algorithm implementation with only observing a single signing process. With a
key recovery on this level, it allows forgery for future data transmissions.

\section{Improving and Reusing our Framework}

For attacks based on memory accesses, we can take away that tools such as
Pin~\cite{pintool} or \texttt{angr}~\cite{angrpaper} will be a significant
factor for analysis. The work of Chabbi~\etal~\cite{pincallpaths} could also be
taken to improve our testing framework as they provide an even more in-depth
view of execution paths in binaries. Using such detailed graphs, more
bytes-to-test could be filtered, speeding up the framework.

On the other side, we can see our framework working as a base-implementation for
other test environments for parallel testing use cases. The environment is very
adaptable and developers can exchange single components without much effort.
Testing inside \texttt{chroot} could be changed to container environments like
Docker~\cite{docker}, to provide more safety or better abstraction of the host
operating system. Also, virtual machines could be used, if the software has to
be tested on other kernels or operating systems. The testing scripts could be
replaced to work with multiple input files for a program instead of different
binaries. This would make it possible to use the framework as a basis for a
fuzzer. Even a combination with fuzzers would be possible, as the framework
could be combined with \texttt{afl}~\cite{aflweb} to provide a parallel fuzzing
framework, which would either allow fuzzing in multiple environments or as a
general way to parallelise \texttt{afl}.

For the part where binaries are instrumented with Intel Pin at the moment,
replacements or other sources for analysis could be used. Like the
QBDI~\cite{qbdi} tool for dynamic instrumentation could be used instead, which
provides a wider range for software architectures, such as ARM or AArch64.
Additional, a complete open source solution for instrumentation is available in
DynamoRIO~\cite{dynrio}, which could also replace Pin.

As other use cases for our framework, we could see it being used as a basis for
malware research, to track the behaviour of malware in different environments,
as it allows setting up many different systems in a short time and tracks
events inside those systems.
%}}}

%% vim:foldmethod=expr
%% vim:fde=getline(v\:lnum)=~'^%%%%\ .\\+'?'>1'\:'='
%%% Local Variables:
%%% mode: latex
%%% mode: auto-fill
%%% mode: flyspell
%%% eval: (ispell-change-dictionary "en_US")
%%% TeX-master: "main"
%%% End:


%\appendix                       %% closes main document, appendix follows until end; only available in book-classes
%\addpart*{Appendix}             %% adding Appendix to tableofcontents

%\addcontentsline{toc}{chapter}{Bibliography}
\printbibliography              %% remove, if using BibTeX instead of biblatex
% \include{further_ressources}  %% this is a suggestion: you have to create this file on demand






%%%% end{document}
\end{document}
%% vim:foldmethod=expr
%% vim:fde=getline(v\:lnum)=~'^%%%%\ .\\+'?'>1'\:'='
%%% Local Variables:
%%% mode: latex
%%% mode: auto-fill
%%% mode: flyspell
%%% eval: (ispell-change-dictionary "en_US")
%%% TeX-master: "main"
%%% End:
