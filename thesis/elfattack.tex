%%%%%% Static Data Flips %%%%%%%%%%%%%%%%%%%%%%%%%%%%%%%%%%%%%%%%%%%%%%%%%%%%{{{
\chapter{Bitflip Attacks on ELF Files}\label{sec:bitflip}

In this chapter, we will discuss our approach to automate the finding bitflips
in ELF files to gain privilege escalation. This is an approach which
Gruss~\etal~\cite{flipinthewall} already have shown by disassembling binaries
and manually looking for them. We start by describing the impact of a single
bitflip. Then, we look at our work and the framework we created to find bitflips
in binaries to change their behaviour to a pre-defined state. We take a look at
the framework design in general. Then we look at the details on how we find bits
to check, how the definition for searching works, how we test the bitflips and
how results are captured and verified.

\section{Changing the Execution Path with a Single Bitflip}

Lorem ipsum dolor sit amet, consetetur sadipscing elitr, sed diam nonumy eirmod
tempor invidunt ut labore et dolore magna aliquyam erat, sed diam voluptua. At
vero eos et accusam et justo duo dolores et ea rebum. Stet clita kasd gubergren,
no sea takimata sanctus est Lorem ipsum dolor sit amet.

\section{Automating the Finding of Feasible Bitflips}

Lorem ipsum dolor sit amet, consetetur sadipscing elitr, sed diam nonumy eirmod
tempor invidunt ut labore et dolore magna aliquyam erat, sed diam voluptua. At
vero eos et accusam et justo duo dolores et ea rebum. Stet clita kasd gubergren,
no sea takimata sanctus est Lorem ipsum dolor sit amet.
%}}}

%% vim:foldmethod=expr
%% vim:fde=getline(v\:lnum)=~'^%%%%\ .\\+'?'>1'\:'='
%%% Local Variables:
%%% mode: latex
%%% mode: auto-fill
%%% mode: flyspell
%%% eval: (ispell-change-dictionary "en_US")
%%% TeX-master: "main"
%%% End:
