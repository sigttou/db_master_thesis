%%%%%% Static Data Flips %%%%%%%%%%%%%%%%%%%%%%%%%%%%%%%%%%%%%%%%%%%%%%%%%%%%{{{
\chapter{Bitflip Attacks on ELF Files}\label{sec:bitflip}

In this chapter, we discuss our approach to automate the finding bitflips in ELF
files to gain privilege escalation. This is an approach which
Gruss~\etal~\cite{flipinthewall} already have shown by disassembling binaries
and manually looking for suitable flips. We start by describing the impact of a
single bitflip on the execution path. Then, we look at our work and the
framework we created to find bitflips in binaries to change their behaviour to a
pre-defined state. We start with a general overview of the framework design and
go on with discussing the details on how we find bits to check, how the
definition for searching works, how we test the bitflips and how results are
captured and verified.

\section{Changing the Execution Path with a Single Bitflip}

Our task is to find a bit in an ELF file which we can flip to change the
behaviour in a way that it benefits us. See the code
in listing~\ref{lst:csimbranch} for example. We want to change the binary in a
way that it will print \texttt{success.} instead of \texttt{fail.}, by just
toggling a single bit in the binary executing this code. We look at the
disassembly of the code in listing~\ref{lst:disasmsimplebranch}. From this, we
point out that we can start with changing the \texttt{1} to a \texttt{0} at
the \texttt{mov} instruction on address \texttt{0x0642}. Besides that, also the
opcode of \texttt{jnz}, namely \texttt{0x75 (1110101)}, can be changed into a
\texttt{jz}, as this is \texttt{0x74 (1110100)}, at address \texttt{0x064d}. For
compiling we used~\texttt{gcc (Ubuntu 7.3.0-16ubuntu3) 7.3.0}~\cite{gccubuntu},
for disassembling~\texttt{radare2 2.3.0-git 17214}~\cite{radare2web}.

\begin{minipage}{\linewidth}
\begin{lstlisting}[style=CStyle,
                   caption={Simple branching code to show an example for a
single bitflip to change the execution path.},
                   label={lst:csimbranch}]
#include <stdio.h>
int main(void)
{
  int x = 0;
  if(x == 1)
    printf("success.\n");
  else
    printf("fail.\n");
  return 0;
}
\end{lstlisting}
\end{minipage}

\begin{minipage}{\linewidth}
\begin{lstlisting}[style=nasm,
                   caption={Disassemby of the main function created by the
code in listing~\ref{lst:csimbranch}. Shows machine code at given address
inside the ELF file, starting at \texttt{0x063a}.},
                   label={lst:disasmsimplebranch}]
0x063a  push rbp
0x063b  mov rbp, rsp
0x063e  sub rsp, 0x10
0x0642  mov dword [local_4h], 0
0x0649  cmp dword [local_4h], 1  ; [0x1:4]=0x2464c45
0x064d  jnz 0x65d
0x064f  lea rdi, str.success.    ; 0x6f4 ; "success."
0x0656  call sym.imp.puts        ; int puts(const char *s)
0x065b  jmp 0x669
0x065d  lea rdi, str.fail.       ; 0x6fd ; "fail."
0x0664  call sym.imp.puts        ; int puts(const char *s)
0x0669  mov eax, 0
0x066e  leave
0x066f  ret
\end{lstlisting}
\end{minipage}

We flip each bit of the instruction opcode \texttt{jnz} to line out what other
outcomes are possible with a single flip. We apply all possible eight flips and
check the resulting output of the binary. Table~\ref{tab:jnzflips} shows the
flips and results. We can see that in three cases the program would print
\texttt{success.}, in one case the output would not change and in four cases the
program would crash for different reasons.

\begin{table}[]
\begin{tabular}{|l|l|l|l|}
\hline
Opcode & Hex Value & Describtion                                           &
Output              \\ \hline
JZ     & 0x74      & Jump short if zero/equal                              &
success.            \\ \hline
JNBE   & 0x77      & Jump short if not below or equal/above                &
success.            \\ \hline
JNO    & 0x71      & Jump short if not overflow                            &
fail.               \\ \hline
JNL    & 0x7D      & Jump short if less or equal/not greater               &
success.            \\ \hline
GS     & 0x65      & GS segment override prefix                            &
Illegal instruction \\ \hline
PUSH   & 0x55      & ($50+r$) Push onto the Stack &
Illegal instruction \\ \hline
XOR    & 0x35      & Logical Exclusive OR                                  &
Segmentation fault  \\ \hline
CMC    & 0xF5      & Complement Carry Flag                                 &
Illegal instruction \\ \hline
\end{tabular}
\caption{Possible opcodes resulting from changing a single bit in
\texttt{JNZ}~\texttt{0x75} and the output when applied in the assembly showed in
listing~\ref{lst:disasmsimplebranch}.}
\label{tab:jnzflips}
\end{table}

From this, we gain the knowledge that at least three bitflips to the opcode give
us our desired behaviour. Additionally, we can add the bitflip to the $1$, which
makes it four. These flips we can find manually quite fast. Now, the question
arises how do we automate this searching process?

\section{Automating the Finding of Feasible Bitflips}

Testing all possible bitflips inside the binary and report each output would be
a way to gain all a list of all flips which would result in the printing of
\texttt{success.}. The binary resulting from the compilation of the small
branching code from listing~\ref{lst:csimbranch} has a size of $8288$ bytes,
which means there are $66304$ bitflips to test. With an execution time of
$0,003$ seconds per run, this means that it would take close to $200$ seconds to
test all bits inside the binary. For our automation, we want to need less time
than that as we target larger binaries.

\subsection{Design of the Testing Framework}

We have implemented a framework to search bits to flip in binaries.
Figure~\ref{fig:frameworkdesign} shows a diagram describing the framework's
design. We split it into multiple parts:

\begin{enumerate}
  \item Accessed memory areas are logged from all the ELF files the program
uses.
  \item Pre-defined filters are applied to the results.
  \item ELF files for each of the bit flips are generated.
  \item Each ELF file gets executed, and the behaviour is checked  against a
pre-defined success state
\end{enumerate}

\begin{figure}
  \centering
  \begin{tikzpicture}
  % Nodes of the framework
  \node (start) [root] {Start};
  \node (conf) [function, right of=start, xshift=2cm] {Load config};
  \node (instr) [function, right of=conf, align=center, xshift=2cm]
                {Instrument \\ binary};
  \node (filter) [function, right of=instr, align=center, xshift=2cm]
                 {Filter \\ memory areas};
  \node (gen) [function, below of=filter, align=center, yshift=-1cm]
              {Generate \\ flipped binaries};
  \node (test) [function, below of=gen, align=center, yshift=-1cm]
               {Test \\ generated files};
  \node (report) [function, below of=test, align=center, yshift=-1cm]
                 {Log results};
  \node (clean) [function, right of=report, align=center, xshift=2cm]
                {Delete \\ generated files};
  \node (dec) [decision, below of=report, align=center, yshift=-1cm]
              {Finished?};
  \node (propclean) [function, left of=dec, align=center, xshift=-2cm]
                    {Cleanup \\ Testing \\ Environment};
  \node (printres) [function, left of=propclean, align=center, xshift=-2cm]
                   {Print Results};
  \node (finish) [finish, left of=printres, xshift=-2cm] {Done};
  % Connections
  \draw [arrow] (start) -- (conf);
  \draw [arrow] (conf) -- (instr);
  \draw [arrow] (instr) -- (filter);
  \draw [arrow] (filter) -- (gen);
  \draw [arrow] (gen) -- (test);
  \draw [arrow] (test) -- (report);
  \draw [arrow] (report) -- (dec);
  \draw [arrow] (dec) -- (propclean) node[near start, above] {Yes};
  \draw [arrow] (propclean) -- (printres);
  \draw [arrow] (printres) -- (finish);
  \draw [arrow] (dec) -| (clean) node[near start, above] {No};
  \draw [arrow] (clean) |- (gen);
  \end{tikzpicture}
  \caption{Flowchart showing each part of the framework and their connections.}
  \label{fig:frameworkdesign}
\end{figure}

\begin{figure}
\centering
\begin{tikzpicture}[%
  grow via three points={one child at (0.5,-0.7) and
  two children at (0.5,-0.7) and (0.5,-1.4)},
  edge from parent path={(\tikzparentnode.south) |- (\tikzchildnode.west)}]
  \node [fsnode] {proj.root}
    child {node[fsnode] {tool\_config.json}}
    child {node [fsnode] {instr\_tool.sh}}
    child {node [fsnode] {run\_tool\_chroot.sh}}
    child {node [fsnode, optional] {tool.out}}
    child {node [fsnode] {chroot\_tmpl/}
      child {node [fsnode] {bionic/}}
      };
  \node [fsnode, xshift=5cm] {chroot.root}
    child {node [fsnode] {media/}
      child {node [fsnode] {success/}}
      child {node [fsnode] {flips/}}
    }
    child [missing] {}
    child [missing] {}
    child {node [fsnode] {tmp/}
      child {node [fsnode] {succ.file}}
    }
    child [missing] {}
    child {node [fsnode, optional] {run\_tool\_chroot.sh}};
\end{tikzpicture}
\caption{File system structure of framework and the chroot loaded for the test
runs, whereas the chroot is created from a templated loaded from the
\texttt{proj.root} and the \texttt{run\_tool\_chroot.sh} is copied over to the
created chroot.}
\label{fig:framfilesys}
\end{figure}

Each part of the framework is designed to work on its own and has a clearly
defined input and output interface. The parts can be adopted or easily used for
other testing purposes. The configuration file for the framework defines how
each step is applied to the application as instrumentation and verification may
differ for each program. The file in listing~\ref{lst:expconfig} shows an
example config file for the framework. Figure~\ref{fig:framfilesys} shows the
layout of the framework in the file system. The configuration file tells the
framework where to find each of the needed files. The template chroot describes
the system used for testing.

\begin{minipage}{\linewidth}
\begin{lstlisting}[style=nasm,
                   caption={JSON style config file for the framework, showing
all parameters used to tweak each part of the framework. Entries
starting with \texttt{CR\_} are used inside the testing \texttt{chroot}.},
                   label={lst:expconfig}]
{
  "instrumenter_call": "./instr_tool.sh",
  "instrumenter_outfile": "tool.out",
  "chroot_template": "chroot_tmpl/bionic",
  "tmp_chroot_folder": "/media/ramdisk/chroot/",
  "folder_with_flips": "/media/ramdisk/flips/",
  "num_of_parallel_checks": 1,
  "CR_exec_file": "./run_tool_chroot.sh",
  "CR_flip_folder": "/media/flips",
  "CR_success_folder": "/media/success/",
  "CR_log_file": "/tmp/succ.file"
}
\end{lstlisting}
\end{minipage}

The framework starts with executing \texttt{instr\_tool.sh} to generate a report
showing all bitflips, additional it can also filter them. After the script's
execution, the \texttt{tool.out}-file should contain a list of all byte
positions to flip and test. Afterwards, the defined number of chroots are
created by copying the template from \texttt{chroot\_tmpl}. The configured flips
are split and separately copied to each \texttt{/media/flips} folder in the
testing chroot. The \texttt{run\_tool\_chroot.sh} file is copied to each testing
environment and executed there. This script runs the tests with all flips and
reports successful flips to the framework

\subsubsection{Instrumenting the Program}

We use Intel Pin~\cite{pintool} for instrumentation. The instrumentation is used
to log every memory access during execution of the program, whereas we want to
log all accesses which happen inside of ELF files, which means the program's
binary and all libraries used by it during runtime. The used pintool is the same
for all test runs. It adds instrumenter calls to all memory accesses. The tools
stores all accesses per file and what base offsets are used. This information is
needed to determine if the accessed area is mapped to a file later on.

We instrument the program with a run defining a failed state, after the run, we
filter the accessed memory areas for each file and check if areas were written
before reading. If so, we can also skip testing those bits, as the program would
overwrite the flips again. Listing~\ref{lst:pinlogcode} shows an example code
snippet of a pintool which logs all memory accesses as it instruments each
instruction and logs the opcode position and its parameters if they address
memory. The listing~\ref{lst:pinlogcode} also shows some API-calls for the PIN
framework. In Pin, the Image~\texttt{img} refers to files. These are most of the
time dynamically loaded libraries. The \texttt{INS\_InsertPredicatedCall}
function is used to insert a function call on the machine code layer. The
function can have an arbitrary number of parameters, as the user have to define
the function with each parameter and the type. Pin then inserts a call to the
given function pointer and manages the handling of the parameters and return
values in a way that the instrumented binary's execution is not affected.

Instrumentation is done via the configured \texttt{instrumenter\_call}. That
entry is executed and expected to generate a file named in the
\texttt{instrumenter\_outfile} configuration parameter. This file needs to
follow an output format for the accesses as in \texttt{<hex:position\_in\_file>
- str:filename}. The framework then applies a filter to that before it tests the
actual bitflips.

\begin{minipage}{\linewidth}
\begin{lstlisting}[style=CStyle,
                   caption={Example C++ code for a pintool logging memory
accesses. At first, the tool stores the instruction bytes, and after the
operands are stored separately, depending on if they are writing or reading
memory.},
                   label={lst:pinlogcode}]
ADDRINT ins_addr = INS_Address(ins);
IMG img = IMG_FindByAddress(ins_addr);
ADDRINT base = IMG_LowAddress(img);
img_offsets[base] = IMG_LoadOffset(img);

for(size_t i = 0; i < INS_Size(ins); i++)
  accesses.insert(std::make_pair(ins_addr + i, base));

UINT32 memOperands = INS_MemoryOperandCount(ins);
for (UINT32 memOp = 0; memOp < memOperands; memOp++)
{
  if(INS_MemoryOperandIsRead(ins, memOp))
  {
    INS_InsertPredicatedCall(
      ins, IPOINT_BEFORE, (AFUNPTR)RecordMemRead,
      IARG_INST_PTR,
      IARG_MEMORYREAD_EA,
      IARG_ADDRINT, base,
      IARG_MEMORYREAD_SIZE,
      IARG_END);
  }
  if(INS_MemoryOperandIsWritten(ins, memOp))
  {
    INS_InsertPredicatedCall(
      ins, IPOINT_BEFORE, (AFUNPTR)RecordMemWrite,
      IARG_INST_PTR,
      IARG_MEMORYWRITE_EA,
      IARG_ADDRINT, base,
      IARG_MEMORYWRITE_SIZE,
      IARG_END);
  }
}
\end{lstlisting}
\end{minipage}

\subsubsection{Filtering Memory Accesses reported by the Instrumentation}

We apply a filter to the memory accesses at two different points. At first, we
already filter inside the written pintool. As we log reading and writing
accesses to memory separately, we can check on reading accesses if the program
has written to the position before, if so, there is no need to flip this
location in the binary.

At second, after the program run, we translate all reported accessed addresses
to positions in ELF files. At this point, every access which does not refer to
the file content is thrown away. We can find this as we know the base address
from the loaded library from the pintool report.

Additionally, we add addresses of ELF structure headers, as remapping sections
or change other property bits inside the ELF structure can affect the way the
binary runs. After applying the filters, we gain a list of bytes to flip.

\subsubsection{Generating and Testing the Flips}

Next, we create a new ELF file for each bit for each of the reported bytes. We
generate those files into a \texttt{tmpfs} file-system placing its memory in the
DRAM to have lower access times than to the hard drive. Depending on the
configured number of parallel checks, chroots are created in the \texttt{tmpfs}
too. The framework makes sure the number of created ELF files does not need more
than the memory available. If there are more files to test than the available
memory allows, the framework takes multiple runs, as seen in
figure~\ref{fig:frameworkdesign}. A test run starts configured
\texttt{CR\_exec\_file} which will copy a flipped file from
\texttt{CR\_flip\_folder} to the original position, then run it and check if the
output matches a defined success state. If the flipped file reaches the success
state, the framework writes the flipped address to \texttt{CR\_log\_file} and
copies the flipped ELF file to \texttt{CR\_success\_folder}. The framework
passes the file to replace to the \texttt{CR\_exec\_file} when it calls it. To
keep track of which bitflip is tested the filename for generated ELF files is
structured as \texttt{<original\_filename>\_<address\_in\_file>\_<bit\_number>}.

\subsection{Applying the Framework to real-world Applications}

In our thesis, we want to apply our tool to more extensive programs, programs
used in the real world and with bitflips in mind which an attacker also would
use. As Gruss~\etal\cite{flipinthewall} already showed, there are bits in the
\texttt{sudoers} library which when toggled, allow using ~\texttt{sudo} without
or with a wrong password to still gain super-user privilege. We want to show,
that our framework can find a more significant number of possible bitflips to
achieve this. Also, we also want to show how we can search for bitflips in the
\texttt{nginx} program to bypass HTTP basic authentication.

\subsubsection{\texttt{sudo} - Privilege Escalation}

For \texttt{sudo}, the task was to find all possible bitflips which allow
switching to \texttt{root}, without knowing the correct user's password.
\texttt{sudo} is a \texttt{setuid}-binary, which changes it's process owner to
\texttt{root} when being executed. We ran into problems with our framework
because of this property of the binary.

\paragraph{Instrumentation} for \texttt{sudo} became more advanced than for
simple programs, because of the way Pin works. Usually, when starting a program
with a pintool, Pin would modify the program's memory to add its internal
instrumenter calls and then execute the program. As \texttt{sudo} is owned by
\texttt{root}, Pin is not allowed to modify the memory when running as regular
user. Running the \texttt{sudo} with Pin attached as \texttt{root} is possible,
but starting \texttt{sudo} as super-user would not trigger the password dialogue
nor trigger any of the authentication check execution paths.

\begin{minipage}{\linewidth}
\begin{lstlisting}[style=CStyle,
                   caption={Code of the pre-loaded library to keep the process
waiting for some milliseconds, which gives enough time for Pin to attach to the
process.},
label=lst:presleeplib]
extern const char *__progname;

__attribute__((constructor)) void init(void)
{
  if(!strcmp(__progname, "sudo_instr"))
  {
    size_t i = 0;
    while(i++ < 600000000);
  }
}
\end{lstlisting}
\end{minipage}

To bypass this, we use the attach-to-process functionality of Pin, which
wouldn't change the context of the program and still run it as a regular user.
The problem for this is, finding the program ID and attach to it takes time,
which allows us only to attach at the point where the password dialogue is
already popped-up. Pre-loading a wait library which keeps the process sleeping
resolves this issue, and we would instrument enough of the code.

Listing~\ref{lst:presleeplib} shows the code used to generate such a pre-loaded
library. We use the counting while-loop instead of a call to the sleep system
call because attaching Pinto a process in sleeping state results in a state,
where the process would never wake up again. The number of increments was chosen
to have enough time to search the process ID and attach to it. The compiler
attribute \texttt{constructor} makes sure the program executes this code on
loading the library before any other code. We use the
\texttt{/etc/ld.so.preload} file to make our library load for any program, to
not have all of them waiting on start-up, we check the name of the binary.

\paragraph{Testing} the \texttt{sudo} program is easy, we call it and give it a
wrong password. There are multiple ways to verify a successful privilege change.
For once, it would be possible to use the \texttt{whoami} program to check if
the system changed the current user to \texttt{root}. On the other side, it is
also possible to verify by reading a file owned by the other user.

\subsubsection{\texttt{nginx} - Basic Authentication}

For \texttt{nginx} we want to achieve an authentication bypass by flipping a
single bit. This means we need to set up a page which is only accessible given a
correct user and password credential pair.

\paragraph{Instrumentation} for \texttt{nginx} is straightforward. We had to
modify the default configuration a little to make sure the server runs as a
single process. This was done by not allowing \texttt{nginx} to run as a system
daemon or use their master process. We also only want to instrument the code run
when an actual request to the protected site happens. Therefore, we again use
the functionality to attach Pin to an already running process. For
instrumentation, we start the web server, attach Pin, sent a request to the
protected site with wrong credentials, wait for the denying answer from the
server and then end instrumentation. This should give us only the bytes used
during a basic authentication request.

\paragraph{Testing} the bitflips in \texttt{nginx} works the same as in the
instrumentation. We copy the flipped ELF, start the server, verify it is running
by requesting a standard page, then requesting the protected site with wrong
credentials, a successful state is if the web server returns the protected site
to a request with wrong credentials. For requesting the protected site, we would
use \texttt{netcat} instead of \texttt{curl} or \texttt{wget}, as we only want
to make sure the content is delivered to us and not if the answer got malformed
in any way by the bitflip.

\subsection{Results given by the Framework}

Looking back at the introduced flips for the code in
listing~\ref{lst:disasmsimplebranch}, we can see that permission checks can be
bypassed in the same way for both programs we looked at. As at some point jump
instructions will point to code executed only in a successful state. We can also
see flips causing a success which change bits in other sections than the code
one. We want to show which ELF files have bits where flips would cause
successes and what areas in ELF files are most promising for flips to be found.

\subsubsection{\texttt{sudo} - Privilege Escalation}

Table~\ref{tab:sudores} shows the number of found flips per file. We see that
the possible flips are spread over multiple files.  The \texttt{sudoers} library
contains most of them, whereas we also can find flips affecting a privilege
change in the linker (\texttt{ld-linux}) and the threading library
(\texttt{libpthread}). As expected, most flips are in the \texttt{.text}
section. Additionally, other main sections such as \texttt{.rodata} contain
flips. The \texttt{.plt} section refers to the Procedure Linkage Table, which is
used to resolve mappings between position-independent functions and their
absolute addresses. Close to the same is the \texttt{.got.plt} section, which
represents the Global Offset Table, which resolves mappings for position
independent code parts.

\begin{table}[]
\begin{tabular}{c|cccc|c}
ELF File & \texttt{.text}  & \texttt{.rodata} & \texttt{.got.plt} &
\texttt{.plt} & Sum of flips:                             \\ \hline
\texttt{ld-linux-x86-64.so.2} & 1   & 0  & 0  & 0  & 1    \\
\texttt{libpthread.so.0}      & 2   & 0  & 0  & 0  & 2    \\
\texttt{sudo}                 & 6   & 0  & 0  & 0  & 6    \\
\texttt{libnns\_compat.so.2}  & 6   & 0  & 1  & 0  & 7    \\
\texttt{libpam.so.0}          & 64  & 0  & 0  & 0  & 64   \\
\texttt{pam\_unix.so}         & 85  & 0  & 0  & 1  & 86   \\
\texttt{sudoers.so}           & 219 & 35 & 5  & 0  & 259  \\ \hline
Sum of flips:                 & 383 & 35 & 6  & 1  & 425
\end{tabular}
\caption{List of ELF files used by the \texttt{sudo(1.8.19p1)} program and the
number of flips causing privilege escalation listed per section.}
\label{tab:sudores}
\end{table}

\begin{minipage}{\linewidth}
\begin{lstlisting}[style=diff,
                   caption={Diff for a bitflip applied to \texttt{libnss} in
order to bypass a user privilege check. The call to \texttt{strcmp} is
replaced because of the offset in the lookuptable being one off.},
label=lst:sudoex]
     351b:      0f 84 90 00 00 00       je     35b1
     3521:      48 89 ee                mov    %rbp,%rsi
-    3524:      e8 47 dc ff ff          callq  1170 <strcmp@plt>
+    3524:      e8 c7 dc ff ff          callq  11f0 <malloc@plt>
     3529:      85 c0                   test   %eax,%eax
     352b:      0f 85 3e ff ff ff       jne    346f
\end{lstlisting}
\end{minipage}

In listing~\ref{lst:sudoex}, we show a diff for \texttt{libnss} where a call to
\texttt{strcmp} is replaced with a call to \texttt{malloc} by a single bitflip,
causing the later \texttt{jne} instruction to behave differently.

\subsubsection{\texttt{nginx} - Basic Authentication}

For the basic authentication bypass in \texttt{nginx} we can report that only
flips inside the program's binary itself were useable. Table~\ref{tab:nginxres}
shows the number of flips found and that all flips are inside the \texttt{.text}
section of the binary.

\begin{table}[]
\centering
\begin{tabular}{c|c}
ELF File               & \texttt{.text} \\ \hline
\texttt{nginx}         & $298$
\end{tabular}
\caption{List of flips inside the \texttt{nginx(nginx/1.10.3)} program causing a
basic authentication bypass, whereas all of them were in the \texttt{.text}
section.}
\label{tab:nginxres}
\end{table}

\begin{minipage}{\linewidth}
\begin{lstlisting}[style=diff,
                   caption={Diff for a bitflip applied to the \texttt{nginx}
binary in order to bypass a credential check. The call to \texttt{test} is
replaced by \texttt{add} which doesn't chage a register the \texttt{setne}
instruction would check.},
label=lst:nginxex]
    3c935:      48 83 c0 01             add    $0x1,%rax
    3c939:      0f b6 10                movzbl (%rax),%edx
-   3c93c:      84 d2                   test   %dl,%dl
+   3c93c:      04 d2                   add    $0xd2,%al
    3c93e:      0f 95 c1                setne  %cl
    3c941:      80 fa 24                cmp    $0x24,%dl
\end{lstlisting}
\end{minipage}

In listing~\ref{lst:nginxex}, we show a diff for the \texttt{nginx} binary
itself, the flip in this example changes a call to \texttt{test} to
\texttt{add}. The instruction afterwards is \texttt{setn}, used to set a byte if
the condition, "not equal" in this case, is met. As \texttt{add} does not affect
the "zero flag"-register, which the \texttt{setn} checks, the code is likely to
behave differently.
%}}}

%% vim:foldmethod=expr
%% vim:fde=getline(v\:lnum)=~'^%%%%\ .\\+'?'>1'\:'='
%%% Local Variables:
%%% mode: latex
%%% mode: auto-fill
%%% mode: flyspell
%%% eval: (ispell-change-dictionary "en_US")
%%% TeX-master: "main"
%%% End:
