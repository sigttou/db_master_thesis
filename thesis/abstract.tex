%%%% Time-stamp: <2013-02-25 10:31:01 vk>

\phantomsection
\addcontentsline{toc}{chapter}{Abstract}
\chapter*{Abstract}
\label{cha:abstract}

Microarchitectural attacks exploit design choices made by hardware vendors.
Cache attacks, or attacks like Meltdown and Spectre, target the CPU.
Microarchitectural attacks either target design choices or exploit bugs
introduced by mistakes in the hardware or the microcode. Rowhammer attacks
exploit a design flaw in DRAM chips. Hardware vendors steadily decreased sizes
and lowered refresh rates of the cells. With specially crafted memory access
routines it is possible to access DRAM cells fast enough to create interference
between cells, which will cause them to change their load, hence, changing their
logical state.

Rowhammer attacks use this phenomenon to target memory areas where bitflips
would benefit the attacker. Researchers have shown that bitflips in page tables
can cause privilege escalation and bypass security mechanisms implemented in
modern operating systems. It has also been shown that similar attacks work for
memory chips used by solid-state disks. Besides flipping bits in page tables,
researchers showed that when flipping bits in executables, the program's
behaviour can change. With such changes, it is possible for an attacker to abuse
programs for privilege escalation, or to bypass authentication-checks.

In this thesis, we present a way of testing binaries for possible execution-path
changes introduced by bitflips. In our work, we pre-define an outcome for a
program and then search for all single bitflips which cause the program to
behave in the desired way. We scan the entire address space of the program,
including all dynamically loaded libraries. We show results for searched
bitflips in programs used for authentication on Linux-based systems. We bypass
user privilege checks, which lead to privilege escalation, or make login
possible without knowing the user\textquotesingle s password. We also show a
bypass of HTTP basic authentication, allowing an attacker to download files
which unauthenticated users are not allowed to access. In addition to searching
for bitflips in executable files, we also look at possible other attack vectors
for Rowhammer. We show that bitflips applied to the runtime of cryptographic
calculations can break assumptions made by the communicating parties and can
even allow key leakage. We apply bitflips to the implementation of AES-GCM in
OpenSSL and show how Rowhammer can be used to cause reusing of nonces.

With our work, we want to increase the awareness of Rowhammer and show how
software security is affected by bitflips. We call on to all vendors of hardware
to not forget to keep their systems secure and do not put lower prices and
higher performance ahead of security, which would harm their users.
\cleardoublepage
\phantomsection
\addcontentsline{toc}{chapter}{Kurzfassung}
\begin{otherlanguage}{ngerman}
\chapter*{Kurzfassung}
\label{cha:kurzfassung}

Angriffe auf die Mikroarchitektur zielen meistens auf Fehler von
Hardwareherstellern ab. Attacken auf Caches nutzen hierbei ein gewolltes
Verhalten des Systems aus. CPU Lücken wie Meltdown und Spectre machen
sich Fehler im Design der Hardware zu Nutzen, welche vom Hersteller
entweder durch Mikrocode Updates oder einem Tausch der CPU berichtigt werden
müssen. Rowhammer macht sich einen Designfehler in DRAM Chips zu Nutze.
Hersteller dieser Speicherchips produzieren immer kleinere Chips mit
geringeren Refresh-Zyklen. Spezielle Reihenfolgen von Speicherzugriffen
ermöglichen es Interferenzen zu erzeugen welche Ladungsänderungen in
benachbarten Speicherzellen verursachen, diese können dadurch ihren logischen
Zustand wechseln.

Rowhammer Angriffe machen sich dieses Verhalten zu Nutze und zielen damit auf
Speicherbereiche ab, welche durch eine Änderung dem Angreifer einen Vorteil
verschaffen. Forscher haben gezeigt, dass es möglich ist mit Bitflips in Page
Tables Privilegien einen Superusers zu bekommen. Ebenso wurde gezeigt, dass
ähnliche Angriffe auch auf Speicherchips in Solid-State-Disks möglich sind.
Neben Flips in Page Tables wurde auch gezeigt, dass Änderungen in ausführbaren
Files Folgen auf das Verhalten des Programms haben, sie ändern dieses durch das
Wechseln eines einzigen Bits. Dies kann zum Beispiel dazu führen dass
Berechtigungsüberprüfungen umgangen werden.

Wir zeigen eine Möglichkeit Programme auf solche Verhaltensänderungen durch
Bitflips zu testen. Wir geben hier ein gewünschtes Verhalten vor und suchen
danach nach allen möglichen Bitflips welche das Verhalten in den gewünschten
Zustand ändern. Im Gegensatz zu früheren Arbeiten haben wir diesen Vorgang
automatisiert um eine größere Anzahl von Programmen abzudecken, zusätzlich
betrachten wir den gesamten Speicher in ausführbaren Files und dynamisch
geladenen Software-Bibliotheken. Wir zeigen die gefundenen Bitflips welche
Sicherheitsüberprüfungen umgehen, wie zum Beispiel Passwortabfragen. Auch
zeigen wir Bitflips, welche es erlauben HTTP-Authentication Checks in einem
Webserver zu umgehen. Zusätzlich zur Untersuchung von statischen Files
betrachten wir auch die Auswirkung von Bitflips auf kryptographische
Algorithmen. Hier untersuchen wir, wie Rowhammer dazu genutzt werden kann um in
der AES-GCM Implementation von OpenSSL eine falsche Verwendung von Noncen zu
verursachen.

Unsere Arbeit soll auf die Sicherheitsrisiken die durch Fehler wie Rowhammer
entstehen hinweisen und als Aufruf an die Hersteller von Hardware dienen, damit
diese nicht billigere Hardware und bessere Performance über die Sicherheit
ihrer Anwender stellen.

\end{otherlanguage}

%\glsresetall %% all glossary entries should be used in long form (again)
%% vim:foldmethod=expr
%% vim:fde=getline(v\:lnum)=~'^%%%%\ .\\+'?'>1'\:'='
%%% Local Variables:
%%% mode: latex
%%% mode: auto-fill
%%% mode: flyspell
%%% eval: (ispell-change-dictionary "en_US")
%%% TeX-master: "main"
%%% End:
