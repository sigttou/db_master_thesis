%%%% Time-stamp: <2013-02-25 10:31:01 vk>

\phantomsection
\addcontentsline{toc}{chapter}{Abstract}
\chapter*{Abstract}
\label{cha:abstract}

Microarchitectural attacks exploit mistakes made by hardware producers, who rely
heavily on a perfect design. Cache-attacks or attacks like Meltdown and Spectre
target the CPU, either they exploit a principle which works as designed and is
therefore vulnerable or abuse bugs introduced by mistakes in the hardware
design or the microcode. Rowhammer makes use of a design flaw made by DRAM
vendors, who were forced to lower quality and therefore decrease the size and
refresh rates of memory cells. With specially crafted memory access routines it
is possible to access DRAM cells fast enough to create interfering loads between
other cells, which will cause them to change their load, hence changing their
logic state.

Rowhammer attacks use this phenomenon to target memory areas where flips would
benefit the attacker. Researchers have shown in the past that flips to
page-tables can cause privilege escalation and bypass security measures placed
by modern operating systems. It also has been shown that similar attacks also
work with memory chips used inside solid-state disks. Besides flipping bits in
page-tables, researchers showed that when flipping bits in executables, the
behaviour of programs changes. With such changes, it is possible for an attacker
to abuse programs for privilege escalation or authentication check bypasses.

We present a way of testing binaries for possible execution path changes applied
by bitflips. In our work, we give a pre-defined outcome for a binary and then
search for all single bitflips which would cause the program to behave in the
desired way. Whereas previous work mostly looked at just op-codes in machine
code, we scan the whole memory provided by the binary. In addition to the
executable, we also check all dynamically loaded libraries used inside the
program. We show results for searched bitflips in the \texttt{sudo} program, for
bypassing user privilege checks, allowing super-user permissions without knowing
the password and in the \texttt{nginx} web server, for bypassing HTTP
authentication, allowing an attacker to load files which he was not allowed to
access. In addition to searching for bitflips in static files, we also looked at
possible other attack vectors for Rowhammer. We show that when bitflips are
applied to the runtime of cryptographic calculations, these could break
assumptions made by the communicating parties and could even allow key leakage.
We apply bitflips to the implementation of AES-GCM inside OpenSSL and show how
Rowhammer can be used to cause nonce-reuse.

With our work, we want to increase the awareness of Rowhammer and show how
software security is affected by bitflip. We call on to all vendors of hardware
to not forget to keep their systems secure and do not put competition and
performance ahead of security, which would harm their users.
\cleardoublepage
\phantomsection
\addcontentsline{toc}{chapter}{Kurzfassung}
\begin{otherlanguage}{ngerman}
\chapter*{Kurzfassung}
\label{cha:kurzfassung}

Angriffe auf die Mikroarchitektur ziehlen meistens auf Fehler von
Hardwareherstellern ab. Attacken auf Caches nutzen hierbei ein gewolltes
Verhalten des Systems aus. Populäre CPU Lücken wie Meltdown und Spectre machen
sich jedoch Fehler im Design der Hardware zu nutzen, welche vom Hersteller
entweder durch Mikrocode Updates oder einem Tausch der CPU berichtigt werden
können. Rowhammer macht sich einen Designfehler in DRAM Chips zu Nutze.
Hersteller dieser Speicherchips wurden im Laufe der Zeit zu immer besserer
Performance und kleineren Chips gezwungen, hierbei wurde aber die
Qualitätskontrolle vernachlässigt. Mit speziellen Reihenfolgen von
Speicherzugriffen ist es möglich Interferenzen zu erzeugen welche eine
Ladungsänderung in benachbarten Speicherzellen verursachen, diese wechseln
dadurch ihren logischen Zustand.

Rowhammer Angriffe machen sich dieses Verhalten zu Nutze und ziehlen damit auf
Speicherbereiche ab welche dem Angriffer einen Vorteil verschaffen. Forscher
haben in der Vergangenheit gezeigt, dass es möglich ist mit Bitflips in
Page-Tables Privilegien einen Super-Users zu bekommen. Ebenso wurde gezeigt
dass ähnliche Angriffe auch auf Speicherchips in Solid-State-Disks möglich
sind. Neben Flips in Page-Tables wurde auch gezeigt, dass Änderungen in
ausführbaren Files Folgen auf das Verhalten des Programms haben, sie ändern
dieses durch das Welchseln eines einzigen Bits. Dies kann zum Beispiel dazu
führen dass Zugriffsabfragen umgangen werden.

Wir zeigen eine Möglichkeit Programme auf solche Verhaltensänderungen durch
Bitflips zu testen. Wir geben hier ein gewünschtes Verhalten vor und suchen
danach nach allen möglichen Bitflips welche das Verhalten in den gewünschten
Zustand ändert. Im Gegensatz zu früheren Arbeiten haben wir diesen Vorgang
automatisiert um eine größere Anzahl von Programmen abzudecken, zusätzlich
betrachten wir den Gesammten Speicher in ausführbaren Files und dynamisch
geladenen Software-Bibliotheken. Wir zeigen die gefundenen Bitflips für das
\texttt{sudo} Programm, welche die Passwortabfrage umgehen und den
\texttt{nginx} Web-Server, wobei wir hier die HTTP-Authentication Checks
umgehen. Zusätzlich zur Untersuchung von statischen Files betracteten wir auch
die Auswirkung von Bitflips auf kryptographische Algorithmen. Hier betrachteten
wir, wie Rowhammer dazu genutzt werden kann um in der AES-GCM Implementation von
OpenSSL eine falsche Verwendung von Noncen zu verursachen.

Unsere Arbeit soll auf die Sicherheitsriskien die durch Fehler wie Rowhammer
entstehen hinweisen und als Aufruf an die Hersteller von Hardware dienen, damit
diese nicht Wettbewerb und Performance über die Sicherheit ihrer Anwender
stellen.

\end{otherlanguage}

%\glsresetall %% all glossary entries should be used in long form (again)
%% vim:foldmethod=expr
%% vim:fde=getline(v\:lnum)=~'^%%%%\ .\\+'?'>1'\:'='
%%% Local Variables:
%%% mode: latex
%%% mode: auto-fill
%%% mode: flyspell
%%% eval: (ispell-change-dictionary "en_US")
%%% TeX-master: "main"
%%% End:
