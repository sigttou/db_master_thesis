%%%% Time-stamp: <2013-02-25 10:31:01 vk>

\phantomsection
\addcontentsline{toc}{chapter}{Abstract}
\chapter*{Abstract}
\label{cha:abstract}

Microarchitectural attacks exploit design choices and mistakes made by hardware
producers. Cache attacks or attacks like Meltdown and Spectre target the CPU.
They either target a design which works as designed or exploit bugs introduced
by mistakes in the hardware or microcode design. Rowhammer exploits a design
flaw by DRAM vendors, who lowered their quality in favour of smaller cell sizes
and lower refresh rates of those. With specially crafted memory access routines
it is possible to access DRAM cells fast enough to create leakage loads between
cells, which will cause them to change their load, hence, changing their logic
state.

Rowhammer attacks use this phenomenon to target memory areas where flips would
benefit the attacker. Researchers have shown in the past that flips in page
tables can cause privilege escalation and bypass security mechanisms implemented
in modern operating systems. It has been shown that similar attacks also work
with memory chips used inside solid-state disks. Besides flipping bits in page
tables, researchers showed that when flipping bits in executables, the program's
behaviour changes. With such changes, it is possible for an attacker to abuse
programs for privilege escalation or authentication-check bypasses.

We present a way of testing binaries for possible execution path changes by
bitflips. In our work, we pre-define an outcome for a binary and then search for
all single bitflips which would cause the program to behave in the desired way.
Whereas previous work mostly looked at just op-codes in machine code, we scan
the entire address space of the program, including all dynamically loaded
libraries. We show results for searched bitflips in the \texttt{sudo} program,
for bypassing user privilege checks, allowing super-user permissions without
knowing the password and in the \texttt{nginx} web server, for bypassing HTTP
authentication, allowing an attacker to load files which unauthenticated users
are not allowed to access. In addition to searching for bitflips executable
files, we also looked at possible other attack vectors for Rowhammer. We show
that when bitflips are applied to the runtime of cryptographic calculations,
these could break assumptions made by the communicating parties and could even
allow key leakage. We apply bitflips to the implementation of AES-GCM inside
OpenSSL and show how Rowhammer can be used to cause nonce-reuse.

With our work, we want to increase the awareness of Rowhammer and show how
software security is affected by bitflip. We call on to all vendors of hardware
to not forget to keep their systems secure and do not put lower prices and
higher performance ahead of security, which would harm their users.
\cleardoublepage
\phantomsection
\addcontentsline{toc}{chapter}{Kurzfassung}
\begin{otherlanguage}{ngerman}
\chapter*{Kurzfassung}
\label{cha:kurzfassung}

Angriffe auf die Mikroarchitektur zielen meistens auf Fehler von
Hardwareherstellern ab. Attacken auf Caches nutzen hierbei ein gewolltes
Verhalten des Systems aus. Populäre CPU Lücken wie Meltdown und Spectre machen
sich jedoch Fehler im Design der Hardware zu Nutzen, welche vom Hersteller
entweder durch Mikrocode Updates oder einem Tausch der CPU berichtigt werden
müssen. Rowhammer macht sich einen Designfehler in DRAM Chips zu Nutze.
Hersteller dieser Speicherchips wurden im Laufe der Zeit zu immer besserer
Performance und kleineren Chips gezwungen, hierbei wurde aber die
Qualitätskontrolle vernachlässigt. Mit speziellen Reihenfolgen von
Speicherzugriffen ist es möglich Interferenzen zu erzeugen welche eine
Ladungsänderung in benachbarten Speicherzellen verursacht, diese wechseln
dadurch ihren logischen Zustand.

Rowhammer Angriffe machen sich dieses Verhalten zu Nutze und zielen damit auf
Speicherbereiche ab, welche durch eine Änderung dem Angreifer einen Vorteil
verschaffen. Forscher haben in der Vergangenheit gezeigt, dass es möglich ist
mit Bitflips in Page Tables Privilegien einen Super-Users zu bekommen. Ebenso
wurde gezeigt, dass ähnliche Angriffe auch auf Speicherchips in
Solid-State-Disks möglich sind. Neben Flips in Page Tables wurde auch gezeigt,
dass Änderungen in ausführbaren Files Folgen auf das Verhalten des Programms
haben, sie ändern dieses durch das Wechseln eines einzigen Bits. Dies kann zum
Beispiel dazu führen dass Berechtigungsüberprüfungen umgangen werden.

Wir zeigen eine Möglichkeit Programme auf solche Verhaltensänderungen durch
Bitflips zu testen. Wir geben hier ein gewünschtes Verhalten vor und suchen
danach nach allen möglichen Bitflips welche das Verhalten in den gewünschten
Zustand ändern. Im Gegensatz zu früheren Arbeiten haben wir diesen Vorgang
automatisiert um eine größere Anzahl von Programmen abzudecken, zusätzlich
betrachten wir den gesamten Speicher in ausführbaren Files und dynamisch
geladenen Software-Bibliotheken. Wir zeigen die gefundenen Bitflips für das
\texttt{sudo} Programm, welche die Passwortabfrage umgehen und den
\texttt{nginx} Web-Server, wobei wir hier die HTTP-Authentication Checks
umgehen. Zusätzlich zur Untersuchung von statischen Files betrachten wir auch
die Auswirkung von Bitflips auf kryptographische Algorithmen. Hier untersuchen
wir, wie Rowhammer dazu genutzt werden kann um in der AES-GCM Implementation von
OpenSSL eine falsche Verwendung von Noncen zu verursachen.

Unsere Arbeit soll auf die Sicherheitsrisiken die durch Fehler wie Rowhammer
entstehen hinweisen und als Aufruf an die Hersteller von Hardware dienen, damit
diese nicht Wettbewerb und Performance über die Sicherheit ihrer Anwender
stellen.

\end{otherlanguage}

%\glsresetall %% all glossary entries should be used in long form (again)
%% vim:foldmethod=expr
%% vim:fde=getline(v\:lnum)=~'^%%%%\ .\\+'?'>1'\:'='
%%% Local Variables:
%%% mode: latex
%%% mode: auto-fill
%%% mode: flyspell
%%% eval: (ispell-change-dictionary "en_US")
%%% TeX-master: "main"
%%% End:
