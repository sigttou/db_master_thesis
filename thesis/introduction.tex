%%%%%% Introduction %%%%%%%%%%%%%%%%%%%%%%%%%%%%%%%%%%%%%%%%%%%%%%%%%%%%%%%%%{{{
\chapter{Introduction}\label{sec:intro}

With the latest releases of Microarchitectural attacks like
Meltdown~\cite{meltdown} and Spectre~\cite{spectre}, the topic of flaws in
hardware implementations became known by the general public. Many media outlets,
like BBC~\cite{bbcmeltdown} and NBC~\cite{nbcmeltdown}, reported on these issues
of modern CPUs. Vendors of x86-architectures, like Intel or AMD, are affected,
but also ARM, and with it, most mobile devices have similar flaws.

Such issues show that these vendors have set performance above security. The
demand for faster hardware is rising all the time, and not only vendors of CPUs
are influenced by this demand. Another field of silicon chip design ran into a
similar problem in the past, namely DRAM chip vendors.

In 2014, Kim~\etal~\cite{rowhammergeneral} released their paper ``RowHammer'',
showing how specially crafted memory-access routines can cause bits in DRAM
chips to flip, without accessing them directly. This work showed how higher
memory densities caused faults where interfering voltages and leaking currents
influence other storage cells. While at first, this was just seen as a stability
issue, Google\textquotesingle s Project Zero showed how Rowhammer can be used
for privilege escalation and sandbox escapes~\cite{projectzerorow}. With reports
like this, the interest in researching the field of Rowhammer increased.
Gruss~\etal~\cite{rowhammerjs} showed that it is not only possible to target
systems by executing native code, but also that Rowhammer can be triggered by
using JavaScript. Van der Veen~\etal\cite{drammer} published work named
``Drammer'', where they show how not only desktop computers are affected by the
Rowhammer bug, but also mobile devices. Earlier this year,
Gruss~\etal~\cite{nethammer}, released a way to trigger bitflips by sending
specially-crafted network requests. Publications like these show that Rowhammer
is an active research topic, where still new findings come up.

This thesis builds on work released by Gruss~\etal~\cite{flipinthewall}, which
states that application-binary code can directly be attacked with Rowhammer.
They show, for example, that a bitflip applied to \texttt{sudo} can result in a
bypass of the password check. They report some bitflips causing this bypass.
They look at the disassembly of the authentication-check code and find opcodes,
which when changed, would result in a different outcome. With our work, we want
to automate, and therefore simplify, this process. With this, we predict a
finding of a higher number of bitflips in a shorter time. In addition to that,
we want to provide a toolset, which allows us to apply similar searches to other
applications. This tool should be able to run lots of tests in parallel and
verify the outcomes. Therefore, we want to make use of modern testing
techniques.

Testing and debugging were always a significant part of software technology, and
with rising sizes of projects and an increasing number of old code bases, it is
more vital than ever. Not only developers are putting much work into these
topics, but also researchers releasing new ways of testing regularly. As the bug
reports of ``american fuzzy lop''~\cite{aflweb} in many open-source software
show, modern approaches for testing like fuzzying make bug searching in unknown
code more successful. Also, the field of proving software\textquotesingle s
correctness got much attention. With symbolic execution techniques, the
possibility to prove each software state on its own got more practical. The
release of the open-source symbolic-execution framework
\texttt{angr}~\cite{angrpaper} made it possible for a wide range of users to
apply symbolic execution to programs. This tool mostly gets used in testing, in
combination with fuzzing, like Stephens~\etal~\cite{driller} showed in their
work ``Driller''. Also, security researchers use \texttt{angr} to find
exploitable code segments and execution paths, such as
Shoshitaishvili~\etal~\cite{firmalice} showed with their work ``Firmalice'',
where \texttt{angr} was used to detect authentication bypasses in firmware
images.

Understanding what programs do, and how they are executed by the CPU, gets
harder with every improvement and change in hardware design. Instrumentation is
a technique to inject code into programs, providing the possibility to collect
runtime information. With tools like Intel Pin~\cite{pintool} it is possible to
check changes to processor registers, log accessed memory and performance
measuring at machine code level.

\section{Goals and Motivation for the Thesis}

As we know from previous work done by Gruss~\etal~\cite{flipinthewall}, there
are bitflips in the ELF files loaded by the \texttt{sudo} program which allow
privilege escalation by providing a password check bypass. However, they only
looked at the code section providing the permission check. They could not claim
to find all flips, and their approach is very time-consuming. We want to
simplify the search by automatic testing of flips. Also, we want to make it
easier for future applications to be tested for possible bitflip outcomes by
providing a test framework.

Common Unix-based operating systems use package management to roll-out
applications to users. Every instance of the operating system then uses the same
binary to execute. With this in mind, a bitflip found in the \texttt{sudo}
application distributed with a major Linux distribution, can be used to attack
all instances of such a system. An attacker, therefore, can use the test
framework to find bitflips in widely distributed binaries.

With our work, we want to present an easy-to-apply framework to search for
bitflips providing a pre-defined outcome. To show how this framework works, we
apply it to real-world applications and compare our results to the ones reported
by Gruss~\etal~\cite{flipinthewall}. Also, we want to show in general how likely
such bitflips are in applications.

\section{Contributions of this Work}

Our contribution to the field of microarchitectural attacks and Rowhammer is
providing a practical analysis of real-world applications and how bitflips can
affect them. We present a framework which can be used to find bitflips changing
a program\textquotesingle s behaviour to a pre-defined outcome. The structure of
the framework is designed to be extensible and adaptable for multiple purposes.

We apply our tool to real-world applications to show the impact bitflips could
have on users of personal computers. On one hand, we show how privilege
escalation bitflips can be found in the \texttt{sudo} program. We show bits,
which when flipped, allow us to skip the password check. On the other hand, we
also analyse the popular \texttt{nginx} web server. For this application, we
show bitflips which permit an attacker to bypass HTTP authentication measures.
We present results for these applications and if there exist bits, which when
flipped, allow us to achieve our set outcome. Besides analysing the bits inside
the program\textquotesingle s executable, we also examine any dynamically loaded
library it uses. By that, we also cover possibilities where external functions
could change the application\textquotesingle s outcome.

In addition to that, we look at possible cryptographic vulnerabilities
introduced by bitflips. As a basis, we take the work by
Böck~\etal~\cite{gcmnonceattack}, who showed how it is possible for web servers
to misuse nonces when using AES-GCM. We build on their approach to bypass the
fixes applied by server software to re-introduce this nonce misusage via
bitflips. Here we look at the current implementation of AES-GCM in the TLS
library OpenSSL. We show that nonce misusage can be reintroduced by bitflips,
and give a probability for them to happen during a Rowhammer attack.

\section{Outline of this Work}

This thesis is structured as follows:
In section~\ref{sec:general}, we describe general terms and technologies our
work is build on or makes use of. We discuss other microarchitectural attacks,
and give an overview of the functionality of programs which our work targets.
In section~\ref{sec:elfattack}, we discuss our work regarding the automatic
bitflip search. We present our testing framework and discuss how bitflips could
be introduced to systems.
In section~\ref{sec:results}, we show our results and present multiple tested
applications and what adaptions had to be applied to the framework for
successful testing.
In section~\ref{sec:dynattack}, we discuss our work regarding Rowhammer attacks
targeting dynamic memory. We thereby show how the OpenSSL implementation of
AES-GCM can be attacked by flipping bits.
In section~\ref{sec:countermeasure}, we discuss countermeasures which could be
applied to improve system security. We discuss countermeasures against
microarchitectural attacks in general, and discuss on what could be done to
reduce the impact of our tests. In section~\ref{sec:futurework}, we show
possible future works, and an overview of directions the research in the field
of microarchitectural attacks could take.
In section~\ref{sec:conclusion}, we close our thesis with a summary and give a
conclusion of our work.
%}}}

%% vim:foldmethod=expr
%% vim:fde=getline(v\:lnum)=~'^%%%%\ .\\+'?'>1'\:'='
%%% Local Variables:
%%% mode: latex
%%% mode: auto-fill
%%% mode: flyspell
%%% eval: (ispell-change-dictionary "en_US")
%%% TeX-master: "main"
%%% End:
