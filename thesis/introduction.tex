%%%%%% Introduction %%%%%%%%%%%%%%%%%%%%%%%%%%%%%%%%%%%%%%%%%%%%%%%%%%%%%%%%%{{{
\chapter{Introduction}\label{sec:intro}

Computers run code from files provided as executables. These files include some
properties for the operating system on how to execute the binary. Besides this
information, the executable contains the actual code encoded as instructions for
the CPU.
Changes to the instructions or properties might change the whole behaviour of
the program to do something completely different from what its original purpose
was.

Rowhammer is a side-effect introduced by dynamic random-access memory (DRAM)
vendors in the past. Yoongnu \etal~\cite{rowhammergeneral} describe this
phenomenon as an effect where memory could change without accessing it.

\section{Goals and Motivation for the Thesis}

\section{Contribution of this Work}

\section{Outline of this Work}

We present this work separated into different chapters. We start with a general
overview of topics in this field. We begin by describing how programs get
executed on modern computers and how operating systems handle application. For
this, we also dig deeper into the general design for executables on Unix like
systems, as we discuss the executable and linkable format (ELF) in more detail.
As ELF files hold machine code executed by CPUs and our work relies heavily on
how machine code is built, we also discuss the design of instructions in modern
CPU architectures.  We then go on and look at different testing techniques in
software development, where we compare and describe several options and check
their advantages and disadvantages. In particular, we work out details about
fuzzing, symbolic execution and instrumentation. For our work, we try to change
the behaviour of programs by modifying their execution path, most of the time
this behaviour is changed to gain improved privilege. Therefore we discuss the
permission model used by Unix-based systems. Also, we also look at details of
how permission switches work on these systems, especially of how the
\texttt{setuid} property works. For permission separation and testing purposes
we also take a look at the \texttt{chroot} provided by most Unix-like operating
systems. We do not only target behaviour changes to gain a higher privilege but
also target changes to bypass permission checks. And besides local attacks, we
also look at remote possibilities. Therefore, we describe the networking and
security principles used by most computers. We look at common web servers and
TLS libraries and how these provide security. As our work targets cryptographic
implementations, we take a detailed look at the advanced encryption standard
(AES) and a variant of it using Galois/counter mode (AES-GCM). Our work relies
heavily on rowhammer, which is a software-based microarchitectural attack.
Therefore, we take a look at these attacks in general and give an overview of
state of the art attacks using similar techniques. As timing plays a vital role
in exploiting side-channels in this area, we take a look at precise timing
measurement methods. We look at recent cache attacks and the impact those have
on modern systems. We close our background overview with a details description
of the rowhammer bug.

We continue with describing our distribution. This is split into two parts, one
being our testing of bitflips in ELF files and the other being attacks against
cryptographic functions during runtime. We start with the ELF analysis and
describe what impact a single bitflip can have to the execution path of a
program. We go on with outlining possibilities to find bitflips which would
change the behaviour in a manner so that it would benefit an attacker. We
describe the design of our automated bitflip-search framework and how we applied
it to real-world applications. We resume by showing the results of our tests for
the applications of \texttt{sudo} and \texttt{nginx}. We also mention how we
would apply these bitflips to a systems by using the rowhammer bug.

In the second part of our contribution we describe the influence of bitflips on
cryptographic implementations, we discuss the problems of nonce misuse and how
this problem could be introduced to AES-GCM implemented in OpenSSL with
bitflips. We give numbers for a likelihood of such a nonce-misuse introduced by
a single bit flip. Also, we show how we tested this issue in a practical setup
with a simple web server using OpenSSL.

We go on with looking at countermeasures for issues in the field of
microarchitectural attacks. We look at possible ways to prevent cache attacks
and rowhammer. Also, we line out ways of reducing the impact of our tests on
real-world setups. We close this section by describing how general system
security could be improved.

As microarchitectural attacks are an on-going and growing research field, we
also look at possible future works. We look at the future of these attacks in
general, how open source architectures could help to prevent some of them. We
also talk about a possible attribution of machine learning to help to find new
attack vectors. We also find possibilities to improve the impact of rowhammer,
by looking at new attacks coming by applying it to further implementations of
cryptography. In the end, we talk about how our framework could be helpful for
future work and how it can be used in various testing and research environments
with just little adaptions.

We close our thesis by giving a conclusion describing our results and provide a
summary of the outcome of our work.
%}}}

%% vim:foldmethod=expr
%% vim:fde=getline(v\:lnum)=~'^%%%%\ .\\+'?'>1'\:'='
%%% Local Variables:
%%% mode: latex
%%% mode: auto-fill
%%% mode: flyspell
%%% eval: (ispell-change-dictionary "en_US")
%%% TeX-master: "main"
%%% End:
