%%%%%% Introduction %%%%%%%%%%%%%%%%%%%%%%%%%%%%%%%%%%%%%%%%%%%%%%%%%%%%%%%%%{{{
\chapter{Introduction}\label{sec:intro}

Microarchitectural attacks are becoming more and more popular. With the latest
contribution in this field, releases like Meltdown~\cite{meltdown} and
Spectre~\cite{spectre} made the topic known to the general public. Many media
outlets reported on the issues of modern CPUs and their design. Where it was
mostly x86 vendors like Intel or AMD, also ARM was affected by such exploiting
techniques.

Issues like these show one more time that vendors set performance above security
for their users. The demand for better releases of hardware is rising all the
time, and not only vendors of CPUs are affected by this demand. Another field of
silicon chip design ran into similar problems in the past, namely DRAM chip
vendors. In 2015 Kim~\etal~\cite{rowhammergeneral} released their paper
"RowHammer", showing how specially crafted memory access routines can cause bits
in the DRAM to flip, without accessing these directly. This work showed how the
demand for higher memory density caused vendors to lower quality management,
causing interfering voltages and currents to influence other memory storage
transistors. Whereas this was just seen as a stability issue, Google's Project
Zero showed how rowhammer could be used for privilege escalation and sandbox
escapes~\cite{projectzerorow}. With results like this research in the field of
rowhammer kicked off, where Gruss~\etal~\cite{nethammer} showed how bitflips
could be triggered by just specially crafted network requests and Van der
Veen~\etal~\cite{drammer} showed how rowhammer could be applied to mobile
platforms.

Testing and debugging always has been a significant part of software technology,
with rising sizes of projects and an increasing number of old code bases it is
more vital than ever. With modern approaches for testing like fuzzying, bug
searching in unknown code got more successful. With techniques like symbolic
execution and instrumentation understanding what programs do at machine code
layer got easier, allowing researchers to either see what happens to the
computing environment or prove that a state of a program is secure.

\section{Goals and Motivation for the Thesis}

With our work, we want to increase the awareness about rowhammer and making the
issue better known. Gruss~\etal~\cite{flipinthewall} have already shown that
current rowhammer defences are not suitable for attackers They show, that the
countermeasures do not prevent them from still abusing bitflips to gain
privilege escalation. We want to take this approach further to present an easy
way to find dangerous bitflips in programs. With our work, we want to increase
the impact of rowhammer once more, supporting the research in better
countermeasures and forcing vendors to built secure and higher quality DRAM
chips again.

With the knowledge gained from already released rowhammer attacks and the
knowledge gained from modern testing approaches, we want to combine the two
fields to get the best possible outcome for finding feasible bitflips.

Whereas securing systems is something wanted by most users and vendors, we want
to show that without flawless hardware components security claims on other
levels are worthless. We want to push the desire of secure code and operating
systems down to hardware layers to have users, and researchers care as much
about security in this field as they already do for their running programs and
internet connections.

\section{Contribution of this Work}

Our contribution to the field of microarchitectural attacks and rowhammer is
providing a practical analysis of real-world applications and how bitflips could
affect them. We present a framework which can be used to find bitflips suitable
to change a program's behaviour in a given way. This framework is extensible and
adaptable to multiple purposes. We show example use cases by finding bitflips
bypassing the password check in the \texttt{sudo} program and bitflips bypassing
HTTP authentication in the very popular \texttt{nginx} web server. We present
the achieving bitflips not only in the main executable file but also in loaded
dynamic libraries.

In addition to that, we contributed to rowhammer research by looking at possible
cryptographic vulnerabilities introduced by bitflips. We look at the work of
Böck~\etal~\cite{gcmnonceattack}, who showed how web servers were reusing nonces
when using AES-GCM. We took their approach to bypass the fixes applied by server
software to re-introduce this nonce misusage via bitflips.

\section{Outline of this Work}

We present this work separated into different chapters. We start with a general
overview of topics in this field. We begin by describing how programs get
executed on modern computers and how operating systems handle application. For
this, we also dig deeper into the general design for executables on Unix like
systems, as we discuss the executable and linkable format (ELF) in more detail.
As ELF files hold machine code executed by CPUs and our work relies heavily on
how machine code is built, we also discuss the design of instructions in modern
CPU architectures.  We then go on and look at different testing techniques in
software development, where we compare and describe several options and check
their advantages and disadvantages. In particular, we work out details about
fuzzing, symbolic execution and instrumentation. For our work, we try to change
the behaviour of programs by modifying their execution path, most of the time
this behaviour is changed to gain improved privilege. Therefore we discuss the
permission model used by Unix-based systems. Also, we also look at details of
how permission switches work on these systems, especially of how the
\texttt{setuid} property works. For permission separation and testing purposes
we also take a look at the \texttt{chroot} provided by most Unix-like operating
systems. We do not only target behaviour changes to gain a higher privilege but
also target changes to bypass permission checks. And besides local attacks, we
also look at remote possibilities. Therefore, we describe the networking and
security principles used by most computers. We look at common web servers and
TLS libraries and how these provide security. As our work targets cryptographic
implementations, we take a detailed look at the advanced encryption standard
(AES) and a variant of it using Galois/counter mode (AES-GCM). Our work relies
heavily on rowhammer, which is a software-based microarchitectural attack.
Therefore, we take a look at these attacks in general and give an overview of
state of the art attacks using similar techniques. As timing plays a vital role
in exploiting side-channels in this area, we take a look at precise timing
measurement methods. We look at recent cache attacks and the impact those have
on modern systems. We close our background overview with a details description
of the rowhammer bug.

We continue with describing our distribution. This is split into two parts, one
being our testing of bitflips in ELF files and the other being attacks against
cryptographic functions during runtime. We start with the ELF analysis and
describe what impact a single bitflip can have to the execution path of a
program. We go on with outlining possibilities to find bitflips which would
change the behaviour in a manner so that it would benefit an attacker. We
describe the design of our automated bitflip-search framework and how we applied
it to real-world applications. We resume by showing the results of our tests for
the applications of \texttt{sudo} and \texttt{nginx}. We also mention how we
would apply these bitflips to a systems by using the rowhammer bug.

In the second part of our contribution we describe the influence of bitflips on
cryptographic implementations, we discuss the problems of nonce misuse and how
this problem could be introduced to AES-GCM implemented in OpenSSL with
bitflips. We give numbers for a likelihood of such a nonce-misuse introduced by
a single bit flip. Also, we show how we tested this issue in a practical setup
with a simple web server using OpenSSL.

We go on with looking at countermeasures for issues in the field of
microarchitectural attacks. We look at possible ways to prevent cache attacks
and rowhammer. Also, we line out ways of reducing the impact of our tests on
real-world setups. We close this section by describing how general system
security could be improved.

As microarchitectural attacks are an on-going and growing research field, we
also look at possible future works. We look at the future of these attacks in
general, how open source architectures could help to prevent some of them. We
also talk about a possible attribution of machine learning to help to find new
attack vectors. We also find possibilities to improve the impact of rowhammer,
by looking at new attacks coming by applying it to further implementations of
cryptography. In the end, we talk about how our framework could be helpful for
future work and how it can be used in various testing and research environments
with just little adaptions.

We close our thesis by giving a conclusion describing our results and provide a
summary of the outcome of our work.
%}}}

%% vim:foldmethod=expr
%% vim:fde=getline(v\:lnum)=~'^%%%%\ .\\+'?'>1'\:'='
%%% Local Variables:
%%% mode: latex
%%% mode: auto-fill
%%% mode: flyspell
%%% eval: (ispell-change-dictionary "en_US")
%%% TeX-master: "main"
%%% End:
