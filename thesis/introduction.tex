%%%%%% Introduction %%%%%%%%%%%%%%%%%%%%%%%%%%%%%%%%%%%%%%%%%%%%%%%%%%%%%%%%%{{{
\chapter{Introduction}\label{sec:intro}

Computers run code from files provided as executables. These files include some
properties for the operating system on how to execute the binary. Besides this
information, the executable contains the actual code encoded as instructions for
the CPU.
Changes to the instructions or properties might change the whole behaviour of
the program to do something completely different from what its original purpose
was.

Rowhammer is a side-effect introduced by dynamic random-access memory (DRAM)
vendors in the past. Yoongnu \etal~\cite{flipbitanal} describe this phenomenon
as an effect where memory could change without accessing it.
%}}}

%% vim:foldmethod=expr
%% vim:fde=getline(v\:lnum)=~'^%%%%\ .\\+'?'>1'\:'='
%%% Local Variables:
%%% mode: latex
%%% mode: auto-fill
%%% mode: flyspell
%%% eval: (ispell-change-dictionary "en_US")
%%% TeX-master: "main"
%%% End:
