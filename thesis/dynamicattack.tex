%%%%%% Dynamic Data Attack %%%%%%%%%%%%%%%%%%%%%%%%%%%%%%%%%%%%%%%%%%%%%%%%%%{{{
\chapter{Bitflip Attacks on Dynamic Data}\label{sec:automate}

In this chapter we want to take a look at data created at the runtime of the
program, we will go into details about OpenSSL and how bitflips can make attacks
like a nonce-misuse possible. We want to apply a similar attack as Böck~\etal
describe in their work about "practical nonce misusage
attacks"~\cite{gcmnonceattack}. We also want to show how other attacks published
could be reintroduced by rowhammer.

\section{Analysis of OpenSSL for possible Nonce Misuse Flips}

We want to look at the implementation of AES-GCM inside OpenSSL. Therefore we
take a look at the structures used by it in listing~\ref{lst:aesstruct}.

\begin{minipage}{\linewidth}
\begin{lstlisting}[style=CStyle,
                   caption={Struct used by OpenSSL to describe AES.},
                   label={lst:aesstruct}]
#include <stdio.h>
int main(void)
{
  int x = 0;
  if(x == 1)
    printf("success.\n");
  else
    printf("fail.\n");
  return 0;
}
\end{lstlisting}
\end{minipage}

\subsection{Possible Nonce Reuse in AES-GCM with a single Bitflip}

Lorem ipsum dolor sit amet, consetetur sadipscing elitr, sed diam nonumy eirmod
tempor invidunt ut labore et dolore magna aliquyam erat, sed diam voluptua. At
vero eos et accusam et justo duo dolores et ea rebum. Stet clita kasd gubergren,
no sea takimata sanctus est Lorem ipsum dolor sit amet.
%}}}

%% vim:foldmethod=expr
%% vim:fde=getline(v\:lnum)=~'^%%%%\ .\\+'?'>1'\:'='
%%% Local Variables:
%%% mode: latex
%%% mode: auto-fill
%%% mode: flyspell
%%% eval: (ispell-change-dictionary "en_US")
%%% TeX-master: "main"
%%% End:
