%%%%%% Dynamic Data Attack %%%%%%%%%%%%%%%%%%%%%%%%%%%%%%%%%%%%%%%%%%%%%%%%%%{{{
\chapter{Bitflip Attacks on Dynamic Data}\label{sec:dynattack}

In this chapter, we want to take a look at data created at the runtime of the
program and how flips to that could benefit an attacker. We will go into details
about OpenSSL and how bitflips can enable for instance nonce-misuse attacks. We
want to apply a similar attack as Böck~\etal describe in their work about
``practical nonce misusage attacks''~\cite{gcmnonceattack}. By this, we want to
show how Rowhammer can introduce new attack vectors to applications.

\section{Analysis of OpenSSL for possible Nonce Misuse Flips}

\begin{figure}
\begin{minipage}{\linewidth}
\begin{lstlisting}[style=CStyle,
                   caption={Struct used by OpenSSL to describe the AES-GCM
context. The IV used is stored in the memory pointed to by \texttt{iv}. Source
is taken from OpenSSL version $1.1.0g$},
                   label={lst:aesstruct}]
typedef struct {
  union {
    double align;
    AES_KEY ks;
  } ks;                       /* AES key schedule to use */
  int key_set;                /* Set if key initialised */
  int iv_set;                 /* Set if an iv is set */
  GCM128_CONTEXT gcm;
  unsigned char *iv;          /* Temporary IV store */
  int ivlen;                  /* IV length */
  int taglen;
  int iv_gen;                 /* It is OK to generate IVs */
  int tls_aad_len;            /* TLS AAD length */
  ctr128_f ctr;
} EVP_AES_GCM_CTX;
\end{lstlisting}
\end{minipage}
\end{figure}

\begin{figure}
\begin{minipage}{\linewidth}
\begin{lstlisting}[style=CStyle,
                   caption={Context struct describing the Cipher used in TLS.
This struct is used as the SSL context inside OpenSSL. Source is taken from
OpenSSL version $1.1.0g$},
                   label={lst:ciphctx}]
struct evp_cipher_ctx_st {
  const EVP_CIPHER *cipher;
  ENGINE *engine;     /* functional reference if
                       * 'cipher' is ENGINE-provided */
  int encrypt;        /* encrypt or decrypt */
  int buf_len;        /* number we have left */
  unsigned char oiv[EVP_MAX_IV_LENGTH]; /* original iv */
  unsigned char iv[EVP_MAX_IV_LENGTH]; /* working iv */
  unsigned char buf[EVP_MAX_BLOCK_LENGTH]; /* saved partial
                                            * block */
  int num;           /* used by cfb/ofb/ctr mode */
  /* FIXME: Should this even exist? It appears unused */
  void *app_data;    /* application stuff */
  int key_len;       /* May change for variable length cipher */
  unsigned long flags; /* Various flags */
  void *cipher_data; /* per EVP data */
  int final_used;
  int block_mask;
  unsigned char final[EVP_MAX_BLOCK_LENGTH]; /* possible final
                                              * block */
} /* EVP_CIPHER_CTX */ ;
\end{lstlisting}
\end{minipage}
\end{figure}

We look at the implementation of AES-GCM inside OpenSSL. We see the general
AES-GCM context struct in Listing~\ref{lst:aesstruct}. Additional to that, we
take a look at the cypher-context struct in Listing~\ref{lst:ciphctx}. The
connection between those two is the initialisation vector \texttt{iv} which is
updated on each call to the GCM function.

To apply the described attack by Böck~\etal~\cite{gcmnonceattack}, we require
the \texttt{iv} to be reused by multiple GCM calls. OpenSSL uses a counter for
the \texttt{iv} for AES-GCM instead of a random value, flipping a bit to zero in
the \texttt{iv} causes a decrement of the counter which makes reuse likely. The
counter starts at a random value and is increased after. Therefore, a flip in
the lower bits is more likely to result in reuse of the nonce.
Böck~\etal~\cite{gcmnonceattack} state that this reuse constitutes breakage of
the cryptography for future messages and the attacker can craft valid
ciphertexts and overtake sessions. For the attack with Rowhammer, we target the
\texttt{iv} array in the \texttt{EVP\_CIPHER\_CTX} struct, as seen in
Listing~\ref{lst:ciphctx}.

\subsection{Likelihood of a Nonce-Misuse introduced by Rowhammer}

Lipp~\etal~\cite{nethammer} show that it is possible to send network requests
which cause memory accesses that induce Rowhammer faults. With this knowledge
and the possibility of nonce-misuse attacks in OpenSSL by bitflips, an attacker
could overtake TLS sessions with only remote access.

Looking at the OpenSSL source code, we can see that for each TLS connection at
least these two context structs are generated. There is one general SSL struct
needed, one AES-GCM context struct and two cypher contexts, as one is used for
sending and one for receiving. The working IVs make \SI{32}{\byte} of this
memory. The sum of the structs for one TLS connection is \SI{1960}{\byte}. If we
fill the DRAM with just these structs, a Rowhammer flip hitting an IV happens
with a probability of about \SI{1.5}{\percent}. Therefore, the attack is already
quite unlikely, as also, only lower bits are allowed to be hit, to make a nonce
counter reuse likely. Also, we cannot just hold these structs in memory, and the
operating system usually limits the number of parallel TLS connections.

\subsection{Analysis of Practical Nonce-Reuse caused by Rowhammer}

\begin{figure}
\begin{minipage}{\linewidth}
\begin{lstlisting}[style=CStyle,
                   caption={Code showing an example for a simple TLS server,
keeping sending a reply until the client disconnects.},
                   label={lst:ssltestcode}]
if (SSL_accept(ssl) <= 0) {
    ERR_print_errors_fp(stderr);
}
else {
  int run = 1;
  while(run)
  {
    if(SSL_write(ssl, reply, strlen(reply)) < 0)
      run = 0;
    sleep(1);
  }
}
SSL_free(ssl);
close(client);
\end{lstlisting}
\end{minipage}
\end{figure}

We set up two versions for the practical analysis of the nonce reuse, where one
uses multiple processes, and one uses multithreading. In both cases, we
implemented a simple endless sending loop which will keep the TLS connection to
any client open and sent a string every second. Listing~\ref{lst:ssltestcode}
shows the code parts used in both cases. The code makes sure the connection will
send the reply every second, as long as the client accepts the write.

For our tests, we set the used cypher suite to
\mbox{\texttt{ECDHE-RSA-AES256-GCM-SHA384}}, which makes sure the described GCM
structs are used by OpenSSL.

For our experiments, we had a memory increase of about \SI{5440}{\byte} for each
forked process. With threading, the increase was slightly less. We never
achieved filling a larger of the memory with TLS structs so that we came near
the \SI{1.5}{\percent}. Thus, the probability in practical implementations is
therefore even lower. If it is possible to get \SI{1000} parallel connections,
with \SI{5440}{\byte} each, with \SI{32}{\byte} in initialisation vectors, it
will result in \SI{5.19}{\mega\byte} of memory representing TLS connections,
with \SI{0.59}{\percent} of this being IVs. In a server setup with
\SI{4}{\giga\byte} of DRAM, the IVs would only make \SI{74.5e-6}{\percent}. Even
if an attacker could gain knowledge about the \SI{4}{\kilo\byte} pages used to
store TLS structs, it would be difficult to hit IVs with Rowhammer. Given this
knowledge, we can conclude that a remote attack on TLS nonces with Nethammer is
very unlikely to succeed.
%}}}

%% vim:foldmethod=expr
%% vim:fde=getline(v\:lnum)=~'^%%%%\ .\\+'?'>1'\:'='
%%% Local Variables:
%%% mode: latex
%%% mode: auto-fill
%%% mode: flyspell
%%% eval: (ispell-change-dictionary "en_US")
%%% TeX-master: "main"
%%% End:
