%%%%%% Countermeasures %%%%%%%%%%%%%%%%%%%%%%%%%%%%%%%%%%%%%%%%%%%%%%%%%%%%%%{{{
\chapter{Countermeasures}\label{sec:countermeasure}

In this chapter, we discuss countermeasures. First, we look at countermeasures
against microarchitectural attacks in general. Secondly, we look at
countermeasures against Rowhammer. In the end we discuss measures that
make our testing less practical, and our results less useful for an
attacker.

\section{Microarchitectural Attacks}

Fixes for microarchitectural attacks can be applied to different layers of a
system. The KAISER patch for kernel page-table isolation by
Gruss~\etal~\cite{kaiserpaper} is applied to the kernel to prevent leakage of
kernel memory by Meltdown~\cite{meltdown}. For other attacks, such as Spectre,
Kocher~\etal~\cite{spectre} state that lower layers, like the
CPU\textquotesingle s microcode, have to be fixed. Kocher~\etal~\cite{spectre}
also state that a long-term solution to prevent microarchitectural attacks is a
fundamental change of instruction set architectures.

\subsection{Rowhammer}

Besides fixing systems to prevent microarchitectural attacks, it is also
possible to prevent the exploitation of microarchitectural flaws by checking
programs for malicious behaviour. Irazoqui~\etal~\cite{mascat} present a
Micro-Architectural Side Channel Attack Trapper (MASCAT), which uses static
analysis of binaries to detect microarchitectural attacks.
Irazoqui~\etal~\cite{mascat} state characteristics of attacks and what code
parts indicate a microarchitectural attack. To detect a possible use of
Rowhammer, MASCAT checks for cache evictions in binaries.
Irazoqui~\etal~\cite{mascat} show that using \texttt{clflush} in a loop is
likely part of a Rowhammer attack. Also, \texttt{monvnti} and \texttt{movntdq}
are listed as suspicious, as these instructions allow direct access to the DRAM
and bypassing the cache. Besides looking for these instructions,
Irazoqui~\etal~\cite{mascat} also state, that a program using self-map
translation to resolve the mapping between physical and virtual memory is likely
to be a threat to the system as knowledge about this mapping is part of
Rowhammer attacks.

Besides statically checking binaries, checking the behaviour of a program also
can detect an ongoing attack. Aweke~\etal~\cite{anvil} present a software-based
solution to detect ongoing Rowhammer attacks, namely ANVIL. ANVIL uses hardware
performance monitoring to monitor DRAM row accesses. If a row is accessed
repeatedly, ANVIL forces refreshing of neighbouring rows.
Aweke~\etal~\cite{anvil} implemented ANVIL as a kernel module for Linux and
state that their detection and prevention system leads to an average slowdown of
\SI{1}{\percent} and a worst-case slowdown of \SI{3.2}{\percent} of the system.

Brasser~\etal~\cite{canttouch} show another software mitigation against
Rowhammer attacks. Their countermeasure prevents an attacker to corrupt kernel
memory from user mode. Brasser~\etal~\cite{canttouch} extend the physical memory
allocator of the Linux kernel to isolate the memory of the kernel and the
userspace on DRAM storage. The mitigation by Brasser~\etal~\cite{canttouch},
CATT, does not prevent bitflips from happening but removes the possibility to
exploit bitflips in kernel memory. CATT splits the memory layout into security
domains and makes sure that memory from different domains is not placed in
neighbouring rows on the DRAM chip. This placing prevents an attacker from using
kernel memory as a target for Rowhammer.

Gruss~\etal~\cite{flipinthewall} show that it is still possible to mount
privilege-escalation attacks with such countermeasures present. For the approach
by Brasser~\etal~\cite{canttouch}, they show that the isolation from kernel
memory does not prevent privilege escalation as they introduce bitflips to ELF
files, which then allow an attacker to gain superuser privileges. For defeating
static analysis and performance counter based mitigations,
Gruss~\etal~\cite{flipinthewall} present attacks based on the SGX enclave.
Gruss~\etal~\cite{flipinthewall} state that this defeats performance counter
monitoring countermeasures as executions inside the enclaves are not monitored
by the CPU\textquotesingle s performance counters.

To prevent exploiting ELF files with bitflips introduces to the storage devices
operating systems can store a hash of the executables and check the checksums
before execution. Bitflips attacks on storage devices are described by
Kurmus~\etal~\cite{rowssdhammer}. Cai~\etal~\cite{rownandhammer} show how MLC
NAND flash memory devices can be targeted to introduce exploitable memory
corruptions to solid-state drives. Checking the hash of the executable before
execution would detect such a change of the ELF file and can prevent the
execution of an exploiting binary.

\section{Making Bitflip Testing Less Practical}

With our testing framework, we can search for exploitable bitflips in ELF files
which are identical for every user of the software distribution. For example,
the same exploitable bitflips have the same position in all installations of the
current Debian operating system. Distributing randomised ELF files would make
the search for exploitable bitflips harder as an attacker would need to run the
search framework for each target.

However, randomised binaries would not comply with the current trend of research
in the field of reproducible builds show. The Debian-close organisation
reproducible-builds~\cite{reporbuilds} works on making building packages for
Debian reproducible. With their techniques, it is possible to produce the same
binary using the same compiler, build configuration and same source code.
Reproducible builds make binary changes detectable and allow a bit-by-bit
verification of the full build chain. Consequently, changes or backdoors
introduced by a malformed compiler or linker are detected.

Reproducible builds can also help to introduce other countermeasures. The
operating system can verify ELF files by using stored checksums.
Suh~\etal~\cite{memintegrity} show how memory integrity verification can bring a
similar check to programs at runtime. Gelbart~\etal~\cite{spee} present a secure
program execution environment (SPEE) providing static verification before
execution and code block verification at runtime. During compilation, SPEE adds
a signature and hashes to each ELF section of the executable. The operating
system checks these additions before performing the execution. To allow checking
during runtime, SPEE adds a security module to the Linux kernel which performs
additional checks during execution. Andre Rein~\cite{drive} present a software
component for dynamic runtime integrity verification and evaluation (DRIVE).
DRIVE adds monitoring for memory changes during the runtime of a program. It
uses data gained from the ELF file to check the memory of the process during the
runtime.

If decryption happens on each access to the file content based in DRAM,
encryption of binaries can be seen as a countermeasure to exploitable bitflips.
If a bit is flipped inside an encrypted ELF file, decryption is likely to
produce garbage machine code for the CPU to execute, meaning the program would
crash in case of an applied bitflip.
%}}}

%% vim:foldmethod=expr
%% vim:fde=getline(v\:lnum)=~'^%%%%\ .\\+'?'>1'\:'='
%%% Local Variables:
%%% mode: latex
%%% mode: auto-fill
%%% mode: flyspell
%%% eval: (ispell-change-dictionary "en_US")
%%% TeX-master: "main"
%%% End:
