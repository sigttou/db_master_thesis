%%%%%% Countermeasures %%%%%%%%%%%%%%%%%%%%%%%%%%%%%%%%%%%%%%%%%%%%%%%%%%%%%%{{{
\chapter{Countermeasures}\label{sec:countermeasure}

In this chapter, we want to look at countermeasures. At first, we look at
measures against Rowhammer and microarchitectural attacks in general. Then we
want to look at measures which would reduce the impact of our work. In the end,
we want to close with measures which could be applied to computer systems in
general to make them more secure.

\section{Microarchitectural Attacks}

For this kind of bugs, it is vital to check who is to blame for it, to know who
needs to define the countermeasure. For attacks like
Foreshadow\cite{foreshadow}, Meltdown\cite{meltdown} and
Spectre\cite{spectre} it is clear that that bug is caused by the CPUs
manufactured by Intel or AMD. Even if there are countermeasures like
Kaiser\cite{kaiserpaper}, which would prevent leakage of kernel memory by
Meltdown, the bug remains in the microcode, and other processes\textquotesingle
memory can still be leaked. Vendors need to address such problems directly, by
either deprecating CPUs or patch the bugs with microcode updates.
Countermeasures applied to other layers might not be as effective and bring a
larger performance drop with them.

\subsection{Cache Attacks}

Cache attacks are mostly timing-attacks, where based on the content of the
cache, accesses to memory take less time. CPU vendors cannot, and should not,
fix this, as the attack is based on the principle of how caches work.
Cache-timing attacks often try to gain information on other processes, by
loading the same library and check what parts are accessed.
Gruss~\etal\cite{gruss2015cache} showed how such an attack could be used to
build a keylogger, as memory accesses differ for different keys. They also show
how to generate cache-attacks automatically without an in-depth knowledge of
the targeted program and system. They also state that a possible removing of
the \texttt{clflush} instruction is not a sufficient countermeasure, as there
are ways to build cache-attacks with Evict+Reload. They propose disabling
shared-memory as a countermeasure, but operating-systems use this technology a
lot to have lower memory-footprints. They propose a solution where security
relevant functions can be marked as not shared, to remain the low
memory-footprint but protect functions which could leak sensitive information.
It would be down to developers of libraries and programs to mark these
functions. Overall there might be no perfect countermeasure to cache-attacks,
but there are ways to lower their impact on security.

\subsection{Rowhammer}

Rowhammer is an issue which occurs because of missing high-quality checks by
vendors, the leakage of bit-charges to change bits in other memory areas is a
hardware fault and needs fixing on this layer. Also, users of current DRAM
chips need to be protected, and their risk of using their hardware should be
minimalised. In their work about Rowhammer, Gruss~\etal\cite{flipinthewall}
analyse various countermeasures. Tools like the Micro-Architectural Side
Channel Attack Trapper (MASCAT) by Irazoqui~\etal\cite{mascat} could be used by
anti-virus software to prevent users from executing software making use of
cache-attacks. With a physical memory separation of kernel memory and user
memory, privilege escalation by applying bitflips to page-tables can be
prevented.

Brasser~\etal\cite{canttouch} showed how such a separation could be built
inside an OS by changing the code of the physical memory allocator. Another
way to prevent Rowhammer would be to detect an ongoing attack during the runtime
of a program. With their work about ANVIL, Aweke~\etal\cite{anvil} showed such
a detection technique. They track DRAM accesses using hardware performance
measures and check for frequently accessed memory rows to detect the so-called
aggressors. As prevention, they refresh nearby rows more often to make them
secure against possible flipping. Their benchmarking shows that they have less
than one per cent false-positives and an average performance drop of about one
per cent.

Operating system vendors could implement another countermeasure for the file
cache. Our work relies on the file cache to be used when we execute the ELF
file. The systems could either recheck file contents before execution, as
in check some checksum, or force loading binaries from the disk instead of
using the cache. In such cases, the performance drop would be rather small, but
it would not prevent attacks using Rowhammer on SSDs.

\section{Reducing the Impact of Bit-Flip Testing}

From looking at our binary analysis and the impact we get from knowing that
operating system ship the same ELF files to all their users, we can conclude
that having different ELF files per instance would drastically lower our impact.
Doing so would mean, an attacker would need to analyse bitflips per target.
Whereas this would be a valid countermeasure, it is not practical. Also, having
different ELF files per system breaks other security improvements. Compiler
developers and developers of wide-spread software currently push the technique
of reproducible builds. Whereas the Debian-close organisation
reproducible-builds\cite{reprobuilds} is one of the most significant
contributors in this field. The advantage of reproducible builds is that all
binaries compiled with the same compiler, same configuration and same
source code will not differ, this makes detection of changes very easy. The
technology allows a bit-by-bit compared verification of the full build chain
used for the program. By this, also changes or backdoors introduced by a
malformed compiler or linker could be detected.

Reproducible builds could also help to define other countermeasures to bit flips
in the binary memory. Whereas the operating system can verify ELF files by using
stored checksums, it would also be possible to use similar techniques to do this
with in-memory code or data at runtime. As Suh~\etal\cite{memintegrity} have
shown in their work about memory integrity verification and encryption for
secure processors. Other work that could be looked at as countermeasure is
DRIVE\cite{drive} by Rein~\etal or SPEE\cite{spee} by Gelbart~\etal. Those apply
code verification to binaries before execution and during runtime.

The encryption of binaries could in general seen as a countermeasure to
bitflips. If decryption happens on each access to the file stored in memory. If
the entire file is stored decrypted it would not improve security. If a bit is
flipped inside an encrypted ELF file, decryption is likely to fail, meaning the
binary could not continue execution and the program would crash in case of an
applied bitflip.

\section{Improving General System Security}

There is no unbreakable, unhackable, totally secure system. Users and vendors
need to be aware of this. If vulnerabilities are found vendors should react to
provide possibilities for their users to be secure. Like provide patches for
either their microcode, operating system kernels or programs itself. Users
should be informed about security updates and apply them to their environment
as soon as possible. For updates to be in place on time, it is vital for
vendors to work closely with operating system developers, to push patches very
soon after or even before public disclosure.

As systems should generally be seen as unsafe and programs running cannot be
trusted, a defensive approach for operating system development might be a right
approach. Defensive coding is usually based on a multiple wall approach. This
allows having defensive structures in place even if some defences are broken.
Whereas developers should find motivation in writing secure code, mistakes
happen, and old software is sometimes run. Bugs like buffer-overflow often lead
to code execution. Operating systems introduced techniques like address space
layout randomisation (ASLR), to make attacks abusing buffer-overflow harder as
an attacker would not know positions of injected code anymore. From this other
attack were designed, using return-oriented programming (ROP) and using known
positions inside shared libraries to craft code from code-pieces inside those.
In their work Ruan~\etal\cite{ropsur}, show a survey of ROP defence mechanisms.
They point out the effectiveness of randomisation in memory, which would also
increase the defence for Rowhammer. Additional there are approaches described
using instrumentation to detect ROP attacks, which could again also be used to
detect the usage of Rowhammer.
%}}}

%% vim:foldmethod=expr
%% vim:fde=getline(v\:lnum)=~'^%%%%\ .\\+'?'>1'\:'='
%%% Local Variables:
%%% mode: latex
%%% mode: auto-fill
%%% mode: flyspell
%%% eval: (ispell-change-dictionary "en_US")
%%% TeX-master: "main"
%%% End:
