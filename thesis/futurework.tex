%%%%%% Future Work %%%%%%%%%%%%%%%%%%%%%%%%%%%%%%%%%%%%%%%%%%%%%%%%%%%%%%%%%%{{{
\chapter{Future Work}\label{sec:futurework}

Based on the current releases in the area of rowhammer and microarchitectural
attacks, in general, it will get more interesting to automate attacks on this
layer to make it possible to target multiple programs and systems without
in-depth knowledge about the target. With automated attacks and vulnerability
search not only the number of attacks rises but also security measures, as
techniques like fuzzing allow testing without knowing too much about the code,
developers can have their code tested and improve stability.

In general, we can see a more connecting work between microarchitectural attacks
and classic exploitation techniques. With improved attacks also awareness of
vulnerabilities such as cache attacks or other side-channels is improved as more
users are affected. Latest attacks such as Meltdown~\cite{meltdown} and
Spectre~\cite{spectre} by Lipp and Gruss~\etal have shown this already. Another
example of a famous microarchitectural attack Foreshadow~\cite{foreshadow},
released in August 2018 by Bulck~\etal. The three attacks are related as they
target the CPU internal code used to execute machine code given to it.

\todo{add details about Meltdown, Spectre, Foreshadow}

Attacks like these show a real-world threat and need to be patched on a
low-system level. Vendors need to publish fixes for the microcode. Operating
systems then need to push these fixes to its users and update the CPU's internal
code. Operating System vendors fixed other bugs, like Meltdown and Spectre by
kernel code changes, which caused a performance drop for the users.

With more attacks in this field targeting end-users, cloud environments, servers
and mobile devices. We can see the number of published work rising in the near
future. Automated exploit technologies will, therefore, influence
microarchitecture attacks even more. We can see researchers pushing more on
fuzzying on hardware-level and microcode-level. As knowledge on about how CPUs
work rises, we predict an increasing number of similar attacks on various other
architectures.

While CPU vendors need to improve their architecture and fix parts of their
CPU's microcode, researchers and developers will try to avoid bugs and CPU based
exploits by pushing open architectures and open implementations more. The most
popular Free Software instruction set architecture is RISC-V. It is built on the
principles of reduced instruction set computing (RISC). RISC-V started in 2010,
and lots of researchers have contributed since then. There are implementations
in place, which implement System-on-Chip designs of RISC-V, such as
lowRISC~\cite{lowrisc}. Other research areas use RISC-V  to implement CPU
designs to improve systems, by adding security or optimising options. Such as
Ming~\etal~\cite{smarts} showed in their work of introducing a memory protection
unit (MPU) in a RISC-V SOC. The MPU is used to encrypt and verify memory in
DRAM, where they use a Rocket core with tagged memory based on a lowRISC
implementation. Tagged memory is a technology where bedsides pointers to memory
also information is stored about the referenced memory. For example, in a 64bit
system, one could use 48 bits for addressing memory and use the other 16 bits to
store information, like access rights, encryption information or a checksum of
the referenced memory area. With increasing attacks on memory safety such as
rowhammer, we can see an increase in such countermeasures like tagged memory.

As machine learning is another fast-growing field in computer science, we can
see that testing and exploitation can learn more and more from already found
bugs. This data can be used to find errors in new code. There also exist bug
prediction models, like Puranik~\etal~\cite{bugprediction} showed in their work.
These models can be applied to code and predict if there is a bug. We can see
that these prediction models will influence microarchitectural attacks, as the
behaviour of the CPU in certain states could be analysed and conclusions about
faulty behaviour could be drawn.

\todo{write about fuzzying on CPU level/opcode level}

For attacks based on memory accesses, we can take away that tools such as
Pin~\cite{pintool} or angr~\cite{angrpaper} will be a significant factor for
analysis. The work of Chabbi~\etal\cite{pincallpaths} could also be taken to
improve our attack as they provide an even more in-depth view of execution paths
in binaries. Using such detailed graphs, more bytes to test could be filtered,
speeding up our framework even more.
%}}}

%% vim:foldmethod=expr
%% vim:fde=getline(v\:lnum)=~'^%%%%\ .\\+'?'>1'\:'='
%%% Local Variables:
%%% mode: latex
%%% mode: auto-fill
%%% mode: flyspell
%%% eval: (ispell-change-dictionary "en_US")
%%% TeX-master: "main"
%%% End:
