%%%%%% Future Work %%%%%%%%%%%%%%%%%%%%%%%%%%%%%%%%%%%%%%%%%%%%%%%%%%%%%%%%%%{{{
\chapter{Future Work}\label{sec:futurework}

In this chapter, we present an outlook for attacks in the field of Rowhammer. We
derive possible future attack vectors from current trends. We also look at
combinations of attacks, where Rowhammer could reintroduce exploits, especially
regarding attacks in cryptography. Finally, we find possibilities to extend our
testing framework in future works.

\section{Future Work regarding Rowhammer}

Kim~\etal~\cite{rowhammergeneral} state that it is not possible to make
definitive claims on how word lines interact in DRAM chips and how the bitflips
happen in detail. Kim~\etal~\cite{rowhammergeneral} also state that there are
possible ways of interaction based on their studies and findings but research of
DRAM chips at the device-level, regarding Rowhammer, is still missing.

While vendors could fix the exploitation in DRAM chips, by increasing refresh
rates or built non-load-leaking chips, there is other hardware which already
faces similar issues, as Cai~\etal~\cite{rownandhammer} show for MLC NAND flash
memory used by solid-state drives.

We show how nonce misusage can be introduced by single bitflips to target an
AES-GCM implementation. Attackers could use this idea to target other
cryptographic implementations as well, especially those using a counter nonce.

\section{Improving and Reusing our Framework}

For attacks based on memory accesses, we can take away that tools such as
Pin~\cite{pintool} or \texttt{angr}~\cite{angrpaper} will be a significant
factor for analysis. The work of Chabbi~\etal~\cite{pincallpaths} could also be
taken to improve our testing framework as they provide an even more in-depth
view of execution paths in binaries. By using such detailed graphs, more
bytes-to-test could be filtered, speeding up the framework.

On the other side, we can view our framework as a base implementation for other
test environments for parallel testing use cases. The environment is easy to
adapt, and developers can exchange single components without much effort.
Testing inside \texttt{chroot} could be changed to container environments like
Docker~\cite{docker}, to provide more safety or better abstraction of the host
operating system. Also, virtual machines could be used, if the software has to
be tested on other kernels or operating systems. The testing scripts could be
replaced to work with multiple input files for a program instead of different
binaries. This would make it possible to use the framework as a basis for a
fuzzer. Even a combination with fuzzers would be possible, as the framework
could be combined with \texttt{afl}~\cite{aflweb} to provide a parallel fuzzing
framework, which would either allow fuzzing in multiple environments or as a
general way to parallelise \texttt{afl}.

Instead of instrumenting binaries with Intel Pin, replacements or other sources
for analysis could be used. For instance, the QBDI~\cite{qbdi} tool for dynamic
instrumentation could be used instead, which supports a wider range of software
architectures, such as $x86$, ARM and AArch64. Additionally, a complete open
source solution for instrumentation is available in DynamoRIO~\cite{dynrio},
which could also replace Pin.
%}}}

%% vim:foldmethod=expr
%% vim:fde=getline(v\:lnum)=~'^%%%%\ .\\+'?'>1'\:'='
%%% Local Variables:
%%% mode: latex
%%% mode: auto-fill
%%% mode: flyspell
%%% eval: (ispell-change-dictionary "en_US")
%%% TeX-master: "main"
%%% End:
