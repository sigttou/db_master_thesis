%%%%%% Future Work %%%%%%%%%%%%%%%%%%%%%%%%%%%%%%%%%%%%%%%%%%%%%%%%%%%%%%%%%%{{{
\chapter{Future Work}\label{sec:futurework}

In this chapter we want to present an outlook for attacks in the field of
microarchitectural attacks, we want to look at these attacks in general first
and then derive possible future outcomes from current trends. We also want to
look at the combination of these attacks with testing techniques used in other
areas. Also, we want to look at attacks in cryptography which Rowhammer could
reintroduce. In the end, we want to find some possibilities for how future
projects or research could use our work.

\section{Microarchitectural Attacks}

Based on the current releases in the area of Rowhammer and microarchitectural
attacks in general, it will get more interesting to automate attacks on this
layer to make it possible to target multiple programs and systems without
in-depth knowledge about the target. With automated attacks and vulnerability
search, not only the number of attacks rises but also security measures, as
techniques like fuzzing allow testing without knowing too much about the code,
developers can have their code tested and improve stability.

In general, we can see a more connecting work between microarchitectural attacks
and classic exploitation techniques. With improved attacks also awareness of
vulnerabilities such as cache-attacks or other side-channels is improved as more
users are affected. Latest attacks, such as Meltdown~\cite{meltdown} and
Spectre~\cite{spectre} by Lipp and Gruss~\etal, have shown this already. Another
example of a famous microarchitectural attack Foreshadow~\cite{foreshadow},
released in August 2018 by Bulck~\etal. The three attacks are related as they
target the CPU internal code used to execute machine code given to it.

Attacks like these show a real-world threat and need to be patched on a
low-system level. Vendors need to publish fixes for the microcode. Operating
systems then need to push these fixes to its users and update the
CPU\textquotesingle s internal code. Some operating system vendors fixed
bugs, like Meltdown and Spectre, by kernel code changes, which caused a
performance drop for the users.

\subsection{Future of Microarchitectural Attacks}

We can see that the number of published work in this field will rise, with more
attacks in this field targeting end-users, cloud environments, servers and
mobile devices. Automated exploit technologies will, therefore, influence
microarchitecture attacks even more. We can see researchers pushing more on
fuzzying on hardware-level and microcode-level. As knowledge on about how CPUs
work rises, we predict an increasing number of similar attacks on various other
architectures.

\subsubsection{Microarchitectural Attacks in JavaScript}

With already a high number of side-channel attacks coming to JavaScript and
attacks like Spectre being run inside web browsers, we can see the number of
microarchitectural attacks in JavaScript rising. Gruss~\etal~\cite{rowhammerjs}
have already shown how the attack of Rowhammer can be ported to JavaScript. With
attacks like this, the number of possible targets for these attacks rises.
Victims only need to open a website and are targeted with Spectre, Rowhammer
and other attacks. In this way, the need for native code execution is not given
any more to run the exploiting code. Additional, attacks could also be ported
to mobile devices and browsers running on Android and iOS operating systems.
Gruss~\etal~\cite{keytimejs} already showed how timing attacks ported to
JavaScript could affect end users on mobile devices. With the rising numbers of
attack vectors in the field, and the number of victims, we can see an increasing
awareness of the issue in the wider user base. Therefore, more pressure is
building up against vendors of operating systems, processing units or DRAM
chips to address these issues and give users a feeling of security back.

\subsubsection{Using Fuzzing to find more Microarchitectural Flaws}

With the increasing complexity of CPUs and their design including more
dependency on microcode, we see an increasing number of bugs coming from this
area. Finding bugs by automation is already a topic in security research, for
microarchitectural attacks we can see this trend continuing.
Domas~\cite{sandsifter} already showed how fuzzying can be applied to the x86
ISA in his work, named \texttt{sandsifter}. The tool is open-source and
published on GitHub~\cite{sandsifterurl}. \texttt{sandsifter} is written in
Python, and the developer built it to audit x86 CPUs. It can be used to find
hidden instructions or hardware bugs. The tool generates machine code, executes
it and checks the behaviour of the CPU during execution. The tool already
succeeded in finding undocumented opcodes, bugs in disassembles and hypervisors
which also were security related. The results and whitepaper were released at
the Blackhat conference in 2017~\cite{sandsifter}. With such a tool already in
place, we can see adaptions of it for other architectures in the near future.
Also, besides only fuzzying instructions, we can also see similar tools being
introduced for memory accesses to fuzz cache behaviour for example.

\subsection{Open Source Architectures}

While CPU vendors need to improve their architecture and fix parts of their
CPU\textquotesingle s microcode, researchers and developers will try to avoid
bugs and CPU-based exploits by pushing open architectures and open
implementations more. The most popular Free Software instruction set
architecture is RISC-V. It is built on the principles of reduced instruction
set computing (RISC). RISC-V started in 2010, and lots of researchers have
contributed since then. There are implementations in place, which provide
System-on-Chip (SOC) designs of RISC-V, such as lowRISC~\cite{lowrisc}. Other
research areas use RISC-V to implement CPU designs to improve systems by
adding security or optimising options. Such as Ming~\etal~\cite{smarts} showed
in their work of introducing a memory protection unit (MPU) in a RISC-V SOC.
The MPU is used to encrypt and verify memory in DRAM, where they use a Rocket
core with tagged memory-based on a lowRISC implementation. Tagged memory is a
technology where bedsides pointers to memory also information is stored about
the referenced memory. For example, in a 64-bit system, one could use 48 bits
for addressing memory, and use the other 16 bits to store information, like
access rights, encryption information, or a checksum of the referenced memory
area. With increasing attacks on memory safety such as Rowhammer, we can see an
increase in such countermeasures, like tagged memory.

\subsection{Using Machine Learning to Detect Bugs}

As machine learning is another fast-growing field in computer science, we can
see that testing and exploitation can learn more and more from already found
bugs. This data can be used to find errors in new code. There also exist bug
prediction models, like Puranik~\etal~\cite{bugprediction} showed in their work.
These models can be applied to code and predict if there is a bug. We can see
that these prediction models will influence microarchitectural attacks, as the
behaviour of the CPU in certain states could be analysed and conclusions about
faulty behaviour can be drawn.

\section{Future Work regarding Rowhammer}

We see Rowhammer as a still open field of research, while vendors could fix the
exploitation in DRAM chips, by increasing refresh rates or built
non-load-leaking chips, there is other hardware which already faces similar
issues, as Kurmus~\etal~\cite{rowssdhammer} showed for SSDs. We can see more
devices and techniques targeted with such attacks.

Also, an in-depth analysis of the Rowhammer is still an open issue in the
research of microarchitectural attacks. Analysing this issue might bring better
knowledge of the hardware and why it has its faults. With this kind of
knowledge is might be possible to target bits better in the DRAM, allowing the
Rowhammer attack to gain more attention and increase their attack vector even
more.

\subsection{Possible Attacks against Cryptography using Rowhammer}

We showed how nonce-misusage could be introduced by single bitflips to target
AES-GCM implementations. Attackers could use this idea to target other
cryptographic implementations too. Especially those using a counter nonce. Also,
from our work and research inside the OpenSSL source code, we can see other
memory areas to be attacked, such as session reuse by flipping bits, or changing
used cyphers if identifiers are enumerated. Such attacks could not only break
encryption or allow key-recovery but also allow to break authenticity and
confidentiality when an attacker flips bits in tags inside message
authentication codes.

Dobraunig~\etal~\cite{noncestat} showed statistical attacks against nonce-based
authenticated encryption schemes. They show fault attacks on various modes of
AES, such as GCM, CCM, EAX or OCB. They show how faults would affect the
algorithm when they are introduced before the MixColumn step of AES. Work on
faults in memory, like Rowhammer, could be used as a fault source here and work
like this could be used to show future attacks using single bitflips. We can see
more upcoming work in that regard in the future of fault attack research.

Another possible field of research would be to check if it is possible to bring
linear or differential properties to S-box calls via bitflips, or if targeting
the functionality of shifting and replacing could break cryptographic
assumptions made by AES.

We already saw a combination between microarchitectural attacks and
cryptographic attacks. Such as Yarom~\etal~\cite{noncerec} showed, that it is
possible to use FLUSH+RELOAD cache attacks to recover ECDSA nonces in OpenSSL.
They show an attack which allows a partial key-recovery in they Elliptic Curve
Algorithm implementation with only observing a single signing process. With a
key recovery on this level, it allows forgery for future data transmissions.

\section{Improving and Reusing our Framework}

For attacks based on memory accesses, we can take away that tools such as
Pin~\cite{pintool} or \texttt{angr}~\cite{angrpaper} will be a significant
factor for analysis. The work of Chabbi~\etal~\cite{pincallpaths} could also be
taken to improve our testing framework as they provide an even more in-depth
view of execution paths in binaries. Using such detailed graphs, more
bytes-to-test could be filtered, speeding up the framework.

On the other side, we can see our framework working as a base-implementation for
other test environments for parallel testing use cases. The environment is very
adaptable and developers can exchange single components without much effort.
Testing inside \texttt{chroot} could be changed to container environments like
Docker~\cite{docker}, to provide more safety or better abstraction of the host
operating system. Also, virtual machines could be used, if the software has to
be tested on other kernels or operating systems. The testing scripts could be
replaced to work with multiple input files for a program instead of different
binaries. This would make it possible to use the framework as a basis for a
fuzzer. Even a combination with fuzzers would be possible, as the framework
could be combined with \texttt{afl}~\cite{aflweb} to provide a parallel fuzzing
framework, which would either allow fuzzing in multiple environments or as a
general way to parallelise \texttt{afl}.

For the part where binaries are instrumented with Intel Pin at the moment,
replacements or other sources for analysis could be used. Like the
QBDI~\cite{qbdi} tool for dynamic instrumentation could be used instead, which
provides a wider range for software architectures, such as ARM or AArch64.
Additional, a complete open source solution for instrumentation is available in
DynamoRIO~\cite{dynrio}, which could also replace Pin.

As other use cases for our framework, we could see it being used as a basis for
malware research, to track the behaviour of malware in different environments,
as it allows setting up many different systems in a short time and tracks
events inside those systems.
%}}}

%% vim:foldmethod=expr
%% vim:fde=getline(v\:lnum)=~'^%%%%\ .\\+'?'>1'\:'='
%%% Local Variables:
%%% mode: latex
%%% mode: auto-fill
%%% mode: flyspell
%%% eval: (ispell-change-dictionary "en_US")
%%% TeX-master: "main"
%%% End:
