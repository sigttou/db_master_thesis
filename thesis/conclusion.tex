%%%%%% Conclusion %%%%%%%%%%%%%%%%%%%%%%%%%%%%%%%%%%%%%%%%%%%%%%%%%%%%%%%%%%%{{{
\chapter{Conclusion}\label{sec:conclusion}

In conclusion, we have shown that the Rowhammer bug is still relevant to system
engineers and that users and vendors need to be aware of this issue.

We have shown a framework to test any binary to find bitflips which change the
execution path to a given, pre-defined outcome. We apply this framework to the
\texttt{sudo} program used to switch privileges at runtime on most Unix-like
operating systems. We show how many possible bitflips sit inside the binary and
libraries used by it, which allow bypassing privilege checks such as the
password prompt for the calling user.

In addition to that, we also apply our testing framework to the popular web
server \texttt{nginx}. In the server, we want to find all flips allowing to
bypass HTTP authentication checks. Such a bypass would allow an attacker to
access pages or files which he could not do before as he was not granted the
access rights in the original configuration.

Besides analysing the impact of bitflips in programs\textquotesingle binary
files, we also look at the consequences of flips applied to memory generated
and used at runtime by programs. As an example of such an attack, we show how
bitflips can introduce misusage of nonces inside the OpenSSL implementation of
AES-GCM. We create nonce re-use using Rowhammer and with it break the
cryptographic security offered by the cypher.

With our work, we want to motivate further research in the field of Rowhammer
attacks and microarchitectural attacks in general. This thesis is a basis for
further improvements in the testing of software being influenced by
side-channels and hardware faults. We name examples for future topics in this
field and how other research fields could also play a role in the development of
microarchitectural software testing.

We present a summary of countermeasures already applied to systems against such
attacks and how developers need to extend them. We also line out that open
architectures such as RISC-V could play a vital role in the future of research
in this field and how researchers can benefit from them.

We want to close with a recommendation for vendors of hardware parts, such as
CPUs or DRAM chips, that besides looking at performance increasements and space
reductions, they should not neglect the quality of their hardware. Not only to
have reliable components in a manner of stability but also to provide
reliability in the sense of security, so that exploits like Meltdown and Spectre
are not possible anymore and that Rowhammer like attacks do not happen in any
case because of a higher quality in the chips.
%}}}

%% vim:foldmethod=expr
%% vim:fde=getline(v\:lnum)=~'^%%%%\ .\\+'?'>1'\:'='
%%% Local Variables:
%%% mode: latex
%%% mode: auto-fill
%%% mode: flyspell
%%% eval: (ispell-change-dictionary "en_US")
%%% TeX-master: "main"
%%% End:
