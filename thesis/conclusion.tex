%%%%%% Conclusion %%%%%%%%%%%%%%%%%%%%%%%%%%%%%%%%%%%%%%%%%%%%%%%%%%%%%%%%%%%{{{
\chapter{Conclusion}\label{sec:conclusion}

In conclusion, we have shown that the Rowhammer bug is still relevant to system
engineers and that users and vendors need to be aware of this issue.

We present a framework to test any binary to find bitflips which change the
execution path to a given, pre-defined outcome. We apply this framework to
programs available for Linux-based operating systems. We gain privilege
escalation by exploiting \texttt{sudo}, bypass the credential check for local
login and remote login, and bypass HTTP basic authentication implemented by the
\texttt{nginx} web server.

For \texttt{sudo}, Gruss~\etal~\cite{flipinthewall} show \num{29} bitflips which
yield privilege escalation, which they find by manual analysis of the ELF files
used by the program. We show that our framework finds over \num{10} times more
bitflips gaining the same outcome in a few hours of testing time.

Besides searching for exploitable bitflips in ELF files, we also look at the
consequences of bitflips applied to memory generated and used at the runtime  of
programs. As an example, we show how bitflips can introduce nonce reuse in the
AES-GCM implementation of OpenSSL.

We present a summary of countermeasures already applied to systems against
Rowhammer and how developers need to extend them. Also, we discuss measures
which make executables or more secure by checking the ELF files before
execution, or check the process\textquotesingle memory during the runtime.

We want to close with a recommendation for vendors of hardware parts, such as
CPUs or DRAM chips, that besides looking at increasing performance and space
reductions, they should not neglect the quality of their hardware. Not only to
have reliable components in a manner of stability, but also to provide
reliability in the sense of security. So that exploits like Meltdown and Spectre
are not possible anymore and that issues like Rowhammer do not occur anymore
because of the higher quality of chips.
%}}}

%% vim:foldmethod=expr
%% vim:fde=getline(v\:lnum)=~'^%%%%\ .\\+'?'>1'\:'='
%%% Local Variables:
%%% mode: latex
%%% mode: auto-fill
%%% mode: flyspell
%%% eval: (ispell-change-dictionary "en_US")
%%% TeX-master: "main"
%%% End:
