%%%%%% Conclusion %%%%%%%%%%%%%%%%%%%%%%%%%%%%%%%%%%%%%%%%%%%%%%%%%%%%%%%%%%%{{{
\chapter{Conclusion}\label{sec:conclusion}

We have shown that the Rowhammer bug is still relevant to system
engineers and that users and vendors need to be aware of this issue.

We present a framework to test any binary to find bitflips which change the
execution path to a given, pre-defined outcome. We apply this framework to
programs available for GNU/Linux operating systems. We gain privilege
escalation by exploiting \texttt{sudo}, bypass the credential check for local
and remote login measures, and bypass HTTP basic authentication implemented by
the \texttt{nginx} web server.

For \texttt{sudo}, Gruss~\etal~\cite{flipinthewall} show \num{29} bitflips which
yield privilege escalation, which they find by manual analysis of the ELF files
used by the program. We show that our framework finds over \num{10} times more
bitflips gaining the same outcome in a few hours of testing time.

Besides searching for exploitable bitflips in ELF files, we also look at the
consequences of bitflips applied to memory generated and used at the runtime of
programs. As an example, we show how bitflips can introduce nonce reuse in the
AES-GCM implementation of OpenSSL.

We present a summary of countermeasures already applied to systems against
Rowhammer and how developers need to extend them. Also, we discuss measures
which make systems more secure by checking the ELF files before
execution, or check the memory of the process during the runtime.

We close with a recommendation for vendors of hardware, such as CPUs or DRAM
chips, that besides increasing performance and space reduction, also the
security of products should be improved. Hardware vendors and researchers should
work together more often to find new microarchitectural flaws in hardware, and
thereby improve the security for users.
%}}}

%% vim:foldmethod=expr
%% vim:fde=getline(v\:lnum)=~'^%%%%\ .\\+'?'>1'\:'='
%%% Local Variables:
%%% mode: latex
%%% mode: auto-fill
%%% mode: flyspell
%%% eval: (ispell-change-dictionary "en_US")
%%% TeX-master: "main"
%%% End:
