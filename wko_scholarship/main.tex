
\documentclass[a4paper]{article}
\usepackage[cm]{fullpage}
\usepackage[utf8]{inputenc}

\newcommand{\etal}{et~al. }

\title{Finden von neuen Rowhammer Lücken \\ \vspace{+2ex}\small{Diplomarbeit}}
\author{David Bidner}
\date{\vspace{-5ex}}

\usepackage{natbib}
\usepackage{graphicx}

\begin{document}
\maketitle

Der Einfluss von Computern, Handys und dem Internet auf das Leben wird immer
größer und diese Technologien werden für vieles im Alltag verwendet.
Banktransaktionen, Einkäufe, E-Mails und mittlerweile können auch Verträge
elektronisch signiert werden. Wir müssen daher auf diese Technologien vertrauen
können. Sowohl sollten sie immer Funktionieren und das gewünschte Verhalten
aufweisen, als auch sicher sein. Persönliche Daten, Passwörter, Fotos und
ähnlich vertrauliche Dinge sollten vor unbefugten Personen geschützt sein.
Fehler, welche die Sicherheit betreffen, können erhebliche Folgen haben, wobei
nicht nur finanzielle Schäden entstehen können sondern auch Identitätsdiebstahl,
oder der Verlust von wertvollen Firmendaten, sind möglich.

\paragraph{}

Benutzer von Handys, Computern und ähnlichen Produkten haben eine gewisse
Erwartung, nicht nur was die Sicherheit betrifft. So wünschen sich Kunden unter
anderem immer schnellere und bessere Hardware. Um diesem Druck gerecht zu werden
müssen Hersteller von Hardware immer besser werden und ihre Designs immer wieder
überarbeiten und sich neue Optimierungen ausdenken. Dass diese Optimierungen
nicht immer perfekt sind, haben Angriffe wie Meltdown~\cite{meltdown}
und Spectre~\cite{spectre} gezeigt, wobei auch der ORF~\cite{orfmeltdown}, der 
britische BBC~\cite{bbcmeltdown} und die amerikanische NBC~\cite{nbcmeltdown} 
berichteten.

\paragraph{}

Diese Angriffe zielten auf Lücken in CPUs von Intel, AMD und ARM ab, welche
von nahezu allen Computern und Handys verwendet werden. Diese Lücken 
ermöglichen Datendiebstahl. Sowohl einfache Benutzer können Daten von anderen
stehlen, als auch Webseiten wird ermöglicht Daten vom PC des Besuchers zu
stehlen, was schwerwiegende Folgen hat. Neben diesen Angriffen auf die CPU
von Systemen, haben Forscher in der Vergangenheit auch Probleme in 
Speicherchips aufgezeigt. 2014 zeigten Kim~\etal\cite{rowhammergeneral} dass 
die RAM Module in vielen PCs ein Problem haben welches den Speicher von anderen
Programmen ändert wenn ein Programm gewisse Speicherzugriffe ausführt.
Normalerweise sollte ein Programm nie den Speicher eines anderen verändern 
dürfen! Dieses Problem wurde Rowhammer genannt.

\paragraph{}

Was Anfangs nur als Stabilitätsproblem gesehen wurde, wurde von Google's Project
Zero~\cite{projectzerorow} als schwerwiegendes Sicherheitsproblem identifiziert.
Mit gezielten Zugriffen und den richtigen Speicheränderungen wird es einem 
Angreifer ermöglicht Administrator zu werden. Wenn dieser Fall eintrifft hat
der Angreifer die volle Kontrolle über das System. Angriffe mittels Rowhammer
können nicht nur von lokalen Benutzer gestartet werden, es ist auch möglich
Systeme über das Öffnen einer Webseite zu attackieren~\cite{rowhammerjs}.
Ebenso betrifft das Problem nicht nur Computer, sonder auch
Handys~\cite{drammer}. Es wurde dieses Jahr bekannt, dass Rowhammer-Angriffe 
über das bloße Senden von Netzwerkpaketen möglich sind, ohne jegliche
Interaktion von Benutzern~\cite{nethammer}. 

\paragraph{}

In meiner Arbeit finde ich weitere mögliche Angriffe und automatisiere die Suche
nach möglichen Angriffszielen. So ist bereits bekannt, dass die Änderung
eines einzigen Bits in der Systemadministratorverwaltung von Computersystemen
einem Angreifer ermöglicht Administrator zu werden, ohne das
richtige Passwort zu kennen~\cite{flipinthewall}. Diese Suche wurde manuell
durchgeführt. Ich zeige, dass das Finden solcher Lücken schnell und 
automatisch möglich ist. Hierbei, greife ich nicht nur die
Administratorverwaltung an, sondern auch Mechanismen zum Einloggen an Systemen
und zur Identifikation. Zusätzlich untersuche ich Mögliche Auswirkungen von 
Rowhammer auf  Verschlüsselungen, welche dadurch gebrochen werden und geheime
Daten preisgeben können.

\paragraph{}

Die Arbeit soll auf diese Sicherheitslücken hinweisen und neben neuen Angriffen
werden auch mögliche Gegenmaßnahmen diskutiert, welche Betriebssystemhersteller
und Hardwareproduzenten einbauen können um diesen Angriffen vorzubeugen.
Meine bisherigen Ergebnisse zeigen, dass sich die Suche von Rowhammer Fehlern
automatisieren lässt, auch kann ich bekannte Angriffe reproduzieren. Zusätzlich
kann ich viele zusätzliche, ausnutzbare Rowhammer Ziele in der 
Administratorverwaltung \texttt{sudo} zeigen. Auch ist es mir möglich einen 
Webserver so zu verändern dass Seiten ausliefert auf welche man eigentlich
keinen Zugriff haben sollte. Angriffe auf das System zum einloggen auf Systemen
funktionieren ebenso, so wurden Ziele in der Software gefunden welche für lokale
Anmeldefunktionen zuständig ist, als auch im System für Anmeldungen über das
Netzwerk. Es ist daher für einen Angreifer möglich sich an einem angegriffenen 
System einzuloggen und Administrator zu werden ohne Zugangsdaten zu haben.

\pagebreak

\bibliographystyle{plain}
\bibliography{references}
\end{document}
