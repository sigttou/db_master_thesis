\documentclass[a4paper]{article}
\usepackage[cm]{fullpage}
\usepackage[utf8]{inputenc}
\pagenumbering{gobble}

\newcommand{\etal}{et~al. }

\title{Finden von neuen Rowhammer Lücken \\ \vspace{+2ex}\small{Diplomarbeit}}
\author{David Bidner}
\date{\vspace{-5ex}}

\usepackage{natbib}
\usepackage{graphicx}

\begin{document}
\maketitle

Der Einfluss von Computern, Handys und dem Internet auf das Leben wird immer
größer und diese Technologien werden für vieles im Alltag verwendet. So benutzen
wir unsere Telefone schon lange nicht mehr nur um zu Telefonieren oder zum
Versenden von SMS. Das Handy ist ein Alltagsgegenstand wie Wohnungsschlüssel und
die Brieftasche, man hat es immer dabei. Computer und Handys werden auch immer
öfter für Einkäufe verwendet, hierbei werden vertrauliche Daten, wie die
Kreditkartennummer an den Verkäufer gesendet. Auch firmeninterne Daten befinden
sich oft auf Computern, oder werden mittels E-Mail auch an das Smartphone
gesendet.

Die Sicherheit von Handys und Computern ist sehr wichtig. Persönliche
Daten, Passwörter, Telefonnummern, E-Mails, private Fotos und ähnlich
vertrauliche Daten müssen geschützt sein. Angreifer die Zugriff auf die
Geräte haben, können sehr leicht Daten auslesen und damit erheblichen
Schaden anrichten. Wobei es zu finanziellem, durch zum Beispiel das
Abhandenkommen von Kreditkartendaten, als auch zu persönlichem Schaden,
wenn z.B. E-Mail Konten übernommen werden, kommen kann.

\paragraph{}

Es wird regelmäßig neue Hardware gekauft, jedoch wird bei einem Neukauf selten
auf höhere Sicherheit geachtet. So muss ein neues Handy besser sein als das
alte. Ein schnellerer Prozessor, mehr Speicher, eine höhere Bildschirmauflösung
sind oft die Anforderungen.
Diese Forderungen üben einen gewissen Druck auf die Hersteller aus. Sie müssen
Optimierungen entwerfen, Chips immer kleiner bauen und dabei noch eine gewisse
Qualität einhalten. Dass dies nicht immer gut geht, hat
Kim~\etal~\cite{rowhammergeneral} in 2014 gezeigt. Es wurden Hardwaremodule für
Arbeitsspeicher gezeigt, welche bei gewissen Abfolgen von Speicherzugriffen auf
Speicherzellen, andere Speicherstellen ändern, was normalerweise niemals möglich
sein sollte! Es hat sich gezeigt, dass an diesem Fehler die steigenden
Geschwindigkeiten und kleiner werdenden Speicherchips Schuld sind.

Was am Anfang nur als Stabilitätsproblem gesehen wurde, wurde von Google's
Project Zero~\cite{projectzerorow} als schwerwiegendes Sicherheitsproblem
identifiziert. Es wurde gezeigt, dass es gezielt möglich ist Speicherstellen so
zu ändern, dass sie für einen potentiellen Angreifer nützlich werden. Die
Methode fremden Speicher mittels speziellen Zugriffen zu ändern wurde
``Rowhammer'' genannt. Es wurde gezeigt, dass durch Rowhammer-Angriffe ein
normaler Nutzer Administrator werden kann, ohne das Passwort zu
wissen~\cite{projectzerorow}. Später wurde auch bekannt, dass Rowhammer
ebenfalls mittels Webseiten funktioniert. Hier reicht es aus, dass der Angreifer
ein Opfer dazu bringt seine Webseite aufzumachen, danach kann der Angreifer
über das Internet Administrator am Computer des Opfers
werden~\cite{rowhammerjs}. Neben Arbeitsspeicherchips in Computern, sind auch
die Speicherchips von Mobiltelefonen betroffen~\cite{drammer}.

\paragraph{}

Aktuelle, bekannte Angriffe sind aufwendig, daher beschäftige ich mich in
meiner Arbeit mit weiteren Zielen für Speicheränderungen. Ich automatisiere die
Suche nach diesen Zielen. So ist bereits bekannt, wie in der
Administratorverwaltung eine einzelne Speicherzelle geändert werden muss um
einen einfachen Benutzer zum Administrator zu machen. Jedoch wurde für diese
Anwendung nur eine kleine Anzahl an möglichen Zielen
präsentiert und die Suche hat sich als sehr zeitaufwendig
herausgestellt~\cite{flipinthewall}. Ich zeige, dass das Finden solcher Ziele
viel schneller und automatisch funktioniert. Zusätzlich zur
Administratorverwaltung, präsentiere ich auch mögliche Angriffe auf die
Anmeldesysteme von Computern oder auf Webserver. Neben Angriffen auf Programme
zeige ich auch mögliche Auswirkungen von Rowhammer auf Verschlüsselungen, welche
dadurch gebrochen werden und geheime Daten preisgeben.

Meine Arbeit zeigt Angriffe, die mittels dem Ändern einer einzigen
Speicherstelle einem Angreifer ermöglichen, dass dieser sich an Systemen
anmelden kann ohne das Passwort zu kennen. Diese Angriffe funktionieren sowohl
lokal als auch über das Internet. Zusätzlich zeige ich über zehn mal mehr Ziele
in der Administratorverwaltung, welche es ermöglichen einem einfachen Nutzer
Administratorrechte zu bekommen, erneut ohne das richtige Passwort zu kennen.
Ich finde ebenfalls Ziele in einem der populärsten Webserver, Angriffe auf
diesen ermöglichen das Zugreifen auf Seiten die nur für einen authentifizierten
Benutzer zugänglich sind. Mit solchen Angriffen und dem Wissen über
Speicherstellen welche zu ändern sind, ist es einem Angreifer möglich sich an
Systemen anzumelden, Administrator zu werden und damit Zugriff auf das komplette
System zu haben. Daten- und Identitätsdiebstahl sind damit ein leichtes
Vorgehen. Die Ziele im Webserver ermöglichen dem Angreifer auf Daten von anderen
zuzugreifen oder Daten von Firmen zu stehlen, die nicht für die Öffentlichkeit
bestimmt sind.

\paragraph{}

Die Arbeit soll auf diese Sicherheitslücken und Angriffe hinweisen und neben
der Suche von neuen Rowhammer-Zielen werden auch Gegenmaßnahmen diskutiert,
welche Hersteller von Betriebssystemen, Arbeitsspeichern und Handys einbauen
sollen um diesen Angriffen vorzubeugen.

\pagebreak

\bibliographystyle{plain}
\bibliography{references}
\end{document}
