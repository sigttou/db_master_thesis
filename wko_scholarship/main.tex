\documentclass[a4paper]{article}
\usepackage[cm]{fullpage}
\usepackage[utf8]{inputenc}

\newcommand{\etal}{et~al. }

\title{Finden von neuen Rowhammer Lücken \\ \vspace{+2ex}\small{Diplomarbeit}}
\author{David Bidner}
\date{\vspace{-5ex}}

\usepackage{natbib}
\usepackage{graphicx}

\begin{document}
\maketitle

Der Einfluss von Computern, Handys und dem Internet auf das alltägliche Leben wird immer größer. Benutzer von solchen Technologien und Produkten haben eine gewisse Erwartung, so wünschen sich Kunden unter anderem immer schnellere und bessere Hardware. Um dieses Druck gerecht zu werden müssen Hersteller von Hardware immer besser werden und ihre Designs immer wieder überarbeiten und sich neue Optimierungen ausdenken. Dass diese Optimierungen nicht immer perfekt sind, haben Veröffentlichungen wie Meltdown~\cite{meltdown} und Spectre~\cite{spectre} von Kocher und Lipp~\etal gezeigt, wobei auch der ORF~\cite{orfmeltdown}, der britische BBC~\cite{bbcmeltdown} und die amerikanische NBC~\cite{nbcmeltdown} berichteten.

\paragraph{}

Wobei diese Fehler von CPU Herstellern, wie Intel, AMD und ARM hervorgerufen wurden und es erlaubten Speicher zu lesen auf welchen man eigentlich keinen Zugriff haben sollte, gab es auch in einer anderen Chip-Design Branche Probleme. DRAM Hersteller wurden durch die steigenden Erwartungen dazu gezwungen immer kleinere und schnellere Chips zu produzieren. Darunter litt jedoch die Qualität. In 2014 zeigte Kim~\etal\cite{rowhammergeneral}, dass es Probleme in DRAM gibt, welche es ermöglichen dass sich Speicherstellen ändern ohne dass auf diese Zugegriffen wird. Dieses Problem wurde Rowhammer genannt.

\paragraph{}

Was Anfangs nur als Stabilitätsproblem gesehen wurde, wurde von Google's Project Zero~\cite{projectzerorow} als schwerwiegender Hardwarefehler aufgezeigt. Diese Gruppe zeigte, dass es war möglich ist mit gezielten Speicherzugriffen Administrator-Rechte zu erhalten. Es folgte ein steigendes Interesse an der Forschung in diesem Bereich. Veröffentlichungen wurden gemacht, welche zeigten dass Rowhammer-Angriffe auch über das bloße öffnen einer Webseite möglich sind~\cite{rowhammerjs}, dass nicht nur Computer betroffen sind, sondern auch Handys~\cite{drammer}, und außerdem wurde dieses Jahr bekannt, dass Rowhammer-Angriffe auch durch das bloße Senden von Netzwerkpaketen möglich sind, ohne jegliche Interaktion von Benutzern~\cite{nethammer}.

\paragraph{}

Daniel Gruss~\etal haben in ihrer Arbeit~\cite{flipinthewall} gezeigt dass die Änderungen eines einzelnen Bits, wie sie bei Rowhammer zu finden sind, verschiedenen Applikationen betreffen können. In der Veröffentlichung zeigen sie wie die Passwort-Abfrage in \texttt{sudo}, ein Programm zur Vergabe von Administrator-Rechten, umgangen werden kann. In meiner Arbeit möchte ich darauf eingehen und diese Herangehensweise verallgemeinern und die Suche von ähnlichen Lücken automatisieren.

\paragraph{}

Ich möchte moderne Testmethoden mit dem bekannten Wissen über Rowhammer kombinieren um so in schneller Zeit mehr Ergebnisse erzielen. Dadurch kann ich auf die Gefahren von Rowhammer und ihren Einfluss auf die Sicherheit von Systemen hinweisen. Diese Tests sollen zeigen dass jede Anwendung und jedes System betroffen ist. 

\paragraph{}

Die Arbeit soll auch dazu dienen Herstellern von Hardware klar zumachen dass sie mehr auf Sicherheit Rücksicht nehmen müssen. Nur durch bessere Sicherheitsmaßnahmen lassen sich Attacken wie Meltdown, Spectre und Rowhammer vermeiden. Neben verschiedenen Rowhammer-Zielen in Anwendungen will ich in meiner Arbeit auch mögliche Gegenmaßnahmen präsentieren, wobei diese nicht nur gegen Rowhammer sind, sondern auch andere Attacken betreffen. Eine umsetzung solcher Schritte kann man besser durch Aufmerksamkeit in der Öffentlichkeit bewirken. Durch Verbesserungen der Hardware und Berichtigungen von solchen Fehlern werden Computer und dadurch die Daten von Nutzern sicherer.

\paragraph{}

Meine bisherigen Ergebnisse zeigen dass sich die Suche von Rowhammer Fehlern automatisieren lässt, auch kann ich die Angriffe von Gruss~\etal\cite{flipinthewall} reproduzieren und sogar mehr Speicherbereiche finden als sie aufgezeigt haben. Zusätzlich wurden Sicherheitsmechanismen in modernen Webservern, wie \texttt{nginx}, umgangen. Auch konnten zielführende Flips in der sicheren Computer Kommunikationssoftware \texttt{ssh} gefunden werden, welche Logins über das Netzwerk ohne Passwort ermöglichten.


\pagebreak

\bibliographystyle{plain}
\bibliography{references}
\end{document}
